\subsection{De Bruijn indices}
\label{section:background_debruijn}
Consider this expression,
\begin{equation*}
  x \colon \iota \in \Gamma \vdash (\lambda \, y \colon \iota \to \iota. y x) \colon \iota.
\end{equation*}

Using an $\alpha$-renaming $y \mapsto_{\alpha} f$, we get a different expression which is
equivalent.
\begin{equation*}
  x \colon \iota \in \Gamma \vdash (\lambda \, f \colon \iota \to \iota. f x) \colon \iota.
\end{equation*}
In fact, $\alpha$-renaming forms an equivalence relation \citep[section~5.3]{pierce_types_2002}, and
so is also called $\alpha$-equivalence \citep{aydemir_engineering_2008}. Like any other equivalence
relation, this induces a quotient set. We say that we have \textit{quotient} inductive definitions
\citep{aydemir_engineering_2008}; we have both an inductive definition of the $\lambda$-calculus,
but also (infinitely) many $\alpha$-equivalence classes. In some proof assistants, this makes proofs
difficult, as they cannot handle quotient definitions well \citep{pitts_locally_2023}.

We can solve this by using \textit{De Bruijn indices}, which eliminate $\alpha$-renamings. De Bruijn
indices, first described by Nicolaas Govert \citet{de_bruijn_lambda_1972}, are a natural number
which reference a variable. The natural number is how many binders from the current reference the
variable is bound at. In our example,
\begin{equation*}
  \iota, \in \Gamma \vdash (\lambda \, 0 1) \colon \iota.
\end{equation*}
Here, $0$ indicates that we are referring to the most recently bound variable, the one bound by the
$\lambda$-abstraction. The other number, $1$, indicates that we should move $1$ binder back. In this
case, we enter the context and see that it is referencing some variable of type $\iota$.

However, if the context changes, the indices would need to change too. For example, if we added
another variable to the context, we would need to reindex the $1$ to a $2$:
\begin{equation*}
  \iota, \iota \to \iota \in \Gamma \vdash (\lambda \, 0 2) \colon \iota,
\end{equation*}
Dealing with de Bruijn indices in the context becomes very difficult, becomes of the above
demonstrated reindexing requirement \citep[section~2.1]{aydemir_engineering_2008}.

\subsection{Locally Nameless Representation}
\label{section:background_locally_nameless}
Using \textit{locally nameless representation}, we can get both purely inductive definitions and
solve the reindexing issue. Free variables use variable names while bound variables use indices. Our
example becomes
\begin{equation*}
  x \colon \iota \in \Gamma \vdash (\lambda \, 0 x) \colon \iota,
\end{equation*}
or with another variable in the context,
\begin{equation*}
  x \colon \iota, y \colon \iota \to \iota \in \Gamma \vdash (\lambda \, 0 x) \colon \iota.
\end{equation*}

Although locally nameless representation was described earlier, our approach of combining it with
cofinite quantification was first described by \citet{aydemir_engineering_2008}.

As part of using locally nameless terms, common operations and properties (called `infrastructure'
by \citet{aydemir_engineering_2008}) need to be defined for the target language, which will be
detailed below. A recent article by Andrew \citet{pitts_locally_2023} uses Agda to explore an
abstraction of this representation called locally nameless sets, and shows that this infrastructure
can be defined in a syntax-agnostic way.

For the development of the System F formalisation in chapter \ref{chapter3}, the infrastructure
required is term and type opening (section \ref{chapter3:opening}), local closure (section
\ref{chapter3:local_closure}), and the various substitutions (section
\ref{chapter3:substitution_syntax}). Term opening and local closure are defined below.

The drawback of using this representation is that we need to define infrastructure and prove some
properties of the syntax (detailed in sections \ref{chapter3:opening} to
\ref{chapter3:substitution_syntax}) before we can prove the properties of the metalangauge. But
usually, this drawback is too small to outweight the benefits.

\paragraph*{Opening and closing.} Two fundamental operations on locally nameless terms are
\textit{term opening} and \textit{term closing} \citep{pitts_locally_2023}. Opening will replace all
occurrences of a bound variable with a free variable, written $[i \to x] L$ for some De Bruijn
index $i$, name $x$, and term $L$. For example,
\begin{equation*}
  [0 \to y] (0 q (\lambda \, 1 t 0)) \quad \mapsto \quad y q (\lambda \, y t 0).
\end{equation*}
Every occurence of the bound variable $0$ is replaced with the free variable $y$. Note how after we
go under a new $\lambda$-abstraction, we have to increment this index to $1$ to keep referring to
the same bound variable.

Closing works similarly, and is the inverse of opening; it replaces all occurrences of a free
variable with a given index. For example,
\begin{equation*}
  [0 \leftarrow y] (y q (\lambda \, y t 0)) \quad \mapsto \quad 0 q (\lambda \, 1 t 0).
\end{equation*}

\paragraph*{Local closure.} A term is said to be \textit{locally closed} up to level $i$ if it
remains unchanged after opening it at index $i$ with an arbitrary string. For some $i \in \iota$ and
term $L$, we write that $L$ is locally closed at level $i$ as $i \succ L$. Formally,
\begin{equation}
  \label{equation:local_closure}
  i \succ L \triangleq \forall j \geq i, \; \cof a , \; [j \to a] L = L.
\end{equation}
If it is locally closed at level $0$, we simply call it \textit{locally closed}. Local closure is
defined for System F in section \ref{chapter3:local_closure}.

As detailed in \citet{aydemir_engineering_2008}, local closure can be defined using universal
quantification instead. For example, equation (\ref{equation:local_closure}) could read $\forall a
\notin \fv(L), \; [j \to a] L = L$ (where $\text{fv} \colon \text{Terms} \to \text{Identifiers}$
returns all identifiers of free variables). The benefit of using cofinite quantification comes when
we use quantified properties in assumptions of theorems.

Suppose we have as an assumption the local closure of terms $L$ and $M$ and we wanted to prove that
$LM$ was locally closed. We would need to combine the two `disallowed' lists together. We would run
into the same issue described in the example given in section \ref{background:cofinite}, simply
replace $P$ and $Q$ with the local closure property.

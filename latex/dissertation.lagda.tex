% UG project example file, February 2024
%
%   Added the "online" option for equal margins, February 2024 [Hiroshi Shimodaira, Iain Murray]
%   A minor change in citation, September 2023 [Hiroshi Shimodaira]
%
% Do not change the first two lines of code, except you may delete "logo," if causing problems.
% Understand any problems and seek approval before assuming it's ok to remove ugcheck.
\documentclass[logo,bsc,singlespacing,parskip,online]{infthesis}
\usepackage{ugcheck}


% Include any packages you need below, but don't include any that change the page
% layout or style of the dissertation. By including the ugcheck package above,
% you should catch most accidental changes of page layout though.

\usepackage{microtype} % recommended, but you can remove if it causes problems
\usepackage[round]{natbib} % recommended for citations

% === Custom packages === %

% Syntax highlighting
\usepackage{minted}
% BNF
\usepackage{simplebnf}
% Inference rules
\usepackage{mathpartir}
% Agda
\usepackage{agda}
\AgdaNoSpaceAroundCode{}
% Hide in output
\usepackage{comment}
% Colour
\usepackage{xcolor}

% Unicode

\usepackage{fontspec}
\usepackage{newunicodechar}
\newfontface{\notosansmono}{NotoSansMono-Regular.ttf}[Path = fonts/]
\newfontface{\ibmplexmath}{IBMPlexMath-Regular.otf}[Path = fonts/]
\newfontface{\hack}{HackNerdFontMono-Regular.ttf}[Path = fonts/]
\newunicodechar{ℕ}{{\notosansmono{ℕ}}}
\newunicodechar{∀}{{\notosansmono{∀}}}
\newunicodechar{≡}{{\notosansmono{≡}}}
\newunicodechar{≥}{{\notosansmono{≥}}}
\newunicodechar{≤}{{\notosansmono{≤}}}
\newunicodechar{≰}{{\notosansmono{≰}}}
\newunicodechar{⊔}{{\notosansmono{⊔}}}
\newunicodechar{≟}{{\notosansmono{≟}}}
\newunicodechar{⇒}{{\hack{⇒}}}
\newunicodechar{≢}{{\notosansmono{≢}}}
\newunicodechar{≮}{{\notosansmono{≮}}}
\newunicodechar{⟪}{{\notosansmono{⟪}}}
\newunicodechar{⟫}{{\notosansmono{⟫}}}
\newunicodechar{⊤}{{\notosansmono{⊤}}}
\newunicodechar{⊥}{{\notosansmono{⊥}}}
\newunicodechar{∣}{{\notosansmono{∣}}}
\newunicodechar{⟨}{{\notosansmono{⟨}}}
\newunicodechar{⟩}{{\notosansmono{⟩}}}
\newunicodechar{∎}{{\notosansmono{∎}}}
\newunicodechar{⌊}{{\notosansmono{⌊}}}
\newunicodechar{⌋}{{\notosansmono{⌋}}}
\newunicodechar{₁}{{\notosansmono{₁}}}
\newunicodechar{₂}{{\notosansmono{₂}}}
\newunicodechar{∃}{{\notosansmono{∃}}}
\newunicodechar{∷}{{\notosansmono{∷}}}
\newunicodechar{∈}{{\notosansmono{∈}}}
\newunicodechar{∉}{{\notosansmono{∉}}}
\newunicodechar{λ}{{\notosansmono{λ}}}
\newunicodechar{И}{{\notosansmono{И}}}
\newunicodechar{∘}{{\notosansmono{∘}}}
\newunicodechar{≠}{{\notosansmono{≠}}}
\newunicodechar{‵}{{\notosansmono{‵}}}
\newunicodechar{ƛ}{{\notosansmono{ƛ}}}
\newunicodechar{≻}{{\ibmplexmath{≻}}}
\newunicodechar{⦃}{{\ibmplexmath{⦃}}}
\newunicodechar{⦄}{{\ibmplexmath{⦄}}}
\newunicodechar{′}{{\notosansmono{′}}}
\newunicodechar{∋}{{\notosansmono{∋}}}
\newunicodechar{⦂}{{\ibmplexmath{⦂}}}
\newunicodechar{∅}{{\hack{∅}}}
\newunicodechar{⊢}{{\hack{⊢}}}
\newunicodechar{ξ}{{\notosansmono{ξ}}}
\newunicodechar{β}{{\notosansmono{β}}}
\newunicodechar{↠}{{\hack{↠}}}
\newunicodechar{ρ}{{\notosansmono{ρ}}}
\newunicodechar{σ}{{\notosansmono{σ}}}
\newunicodechar{⊆}{{\notosansmono{⊆}}}
\newunicodechar{⊎}{{\notosansmono{⊎}}}
\newunicodechar{∧}{{\notosansmono{∧}}}

% Maths:

\usepackage{mathtools}
\usepackage{amssymb}
\usepackage{enumitem}
\usepackage{hyperref}
\usepackage{xurl}
\DeclareMathOperator{\lcm}{lcm}
\DeclareMathOperator{\Real}{Re}
\DeclareMathOperator{\Imag}{Im}
\DeclareMathOperator{\complex}{\mathbb{C}}
\DeclareMathOperator{\reals}{\mathbb{R}}
\DeclareMathOperator{\nat}{\mathbb{N}}
\DeclareMathOperator{\integer}{\mathbb{Z}}
\DeclareMathOperator{\rational}{\mathbb{Q}}
\DeclareMathOperator{\Log}{Log}
\DeclareMathOperator{\Arg}{Arg}
\DeclareMathOperator{\cof}{\text{И}}

% Use minted for Agda

\let\oldcode\code
\NewCommandCopy{\mintedcopy}{\minted}
\NewCommandCopy{\endmintedcopy}{\endminted}
% Adapted from https://tex.stackexchange.com/a/488451/202867
\renewenvironment{code}{\mintedcopy[breaklines,breaksymbolleft=\;]{agda}}{\endmintedcopy}

\begin{document}
\begin{preliminary}

\title{System F in Agda via Pitts}

\author{Charlotte Ausel}

% CHOOSE YOUR DEGREE a):
% please leave just one of the following un-commented
% \course{Artificial Intelligence}
%\course{Artificial Intelligence and Computer Science}
%\course{Artificial Intelligence and Mathematics}
%\course{Artificial Intelligence and Software Engineering}
%\course{Cognitive Science}
%\course{Computer Science}
%\course{Computer Science and Management Science}
\course{Computer Science and Mathematics}
%\course{Computer Science and Physics}
%\course{Software Engineering}
%\course{Master of Informatics} % MInf students

% CHOOSE YOUR DEGREE b):
% please leave just one of the following un-commented
%\project{MInf Project (Part 1) Report}  % 4th year MInf students
%\project{MInf Project (Part 2) Report}  % 5th year MInf students
\project{4th Year Project Report}        % all other UG4 students


\date{\today}

\abstract{
This skeleton demonstrates how to use the \texttt{infthesis} style for
undergraduate dissertations in the School of Informatics. It also emphasises the
page limit, and that you must not deviate from the required style.
The file \texttt{skeleton.tex} generates this document and should be used as a
starting point for your thesis. Replace this abstract text with a concise
summary of your report.
}

\maketitle

\newenvironment{ethics}
   {\begin{frontenv}{Research Ethics Approval}{\LARGE}}
   {\end{frontenv}\newpage}

\begin{ethics}
This project was planned in accordance with the Informatics Research
Ethics policy. It did not involve any aspects that required approval
from the Informatics Research Ethics committee.

\standarddeclaration
\end{ethics}


\begin{acknowledgements}
Any acknowledgements go here.
\end{acknowledgements}


\tableofcontents
\end{preliminary}


\chapter{Introduction}

TODO.

This document is a literate Agda file and uses {\color{violet}colour}. Please see appendix
\ref{appendix:compilation_instructions} for details.

\begin{code}
module dissertation where
\end{code}

\chapter{Background}

\section{Agda}
Agda is a dependently-typed functional programming language based on Martin--Löf type theory, which
makes it suitable as a proof-assistant using intuitionistic logic \citep{norell_towards_2007}.

In Agda, most definitions are done inductively, that is, using recursion. For example, following the
Peano axioms for the natural numbers $\nat$ \citep{boolos_freges_1995}, we may define them like so.

\begin{code}
module Example where
  data ℕ : Set where
    zero : ℕ
    suc  : ℕ → ℕ
\end{code}


\paragraph*{Propositions as types.} First noted by William A. Howard in 1969
\citep{howard_formulae-as-types_1980}, there is a direct correspondence between proofs and programs.
\citet{wadler_propositions_2015} calls this correspondence \textit{propositions as types}, known as
the Curry-Howard correspondence by some authors. Thanks to this correspondence, a correct Agda
function with the appropriate type signature matching the claim (proposition) suffices for a proof.
For example, a proof that addition is associative would use recursion, which corresponds to
induction, as shown below.

\begin{code}
  -- We can import from the standard library, here we're using
  -- the reflexive and congruence properties of equality.
  open import Relation.Binary.PropositionalEquality
    using (_≡_; refl; cong)

  _+_ : ℕ → ℕ → ℕ
  zero  + m = m
  suc n + m = suc (n + m)

  +-assoc : ∀ (m n p : ℕ) → (m + n) + p ≡ m + (n + p)
  +-assoc zero    n p = refl
  +-assoc (suc m) n p = cong suc (+-assoc m n p)
\end{code}

Imports from the standard library \citep{the_agda_community_agda_2024} are omitted, but are
available in the full source file (see appendix \ref{appendix:compilation_instructions}).
\begin{comment}
\begin{code}
-- Data types (naturals, strings, characters)
open import Data.Nat using (ℕ; zero; suc; _<_; _≥_; _≤_; _≤?_; _<?_; z≤n; s≤s; _⊔_)
  renaming (_≟_ to _≟ℕ_)
open import Data.Nat.Properties using (≤-refl; ≤-trans; ≤-<-trans; <-≤-trans; ≤-antisym; ≤-total;
  ≤-pred; +-mono-≤; <-trans; n≤1+n; m≤n⇒m≤1+n; n≮n; <⇒≢; ≰⇒>; ≮⇒≥; ≤∧≢⇒<; 1+n≢n)
open import Data.String using (String; fromList; toList) renaming (_≟_ to _≟str_;
  _++_ to _++str_; length to str-length)
open import Data.Char using (Char)
open import Data.Char.Properties using () renaming (_≟_ to _≟char_)

-- Function manipulation.
open import Function using (_∘_; flip; it; id; case_returning_of_)

-- Relations and predicates/decidability.
import Relation.Binary.PropositionalEquality as Eq
open Eq using (_≡_; _≢_; refl; sym; trans; cong; cong-app; cong₂)
open Eq.≡-Reasoning using (begin_; step-≡-∣; step-≡-⟩; _∎)
open import Relation.Binary.Definitions using (DecidableEquality)
open import Relation.Nullary.Decidable using (Dec; yes; no; True; False; toWitnessFalse;
  toWitness; fromWitness; ¬?; ⌊_⌋; From-yes)
open import Relation.Unary using (Decidable)
open import Relation.Binary using () renaming (Decidable to BinaryDecidable)
open import Relation.Nullary.Negation using (¬_; contradiction)
open import Data.Empty using (⊥-elim)

-- Products and exists quantifier.
open import Data.Product using (_×_; proj₁; proj₂; ∃-syntax) renaming (_,_ to ⟨_,_⟩)
open import Data.Sum using (_⊎_; inj₁; inj₂; [_,_])

-- Lists.
open import Data.List using (List; []; _∷_; _++_; length; filter; foldr; head; replicate)
open import Data.List.Properties using (≡-dec)
import Data.List.Membership.DecPropositional as DecPropMembership
open import Data.List.Relation.Unary.All using (All; all?; lookup)
  renaming (fromList to All-fromList; toList to All-toList; map to All-map)
open import Data.List.Relation.Unary.Any using (Any; here; there)
open import Data.List.Extrema Data.Nat.Properties.≤-totalOrder using (max; xs≤max)

-- Import list membership using List Char comparisons.
_≟lchar_ : ∀ (xs ys : List Char) → Dec (xs ≡ ys)
xs ≟lchar ys = ≡-dec (_≟char_) xs ys

open DecPropMembership _≟lchar_ using (_∈_; _∉_; _∈?_)
\end{code}
\end{comment}

\section{Cofinite Quantification}

Often, we may want to prove a property for \textit{almost} all members of a set; a finite subset is
chosen for which the property is not proven (or disproven), while the property holds for all members
not in this subset. For example, for \textit{almost} all natural numbers $n$, $n \geq 2$. Or in
number theory, \textit{almost} all natural numbers can be written as a product of primes. To prove
this, one would need to give a set of numbers which the proof will ignore (in the example, $\{ (0),
1 \}$), and then prove it for all $n \in \nat$ with the extra assumption that $n \not \in \{0, 1\}$.

While formalising this notion of exception isn't usually useful in mathematics, we will find this to
be very useful later on. This can be viewed as a special kind of quantification, called
\textit{cofinite quantification} \citep{aydemir_engineering_2008}.

This quantification uses the symbol $\cof$. We write, for some set $X$ and property $P$,
\begin{equation}
  \label{equation:cofinite}
  \cof x \in X, P(x) \iff
  \exists S \subsetneq X, \, S \text{ is finite}, \, \forall x \in X \setminus S, \, P(x).
\end{equation}

While the Agda implementation that \citet{pitts_locally_2023} gives uses an arbitrary set, I chose
to use the specific \texttt{List Char}, since we will only quantify over strings.
\begin{code}
module cofinite where
  open import plfa_adaptions using (∉∷[]⇒≢)
\end{code}
\begin{comment}
\begin{code}
  -- Data types (naturals, strings, characters)
  open import Data.Nat using (ℕ; zero; suc; _≤_; z≤n; s≤s)
  open import Data.Nat.Properties using (≤-refl; ≤-trans; ≤-antisym; ≤-total)
  open import Data.Char using (Char)
  open import Data.Char.Properties using () renaming (_≟_ to _≟char_)

  -- Relations and predicates/decidability.
  import Relation.Binary.PropositionalEquality as Eq
  open Eq using (_≡_; _≢_; refl; sym; trans; cong)
  open import Relation.Binary.Definitions using (DecidableEquality)
  open import Relation.Nullary.Decidable using (Dec)
  open import Relation.Unary using (Decidable)
  open import Relation.Binary using () renaming (Decidable to BinaryDecidable)
  open import Relation.Nullary.Negation using (contradiction)

  -- Lists.
  open import Data.List using (List; []; _∷_; _++_; length; filter; map; foldr; head; replicate)
  open import Data.List.Properties using (≡-dec)
  import Data.List.Membership.DecPropositional as DecPropMembership
  open import Data.List.Relation.Unary.All using (All; all?; lookup)
    renaming (fromList to All-fromList; toList to All-toList)
  open import Data.List.Relation.Unary.Any using (Any; here; there)
  open import Data.List.Extrema Data.Nat.Properties.≤-totalOrder using (max; xs≤max)

  -- Import list membership using List Char comparisons.
  private
    _≟lchar_ : ∀ (xs ys : List Char) → Dec (xs ≡ ys)
    xs ≟lchar ys = ≡-dec (_≟char_) xs ys

  open DecPropMembership _≟lchar_ using (_∈_; _∉_; _∈?_)
\end{code}
\end{comment}
\begin{code}
  record Cof (P : List Char → Set) : Set where
    inductive
    eta-equality
    constructor И⟨_,_⟩
    field
      Иe₁ : List (List Char)
      Иe₂ : (a : List Char) {_ : a ∉ Иe₁} → P a
  open Cof public
  
  Cof-syntax : (P : List Char → Set) → Set
  Cof-syntax = Cof
  syntax Cof-syntax (λ a → P) = И a , P
\end{code}

For a simple example, I can show \textit{almost} all strings have a length greater or equal to $1$
(with the exception being the empty string).

\begin{code}
  simple-cof : {s : List Char} → И s , (1 ≤ length s)
  simple-cof = И⟨ [] ∷ [] , (
    λ{[] {a∉}  → contradiction refl (∉∷[]⇒≢ a∉)
    ; (x ∷ xs) → s≤s z≤n}) ⟩
\end{code}


\section{The Lambda Calculus and System F}
\subsection{The lambda calculus.}
The $\lambda$-calculus (pronounced \textit{lambda calculus}), is a theoretical model of computation
developed by Alonzo Church in the 1930s \citep{church_set_1932} and is the basis for System F. It
looks and works similarly to familiar functional programming languages, yet its definition is as
minimal as possible while still being Turing-complete. Although Turing machines would be invented
after the $\lambda$-calculus \citep{turing_computable_1937}, `turing-complete' has become a
shorthand for `universal method of computation'. Such a universal method was not Church's initial
goal, but is why we're still interested in the $\lambda$-calculus.

The $\lambda$-calculus includes three reductions. $\alpha$-conversion is the renaming of variables
such that the semantics of the program are not changed; terms which are semantically identical but
use different variable names are called $\alpha$-equivalent, and as the name would imply, this is an
equivalence relation \citep{pierce_types_2002}.

$\beta$-reduction describes function application. The evaluation rules for the $\lambda$-calculus
are as follows \citep{wadler_programming_2022} (where $N[M/x]$ means replacing the term $M$ for any occurrence of $x$ in the term $N$).

\begin{equation}
\label{equation:untyped_beta_rules}
\begin{split}
\inferrule{ }{(\lambda \, x. N)M \mapsto_{\beta} N[M/x]} \; (\beta) \quad
\inferrule{L \mapsto_{\beta} L'}{LM \mapsto_{\beta} L'M} \; (\xi_1) \quad\\
\inferrule{M \mapsto_{\beta} M'}{LM \mapsto_{\beta} LM'} \; (\xi_2) \quad
\inferrule{N \mapsto_{\beta} N'}{(\lambda \, x. N) \mapsto_{\beta} (\lambda \, x. N')} \; (\zeta)
\end{split}
\end{equation}

The last reduction is $\eta$-reduction; for all $\lambda$-abstractions $\lambda \, x. L x$, we have
that $\lambda \, x. f x \mapsto_{\eta} f$.

If further $\beta$-reducations do not simplify a term, we say that it is in its \textit{normal
form}. We shall represent an arbitrary number of successive $\beta$-reductions using the Kleene star
$\mapsto_{\beta^{\star}}$. The Church-Rosser theorem states that for any term $L$, if it
$\beta$-reduces to two terms $M$ and $N$, then there exists a common term $L'$ which both $M$ and
$N$ eventually $\beta$-reduce to \citep{church_properties_1936}. This will be an important
consideration when choosing an evaluation strategy (see section \ref{background:evaluation_strategy}.)

Notably, some terms may not have a normal form. A well-known example is \textit{little omega}
$\omega \triangleq \lambda \, x. (x x)$ and \textit{omega} (sometimes called the \textit{omega
combinator}) $\Omega \, \triangleq \, \omega \omega$. When $\Omega$ is applied to itself, it doesn't
reduce any further.

\begin{equation*}
  (\lambda \, x. (x x)) (\lambda \, x. (x x)) \quad
  \mapsto_{\beta} \quad (\lambda \, x. (x x)) (\lambda \, x. (x x))
\end{equation*}

The \textit{y-combinator} $\mathcal{Y} \triangleq (\lambda \, f. (\lambda \, x. f (x x )) (\lambda
\, x. f (xx)))$ allows for recursion. It will $\beta$-reduce to the argument applied to itself,
$\mathcal{Y} L \mapsto_{\beta^{\star}} L (\mathcal{Y} L)$.

We say that these terms \textit{do not have a normal form}.

\subsection{The simply-typed lambda-calculus}
The simply-typed $\lambda$-calculus (STLC, sometimes given the symbol $\lambda^{\rightarrow}$) is an
extension of the untyped $\lambda$-calculus developed by Alonzo \citet{church_formulation_1940}. It
requires each term to have a \textit{type}. In the STLC, terms which do not have a normal form
cannot be given a type, thus, all expressions will eventually reduce to an irreducible
expression---their normal form. This is also called \textit{strong normalisation}
\citep{pierce_types_2002}. However, this means we can longer represent all of the terms of the
untyped $\lambda$-calculus (such as $\Omega$), thus, Turing-completeness is lost.

In the STLC, some base types need to be chosen. These are indeterminates which are not given
definitions. While Church originally used the symbols $\iota$ and $\sigma$ for base types, we will
use $\nat$, for reasons given in section \ref{background:evaluation_strategy}. Without base types,
our computational model becomes \textit{degenerate} (that is, there are no terms)
\citep{pierce_types_2002}. As well as the base type, we also have a function (or arrow) type. For
some types $\tau$ and $\sigma$, writing $\tau \to \sigma$ means that this term can be applied using
a term of type $\tau$ and result in a type $\sigma$.

We will also need a \textit{type context} (also called \textit{type environment}), which will
usually be given the symbol $\Gamma$ or $\Delta$ and is separated using $\vdash$. This is a partial
map from variables to types $\Gamma \colon V \to T$. Types are written next to terms using a colon.
For example, if the context contained the mapping $x \colon \nat$, then we could write a function
which applies $x$ to its argument;
\begin{equation*}
  x \colon \nat \in \Gamma \vdash (\lambda \, f \colon \nat \to \nat . f x) \colon \nat.
\end{equation*}

Sometimes we may choose to omit the type of the bound variable for clarity, so we could write
$(\lambda \, f. \lambda \, x. fx) \colon (\nat \to \nat) \to \nat$.

In the STLC, we can't give a type to little omega, since we can't give a type to both the argument
and the argument applied to itself ($(\lambda \, x  \colon ? . x x) \colon ??$). The typing rules for
STLC are given in section \ref{appendix:type_judgements}.

\subsection{System F.}
System F has been the formal background to what many modern programming
languages call \textit{generics}. For an anachronistic example, in Rust, we could write a function
which applies a function twice to an argument.

\begin{minted}[samepage]{rust}
fn twice<T, F>(f: F, x: T) -> T
where
    F: Fn(T) -> T,
{
    f(f(x))
}
\end{minted}

Since we used a \textit{type parameter} in the function's type signature (here \texttt{T}), we can
use any appropriate function which has the type signature $\texttt{T} \to \texttt{T}$. One such
function is \texttt{u64::isqrt}, the (flooring) square root function. If this function was invoked
with \texttt{twice(u64::isqrt, 81)}, its output would be \texttt{3}. In this case, the compiler can
infer that the type for \texttt{T} should be \texttt{u64}, so we don't need to specify it
explicitly.

System F is the STLC equipped with \textit{polymorphic types}, another term for type parameters. It
was independently discovered by Jean-Yves \citet{girard_interpretation_1972} and John
\citet{goos_towards_1974}. We can write this \textit{twice} function like so in System F:

\begin{equation*}
  (\Lambda \, T. \lambda \, f \colon T \to T . \lambda \, x \colon T . f (f x))
  \colon \forall T . (T \to T) \to T \to T.
\end{equation*}

Like in the Rust example, if we wanted to use this function, we would usually need to explicitly
perform a type application to specify what type we're using, though if it can be understood by the
reader from context, we will omit it. We write type applications using [square brackets]. For
instance,
\begin{align*}
  s \colon \nat \to \nat, z \colon \nat \in \Gamma \vdash
  &(\Lambda \, T. \lambda \, f \colon T \to T . \lambda \, x \colon T . f (f x))[\nat] s z \colon \nat\\
  &\mapsto_{\beta} (\lambda \, f \colon \nat \to \nat . \lambda \, x \colon \nat . f (f x)) s z \colon \nat\\
  &\mapsto_{\beta^{\star}} s s z \colon \nat.
\end{align*}


\section{Evaluation strategy}
\label{background:evaluation_strategy}
The idea of reducible expressions, called a \textit{redex}, was introduced by Alonzo Church
\cite[p.~56]{pierce_types_2002}. We refer to any expression of the form $(\lambda \, x. M) N$ as a
redex, since we can perform a $\beta$-reduction (see equation \ref{equation:untyped_beta_rules}).

There are several ways of evaluating the $\lambda$-calculus. One important decision is whether to
treat $\lambda$-abstractions as values, or to reduce these if they are a redex.

When evaluating, we need to consider some terms to be \textit{values}. These cannot be reduced any
further. One method is to consider $\lambda$-abstractions to be values and to only evaluate
\textit{closed} terms, that is, terms without any free variables. This is called weak-head
reduction, and the values are called weak-head normal forms \citep{wadler_programming_2022}. This
method is also used by popular programming languages such as Haskell \citep{hutchison_sharing_2005}.
The definition of $\lambda$-calculus as Church first described it uses what is nowadays referred to
as \textit{full normalisation} \citep{wadler_programming_2022}. The reduction rule which is absent
in weak-head reduction is the $\zeta$ rule (see equation \ref{equation:untyped_beta_rules}), also
called \textit{reducing under a $\lambda$-abstraction}.

Using weak-head reduction, since evaluation stops at $\lambda$-abstractions, the Church-Rosser
property no longer applies (and, in fact, the system is no longer Turing complete). One of the early
explorations of weak-head reduction was made by \citet{cagman_combinatory_1998}, comparing it to
combinatory logic. They presented the necessity of another reduction rule which dictates when one
can reduce under a $\lambda$-abstraction, which restores the Church-Rosser property
\citep{hutchison_sharing_2005}.

However, another, simpler, method is to introduce primitives \citep{wadler_programming_2022}. These
primitives are a small subset of Dana Scott and Gordon Plotkin's Programming Computable Functions
(PCF) \citep{plotkin_lcf_1977}; we introduce a value called \texttt{‵zero} of type $\nat$ and a
function \texttt{‵suc} of type $\nat \to \nat$ (since these are modelled after the naturals, we use
the base type $\nat$). Further description is in section \ref{appendix:stlc_terms}.

Weak-head reduction was used by \citet{chargueraud_locally_2012} and
\citet[chapter~Lambda]{wadler_programming_2022}, so we choose to follow the same approach here.


\section{Locally Nameless Representation}

\subsection{De Bruijn indices}
\label{section:background_debruijn}
Consider this expression,
\begin{equation*}
  x \colon \nat \in \Gamma \vdash (\lambda \, y \colon \nat \to \nat. y x) \colon \nat.
\end{equation*}

Suppose we move this expression into a context where $y$ is already defined,
\begin{equation*}
  x \colon \nat, y \colon \nat \to \nat \in \Gamma \vdash (\lambda \, y \colon \nat \to \nat. y x) \colon \nat.
\end{equation*}

It's unclear whether we are referring to the bound $y$ or the free $y$. One way to resolve this is
to always choose the most recently declared variable. This is called \textit{shadowing}, since the
older definitions are `in the shadow of' the most recent occurence \citep{wadler_programming_2022}.
Another approach is to use an $\alpha$-conversion. Here we can apply $y \mapsto_{\alpha} f$ to the
$\lambda$-abstraction,
\begin{equation*}
  x \colon \nat, y \colon \nat \to \nat \in \Gamma \vdash (\lambda \, f. f x) \colon \nat,
\end{equation*}
which solves the problem. We can add further assumptions to our context without affecting the
semantics of the expression \citep{pitts_locally_2023} (for example, adding $k \colon \nat$ to
$\Gamma$ doesn't cause any name conflicts). I shall call this property weakening-invariance (taken
from proof theory, where extra assumptions make a theorem weaker \citep{buss_handbook_1998}).

Since we have these $\alpha$-equivalent expressions, we can say that we have \textit{quotient}
inductive definitions \citep{aydemir_engineering_2008}; we have both an inductive definition of the
$\lambda$-calculus, but also (infinitely) many $\alpha$-equivalence classes which induce a quotient
set. In some proof assistants, this makes proofs difficult, as they don't handle quotient
definitions well \citep{pitts_locally_2023}.

We can solve this by using \textit{De Bruijn indices}, which ensure variables in the context don't
cause name conflicts and accidental shadowing with bound variables. In fact, $\alpha$-conversions
cannot be performed anymore. In our example,
\begin{equation*}
  x \colon \nat, y \colon \nat \to \nat \in \Gamma \vdash (\lambda \, 0 2) \colon \nat.
\end{equation*}

However, if the context changes, the indices would need to change too. For example, if we remove the
$y$, then we need to reindex the $2$ to a $1$:
\begin{equation*}
  x \colon \nat \in \Gamma \vdash (\lambda \, 0 1) \colon \nat,
\end{equation*}
We have lost weakening-invariance \citep{aydemir_engineering_2008}. This leaves a demand for a
syntax which solves both the issue of quotient definitions and weakening-invariance.

\subsection{Locally Nameless Representation}
Using \textit{locally nameless representation}, we can get both purely inductive definitions
\textit{and} weakening-invariance. Free variables use variable names while bound variables use
indices. Our example becomes
\begin{equation*}
  x \colon \nat, \in \Gamma \vdash (\lambda \, 0 x) \colon \nat,
\end{equation*}
or with another variable in the context,
\begin{equation*}
  x \colon \nat, y \colon \nat \to \nat \in \Gamma \vdash (\lambda \, 0 x) \colon \nat.
\end{equation*}

As part of using locally nameless terms, common operations and properties (called `infrastructure'
by \citet{aydemir_engineering_2008}) need to be defined for the target language. A recent article by
Andrew \citet{pitts_locally_2023} uses Agda to explore an abstraction of this representation called
locally nameless sets, and shows that this infrastructure can be defined in a syntax-agnostic way.

The drawback of using this representation is that we need to define infrastructure and prove some
properties of the syntax before we can prove the properties of the metalangauge. But usually, this
drawback is too small to outweight the benefits.

\paragraph*{Opening and closing.} Two fundamental operations on locally nameless terms are
\textit{term opening} and \textit{term closing} \citep{pitts_locally_2023}. Opening will replace all
occurrences of a bound variable with a free variable, written $[i \to x] L$ for some De Bruijn
index $i$, name $x$, and term $L$. For example,
\begin{equation*}
  [0 \to y] (0 q (\lambda \, 1 t 0)) \mapsto y q (\lambda \, y t 0).
\end{equation*}
Every occurence of the bound variable $0$ is replaced with the free variable $y$. Note how after we
go under a new $\lambda$-abstraction, we have to increment this index to $1$ to keep referring to
the same bound variable.

Closing works similarly, and is the inverse of opening; it replaces all occurrences of a free
variable to a given index. For example,
\begin{equation*}
  [0 \leftarrow y] (y q (\lambda \, y t 0)) \mapsto 0 q (\lambda \, 1 t 0).
\end{equation*}

\paragraph*{Local closure.} A term is said to be \textit{locally closed} up to level $i$ if it
remains unchanged after opening it at index $i$ with an arbitrary string. For some $i \in \nat$ and
term $L$, we write that $L$ is locally closed at level $i$ as $i \succ L$. Formally,
\begin{equation}
  \label{equation:local_closure}
  i \succ L \triangleq \forall j \geq i, \; \cof a , \; [j \to a] L = L.
\end{equation}
If it is locally closed at level $0$, we simply call it \textit{locally closed}. Local closure is
defined for System F in section \ref{chapter4:local_closure}.

\paragraph*{Induction Principle.} Often with locally nameless terms, the induction principle is
similar to how it is for named terms in all cases but the $\lambda$-abstraction case. This is
because it is often defined using cofinite quantification and opening. As an example, take type
judgements for the STLC, defined in appendix \ref{appendix:type_judgements}. In regular notation,
these are:
\begin{equation}
\begin{gathered}
  \inferrule
    { }
    {\Gamma \vdash \texttt{‵zero} \colon \nat}
    \; (\vdash\texttt{zero})
  \quad
  \inferrule{x \colon A \in \Gamma}
    {\Gamma \vdash x \colon A}
    \; (\vdash\text{free})
  \quad
  \inferrule
    {\cof x , \; (\Gamma , \, x \colon A \vdash [0 \to x] L \colon B)}
    {\Gamma \vdash \lambda \, L \colon A \to B}
    \; (\vdash \lambda)
  \\
  \inferrule
    {\Gamma \vdash L \colon (A \to B) \\ \Gamma \vdash M \colon A}
    {\Gamma \vdash LM \colon B}
    \; (\vdash\text{app})
  \quad
  \quad
  \inferrule
    {\Gamma \vdash L \colon \nat}
    {\Gamma \vdash \texttt{‵suc} \, L \colon \nat}
    \; (\vdash\texttt{suc})
\end{gathered}
\end{equation}

The only difference to the named representation is the rule for $\lambda$-abstractions
($\vdash\lambda$). Like with the named representation, we intend to prove that the term $L$ is
well-typed with a free variable $x$ added to the context in place of the bound variable
\citep[chapter~Lambda]{wadler_programming_2022}. However, to replace the bound varaible $0$ with
$x$, we first need to open the term $L$ with the name $x$ at index $0$. Furthermore, we are
replacing the bound $0$ with an arbitrary name $x$, so to make sure that the choice of $x$ is
arbitrary, we need cofinite quantification (we cannot use $\forall$ because otherwise we would run
into unintentional shadowing of free variables in the context).

So, when proving the $\lambda$-abstraction case, the induction principle would allow us to assume
that the term $L$ is well-typed when, for cofinite $x$, the term $L$ is opened $[0 \to x] L$. Or in
other words, given a property $P$ over type judgements, we need to prove that $\cof x , \; P(\Gamma
, \, x \colon A \vdash [0 \to x] L \colon B) \implies P(\Gamma \vdash \lambda \, L \colon A \to B)$.
An example of this is given in chapter \ref{chapter3:stlc_sub_and_eval}.


\section{Prior research}
System F was previously formalised in Agda by \citet{hutton_system_2019}. However, the authors
formalised a variant of System F with language extensions known as \textit{System F$_{\omega \mu}$}.
They also used a different approach, opting to make use of De Bruijn indices only. This paper will
use the locally nameless representation, which hasn't been used for System F in Agda before.

I have previously submitted some work on STLC using locally nameless representation for a course
called \textit{Types and Semantics of Programming Languages} (TSPL). I have presented my prior work
in appendix \ref{appendix:tspl}. There are some new additions to this work which are presented in
the next chapter.

\chapter{Substitution and Evaluation in STLC}
\label{chapter3:stlc_sub_and_eval}
\begin{code}
module stlc where
\end{code}
\begin{comment}
  \begin{code}
  -- Data types (naturals, strings, characters)
  open import Data.Nat using (ℕ; zero; suc; _<_; _≥_; _≤_; _≤?_; _<?_; z≤n; s≤s; _⊔_)
    renaming (_≟_ to _≟ℕ_)
  open import Data.Nat.Properties using (≤-refl; ≤-trans; ≤-<-trans; <-≤-trans; ≤-antisym; ≤-total;
    +-mono-≤; n≤1+n; m≤n⇒m≤1+n; suc-injective; <⇒≢; ≰⇒>; ≮⇒≥)
  open import Data.String using (String; fromList) renaming (_≟_ to _≟str_; _++_ to _++str_;
    length to str-length; toList to ⟪_⟫)
  open import Data.Char using (Char)
  open import Data.Char.Properties using () renaming (_≟_ to _≟char_)
  
  -- Function manipulation.
  open import Function using (_∘_; flip; it; id; case_returning_of_)
  
  -- Relations and predicates/decidability.
  import Relation.Binary.PropositionalEquality as Eq
  open Eq using (_≡_; _≢_; refl; sym; trans; cong; cong-app; cong₂)
  open Eq.≡-Reasoning using (begin_; step-≡-∣; step-≡-⟩; _∎)
  open import Relation.Binary.Definitions using (DecidableEquality)
  open import Relation.Nullary.Decidable using (Dec; yes; no; True; False; toWitnessFalse;
    toWitness; fromWitness; ¬?; ⌊_⌋; From-yes)
  open import Relation.Unary using (Decidable)
  open import Relation.Binary using () renaming (Decidable to BinaryDecidable)
  open import Relation.Nullary.Negation using (¬_; contradiction)
  open import Data.Empty using (⊥-elim)
  
  -- Products and exists quantifier.
  open import Data.Product using (_×_; proj₁; proj₂; ∃-syntax) renaming (_,_ to ⟨_,_⟩)
  open import Data.Sum using (_⊎_; inj₁; inj₂)
  
  -- Lists.
  open import Data.List using (List; []; _∷_; _++_; length; filter; map; foldr; head; replicate)
  open import Data.List.Properties using (≡-dec)
  import Data.List.Membership.DecPropositional as DecPropMembership
  open import Data.List.Relation.Unary.All using (All; all?; lookup)
    renaming (fromList to All-fromList; toList to All-toList)
  open import Data.List.Relation.Unary.Any using (Any; here; there)
  open import Data.List.Extrema Data.Nat.Properties.≤-totalOrder using (max; xs≤max)
  
  -- Import list membership using List Char comparisons.
  private
    _≟lchar_ : ∀ (xs ys : List Char) → Dec (xs ≡ ys)
    xs ≟lchar ys = ≡-dec (_≟char_) xs ys
  
  open DecPropMembership _≟lchar_ using (_∈_; _∉_; _∈?_)
  \end{code}
\end{comment}
The remaining properties of STLC which were not proven in \ref{appendix:tspl} include some
substitution properties and creating an evaluator. Since it is a necessary prelude to System F, this
new work is presented here. As explained in section \ref{background:evaluation_strategy}, one
evaluation strategy is weak-head reduction, which is also used by
\citet[chapter~Properties]{wadler_programming_2022} and \citet[section~5]{chargueraud_locally_2012}.
Thus, evaluation will be restricted to closed terms only (those without free variables).
\begin{code}
  open import plfa_adaptions
  open import tspl_prior_work
  open import cofinite
\end{code}

To avoid name conflicts with Agda's reserved keywords, the syntax of the metalangauge is different
to how it may be commonly written. We will use substitutes, like ƛ for $\lambda$, or $\cdot$ for
function application (which is usually left implicit).

\section{Substitution Preserves Types}
A good example of the induction principle for cofinite quantification is the following proof, that
all well-typed expressions are locally closed using induction on the type judgement.

\begin{code}
  ⊢⇒lc : ∀ {Γ t A} → Γ ⊢ t ⦂ A → LocallyClosed t
  ⊢⇒lc {Γ} {t} {A} (⊢free Γ∋A) = free-lc
  ⊢⇒lc {Γ} {ƛ t} {A} (⊢ƛ И⟨ Иe₁ , Иe₂ ⟩) j =
    И⟨ Иe₁ , (λ a {a∉} → cong ƛ_
      (open-rec-lc-lemma
        (λ ())
        (open-rec-lc (⊢⇒lc (Иe₂ a {a∉}))))) ⟩
  ⊢⇒lc {Γ} {t₁ · t₂} (⊢· ⊢A⇒B ⊢A) _ =
    И⟨ domain Γ , (λ _ → cong₂ _·_
      (open-rec-lc (⊢⇒lc ⊢A⇒B)) (open-rec-lc (⊢⇒lc ⊢A))) ⟩
  ⊢⇒lc {Γ} {‵zero} ⊢zero = ‵zero-≻
  ⊢⇒lc {Γ} {‵suc t} (⊢suc ⊢t) j =
    И⟨ domain Γ , (λ a {a∉} →
      cong ‵suc_ (open-rec-lc (⊢⇒lc ⊢t))) ⟩
\end{code}

Recall the definition of local closure (see equation \ref{equation:local_closure}). In Agda,
\begin{minted}{agda}
  _≻_ : ℕ → Term → Set
  i ≻ t = (j : ℕ) ⦃ _ : j ≥ i ⦄ → И a , ([ j —→ a ] t ≡ t)
\end{minted}
Given a $j \in \nat$, we need to show that for cofinite $a$, opening $t$ with $[j \to a]$ will leave
$t$ unchanged. This is why each case includes the extra \texttt{j} argument.

For all except the $\lambda$-abstraction case, the inductive hypothesis is straight-forward. For
$\lambda$-abstractions, we are given a list \texttt{Иe₁} and a property \texttt{Иe₂} which states
that $(\Gamma , \, x \colon A \vdash [0 \to x] t \colon B)$ for some \texttt{List Char} $x \not \in
\texttt{Иe₁}$. We need to provide a proof that $\lambda \, t$ is locally closed. By including the
argument \texttt{j}, we now only need to show that $\cof a, \; (\lambda \, ([j + 1 \to a] t) =
\lambda \, t)$. The inductive hypothesis allows us to call \texttt{⊢⇒lc} on \texttt{Иe₂}, or in
other words, it implies that for cofinite $a$, the term $[0 \to a] t$ is locally closed. After
applying the inductive hypothesis, the proof is a consequence of a previous lemma
\texttt{open-rec-lc} which was proven in appendix \ref{appendix:tspl}.

Now we need show that substituting preserves types. As mentioned, we restrict ourselves to only
closed terms for substitution.
\begin{code}
  {-# TERMINATING #-}
  subst : ∀ {Γ x t u A B}
    → ∅ ⊢ u ⦂ A
    → Γ , x ⦂ A ⊢ t ⦂ B
      --------------------
    → Γ ⊢ [ x := u ] t ⦂ B
  subst {x = y} ⊢u (⊢free {x = x} (H refl)) with y ≟lchar y
  ... | yes _   = weaken ⊢u
  ... | no  y≢y = contradiction refl y≢y
  subst {x = y} ⊢u (⊢free {x = x} (T x≢y ∋x)) with y ≟lchar x
  ... | yes y≡x = contradiction (sym y≡x) x≢y
  ... | no  _   = ⊢free ∋x
  subst {x = x} {t = ƛ t} ⊢u (⊢ƛ И⟨ Иe₁ , Иe₂ ⟩) =
    ⊢ƛ И⟨ x ∷ Иe₁
        , (λ a {a∉} →
          let a≢y   = ∉∷[]⇒≢ (proj₁ (∉-++ a∉))
              a∉Иe₁ = proj₂ (∉-++ a∉)
          in subst-open-context
            {t = t}
            (sym-≢ a≢y)
            (⊢⇒lc ⊢u)
            (subst ⊢u (swap a≢y (Иe₂ a {a∉Иe₁}))) )
        ⟩
  subst ⊢u (⊢· ⊢t₁ ⊢t₂) = ⊢· (subst ⊢u ⊢t₁) (subst ⊢u ⊢t₂)
  subst ⊢u ⊢zero = ⊢zero
  subst ⊢u (⊢suc ⊢t) = ⊢suc (subst ⊢u ⊢t)
\end{code}

The property is proven by induction on the type judgement of the term \texttt{M}. Agda cannot
determine the termination of this function, and the problematic call is when the type judgement is a
$\lambda$-abstraction \texttt{ƛ M}. Specifically, it highlights the problematic code to be
\texttt{subst ⊢u (swap a≢y (Иe₂ a {a∉Иe₁}))}. Thus, I will only detail that
step\footnote{Technically, the Agda compiler also highlits a problematic call for the application
case, but this is caused by the problematic $\lambda$-abstraction case, so is irrelevant.}.

Since we are inducting on the type judgement, the inductive hypothesis for a term \texttt{ƛ M} of
type \texttt{A ⇒ A'} states that the property $P$ holds for $P(\texttt{И⟨ Иe₁ , Иe₂ ⟩})$. Let
\texttt{b} be an appropriate \texttt{List Char} to supply to \texttt{Иe₂}, then it will return a
proof that \texttt{Γ , x ⦂ A , b ⦂ A' ⊢ [ 0 —→ b ] t ⦂ B}. In this case, the \texttt{subst} function
is called on \texttt{Иe₂} (with a \texttt{swap} function applied, but since this function doesn't
call \texttt{subst} and only operates on the context, this call is irrelevant to this termination
issue). Since we are deconstructing the type judgement and are calling \texttt{subst} on the term
\texttt{Иe₂} which makes up the input type judgement, this function call corresponds to the
inductive hypothesis, and is thus valid.

Substituting a term for an index is similar to the definition of the free-variable substitution
(defined in appendix \ref{appendix:substitution_proofs}). This is, confusingly, also called
`opening' by \citet{chargueraud_locally_2012}.
\begin{code}
  [_:→_]_ : ℕ → Term → Term → Term
  [ k :→ u ] (free x) = free x
  [ k :→ u ] (bound i) with k ≟ℕ i
  ... | yes _ = u
  ... | no  _ = bound i
  [ k :→ u ] (ƛ t) = ƛ [ (suc k) :→ u ] t
  [ k :→ u ] (t₁ · t₂) = [ k :→ u ] t₁ · [ k :→ u ] t₂
  [ k :→ u ] ‵zero = ‵zero
  [ k :→ u ] (‵suc t) = ‵suc ([ k :→ u ] t)
\end{code}

Using an index $i$ to open with $x \in \texttt{List Char}$ is the same as using the index
substitution with the term \texttt{free $x$}.
% Unused.
\begin{code}
  —→≡:→free : ∀ {i : ℕ} {x : List Char} (t : Term)
    → [ i —→ x ] t ≡ [ i :→ free x ] t
  —→≡:→free {i} {x} (free y) = refl
  —→≡:→free {i} {x} (bound k) with i ≟ℕ k
  ... | yes _ = refl
  ... | no  _ = refl
  —→≡:→free {i} {x} (ƛ t) = cong ƛ_ (—→≡:→free t)
  —→≡:→free {i} {x} (t₁ · t₂) =
    cong₂ _·_ (—→≡:→free t₁) (—→≡:→free t₂)
  —→≡:→free {i} {x} ‵zero = refl
  —→≡:→free {i} {x} (‵suc t) = cong ‵suc_ (—→≡:→free t)
\end{code}

There are quite a few more properties of index substitution which \citet{chargueraud_locally_2012}
proves, but the only relevant one for evaluation is \texttt{subst-intro}. It proves that
substituting a term for an index is the same as first opening the term with an $x \in \texttt{List
Char}$ and then using the free variable substitution using this $x$.
\begin{code}
  subst-intro : ∀ {x : List Char} {i : ℕ} (t u : Term)
    → x # t
    → [ i :→ u ] t ≡ [ x := u ] ([ i —→ x ] t)
  subst-intro {x} (free y) u x#t with x ≟lchar y
  ... | yes refl with () ← x#t
  ... | no  x≢y  = refl
  subst-intro {x} {i} (bound j) u x#t with i ≟ℕ j
  ... | no  i≢j  = refl
  ... | yes refl with x ≟lchar x
  ...   | yes refl = refl
  ...   | no  x≢x  = contradiction refl x≢x
  subst-intro (ƛ t) u x#ƛt = cong ƛ_ (subst-intro t u (#-ƛ t x#ƛt))
  subst-intro {x} (t₁ · t₂) u x#t =
    let ⟨ x#t₁ , x#t₂ ⟩ = #-· t₁ t₂ x#t in
      cong₂ _·_ (subst-intro t₁ u x#t₁) (subst-intro t₂ u x#t₂)
  subst-intro ‵zero u x#t = refl
  subst-intro (‵suc t) u x#st =
    cong ‵suc_ (subst-intro t u (#-‵suc t x#st))
\end{code}

Since we need to replace bound variables for free ones to perform a $\beta$-reduction, we should
prove that this substitution preserves types.
\begin{code}
  subst-op : ∀ {Γ t u A B}
    → ∅ ⊢ u ⦂ A
    → Γ ⊢ ƛ t ⦂ A ⇒ B
      --------------------
    → Γ ⊢ [ 0 :→ u ] t ⦂ B
  subst-op {t = t} {u = u} ⊢u (⊢ƛ И⟨ Иe₁ , Иe₂ ⟩) =
    let x                  = fresh (fv t ++ Иe₁)
        ⟨ x∉fv-t , x∉Иe₁ ⟩ = ∉-++ {xs = fv t} {ys = Иe₁}
                                (fresh-correct (fv t ++ Иe₁))
    in ≡-with-⊢ (subst ⊢u (Иe₂ x {x∉Иe₁}))
      (sym (subst-intro t u (∉fv⇒# x t (x∉fv-t))))
\end{code}

\section{Substitution Commutes}
Here we prove a property which...

None of these functions are necessary for evaluation, so we can postulate function extensionality,
that is, for some sets $A$ and $B$ and functions $f, g \colon A \to B$, 
\begin{equation*}
  (\forall x \in A, \, f(x) = g(x)) \implies f = g.
\end{equation*}

This is the 4\textsuperscript{th} axiom in the elementary
theory of the category of sets \citep{tom_leinster_rethinking_2014}. It's known that this does not
cause an inconsistency in Agda's system of logic \citep{wadler_programming_2022}, however, since it
needs to be postulated, we cannot call this function. So, we leave it only defined in this module.

\begin{code}
  -- module substitution_commutes where
  --   private postulate
  --     extensionality : ∀ {A B : Set} {f g : A → B}
  --       → (∀ (x : A) → f x ≡ g x)
  --         -----------------------
  --       → f ≡ g

  --   free-inferred : ∀ {x : List Char} → Term
  --   free-inferred {x} = free x

  --   ctx-weaken : ∀ {Γ x y A B} → x ≢ y → Γ , y ⦂ B ∋ x ⦂ A → Γ ∋ x ⦂ A
  --   ctx-weaken x≢y (H refl) = contradiction refl x≢y
  --   ctx-weaken x≢y (T _ ∋x) = ∋x

  --   exts : ∀ {Γ Δ x y A B}
  --     → (Γ ∋ x ⦂ A        → ∃[ L ]        (Δ ⊢ L ⦂ A))
  --       ------------------------------------------------------------
  --     → (Γ , y ⦂ B ∋ x ⦂ A → ∃[ L ] (Δ , y ⦂ B ⊢ L ⦂ A))
  --   exts σ (H refl) = ⟨ free-inferred , (⊢free (H refl)) ⟩
  --   exts σ (T x≢y x) = ⟨ proj₁ (σ x) , rename (T {!!}) (proj₂ (σ x)) ⟩

    -- subst : ∀ {Γ Δ}
    --   → (∀ {x A} → Γ ∋ x ⦂ A → Δ ⊢ A)
    --     -----------------------
    --   → (∀ {x A} → Γ ⊢ A → Δ ⊢ A)
    -- subst σ (` x)          =  σ x
\end{code}

\section{Evaluation}
Using weak-head reduction, only $\lambda$-abstractions are values, together with the two primitives
that were introduced.
\begin{code}
  data Value : Term → Set where
    V-ƛ : ∀ {t} → Value (ƛ t)
    V-zero : Value ‵zero
    V-suc : ∀ {t} → Value t → Value (‵suc t)
\end{code}

\citet{chargueraud_locally_2012} adds another requirement for $\lambda$-abstractions: $1 \succ M$,
or in other words, that $\lambda M$ is locally closed. However, since we are only evaluating
well-typed terms, and all well-typed terms are locally closed (see section
\ref{appendix:type_judgements}), this requirement isn't necessary here.

We follow the rules for small-step reduction given in \citet{chargueraud_locally_2012}. These are
encoded in Agda below.
\begin{code}
  infix 4 _—→_
  data _—→_ : Term → Term → Set where
    ξ₁ : ∀ {t₁ t₁' t₂}
      → t₁ —→ t₁'
      → LocallyClosed t₂
        -------------------
      → t₁ · t₂ —→ t₁' · t₂

    ξ₂ : ∀ {t₁ t₂ t₂'}
      → t₂ —→ t₂'
        ---------
      → t₁ · t₂ —→ t₁ · t₂'

    ξ-suc : ∀ {t t'}
      → t —→ t'
        ------------------
      → ‵suc t —→ ‵suc t'

    β : ∀ {t u}
      → 1 ≻ t
      → Value u
        -------
      → (ƛ t) · u —→ [ 0 :→ u ] t
\end{code}
Once again, the requirements for local closure could be removed, but they are kept here to follow
the rules presented in \citet{chargueraud_locally_2012}.

Following \citet{wadler_programming_2022}, we define some convenience functions, namely, reflexive
and transitive closure properties which will help reason about taking a reduction step. These follow
similar syntax to how equality reasoning is written in the Agda standard library
\citep{the_agda_community_agda_2024}.
\begin{comment}
\begin{code}
  infix  2 _—↠_
  infix  1 begin'_
  infixr 2 _—→⟨_⟩_
  infix  3 _∎'
\end{code}
\end{comment}
\begin{code}
  data _—↠_ : Term → Term → Set where
    _∎' : ∀ M
        ---------
      → M —↠ M

    step—→ : ∀ L {M N}
      → M —↠ N
      → L —→ M
        ---------
      → L —↠ N

  pattern _—→⟨_⟩_ L L—→M M—↠N = step—→ L M—↠N L—→M

  begin'_ : ∀ {M N}
    → M —↠ N
      ------
    → M —↠ N
  begin' M—↠N = M—↠N
\end{code}

There are two important properties which are required to implement evaluation. Progress (that terms
can always take a step, or are a value and are thus finished reducing) is presented below.
\begin{code}
  data Progress (t : Term) : Set where
    step : ∀ {t'}
      → t —→ t'
        ----------
      → Progress t

    done :
        Value t
        ----------
      → Progress t

  progress : ∀ {t A}
    → ∅ ⊢ t ⦂ A
      ----------
    → Progress t
  progress (⊢ƛ x) = done V-ƛ
  progress (⊢· ⊢t₁ ⊢t₂) with progress ⊢t₁
  ... | step t₁→t₁' = step (ξ₁ t₁→t₁' (⊢⇒lc ⊢t₂))
  ... | done V-ƛ with progress ⊢t₂
  ...   | step t₂→t₂' = step (ξ₂ t₂→t₂')
  ...   | done val    = step (β (i≻ƛt⇒si≻t (⊢⇒lc ⊢t₁)) val)
  progress ⊢zero = done V-zero
  progress (⊢suc ⊢t) with progress ⊢t
  ... | step t→t' = step (ξ-suc t→t')
  ... | done val  = done (V-suc val)
\end{code}

And preservation, that types are preserved when reducing.
\begin{code}
  preserve : ∀ {t t' A}
    → ∅ ⊢ t ⦂ A
    → t —→ t'
      ----------
    → ∅ ⊢ t' ⦂ A
  preserve (⊢· ⊢t₁ ⊢t₂) (ξ₁ t→t' _) = ⊢· (preserve ⊢t₁ t→t') ⊢t₂
  preserve (⊢· ⊢t₁ ⊢t₂) (ξ₂ t→t') = ⊢· ⊢t₁  (preserve ⊢t₂ t→t')
  preserve (⊢· ⊢t₁ ⊢t₂) (β x x₁) = subst-op ⊢t₂ ⊢t₁
  preserve (⊢suc ⊢t) (ξ-suc t→t') = ⊢suc (preserve ⊢t t→t')
\end{code}

Since STLC is not Turing complete \citep{church_formulation_1940}, we don't need to worry about
programs which don't terminate. Still, the evaluation function needs to receive a timeout argument,
as otherwise, Agda cannot determine that it would terminate. Thus, we define a record which limits
evaluation to a certain number of reduction steps. Then we can use the preserve and progress
properties to make an \texttt{eval} function. This is the same definition as
\citet{wadler_programming_2022} uses, so the explanation is ommitted here.
\begin{code}
  record Gas : Set where
    eta-equality
    constructor gas
    field
      amount : ℕ

  data Finished (t : Term) : Set where
    done : Value t → Finished t
    out-of-gas : Finished t

  data Steps (t : Term) : Set where
    steps : ∀ {t'} → t —↠ t' → Finished t' → Steps t

  eval : ∀ {t A} → Gas → ∅ ⊢ t ⦂ A → Steps t
  eval {t} (gas zero) ⊢t = steps (t ∎') out-of-gas
  eval {t} (gas (suc n)) ⊢t with progress ⊢t
  ... | done V-t = steps (t ∎') (done V-t)
  ... | step {t'} t→t' with eval (gas n) (preserve ⊢t t→t')
  ...   | steps t'→u fin-u = steps (t —→⟨ t→t' ⟩ t'→u) fin-u
\end{code}

We provide an example for evaluation. First, we require a type derivation for $2+2$. We would show
that it evaluates to $4$, but because the evaluation proof requires more than eleven thousand lines
of code, it is omitted. But we encourage the reader to try it out for themselves. The proofs of
\texttt{⊢two} and \texttt{⊢plus} are long and are omitted (but present in the source file).
\begin{code}
  two : Term
  two = ƛ ƛ bound 1 · (bound 1 · bound 0)

  plus : Term
  plus = ƛ ƛ ƛ ƛ bound 3 · bound 1 · (bound 2 · bound 1 · bound 0)

  suc' : Term
  suc' = ƛ ‵suc (bound 0)

  ⊢two : ∅ ⊢ two ⦂ (‵ℕ ⇒ ‵ℕ) ⇒ ‵ℕ ⇒ ‵ℕ
  -- omitted.

  ⊢plus : ∀ {Γ A} → Γ ⊢ plus ⦂
    ((A ⇒ A) ⇒ A ⇒ A) ⇒ ((A ⇒ A) ⇒ A ⇒ A) ⇒ ((A ⇒ A) ⇒ A ⇒ A)
  -- omitted

  ⊢suc' : ∀ {Γ} → Γ ⊢ suc' ⦂ ‵ℕ ⇒ ‵ℕ
  ⊢suc' = ⊢ƛ И⟨ [] , (λ _ → ⊢suc (⊢free H′)) ⟩

  ⊢2+2 : ∅ ⊢ plus · two · two · suc' · ‵zero ⦂ ‵ℕ
  ⊢2+2 = ⊢· (⊢· (⊢· (⊢· ⊢plus  ⊢two) ⊢two) ⊢suc') ⊢zero

  -- Using Emacs, normalise "eval (gas 100) ⊢2+2" by pressing
  -- C-c C-n.
\end{code}
\begin{comment}
\begin{code}
  ⊢two = ⊢ƛ
    И⟨ []
    , (λ a → ⊢ƛ
      И⟨ (a ∷ [])
      , (λ b {b∉} →
        ⊢·
        (⊢free (T (sym-≢ (∉∷[]⇒≢ b∉)) H′))
        (⊢· (⊢free (T (sym-≢ (∉∷[]⇒≢ b∉)) H′)) (⊢free (H′)))) ⟩) ⟩

  ⊢plus = ⊢ƛ
    И⟨ []
    , (λ a → ⊢ƛ
      И⟨ a ∷ []
      , (λ b {b∉} → ⊢ƛ
        И⟨ a ∷ b ∷ []
        , (λ c {c∉} → ⊢ƛ
          И⟨ a ∷ b ∷ c ∷ []
          , (λ d {d∉} →
          ⊢·
            (⊢·
              (⊢free (T (a≢d d∉) (T (a≢c c∉) (T (a≢b b∉) H′))))
              (⊢free (T (c≢d d∉) (H′))))
            (⊢·
              (⊢·
                (⊢free (T (b≢d d∉) (T (b≢c c∉) H′)))
                (⊢free (T (c≢d d∉) H′)))
              (⊢free H′))) ⟩) ⟩) ⟩) ⟩
    where
      a≢d : ∀ {a b c d} → d ∉ a ∷ b ∷ c ∷ [] → a ≢ d
      a≢d d∉ = sym-≢ (∉∷[]⇒≢ (proj₁ (∉-++ d∉)))
      a≢c : ∀ {a b c} → c ∉ a ∷ b ∷ [] → a ≢ c
      a≢c c∉ = sym-≢ (∉∷[]⇒≢ (proj₁ (∉-++ c∉)))
      a≢b : ∀ {a b} → b ∉ a ∷ [] → a ≢ b
      a≢b b∉ = sym-≢ (∉∷[]⇒≢ b∉)
      c≢d : ∀ {a b c d} → d ∉ a ∷ b ∷ c ∷ [] → c ≢ d
      c≢d {a} {b} d∉ =
        sym-≢ (∉∷[]⇒≢ (proj₂ (∉-++ {xs = a ∷ b ∷ []} d∉)))
      b≢d : ∀ {a b c d} → d ∉ a ∷ b ∷ c ∷ [] → b ≢ d
      b≢d {a} {b} d∉ =
        sym-≢ (∉∷[]⇒≢ (proj₂ (
          ∉-++
            {xs = a ∷ []}
            (proj₁ (∉-++ {xs = a ∷ b ∷ []} d∉)))))
      b≢c : ∀ {a b c} → c ∉ a ∷ b ∷ [] → b ≢ c
      b≢c {a} c∉ = sym-≢ (∉∷[]⇒≢ (proj₂ (∉-++ {xs = a ∷ []} c∉)))
\end{code}
\end{comment}


\chapter{System F}
Combining System F with the locally nameless representation was previously described by
\citet{chargueraud_locally_2012}.
\begin{code}
module chapter4 where
  open import cofinite
  open import plfa_adaptions using (All-++; ++-All; ∉-++; ++-∉; ∈-++ˡ; ∈-++ʳ; ∈-swap;
    ∉y∷ys⇒≢y; ∉y∷ys⇒∉ys; m+1≤n⇒m≤n; ∉∷[]⇒≢; ≢∧∉⇒∉∷; ∈∷[]⇒≡; ∉y∷ys⇒∉y∷[];
    ∈y∷ys∧≢y⇒∈ys; ∈-≡; ++-idʳ)
  open import tspl_prior_work
    using (suc-preserves-≢; sym-≢; fresh; fresh-correct)
\end{code}

\section{Syntax of types and terms}
In System F, types become part of the syntax, as they now become a part of how terms are built.
Note, some authors use $\Pi$ for the $\forall$ type instead, such as \citet{hutton_system_2019}, and
some use the lowercase $\lambda$ for both term and type abstraction, such as
\citet{pierce_types_2002}.

\begin{code}
  data Type : Set where
    ‵ℕ      : Type               -- Base type.
    t-fr     : List Char → Type   -- Free type variables.
    t-#      : ℕ → Type           -- Bound type variables.
    _⇒_      : Type → Type → Type -- Arrow types.
    t-∀_     : Type → Type -- "For all" type.

  ⇒-inj : ∀ {A B A' B'}→ A ⇒ B ≡ A' ⇒ B' → (A ≡ A') × (B ≡ B')
  ⇒-inj refl = ⟨ refl , refl ⟩

  ∀-inj : ∀ {A B}
    → t-∀ A ≡ t-∀ B → A ≡ B
  ∀-inj refl = refl

  data Term : Set where
    fr     : List Char → Term
    #      : ℕ → Term
    -- λ terms now have a type bound. E.g. λ: T. M
    ƛ_     : Term → Term
    _·_    : Term → Term → Term
    Λ_     : Term → Term
    _[_]   : Term → Type → Term
    ‵zero  : Term
    ‵suc_  : Term → Term

  ƛ-inj : ∀ {L M}
    → ƛ L ≡ ƛ M → (L ≡ M)
  ƛ-inj refl = refl
  ·-inj : ∀ {L M L' M'}
    → L · M ≡ L' · M' → (L ≡ L') × (M ≡ M')
  ·-inj refl = ⟨ refl , refl ⟩
  Λ-inj : ∀ {L M}
    → Λ L ≡ Λ M → L ≡ M
  Λ-inj refl = refl
  []-inj : ∀ {L M A B}
    → L [ A ] ≡ M [ B ] → (L ≡ M) × (A ≡ B)
  []-inj refl = ⟨ refl , refl ⟩
  ‵suc-inj : ∀ {L M}
    → ‵suc L ≡ ‵suc M → L ≡ M
  ‵suc-inj refl = refl
\end{code}

Since there are two different types of free and bound variables now, outside of Agda code, I shall
refer to free variables in lowercase and free type variables in UPPERCASE, and bound variables using
regular arabic numerals ($0$) and bound type variables using bold arabic numerals ($\mathbf{0}$). So
for example,

\begin{equation*}
  \text{id} \triangleq (\Lambda \lambda \colon \mathbf{0}. 0) \colon \forall \mathbf{0} \to \mathbf{0}
\end{equation*}
\begin{code}
  id' : Term
  id' = Λ ƛ # 0
\end{code}

More examples are given after we define type judgements in equation \ref{equation:twice_big_omega}
in section \ref{chapter4:type_judgements}.

\section{Opening}
In System F, we only require the opening operation \citep{chargueraud_locally_2012}. There are three
kinds of opening:
\begin{itemize}
  \item opening a kind at an index by replacing the bound type variable at that index with a free
        type variable,
  \item opening a term at an index by replacing the bound type variable at that index in the term
        with a free type variable, and
  \item opening a term at an index by replacing the bound variable at that index with a free
        variable.
\end{itemize}

\begin{code}
  ty-ty[_—→_]_ : ℕ → List Char → Type → Type
  ty-ty[ i —→ x ] ‵ℕ = ‵ℕ
  ty-ty[ i —→ x ] (t-fr y) = t-fr y
  ty-ty[ i —→ x ] (t-# n) with i ≟ℕ n
  ... | yes _ = t-fr x
  ... | no  _ = t-# n
  ty-ty[ i —→ x ] (A ⇒ B) = (ty-ty[ i —→ x ] A) ⇒ (ty-ty[ i —→ x ] B)
  ty-ty[ i —→ x ] t-∀ A = t-∀ (ty-ty[ suc i —→ x ] A)

  ty-tm[_—→_]_ : ℕ → List Char → Term → Term
  ty-tm[ i —→ x ] (fr y) = fr y
  ty-tm[ i —→ x ] (# n) = # n
  ty-tm[ i —→ x ] (ƛ L) = ƛ (ty-tm[ i —→ x ] L)
  ty-tm[ i —→ x ] (L · M) = (ty-tm[ i —→ x ] L) · (ty-tm[ i —→ x ] M)
  ty-tm[ i —→ x ] (Λ L) = Λ ty-tm[ suc i —→ x ] L
  ty-tm[ i —→ x ] (L [ A ]) = (ty-tm[ i —→ x ] L) [ (ty-ty[ i —→ x ] A) ]
  ty-tm[ i —→ x ] ‵zero = ‵zero
  ty-tm[ i —→ x ] (‵suc L) = ‵suc ty-tm[ i —→ x ] L

  tm-tm[_—→_]_ : ℕ → List Char → Term → Term
  tm-tm[ i —→ x ] (fr y) = fr y
  tm-tm[ i —→ x ] (# n) with i ≟ℕ n
  ... | yes _ = fr x
  ... | no  _ = # n
  tm-tm[ i —→ x ] (ƛ L) = ƛ tm-tm[ (suc i) —→ x ] L
  tm-tm[ i —→ x ] (L · M) = (tm-tm[ i —→ x ] L) · (tm-tm[ i —→ x ] M)
  tm-tm[ i —→ x ] (Λ L) = Λ (tm-tm[ i —→ x ] L)
  tm-tm[ i —→ x ] (L [ A ]) = (tm-tm[ i —→ x ] L) [ A ]
  tm-tm[ i —→ x ] ‵zero = ‵zero
  tm-tm[ i —→ x ] (‵suc L) = ‵suc tm-tm[ i —→ x ] L
\end{code}

\begin{comment}
However, I may use a different approach at some point and need to use closing. Just in case, I'll
define it here.
\begin{code}
  ty-ty[_←—_]_ : ℕ → List Char → Type → Type
  ty-ty[ i ←— x ] ‵ℕ = ‵ℕ
  ty-ty[ i ←— x ] (t-fr y) with x ≟lchar y
  ... | yes _ = t-# i
  ... | no  _ = t-fr y
  ty-ty[ i ←— x ] (t-# n) = t-# n
  ty-ty[ i ←— x ] (A ⇒ B) = (ty-ty[ i ←— x ] A) ⇒ (ty-ty[ i ←— x ] B)
  ty-ty[ i ←— x ] t-∀ A = t-∀ (ty-ty[ i ←— x ] A)

  ty-tm[_←—_]_ : ℕ → List Char → Term → Term
  ty-tm[ i ←— x ] (fr y) = fr y
  ty-tm[ i ←— x ] (# n) = # n
  ty-tm[ i ←— x ] (ƛ L) = ƛ (ty-tm[ i ←— x ] L)
  ty-tm[ i ←— x ] (L · M) = (ty-tm[ i ←— x ] L) · (ty-tm[ i ←— x ] M)
  ty-tm[ i ←— x ] (Λ L) = Λ (ty-tm[ suc i ←— x ] L)
  ty-tm[ i ←— x ] (L [ A ]) = (ty-tm[ i ←— x ] L) [ (ty-ty[ i ←— x ] A) ]
  ty-tm[ i ←— x ] ‵zero = ‵zero
  ty-tm[ i ←— x ] (‵suc L) = ‵suc ty-tm[ i ←— x ] L

  tm-tm[_←—_]_ : ℕ → List Char → Term → Term
  tm-tm[ i ←— x ] (fr y) with x ≟lchar y
  ... | yes _ = # i
  ... | no  _ = fr y
  tm-tm[ i ←— x ] (# n) = # n
  tm-tm[ i ←— x ] (ƛ L) = ƛ tm-tm[ (suc i) ←— x ] L
  tm-tm[ i ←— x ] (L · M) = (tm-tm[ i ←— x ] L) · (tm-tm[ i ←— x ] M)
  tm-tm[ i ←— x ] (Λ L) = Λ (tm-tm[ i ←— x ] L)
  tm-tm[ i ←— x ] (L [ A ]) = (tm-tm[ i ←— x ] L) [ A ]
  tm-tm[ i ←— x ] ‵zero = ‵zero
  tm-tm[ i ←— x ] (‵suc L) = ‵suc tm-tm[ i ←— x ] L
\end{code}

We also have this lemma we'll need later [TODO: reword].
\begin{code}
  tm-tm-swap-ty-tm : ∀ {L i j x y}
    → (tm-tm[ i —→ x ] (ty-tm[ j —→ y ] L))
      ≡ (ty-tm[ j —→ y ] (tm-tm[ i —→ x ] L))
  tm-tm-swap-ty-tm {fr x} = refl
  tm-tm-swap-ty-tm {# k} {i} {j}  with i ≟ℕ j
  ... | yes refl with i ≟ℕ k
  ...   | yes refl = refl
  ...   | no  i≢k  = refl
  tm-tm-swap-ty-tm {# k} {i} {j} | no  i≢j with i ≟ℕ k
  ...   | yes refl = refl
  ...   | no  i≢k  = refl
  tm-tm-swap-ty-tm {ƛ L} {i} {j} {x} {y} =
    cong ƛ_ (tm-tm-swap-ty-tm {L})
  tm-tm-swap-ty-tm {L · M} = cong₂ _·_
    (tm-tm-swap-ty-tm {L})
    (tm-tm-swap-ty-tm {M})
  tm-tm-swap-ty-tm {Λ L} = cong Λ_ (tm-tm-swap-ty-tm {L})
  tm-tm-swap-ty-tm {L [ A ]} {i} {j} {x} {y} =
    cong (_[ ty-ty[ j —→ y ] A ]) (tm-tm-swap-ty-tm {L})
  tm-tm-swap-ty-tm {‵zero} = refl
  tm-tm-swap-ty-tm {‵suc L} = cong ‵suc_ (tm-tm-swap-ty-tm {L})
\end{code}
\end{comment}

\section{Local closure}
\label{chapter4:local_closure}
Like before, a term or kind locally closed at level $i$ if it remains unchanged after opening it at
$i$, see equation \ref{equation:local_closure}.

\begin{comment}
\begin{code}
  -- _≻k_ : ℕ → Type → Set
  -- i ≻k A = (j : ℕ) ⦃ _ : j ≥ i ⦄ → И a , (ty-ty[ j —→ a ] A ≡ A)

  -- _≻t_ : ℕ → Term → Set
  -- i ≻t L = (j : ℕ) ⦃ _ : j ≥ i ⦄ → И a , (tm-tm[ j —→ a ] L ≡ L)

  -- K-LocallyClosed : Type → Set
  -- K-LocallyClosed A = 0 ≻k A

  -- T-LocallyClosed : Term → Set
  -- T-LocallyClosed T = 0 ≻t T
\end{code}
\end{comment}

To make future proofs simpler, we will make use of Agda's instances, which are similar to
typeclasses in Haskell or traits in Rust or Scala. We will follow how \citet{pitts_locally_2023}
defined locally nameless sets, but require fewer of the locally nameless set axioms to be fulfilled
(as we'll only need axioms 1 and 5).

Thanks to using instances, we can prove the local closure properties for both kinds of local
closure.
\begin{code}
  record Lns (A : Set) (B : Set) : Set where
    infix 5 [_—→_]_
    field
      -- We only need opening.
      [_—→_]_ : ℕ → List Char → B → B
      ax1 : ∀ (i : ℕ) (a b : List Char) (L : B)
        → [ i —→ a ] ([ i —→ b ] L) ≡ [ i —→ b ] L
      ax5 : ∀ (i j : ℕ) (a b : List Char) (L : B)
        → (i≢j : i ≢ j)
        → [ i —→ a ] ([ j —→ b ] L) ≡ [ j —→ b ] ([ i —→ a ] L)

  open Lns ⦃ ... ⦄


  -- Local closure definition.
  _≻_ : ∀ {A B : Set} ⦃ _ : Lns A B ⦄ → ℕ → B → Set
  i ≻ L = (j : ℕ) ⦃ _ : j ≥ i ⦄ → И a , ([ j —→ a ] L ≡ L)
  LocallyClosed : ∀ {A B : Set} ⦃ _ : Lns A B ⦄ → B → Set
  LocallyClosed L = 0 ≻ L
  -- Local closure lemmas.
  lemma2·6 : ∀ {A B : Set} ⦃ _ : Lns A B ⦄ {i j : ℕ} {L : B}
    → j ≥ i   → i ≻ L
      ---------------
    → j ≻ L
  lemma2·6 j≥i i≻L k = i≻L k ⦃ ≤-trans j≥i it ⦄

  lemma2·7-1 : ∀ {A B : Set} ⦃ _ : Lns A B ⦄ {i : ℕ} {x y : List Char} {L : B}
    → [ i —→ x ] L ≡ L
      ----------------
    → [ i —→ y ] L ≡ L
  lemma2·7-1 {_} {_} {i} {x} {y} {L} assump =
    begin
      [ i —→ y ] L
    ≡⟨ sym (cong ([ i —→ y ]_) assump) ⟩
      [ i —→ y ] ([ i —→ x ] L)
    ≡⟨ ax1 i y x L ⟩
      [ i —→ x ] L
    ≡⟨ assump ⟩
      L
    ∎

  lemma2·7-2 : ∀ {A B : Set} ⦃ _ : Lns A B ⦄ {i j : ℕ} {x : List Char} {L : B}
    → j ≥ i → i ≻ L
      ----------------
    → [ j —→ x ] L ≡ L
  lemma2·7-2 {j = j} j≥i i≻L =
    let И⟨ Иe₁ , Иe₂ ⟩ = i≻L j ⦃ j≥i ⦄ in
      lemma2·7-1 (Иe₂ (fresh Иe₁) {fresh-correct Иe₁})

  lemma2·13 : ∀ {A B : Set} ⦃ _ : Lns A B ⦄ {L : B} {x : List Char} {i : ℕ} (j : ℕ)
    → j ≥ i      → i ≻ L
      ------------------
    → i ≻ ([ j —→ x ] L)
  lemma2·13 {_} {_} {L} {x} j j≥i i≻L k
    with j ≟ℕ k | Иe₁ (i≻L j ⦃ j≥i ⦄)
  ... | yes refl | l = И⟨ l , (λ a → ax1 j a x L) ⟩
  ... | no  j≢k  | l = И⟨ l , (λ a →
    begin
      [ k —→ a ] ([ j —→ x ] L)
    ≡⟨ ax5 k j a x L (sym-≢ j≢k) ⟩
      [ j —→ x ] ([ k —→ a ] L)
    ≡⟨ cong [ j —→ x ]_ (lemma2·7-2 it i≻L) ⟩
      [ j —→ x ] L
    ∎) ⟩

  open-rec-lc-lemma : ∀ {L i j u v}
    → i ≢ j
    → tm-tm[ i —→ u ] (tm-tm[ j —→ v ] L) ≡ tm-tm[ j —→ v ] L
    → tm-tm[ i —→ u ] L ≡ L
  open-rec-lc-lemma {fr x} i≢j assump = refl
  open-rec-lc-lemma {# k} {i} {j} i≢j assump
    with i ≟ℕ j | i ≟ℕ k
  ... | yes refl | _ = contradiction refl i≢j
  ... | no _     | no _ = refl
  ... | no _     | yes refl with j ≟ℕ k
  ...   | yes refl = contradiction refl i≢j
  ...   | no j≢k with k ≟ℕ k
  ...     | yes refl with () ← assump
  ...     | no  k≢k  = contradiction refl k≢k
  open-rec-lc-lemma {ƛ L} {i} {j} i≢j assump
    rewrite open-rec-lc-lemma {L} {suc i} {suc j}
      (suc-preserves-≢ i≢j)
      (ƛ-inj assump)
    = refl
  open-rec-lc-lemma {L · M} i≢j assump rewrite
      open-rec-lc-lemma {L} i≢j (proj₁ (·-inj assump))
    | open-rec-lc-lemma {M} i≢j (proj₂ (·-inj assump))
    = refl
  open-rec-lc-lemma {Λ L} i≢j assump
    rewrite open-rec-lc-lemma {L} i≢j (Λ-inj assump) = refl
  open-rec-lc-lemma {L [ x ]} i≢j assump
    rewrite open-rec-lc-lemma {L} i≢j (proj₁ ([]-inj assump)) = refl
  open-rec-lc-lemma {‵zero} i≢j assump = refl
  open-rec-lc-lemma {‵suc L} i≢j assump
    rewrite open-rec-lc-lemma {L} i≢j (‵suc-inj assump) = refl


  open-rec-lc-lemma-ty : ∀ {A i j u v}
    → i ≢ j
    → ty-ty[ i —→ u ] (ty-ty[ j —→ v ] A) ≡ ty-ty[ j —→ v ] A
    → ty-ty[ i —→ u ] A ≡ A
  open-rec-lc-lemma-ty {‵ℕ} i≢j assump = refl
  open-rec-lc-lemma-ty {t-fr x} i≢j assump = refl
  open-rec-lc-lemma-ty {t-# k} {i} {j} i≢j assump with i ≟ℕ k
  ... | no  i≢k  = refl
  ... | yes refl with j ≟ℕ k
  ...   | yes refl = contradiction refl i≢j
  ...   | no  j≢k with k ≟ℕ k
  ...     | yes refl with () ← assump
  ...     | no  k≢k = contradiction refl k≢k
  open-rec-lc-lemma-ty {A ⇒ B} i≢j assump rewrite
      open-rec-lc-lemma-ty {A} i≢j (proj₁ (⇒-inj assump))
    | open-rec-lc-lemma-ty {B} i≢j (proj₂ (⇒-inj assump))
    = refl
  open-rec-lc-lemma-ty {t-∀ A} {i} {j} i≢j assump
    rewrite open-rec-lc-lemma-ty {A} {suc i} {suc j}
      (suc-preserves-≢ i≢j)
      (∀-inj assump)
        = refl

  open-rec-lc-lemma-ty-tm : ∀ {L i j u v}
    → i ≢ j
    → ty-tm[ i —→ u ] (ty-tm[ j —→ v ] L) ≡ ty-tm[ j —→ v ] L
    → ty-tm[ i —→ u ] L ≡ L
  open-rec-lc-lemma-ty-tm {fr x} i≢j assump = refl
  open-rec-lc-lemma-ty-tm {# k} i≢j assump = refl
  open-rec-lc-lemma-ty-tm {ƛ L} i≢j assump rewrite
    open-rec-lc-lemma-ty-tm {L} i≢j (ƛ-inj assump) = refl
  open-rec-lc-lemma-ty-tm {L · M} i≢j assump rewrite
      open-rec-lc-lemma-ty-tm {L} i≢j (proj₁ (·-inj assump))
    | open-rec-lc-lemma-ty-tm {M} i≢j (proj₂ (·-inj assump)) = refl
  open-rec-lc-lemma-ty-tm {Λ L} i≢j assump rewrite
    open-rec-lc-lemma-ty-tm {L} (suc-preserves-≢ i≢j) (Λ-inj assump) = refl
  open-rec-lc-lemma-ty-tm {L [ A ]} i≢j assump rewrite
      open-rec-lc-lemma-ty-tm {L} i≢j (proj₁ ([]-inj assump))
    | open-rec-lc-lemma-ty {A} i≢j (proj₂ ([]-inj assump)) = refl
  open-rec-lc-lemma-ty-tm {‵zero} i≢j assump = refl
  open-rec-lc-lemma-ty-tm {‵suc L} i≢j assump rewrite
    open-rec-lc-lemma-ty-tm {L} i≢j (‵suc-inj assump) = refl

  open-rec-lc-lemma-ty-tm-tm-tm : ∀ {L i j u v}
    → ty-tm[ i —→ u ] (tm-tm[ j —→ v ] L) ≡ tm-tm[ j —→ v ] L
    → ty-tm[ i —→ u ] L ≡ L
  open-rec-lc-lemma-ty-tm-tm-tm {fr x} assump = refl
  open-rec-lc-lemma-ty-tm-tm-tm {# k} assump = refl
  open-rec-lc-lemma-ty-tm-tm-tm {ƛ L} assump rewrite
    open-rec-lc-lemma-ty-tm-tm-tm {L} (ƛ-inj assump) = refl
  open-rec-lc-lemma-ty-tm-tm-tm {L · M} assump rewrite
      open-rec-lc-lemma-ty-tm-tm-tm {L} (proj₁ (·-inj assump))
    | open-rec-lc-lemma-ty-tm-tm-tm {M} (proj₂ (·-inj assump))
    = refl
  open-rec-lc-lemma-ty-tm-tm-tm {Λ L} assump rewrite
    open-rec-lc-lemma-ty-tm-tm-tm {L} (Λ-inj assump) = refl
  open-rec-lc-lemma-ty-tm-tm-tm {L [ A ]} assump rewrite
      proj₂ ([]-inj assump)
    | open-rec-lc-lemma-ty-tm-tm-tm {L} (proj₁ ([]-inj assump))
    = refl
  open-rec-lc-lemma-ty-tm-tm-tm {‵zero} assump = refl
  open-rec-lc-lemma-ty-tm-tm-tm {‵suc L} assump rewrite
    open-rec-lc-lemma-ty-tm-tm-tm {L} (‵suc-inj assump) = refl
\end{code}

Now we simply prove axioms 1 and 5 for both kinds of opening, and we get the lemmas above for both.
\begin{code}
  ax1-type : ∀ (i : ℕ) (a b : List Char) (A : Type)
    → ty-ty[ i —→ a ] (ty-ty[ i —→ b ] A) ≡ ty-ty[ i —→ b ] A
  ax1-type i a b ‵ℕ = refl
  ax1-type i a b (t-fr x) = refl
  ax1-type i a b (t-# k) with i ≟ℕ k
  ... | yes _   = refl
  ... | no  i≢k with i ≟ℕ k
  ... | yes refl = contradiction refl i≢k
  ... | no  _    = refl
  ax1-type i a b (A ⇒ B)
    rewrite ax1-type i a b A | ax1-type i a b B = refl
  ax1-type i a b (t-∀ A) rewrite ax1-type (suc i) a b A = refl

  ax5-type : ∀ (i j : ℕ) (a b : List Char) (A : Type)
    → (i≢j : i ≢ j)
    → ty-ty[ i —→ a ] (ty-ty[ j —→ b ] A)
      ≡ ty-ty[ j —→ b ] (ty-ty[ i —→ a ] A)
  ax5-type i j a b ‵ℕ i≢j = refl
  ax5-type i j a b (t-fr x) i≢j = refl
  ax5-type i j a b (t-# k) i≢j with j ≟ℕ k
  ... | yes refl with i ≟ℕ k
  ... |   yes refl = contradiction refl i≢j
  ... |   no  i≢k  with j ≟ℕ j
  ... |     yes refl = refl
  ... |     no  j≢j  = contradiction refl j≢j
  ax5-type i j a b (t-# k) i≢j | no j≢k with i ≟ℕ k
  ... | yes refl = refl
  ... | no  i≢k  with j ≟ℕ k
  ... |   yes refl = contradiction refl j≢k
  ... |   no  _    = refl
  ax5-type i j a b (A ⇒ B) i≢j
    rewrite ax5-type i j a b A i≢j | ax5-type i j a b B i≢j = refl
  ax5-type i j a b (t-∀ A) i≢j
    rewrite ax5-type (suc i) (suc j) a b A (suc-preserves-≢ i≢j)
          = refl

  instance
    LnsType : Lns Type Type
    LnsType = record
      { [_—→_]_ = ty-ty[_—→_]_
      ; ax1 = ax1-type
      ; ax5 = ax5-type }
\end{code}

The proof for \texttt{Term} is very similar and omitted here (see the source file for the full proof).
\begin{code}
  ax1-term : ∀ (i : ℕ) (a b : List Char) (L : Term)
    → tm-tm[ i —→ a ] (tm-tm[ i —→ b ] L) ≡ tm-tm[ i —→ b ] L
  -- omitted.
  ax5-term : ∀ (i j : ℕ) (a b : List Char) (L : Term)
    → (i≢j : i ≢ j)
    → tm-tm[ i —→ a ] (tm-tm[ j —→ b ] L)
        ≡ tm-tm[ j —→ b ] (tm-tm[ i —→ a ] L)
  -- omitted.

  instance
    LnsTerm : Lns Term Term
    LnsTerm = record
      { [_—→_]_ = tm-tm[_—→_]_
      ; ax1 = ax1-term
      ; ax5 = ax5-term }

  ax1-ty-tm : ∀ (i : ℕ) (a b : List Char) (L : Term)
    → ty-tm[ i —→ a ] (ty-tm[ i —→ b ] L) ≡ ty-tm[ i —→ b ] L
  -- omitted.
  ax5-ty-tm : ∀ (i j : ℕ) (a b : List Char) (L : Term)
    → (i≢j : i ≢ j)
    → ty-tm[ i —→ a ] (ty-tm[ j —→ b ] L)
        ≡ ty-tm[ j —→ b ] (ty-tm[ i —→ a ] L)
  -- omitted.

  instance
    LnsTyTm : Lns Type Term
    LnsTyTm = record
      { [_—→_]_ = ty-tm[_—→_]_
      ; ax1 = ax1-ty-tm
      ; ax5 = ax5-ty-tm }
\end{code}

\begin{comment}
\begin{code}
  ax1-term i a b (fr x) = refl
  ax1-term i a b (# k) with i ≟ℕ k
  ... | yes refl = refl
  ... | no  i≢k  with i ≟ℕ k
  ... |   yes refl = contradiction refl i≢k
  ... |   no  _    = refl
  ax1-term i a b (ƛ L) rewrite ax1-term (suc i) a b L = refl
  ax1-term i a b (L · M)
    rewrite ax1-term i a b L | ax1-term i a b M = refl
  ax1-term i a b (Λ L) rewrite ax1-term i a b L = refl
  ax1-term i a b (L [ A ]) rewrite ax1-term i a b L = refl
  ax1-term i a b ‵zero = refl
  ax1-term i a b (‵suc L) rewrite ax1-term i a b L = refl

  ax5-term i j a b (fr x) i≢j = refl
  ax5-term i j a b (# k) i≢j with j ≟ℕ k
  ... | yes refl with i ≟ℕ j
  ... |   yes refl = contradiction refl i≢j
  ... |   no  i≢j with j ≟ℕ j
  ... |     yes refl = refl
  ... |     no  j≢j  = contradiction refl j≢j
  ax5-term i j a b (# k) i≢j | no j≢k with i ≟ℕ k
  ... | yes refl = refl
  ... | no  i≢k with j ≟ℕ k
  ... |   yes refl = contradiction refl j≢k
  ... |   no  j≢k  = refl
  ax5-term i j a b (ƛ L) i≢j rewrite
    ax5-term (suc i) (suc j) a b L (suc-preserves-≢ i≢j) = refl
  ax5-term i j a b (L · M) i≢j rewrite
      ax5-term i j a b L i≢j
    | ax5-term i j a b M i≢j = refl
  ax5-term i j a b (Λ L) i≢j
    rewrite ax5-term i j a b L i≢j = refl
  ax5-term i j a b (L [ A ]) i≢j
    rewrite ax5-term i j a b L i≢j = refl
  ax5-term i j a b ‵zero i≢j = refl
  ax5-term i j a b (‵suc L) i≢j rewrite ax5-term i j a b L i≢j
    = refl

  ax1-ty-tm i a b (fr x) = refl
  ax1-ty-tm i a b (# k) = refl
  ax1-ty-tm i a b (ƛ L) rewrite
    ax1-ty-tm i a b L = refl
  ax1-ty-tm i a b (L · M) rewrite
    ax1-ty-tm i a b L | ax1-ty-tm i a b M = refl
  ax1-ty-tm i a b (Λ L)
    rewrite ax1-ty-tm (suc i) a b L = refl
  ax1-ty-tm i a b (L [ A ])
    rewrite ax1-type i a b A | ax1-ty-tm i a b L = refl
  ax1-ty-tm i a b ‵zero = refl
  ax1-ty-tm i a b (‵suc L)
    rewrite ax1-ty-tm i a b L = refl

  ax5-ty-tm i j a b (fr x) i≢j = refl
  ax5-ty-tm i j a b (# k) i≢j = refl
  ax5-ty-tm i j a b (ƛ L) i≢j rewrite
    ax5-ty-tm i j a b L i≢j = refl
  ax5-ty-tm i j a b (L · M) i≢j
    rewrite ax5-ty-tm i j a b L i≢j | ax5-ty-tm i j a b M i≢j = refl
  ax5-ty-tm i j a b (Λ L) i≢j
    rewrite ax5-ty-tm (suc i) (suc j) a b L (suc-preserves-≢ i≢j) = refl
  ax5-ty-tm i j a b (L [ A ]) i≢j
    rewrite ax5-type i j a b A i≢j | ax5-ty-tm i j a b L i≢j = refl
  ax5-ty-tm i j a b ‵zero i≢j = refl
  ax5-ty-tm i j a b (‵suc L) i≢j
    rewrite ax5-ty-tm i j a b L i≢j = refl
\end{code}
\end{comment}

Sometimes, we want to be specific about which kind of local closure is being used. When it cannot be
inferred from context which one is meant, we specify it with these auxilary functions.

\begin{code}
  _≻ty_ : ℕ → Type → Set
  i ≻ty A = _≻_ ⦃ LnsType ⦄ i A

  _≻tm_ : ℕ → Term → Set
  i ≻tm L = _≻_ ⦃ LnsTerm ⦄ i L

  _≻ty-tm_ : ℕ → Term → Set
  i ≻ty-tm L = _≻_ ⦃ LnsTyTm ⦄ i L

  Ty-LocallyClosed : Type → Set
  Ty-LocallyClosed A = LocallyClosed ⦃ LnsType ⦄ A

  Tm-LocallyClosed : Term → Set
  Tm-LocallyClosed L = LocallyClosed ⦃ LnsTerm ⦄ L

  Ty-Tm-LocallyClosed : Term → Set
  Ty-Tm-LocallyClosed L = LocallyClosed ⦃ LnsTyTm ⦄ L
\end{code}

\begin{code}
  n≻‵ℕ : ∀ {n} → n ≻ ‵ℕ
  n≻‵ℕ j = И⟨ [] , (λ _ → refl) ⟩
  ty-fr-lc : ∀ {A} → Ty-LocallyClosed (t-fr A)
  ty-fr-lc j = И⟨ [] , (λ _ → refl) ⟩
  #-never-lc : ∀ {n} → ¬ Ty-LocallyClosed (t-# n)
  #-never-lc {n} lc-#n with lc-#n n ⦃ z≤n ⦄
  ... | И⟨ Иe₁ , Иe₂ ⟩ with n ≟ℕ n
  ...   | yes refl with () ← Иe₂ (fresh Иe₁) {fresh-correct Иe₁}
  ...   | no  n≢n  = contradiction refl n≢n
  fr⇒fr-lc : ∀ {A} → Ty-LocallyClosed (t-fr A ⇒ t-fr A)
  fr⇒fr-lc j = И⟨ [] , (λ _ → refl) ⟩
  ⇒-≻ : ∀ {A B i} → i ≻ (A ⇒ B) → (i ≻ A) × (i ≻ B)
  ⇒-≻ {A} {B} {i} i≻A⇒B = ⟨ i≻A , i≻B ⟩
    where
      i≻A : i ≻ A
      i≻A j = let И⟨ Иe₁ , Иe₂ ⟩ = i≻A⇒B j
        in И⟨ Иe₁ , (λ a {a∉} → proj₁ (⇒-inj (Иe₂ a {a∉}))) ⟩
      i≻B : i ≻ B
      i≻B j = let И⟨ Иe₁ , Иe₂ ⟩ = i≻A⇒B j
        in И⟨ Иe₁ , (λ a {a∉} → proj₂ (⇒-inj (Иe₂ a {a∉}))) ⟩
  ≻-⇒ : ∀ {A B i} → i ≻ A → i ≻ B → i ≻ (A ⇒ B)
  ≻-⇒ i≻A i≻B j =
    let И⟨ A-Иe₁ , A-Иe₂ ⟩ = i≻A j
        И⟨ B-Иe₁ , B-Иe₂ ⟩ = i≻B j
    in И⟨ A-Иe₁ ++ B-Иe₁ , (λ a {a∉} →
      let ⟨ a∉A , a∉B ⟩ = ∉-++ a∉
      in cong₂ _⇒_ (A-Иe₂ a {a∉A}) (B-Иe₂ a {a∉B})) ⟩
  i≻∀A⇒si≻A : ∀ {A i} → i ≻ (t-∀ A) → (suc i) ≻ A
  i≻∀A⇒si≻A {A} i≻∀ (suc j) =
    let И⟨ Иe₁ , Иe₂ ⟩ = i≻∀ j ⦃ ≤-pred it ⦄
    in И⟨ Иe₁ , (λ a {a∉} → ∀-inj (Иe₂ a {a∉})) ⟩
  si≻A⇒i≻∀A : ∀ {A i} → (suc i) ≻ A → i ≻ (t-∀ A)
  si≻A⇒i≻∀A {A} {i} si≻A j =
    let И⟨ Иe₁ , Иe₂ ⟩ = si≻A (suc j) ⦃ s≤s it ⦄
    in И⟨ Иe₁ , (λ a {a∉} → cong t-∀_ (Иe₂ a {a∉})) ⟩
  i≻ty-tmƛ : ∀ {L i} → i ≻ty-tm (ƛ L) → (i ≻ty-tm L)
  i≻ty-tmƛ {L} {i} i≻ j = let И⟨ Иe₁ , Иe₂ ⟩ = i≻ j
    in И⟨ Иe₁ , (λ a {a∉} → ƛ-inj (Иe₂ a {a∉})) ⟩
\end{code}

\section{Free variables}
I don't think this is needed...
\begin{code}
  ftv-ty : Type -> List (List Char)
  ftv-ty ‵ℕ = []
  ftv-ty (t-fr x) = x ∷ []
  ftv-ty (t-# i) = []
  ftv-ty (A ⇒ B) = ftv-ty A ++ ftv-ty B
  ftv-ty (t-∀ A) = ftv-ty A

  ftv-tm : Term → List (List Char)
  ftv-tm (fr x) = []
  ftv-tm (# i) = []
  ftv-tm (ƛ L) = ftv-tm L
  ftv-tm (L · M) = ftv-tm L ++ ftv-tm M
  ftv-tm (Λ L) = ftv-tm L
  ftv-tm (L [ A ]) = ftv-tm L ++ ftv-ty A
  ftv-tm ‵zero = []
  ftv-tm (‵suc L) = ftv-tm L

  fv-tm : Term → List (List Char)
  fv-tm (fr x) = x ∷ []
  fv-tm (# i) = []
  fv-tm (ƛ L) = fv-tm L
  fv-tm (L · M) = fv-tm L ++ fv-tm M
  fv-tm (Λ L) = fv-tm L
  fv-tm (L [ A ]) = fv-tm L
  fv-tm ‵zero = []
  fv-tm (‵suc L) = fv-tm L
\end{code}

\section{Substitution of Types and Terms}
As before, we can substitute a term for a free variable.
\begin{code}
  ty-ty[_:=_]_ : List Char → Type → Type → Type
  ty-ty[ x := T ] ‵ℕ = ‵ℕ
  ty-ty[ x := T ] (t-fr y) with x ≟lchar y
  ... | yes refl = T
  ... | no  _    = t-fr y
  ty-ty[ x := T ] (t-# i) = t-# i
  ty-ty[ x := T ] (A ⇒ B) = (ty-ty[ x := T ] A) ⇒ (ty-ty[ x := T ] B)
  ty-ty[ x := T ] (t-∀ A) = t-∀ (ty-ty[ x := T ] A)

  ty-tm[_:=_]_ : List Char → Type → Term → Term
  ty-tm[ x := T ] (fr y) = fr y
  ty-tm[ x := T ] (# i) = # i
  ty-tm[ x := T ] (ƛ L) = ƛ (ty-tm[ x := T ] L)
  ty-tm[ x := T ] (L · M) = (ty-tm[ x := T ] L) · (ty-tm[ x := T ] M)
  ty-tm[ x := T ] (Λ L) = Λ (ty-tm[ x := T ] L)
  ty-tm[ x := T ] (L [ A ]) = (ty-tm[ x := T ] L) [ ty-ty[ x := T ] A ]
  ty-tm[ x := T ] ‵zero = ‵zero
  ty-tm[ x := T ] (‵suc L) = ‵suc ty-tm[ x := T ] L

  tm-tm[_:=_]_ : List Char → Term → Term → Term
  tm-tm[ x := N ] (fr y) with x ≟lchar y
  ... | yes refl = N
  ... | no  _    = fr y
  tm-tm[ x := N ] (# i) = # i
  tm-tm[ x := N ] (ƛ L) = ƛ tm-tm[ x := N ] L
  tm-tm[ x := N ] (L · M) = (tm-tm[ x := N ] L) · (tm-tm[ x := N ] M)
  tm-tm[ x := T ] (Λ L) = Λ (tm-tm[ x := T ] L)
  tm-tm[ x := T ] (L [ A ]) = (tm-tm[ x := T ] L) [ A ]
  tm-tm[ x := N ] ‵zero = ‵zero
  tm-tm[ x := N ] (‵suc L) = ‵suc tm-tm[ x := N ] L

  :=-≻ : ∀ {A x C i j} → j ≥ i → i ≻ty A → i ≻ty C → j ≻ty (ty-ty[ x := C ] A)
  :=-≻ {‵ℕ} j≥i i≻A i≻C = n≻‵ℕ
  :=-≻ {t-fr y} {x} j≥i i≻A i≻C with x ≟lchar y
  ... | yes refl = lemma2·6 ⦃ LnsType ⦄ j≥i i≻C
  ... | no  x≢y  = lemma2·6 ⦃ LnsType ⦄ j≥i i≻A
  :=-≻ {t-# k} j≥i i≻A i≻C = lemma2·6 ⦃ LnsType ⦄ j≥i i≻A
  :=-≻ {A ⇒ B} j≥i i≻ i≻C k =
    let ⟨ i≻A , i≻B ⟩ = ⇒-≻ i≻
        И⟨ A-Иe₁ , A-Иe₂ ⟩ = (:=-≻ j≥i i≻A i≻C) k
        И⟨ B-Иe₁ , B-Иe₂ ⟩ = (:=-≻ j≥i i≻B i≻C) k
    in И⟨ A-Иe₁ ++ B-Иe₁ , (λ a {a∉} →
      let ⟨ a∉A , a∉B ⟩ = ∉-++ a∉
      in cong₂ _⇒_ (A-Иe₂ a {a∉A}) (B-Иe₂ a {a∉B})) ⟩
  :=-≻ {t-∀ A} {i = i}  j≥i i≻A i≻C k =
    let И⟨ Иe₁ , Иe₂ ⟩ = (:=-≻ (s≤s j≥i) (i≻∀A⇒si≻A i≻A) (lemma2·6 ⦃ LnsType ⦄ (n≤1+n i) i≻C)) (suc k) ⦃ s≤s it ⦄
    in И⟨ Иe₁ , (λ a {a∉} → cong t-∀_ (Иe₂ a {a∉})) ⟩

  :=⇒-≻ : ∀ {X C i j} (A B : Type) → j ≥ i → i ≻ty (A ⇒ B) → i ≻ty C → j ≻ty ((ty-ty[ X := C ] A) ⇒ (ty-ty[ X := C ] B))
  :=⇒-≻ {i = i} A B j≥i i≻A⇒B i≻C = ≻-⇒ (:=-≻ j≥i i≻A i≻C) (:=-≻ j≥i i≻B i≻C)
    where
      i≻A : i ≻ A
      i≻A = proj₁ (⇒-≻ i≻A⇒B)
      i≻B : i ≻ B
      i≻B = proj₂ (⇒-≻ i≻A⇒B)

  :=∀-≻ : ∀ {X C i j} (A : Type) → j ≥ i → i ≻ty (t-∀ A) → i ≻ty C → j ≻ty (ty-ty[ X := C ] (t-∀ A))
  :=∀-≻ {i = i} ‵ℕ j≥i i≻∀A i≻C j = И⟨ [] , (λ _ → refl) ⟩
  :=∀-≻ {X} {i = i} (t-fr Y) j≥i i≻∀A i≻C with X ≟lchar Y
  ... | no  X≢Y  = λ _ → И⟨ [] , (λ _ → refl) ⟩
  ... | yes refl = λ k →
    let И⟨ Иe₁ , Иe₂ ⟩ = i≻C (suc k) ⦃ m≤n⇒m≤1+n (≤-trans j≥i it) ⦄
    in И⟨ Иe₁ , (λ a {a∉} → cong t-∀_ (Иe₂ a {a∉})) ⟩
  :=∀-≻ {i = i} (t-# n) j≥i i≻∀A i≻C = lemma2·6 ⦃ LnsType ⦄ j≥i i≻∀A
  :=∀-≻ {i = i} (A ⇒ B) j≥i i≻∀ i≻C = si≻A⇒i≻∀A (:=⇒-≻ {i = suc i} A B (s≤s j≥i) si≻A⇒B (lemma2·6 ⦃ LnsType ⦄ (n≤1+n i) i≻C) )
    where
      si≻A : (suc i) ≻ A
      si≻A = proj₁ (⇒-≻ (i≻∀A⇒si≻A i≻∀))
      si≻B : (suc i) ≻ B
      si≻B = proj₂ (⇒-≻ (i≻∀A⇒si≻A i≻∀))
      si≻A⇒B : (suc i) ≻ (A ⇒ B)
      si≻A⇒B = ≻-⇒ si≻A si≻B
  :=∀-≻ {X} {i = i} (t-∀ A) j≥i i≻∀∀A i≻C k =
    let И⟨ Иe₁ , Иe₂ ⟩ = (:=∀-≻ {X} A (s≤s j≥i) si≻∀A (lemma2·6 ⦃ LnsType ⦄ (n≤1+n i) i≻C)) (suc k) ⦃ s≤s it ⦄
    in И⟨ Иe₁ , (λ a {a∉} → cong t-∀_ (Иe₂ a {a∉})) ⟩
    where
      si≻∀A : (suc i) ≻ (t-∀ A)
      si≻∀A = i≻∀A⇒si≻A i≻∀∀A

  :=-∉-idempotent : ∀ {A X B} → X ∉ ftv-ty A → (ty-ty[ X := B ] A) ≡ A
  :=-∉-idempotent {‵ℕ} X∉A = refl
  :=-∉-idempotent {t-fr Y} {X} X∉A with X ≟lchar Y
  ... | yes refl = contradiction refl (∉∷[]⇒≢ X∉A)
  ... | no  X≢Y  = refl
  :=-∉-idempotent {t-# k} X∉A = refl
  :=-∉-idempotent {A ⇒ B} {X} {B = C} X∉ = cong₂ _⇒_
    (:=-∉-idempotent {A} (proj₁ (∉-++ X∉)))
    (:=-∉-idempotent {B} (proj₂ (∉-++ X∉)))
  :=-∉-idempotent {t-∀ A} X∉A = cong t-∀_ (:=-∉-idempotent {A} X∉A)
\end{code}

And we also have the alternative substitution; substituting a term for a bound variable.
\begin{code}
  ty-ty[_:→_]_ : ℕ → Type → Type → Type
  ty-ty[ k :→ T ] ‵ℕ = ‵ℕ
  ty-ty[ k :→ T ] (t-fr x) = t-fr x
  ty-ty[ k :→ T ] (t-# i) with k ≟ℕ i
  ... | yes refl = T
  ... | no  _    = t-# i
  ty-ty[ k :→ T ] (A ⇒ B) = (ty-ty[ k :→ T ] A) ⇒ (ty-ty[ k :→ T ] B)
  ty-ty[ k :→ T ] (t-∀ A) = t-∀ (ty-ty[ (suc k) :→ T ] A)

  ty-tm[_:→_]_ : ℕ → Type → Term → Term
  ty-tm[ k :→ T ] (fr x) = fr x
  ty-tm[ k :→ T ] (# i) = # i
  ty-tm[ k :→ T ] (ƛ L) = ƛ ty-tm[ k :→ T ] L
  ty-tm[ k :→ T ] (L · M) = (ty-tm[ k :→ T ] L) · (ty-tm[ k :→ T ] M)
  ty-tm[ k :→ T ] (Λ L) = Λ ty-tm[ suc k :→ T ] L
  ty-tm[ k :→ T ] (L [ A ]) = (ty-tm[ k :→ T ] L) [ ty-ty[ k :→ T ] A ]
  ty-tm[ k :→ T ] ‵zero = ‵zero
  ty-tm[ k :→ T ] (‵suc L) = ‵suc ty-tm[ k :→ T ] L

  tm-tm[_:→_]_ : ℕ → Term → Term → Term
  tm-tm[ k :→ N ] (fr x) = fr x
  tm-tm[ k :→ N ] (# i) with k ≟ℕ i
  ... | yes refl = N
  ... | no  _    = # i
  tm-tm[ k :→ N ] (ƛ L) = ƛ tm-tm[ suc k :→ N ] L
  tm-tm[ k :→ N ] (L · M) = (tm-tm[ k :→ N ] L) · (tm-tm[ k :→ N ] M)
  tm-tm[ k :→ N ] (Λ L) = Λ tm-tm[ k :→ N ] L
  tm-tm[ k :→ N ] (L [ A ]) = (tm-tm[ k :→ N ] L) [ A ]
  tm-tm[ k :→ N ] ‵zero = ‵zero
  tm-tm[ k :→ N ] (‵suc L) = ‵suc tm-tm[ k :→ N ] L

  ≻⇒:→-idempotent : ∀ {C i j}
    (A : Type)
    → j ≥ i
    → i ≻ty A
    → ty-ty[ j :→ C ] A ≡ A
  ≻⇒:→-idempotent ‵ℕ j≥i i≻A = refl
  ≻⇒:→-idempotent (t-fr x) j≥i i≻A = refl
  ≻⇒:→-idempotent {C} {i} {j} (t-# n) j≥i i≻A with j ≟ℕ n
  ... | no  j≢n  = refl
  ... | yes refl with i≻A j ⦃ j≥i ⦄
  ...   | И⟨ Иe₁ , Иe₂ ⟩ with n ≟ℕ n
  ...     | yes refl with () ← Иe₂ (fresh Иe₁) {fresh-correct Иe₁}
  ...     | no  n≢n  = contradiction refl n≢n
  ≻⇒:→-idempotent (A ⇒ B) j≥i i≻ = let ⟨ i≻A , i≻B ⟩ = ⇒-≻ i≻
    in cong₂ _⇒_ (≻⇒:→-idempotent A j≥i i≻A) (≻⇒:→-idempotent B j≥i i≻B)
  ≻⇒:→-idempotent {i = i} {j = j} (t-∀ A) j≥i i≻ = cong t-∀_
    (≻⇒:→-idempotent A (s≤s j≥i) (i≻∀A⇒si≻A i≻))
\end{code}

\section{Type Environments}
Type environments work as they did for the STLC. Note that since we're implementing System F with
the \texttt{‵zero} and \texttt{‵suc} primitives only (and no other language extensions), we don't
need to add free type variables to the context \citep{pierce_types_2002}. This would, however, be
necessary if one wanted to implement System F$_{<:}$ \citep{chargueraud_locally_2012} or System
F$_{\omega}$ \citep{hutton_system_2019}.

\begin{code}
  data Context : Set where
    ∅ : Context
    _,_⦂_ : Context → List Char → Type → Context
    _,_ : Context → List Char → Context
\end{code}

Contexts are isomorphic to lists [TODO: CITE PLFA],
so we can also concatenate them and define map over the types of free variables.
We will need this later in proofs of substitution.

\begin{code}
  _+_ : Context → Context → Context
  Γ + ∅ = Γ
  Γ + (Δ , x ⦂ A) = (Γ + Δ) , x ⦂ A
  Γ + (Δ , X) = (Γ + Δ) , X

  map : ∀ (f : Type → Type) → Context → Context
  map f ∅ = ∅
  map f (Γ , x ⦂ A) = (map f Γ) , x ⦂ f A
  map f (Γ , X) = (map f Γ) , X
\end{code}

\paragraph*{Ok environments.} We need a predicate to tell when an environment doesn't contain any
bound type variables. Since the STLC only contains base and function types, this wasn't necessary
before. \citet{chargueraud_locally_2012} requires that any references free type variables
exist in the context, but because these aren't present in our contexts, this
restriction is unneccessary.

\begin{code}
  domain-fv : Context → List (List Char)
  domain-fv ∅ = []
  domain-fv (ctx , x ⦂ x₁) = x ∷ domain-fv ctx
  domain-fv (ctx , X) = domain-fv ctx

  domain-ftv : Context → List (List Char)
  domain-ftv ∅ = []
  domain-ftv (ctx , x ⦂ A) = domain-ftv ctx
  domain-ftv (ctx , X) = X ∷ domain-ftv ctx

  domain-all-ftv : Context → List (List Char)
  domain-all-ftv ∅ = []
  domain-all-ftv (ctx , x ⦂ A) = (ftv-ty A) ++ domain-all-ftv ctx
  domain-all-ftv (ctx , X) = X ∷ domain-all-ftv ctx

  weaken-dom-all-ftv : ∀ {X} (Γ : Context) → X ∉ domain-all-ftv Γ → X ∉ domain-ftv Γ
  weaken-dom-all-ftv {X} ∅ _ = λ ()
  weaken-dom-all-ftv {X} (Γ , x ⦂ A) X∉ftv-A++all-Γ =
    weaken-dom-all-ftv Γ (proj₂ (∉-++ X∉ftv-A++all-Γ))
  weaken-dom-all-ftv {X} (Γ , Y) X∉Y∷all-Γ =
    ++-∉
      (∉y∷ys⇒∉y∷[] X∉Y∷all-Γ)
      (weaken-dom-all-ftv Γ (∉y∷ys⇒∉ys X∉Y∷all-Γ))

  _⊆_ : List (List Char) → List (List Char) → Set
  xs ⊆ ys = All (λ x → x ∈ ys) xs

  ≡-⊆ : ∀ {xs ys zs} → xs ⊆ zs → xs ≡ ys → ys ⊆ zs
  ≡-⊆ xs⊆ refl = xs⊆

  ++-⊆ : ∀ {xs ys zs : List (List Char)} → xs ⊆ zs → ys ⊆ zs → (xs ++ ys) ⊆ zs
  ++-⊆ {[]} {ys} {zs} All.[] ys⊆zs = ys⊆zs
  ++-⊆ {x ∷ xs} {ys} {zs} (x∈zs All.∷ xs⊆zs) ys⊆zs = x∈zs All.∷ ++-⊆ xs⊆zs ys⊆zs

  ⊆-++ : ∀ {xs ys zs : List (List Char)} → (xs ++ ys) ⊆ zs → (xs ⊆ zs) × (ys ⊆ zs)
  ⊆-++ {[]} {ys} All.[] = ⟨ All.[] , All.[] ⟩
  ⊆-++ {[]} {ys} (px All.∷ xs++ys⊆) =
    ⟨ All.[] , (px All.∷ xs++ys⊆) ⟩
  ⊆-++ {x ∷ xs} {ys} (px All.∷ xs++ys⊆) =
    ⟨ (px All.∷ ⊆-++ xs++ys⊆ .proj₁)
    , (⊆-++ xs++ys⊆ .proj₂) ⟩

  ⊆⇒⊆∷ : ∀ {xs ys a} → xs ⊆ ys → xs ⊆ (a ∷ ys)
  ⊆⇒⊆∷ {xs} {ys} {a} xs⊆ys = All-map (λ px → ∈-swap {ys = a ∷ []} (∈-++ˡ px)) xs⊆ys

  domain-ftv-map-idempotent : ∀ {Γ f} → domain-ftv Γ ≡ (domain-ftv (map f Γ))
  domain-ftv-map-idempotent {∅} = refl
  domain-ftv-map-idempotent {Γ , x ⦂ A} = domain-ftv-map-idempotent {Γ}
  domain-ftv-map-idempotent {Γ , X} = cong (X ∷_) (domain-ftv-map-idempotent {Γ})

  domain-++ : ∀ (Γ Δ : Context) → (domain-ftv Δ) ++ (domain-ftv Γ) ≡ domain-ftv (Γ + Δ)
  domain-++ Γ ∅ = refl
  domain-++ Γ (Δ , x ⦂ A) = domain-++ Γ Δ
  domain-++ Γ (Δ , Y) = cong (Y ∷_) (domain-++ Γ Δ)

  domain-ftv-++ʳ : ∀ {X} (Γ Δ : Context) → X ∈ (domain-ftv Δ) → X ∈ domain-ftv (Γ + Δ)
  domain-ftv-++ʳ {X} Γ Δ X∈Δ = ∈-≡ (∈-++ˡ X∈Δ) (domain-++ Γ Δ)

  x∈xs∧xs⊆ys⇒x∈ys : ∀ {x xs ys} → x ∈ xs → xs ⊆ ys → x ∈ ys
  x∈xs∧xs⊆ys⇒x∈ys {x} {x' ∷ xs} {ys} x∈xs (px All.∷ xs⊆ys) with x ≟lchar x'
  ... | yes refl = px
  ... | no  x≢x' = x∈xs∧xs⊆ys⇒x∈ys (∈y∷ys∧≢y⇒∈ys x∈xs x≢x') xs⊆ys

  data Ok : Context → Set where
    ok-∅ : Ok ∅
    ok-∷fv : ∀ {Γ A x} → Ok Γ → Ty-LocallyClosed A → ftv-ty A ⊆ domain-ftv Γ → Ok (Γ , x ⦂ A)
    ok-∷ftv : ∀ {Γ X} → Ok Γ → X ∉ domain-all-ftv Γ → Ok (Γ , X)

  _ : Ok (((∅ , ('T' ∷ [])) , ('Q' ∷ [])) , ('x' ∷ []) ⦂ (t-fr ('T' ∷ [])))
  _ = ok-∷fv (ok-∷ftv (ok-∷ftv ok-∅ (λ ())) (λ Q∈ → contradiction (∈∷[]⇒≡ Q∈) λ ())) ty-fr-lc ((there (here refl)) All.∷ All.[])

  _ : Ok ((∅ , ('T' ∷ [])) + (∅ , ('x' ∷ []) ⦂ (t-fr ('T' ∷ []))))
  _ = ok-∷fv (ok-∷ftv ok-∅ (λ ())) ty-fr-lc ((here refl) All.∷ All.[])

  ok-+ : ∀ {Γ Δ} → Ok (Γ + Δ) → Ok Γ
  ok-+ {Γ} {∅} okΓ = okΓ
  ok-+ {Γ} {Δ , x ⦂ A} (ok-∷fv ok+ _ _) = ok-+ ok+
  ok-+ {Γ} {Δ , X} (ok-∷ftv ok+ _) = ok-+ ok+

  extract-⊆ : ∀ {Γ x A} → Ok (Γ , x ⦂ A) → ftv-ty A ⊆ domain-ftv Γ
  extract-⊆ (ok-∷fv okΓ lc-A A⊆Γ) = A⊆Γ
\end{code}

Accessing the context is done in two different ways:
\begin{itemize}
  \item If a free variable is required, then the accessor is the constructor \texttt{\_∋\_⦂\_}.
  \item If a free type variable is required, then the accessor is the constructor \texttt{\_∋\_}.
\end{itemize}
\begin{code}
  infix 4 _,_
  data _∋_ : Context → List Char → Set where
    Z : ∀ {Γ X} → (Γ , X) ∋ X
    S : ∀ {Γ X Y} → Γ ∋ X → (Γ , Y) ∋ X
    S⦂ : ∀ {Γ X y B} → Γ ∋ X → (Γ , y ⦂ B) ∋ X
  infix 4 _∋_⦂_

  data _∋_⦂_ : Context → List Char → Type → Set where
    H : ∀ {Γ x y A}
      → x ≡ y
        --------------------
      → (Γ , x ⦂ A) ∋ y ⦂ A

    T : ∀ {Γ x y A B}
      → x ≢ y
      → (Γ ∋ x ⦂ A)
        -----------------
      → (Γ , y ⦂ B) ∋ x ⦂ A

    T⦂ : ∀ {Γ x Y A}
      → (Γ ∋ x ⦂ A)
        -----------------
      → (Γ , Y) ∋ x ⦂ A

  H′ : ∀ {Γ x A} → (Γ , x ⦂ A) ∋ x ⦂ A
  H′ = H refl

  T′ : ∀ {Γ x y A B}
    → {x≢y : False (x ≟lchar y)}
    → Γ ∋ x ⦂ A
      ------------------
    → (Γ , y ⦂ B) ∋ x ⦂ A
  T′ { x≢y = x≢y } x = T (toWitnessFalse x≢y) x
\end{code}

As for the STLC, we provide some convenience functions with the \texttt{′} character to find the
relevant evidence automatically.

\begin{code}
  ∋⇒∈ : ∀ {Γ X} → Γ ∋ X → X ∈ domain-ftv Γ
  ∋⇒∈ Z = here refl
  ∋⇒∈ (S ∋X) = there (∋⇒∈ ∋X)
  ∋⇒∈ (S⦂ ∋X) = ∋⇒∈ ∋X

  ∋⦂⇒∈ : ∀ {Γ x X} → Γ ∋ x ⦂ t-fr X → X ∈ domain-all-ftv Γ
  ∋⦂⇒∈ {Γ , y ⦂ Y} (H refl) = here refl
  ∋⦂⇒∈ {Γ , y ⦂ Y} (T x ∋x) = ∈-++ʳ (∋⦂⇒∈ ∋x)
  ∋⦂⇒∈ {Γ , Y} (T⦂ ∋x) = ∈-++ʳ (∋⦂⇒∈ ∋x)

  ∈⇒∋ : ∀ {Γ X} → X ∈ domain-ftv Γ → Γ ∋ X
  ∈⇒∋ {Γ , x ⦂ A} X∈Γ = S⦂ (∈⇒∋ X∈Γ)
  ∈⇒∋ {Γ , Y} {X = X} (here X≡Y) with X ≟lchar Y
  ... | yes refl = Z
  ... | no  X≢Y  = contradiction X≡Y X≢Y
  ∈⇒∋ {Γ , Y} (there X∈Γ) = S (∈⇒∋ X∈Γ)

  ∉-domain-all-∋ : ∀ {Γ X x A} → X ∉ domain-all-ftv Γ → Γ ∋ x ⦂ A → X ∉ ftv-ty A
  ∉-domain-all-∋ X∉Γ (H x) = proj₁ (∉-++ X∉Γ)
  ∉-domain-all-∋ X∉Γ (T _ ∋x) = ∉-domain-all-∋ (proj₂ (∉-++ X∉Γ)) ∋x
  ∉-domain-all-∋ X∉Γ (T⦂ ∋x) = ∉-domain-all-∋ (proj₂ (∉-++ X∉Γ)) ∋x

  ⊆-change-ctx : ∀ {Γ A Δ}
    → ftv-ty A ⊆ domain-ftv Γ
    → (∀ {X} → Γ ∋ X → Δ ∋ X)
    → ftv-ty A ⊆ domain-ftv Δ
  ⊆-change-ctx {Γ} {A} A⊆Γ ρ = All-map (λ px → ∋⇒∈ (ρ (∈⇒∋ px))) A⊆Γ

  -- ⊆⇒⊆∷ : ∀ {Γ A X}
  --   → ftv-ty A ⊆ domain-ftv Γ
  --   → ftv-ty A ⊆ (X ∷ domain-ftv Γ)
  -- ⊆⇒⊆∷ {Γ} {A} {X} A⊆Γ = All-map (λ px → (∈-swap {ys = X ∷ []} (∈-++ˡ {xs = domain-ftv Γ} {ys = X ∷ []} px))) A⊆Γ

  ≡-with-∋-ty : ∀ {Γ x A B} → Γ ∋ x ⦂ A → A ≡ B → Γ ∋ x ⦂ B
  ≡-with-∋-ty ∋x refl = ∋x

  ≡-with-∋-ctx : ∀ {Γ Δ x A} → Γ ∋ x ⦂ A → Γ ≡ Δ → Δ ∋ x ⦂ A
  ≡-with-∋-ctx ∋x refl = ∋x

  ⇒-:= : ∀ {Γ X x C} (A B : Type) → Γ ∋ x ⦂ (ty-ty[ X := C ] (A ⇒ B)) → Γ ∋ x ⦂ ((ty-ty[ X := C ] A) ⇒ (ty-ty[ X := C ] B))
  ⇒-:= _ _ []⇒ = []⇒
  ∀-:= : ∀ {Γ X x C} (A : Type) → Γ ∋ x ⦂ (ty-ty[ X := C ] (t-∀ A)) → Γ ∋ x ⦂ (t-∀ (ty-ty[ X := C ] A))
  ∀-:= _ []∀ = []∀

  ∋-map-ftv : ∀ {Γ X x A C}
    (Δ : Context)
    → Ok ((Γ , X) + Δ)
    → ((Γ , X) + Δ) ∋ x ⦂ A
    → Γ + (map (ty-ty[ X := C ]_) Δ) ∋ x ⦂ (ty-ty[ X := C ] A)
  ∋-map-ftv ∅ (ok-∷ftv okΓ X∉Γ) (T⦂ ∋x) =
    ≡-with-∋-ty ∋x (sym (:=-∉-idempotent (∉-domain-all-∋ X∉Γ ∋x)))
  ∋-map-ftv {x = x} (Δ , y ⦂ B) _ (H refl) = H′
  ∋-map-ftv {x = x} (Δ , y ⦂ B) (ok-∷fv okΓ+Δ _ _) (T x≢y ∋x) = T x≢y (∋-map-ftv Δ okΓ+Δ ∋x)
  ∋-map-ftv (Δ , Y) (ok-∷ftv okΓ,X+Δ Y∉) (T⦂ ∋x) = T⦂ (∋-map-ftv Δ okΓ,X+Δ ∋x)

\end{code}

Here are some examples of accessing the context.
\begin{code}
  example-context : Context
  example-context = ((∅ , ('f' ∷ []) ⦂ (‵ℕ ⇒ ‵ℕ)) , ('T' ∷ [])) , ('x' ∷ []) ⦂ t-fr ('T' ∷ [])

  _ : example-context ∋ ('f' ∷ []) ⦂ (‵ℕ ⇒ ‵ℕ)
  _ = T′ (T⦂ H′)

  _ : example-context ∋ ('T' ∷ [])
  _ = S⦂ Z
\end{code}

\begin{code}
\end{code}

\section{Type judgements}
\label{chapter4:type_judgements}
Modifying the ones presented in \citet{chargueraud_locally_2012}, since he uses subtyping, System F
has these type judgements (some $x \in \texttt{List Char}$, types $A, B, T$, context $\Gamma$, and
terms $L, M$):
\begin{equation}
\begin{gathered}
  \inferrule
    { }
    {\Gamma \vdash \texttt{‵zero} \colon \nat}
    \; (\vdash\texttt{zero})\quad
  \inferrule
    {\Gamma \vdash L \colon \nat}
    {\Gamma \vdash \texttt{‵suc} \, L \colon \nat}
    \; (\vdash\texttt{suc})\quad
  \inferrule
    {0 \succ A\\x \colon A \in \Gamma}
    {\Gamma \vdash x \colon A}
    \; (\vdash\text{free})
  \\
  \inferrule
    {0 \succ A\\\cof x , \; (\Gamma , \, x \colon A \vdash [0 \to x] L \colon B)}
    {\Gamma \vdash (\lambda \colon A. L) \colon A \to B}
    \; (\vdash\lambda)\quad
  \inferrule
    {\Gamma \vdash L \colon A \to B\\\Gamma \vdash M \colon A}
    {\Gamma \vdash L M \colon B}
    \; (\vdash\text{app})
  \\
  \inferrule
    {\cof T, \; (\Gamma \vdash ([0 \to T] L) \colon [0 \to T] B)}
    {\Gamma \vdash (\Lambda \, L) \colon \forall B}
    \; (\vdash\Lambda)
  \quad
  \inferrule
    {0 \succ A\\\Gamma \vdash L \colon \forall B}
    {\Gamma \vdash L [ A ] \colon [0 \to A] B}
    \; (\vdash[])
\end{gathered}
\end{equation}

The first five rules are the same as they were for the STLC (with the exception of the locally
closed requirement ($0 \succ A$) and that the ($\vdash \lambda$) case was adapted to the new syntax,
it includes a `$\colon A$' after the $\lambda$ in the consequent). Notice how the two new rules for
type abstraction are similar to the two rules for $\lambda$-abstraction. For the ($\vdash\Lambda$)
judgement, both the term and the type need to be opened with the cofinite type variable $T$ because
bound type variables can be referenced in both the term and the type. For the ($\vdash[]$)
judgement, if one were to rewrite the antecedant type in fully named syntax, it would be $\Gamma
\vdash L \colon \forall A. B$, which more closely mirrors the ($\vdash\text{app}$) judgement. There
is only one part to the antecedant because we don't keep track of free type variables in the
context.

While \citet{chargueraud_locally_2012} requires the entire context to be well-formed (and calls this
predicate `Ok'), this is not strictly required. Since we don't add free type variables to the
context, we only need to make sure that the types which we use are locally closed.

\begin{code}
  infix  4 _⊢_⦂_
  data _⊢_⦂_ : Context → Term → Type → Set where
    ⊢free : ∀ {Γ x A}
      → Ok Γ
      → Ty-LocallyClosed A
      → Γ ∋ x ⦂ A
        ------------------
      → Γ ⊢ fr x ⦂ A

    ⊢ƛ : ∀ {Γ L A B}
      → Ty-LocallyClosed A
      → И x , (Γ , x ⦂ A ⊢ tm-tm[ 0 —→ x ] L ⦂ B)
        ----------------------------------------
      → Γ ⊢ ƛ L ⦂ (A ⇒ B)

    ⊢· : ∀ {Γ L M A B}
      → Γ ⊢ L ⦂ (A ⇒ B)
      → Γ ⊢ M ⦂ A
        ---------------
      → Γ ⊢ L · M ⦂ B

    ⊢Λ : ∀ {Γ L B}
      → И T , ((Γ , T) ⊢ ty-tm[ 0 —→ T ] L ⦂ (ty-ty[ 0 —→ T ] B))
        ----------------------------------------------------------
      → Γ ⊢ Λ L ⦂ t-∀ B

    ⊢[] : ∀ {Γ L A B}
      → Ty-LocallyClosed A
      → ftv-ty A ⊆ domain-ftv Γ
      → Γ ⊢ L ⦂ t-∀ B
        ---------------
      → Γ ⊢ L [ A ] ⦂ ty-ty[ 0 :→ A ] B

    ⊢zero : ∀ {Γ}
      → Ok Γ
        --------------
      → Γ ⊢ ‵zero ⦂ ‵ℕ

    ⊢suc : ∀ {Γ L}
      → Γ ⊢ L ⦂ ‵ℕ
        ---------------
      → Γ ⊢ ‵suc L ⦂ ‵ℕ
\end{code}

We can give a type to \textit{twice} and $\Omega$ now.
\begin{align}
\label{equation:twice_big_omega}
  \text{twice} &\triangleq (\Lambda \lambda \colon \mathbf{0} \to \mathbf{0}. \lambda \colon \mathbf{0}. 1(10))
    \colon \forall (\mathbf{0} \to \mathbf{0}) \to \mathbf{0} \to \mathbf{0}\\
  \Omega &\triangleq (\lambda \colon (\forall \mathbf{0} \to \mathbf{0}). (0 [\forall \mathbf{0} \to \mathbf{0}]) 0)
    \colon (\forall \mathbf{0} \to \mathbf{0}) \to (\forall \mathbf{0} \to \mathbf{0}) \notag
\end{align}
\begin{code}
  module example where
    twice : ∅ ⊢ (Λ ƛ (ƛ ((# 1) · ((# 1) · (# 0)))))
        ⦂ t-∀ (((t-# 0) ⇒ (t-# 0))
          ⇒ ((t-# 0) ⇒ (t-# 0)))
    twice = ⊢Λ И⟨ [] , (λ X {X∉} → ⊢ƛ fr⇒fr-lc И⟨ [] , (λ f {f∉} →
      ⊢ƛ ty-fr-lc И⟨ (f ∷ []) , (λ x {x∉} → ⊢·
        (⊢free ok-ctx fr⇒fr-lc (T (f≢x x∉) H′))
        (⊢· (⊢free ok-ctx fr⇒fr-lc (T (f≢x x∉) H′)) (⊢free ok-ctx ty-fr-lc H′))) ⟩) ⟩) ⟩
      where
        f≢x : ∀ {f x} → x ∉ f ∷ [] → f ≢ x
        f≢x x∉ = sym-≢ (∉∷[]⇒≢ x∉)
        ok-ctx : ∀ {f X x} → Ok (((∅ , X) , f ⦂ (t-fr X ⇒ t-fr X)) , x ⦂ t-fr X)
        ok-ctx = ok-∷fv
                   (ok-∷fv (ok-∷ftv ok-∅ (λ ()))
                     fr⇒fr-lc
                     (here refl All.∷ here refl All.∷ All.[]))
                   ty-fr-lc
                   ((here refl) All.∷ All.[])

    big-omega : ∅ ⊢ (ƛ (((# 0) [ t-∀ ((t-# 0) ⇒ (t-# 0)) ]) · (# 0)))
        ⦂ (t-∀ ((t-# 0) ⇒ (t-# 0)))
          ⇒ (t-∀ ((t-# 0) ⇒ (t-# 0)))
    big-omega =
      ⊢ƛ lc-forall И⟨ [] , (λ x {x∉} → ⊢· (⊢[] lc-forall All.[] (⊢free ok-ctx lc-forall H′)) (⊢free ok-ctx lc-forall H′)) ⟩
        where
          lc-forall : Ty-LocallyClosed (t-∀ (t-# 0 ⇒ t-# 0))
          lc-forall j = И⟨ [] , (λ _ → refl) ⟩
          ok-ctx : ∀ {x} →  Ok (∅ , x ⦂ (t-∀ (t-# 0 ⇒ t-# 0)))
          ok-ctx = ok-∷fv ok-∅ lc-forall All.[]
\end{code}

\section{Typing properties}
Once again, these properties are similar to those in \citet{wadler_programming_2022}.
\begin{code}
  ⊢⇒Ok : ∀ {Γ L A} → Γ ⊢ L ⦂ A → Ok Γ
  ⊢⇒Ok (⊢free okΓ lc-A ∋x) = okΓ
  ⊢⇒Ok (⊢ƛ lc-A И⟨ Иe₁ , Иe₂ ⟩) with ⊢⇒Ok (Иe₂ (fresh Иe₁) {fresh-correct Иe₁})
  ... | ok-∷fv OkΓ _ _ = OkΓ
  ⊢⇒Ok (⊢· ⊢L ⊢LM) = ⊢⇒Ok ⊢L
  ⊢⇒Ok (⊢Λ И⟨ Иe₁ , Иe₂ ⟩) with ⊢⇒Ok (Иe₂ (fresh Иe₁) {fresh-correct Иe₁})
  ... | ok-∷ftv OkΓ _ = OkΓ
  ⊢⇒Ok (⊢[] lc-A _ ⊢L) = ⊢⇒Ok ⊢L
  ⊢⇒Ok (⊢zero OkΓ) = OkΓ
  ⊢⇒Ok (⊢suc ⊢L) = ⊢⇒Ok ⊢L

  -- Extending contexts.
  ext-tm : ∀ {Γ Δ}
    → (∀ {x A}     →        Γ ∋ x ⦂ A → Δ ∋ x ⦂ A)
      -----------------------------------------------------
    → (∀ {x y A B} → Γ , y ⦂ B ∋ x ⦂ A → Δ , y ⦂ B ∋ x ⦂ A)
  ext-tm ρ (H refl) = H′
  ext-tm ρ (T x≢y ∋x) = T x≢y (ρ ∋x)

  ext-ty : ∀ {Γ Δ}
    → (∀ {X}     →        Γ ∋ X → Δ ∋ X)
      -----------------------------------------------------
    → (∀ {X Y} → (Γ , Y) ∋ X → (Δ , Y) ∋ X)
  ext-ty ρ Z = Z
  ext-ty ρ (S ∋X) = S (ρ ∋X)

  ext-tm-ty : ∀ {Γ Δ}
    → (∀ {x A}     →        Γ ∋ x ⦂ A → Δ ∋ x ⦂ A)
      -----------------------------------------------------
    → (∀ {x Y A} → (Γ , Y) ∋ x ⦂ A → (Δ , Y) ∋ x ⦂ A)
  ext-tm-ty ρ (T⦂ ∋x) = T⦂ (ρ ∋x)

  ext-ty-tm : ∀ {Γ Δ}
    → (∀ {X}     →        Γ ∋ X → Δ ∋ X)
      -----------------------------------------------------
    → (∀ {X y A} → (Γ , y ⦂ A) ∋ X → (Δ , y ⦂ A) ∋ X)
  ext-ty-tm ρ (S⦂ ∋X) = S⦂ (ρ ∋X)

  -- Renaming (a.k.a. rebasing) contexts.
  rename : ∀ {Γ Δ}
    → Ok Δ
    → (∀ {x A} → Γ ∋ x ⦂ A → Δ ∋ x ⦂ A)
    → (∀ {X} → Γ ∋ X → Δ ∋ X)
      ---------------------------------
    → (∀ {L A} → Γ ⊢ L ⦂ A → Δ ⊢ L ⦂ A)
  rename okΔ ρ-tm ρ-ty (⊢free okΓ lc-A x) = ⊢free okΔ lc-A (ρ-tm x)
  rename {Γ} {Δ} okΔ ρ-tm ρ-ty {_} {A ⇒ B} (⊢ƛ lc-A И⟨ Иe₁ , Иe₂ ⟩) =
    ⊢ƛ lc-A И⟨ Иe₁ , (λ a {a∉} →
      rename (OkΔ,x a a∉) (ext-tm ρ-tm) (ext-ty-tm ρ-ty) (Иe₂ a {proj₂ (∉-++ a∉)})) ⟩
    where
      OkΔ,x : (x : List Char) → x ∉ Иe₁ → Ok (Δ , x ⦂ A)
      OkΔ,x x x∉ with ⊢⇒Ok (Иe₂ x {x∉})
      ... | ok-∷fv OkΓ lc-A ftvA⊆Γ = ok-∷fv okΔ lc-A (⊆-change-ctx {Γ} {A} ftvA⊆Γ ρ-ty)
  rename okΔ ρ-tm ρ-ty (⊢· ⊢A⇒B ⊢A) = ⊢· (rename okΔ ρ-tm ρ-ty ⊢A⇒B) (rename okΔ ρ-tm ρ-ty ⊢A)
  rename {Γ} {Δ} okΔ ρ-tm ρ-ty (⊢Λ И⟨ Иe₁ , Иe₂ ⟩) =
    ⊢Λ И⟨ Иe₁ ++ domain-all-ftv Δ , (λ T {T∉} →
      let ⟨ T∉Иe₁ , T∉Δ ⟩ = ∉-++ T∉
      in rename (ok-∷ftv okΔ T∉Δ) (ext-tm-ty ρ-tm) (ext-ty ρ-ty) (Иe₂ T {T∉Иe₁}) ) ⟩
  rename okΔ ρ-tm ρ-ty (⊢[] lc A⊆Γ ⊢L) = ⊢[] lc (All-map (λ px → ∋⇒∈ (ρ-ty (∈⇒∋ px))) A⊆Γ) (rename okΔ ρ-tm ρ-ty ⊢L)
  rename okΔ ρ-tm ρ-ty (⊢zero okΓ) = ⊢zero okΔ
  rename okΔ ρ-tm ρ-ty (⊢suc ⊢L) = ⊢suc (rename okΔ ρ-tm ρ-ty ⊢L)

  -- Weakening contexts.
  weaken : ∀ {Γ L A}
    → Ok Γ
    → ∅ ⊢ L ⦂ A
      ---------
    → Γ ⊢ L ⦂ A
  weaken okΓ ⊢L = rename okΓ (λ ()) (λ ()) ⊢L

  -- Swapping variables in a context.
  swap : ∀ {Γ x y L A B C}
    → x ≢ y
    → (Γ , y ⦂ B) , x ⦂ A ⊢ L ⦂ C
      -------------------------
    → (Γ , x ⦂ A) , y ⦂ B ⊢ L ⦂ C
  swap {Γ} {x} {y} {L} {A} {B} {C} x≢y ⊢L with ⊢⇒Ok ⊢L
  ... | ok-∷fv (ok-∷fv okΓ lc-B B⊆Γ) lc-A A⊆Γ =
    rename (ok-∷fv (ok-∷fv okΓ lc-A A⊆Γ) lc-B B⊆Γ) ρ-tm ρ-tm-ty ⊢L
    where
      ρ-tm : ∀ {z C}
        → (Γ , y ⦂ B) , x ⦂ A ∋ z ⦂ C
          -------------------------
        → (Γ , x ⦂ A) , y ⦂ B ∋ z ⦂ C
      ρ-tm (H refl) = T x≢y H′
      ρ-tm (T z≢x (H refl)) = H′
      ρ-tm (T z≢x (T z≢y ∋z)) = T z≢y (T z≢x ∋z)
      ρ-tm-ty : ∀ {X}
        → ((Γ , y ⦂ B) , x ⦂ A) ∋ X
          -------------------------
        → ((Γ , x ⦂ A) , y ⦂ B) ∋ X
      ρ-tm-ty (S⦂ (S⦂ ∋X)) = S⦂ (S⦂ ∋X)

  swap-ty : ∀ {Γ X Y L C}
    → ((Γ , Y) , X) ⊢ L ⦂ C
      -------------------------
    → ((Γ , X) , Y) ⊢ L ⦂ C
  swap-ty {Γ} {X} {Y} {L} {C} ⊢L with ⊢⇒Ok ⊢L
  ... | ok-∷ftv (ok-∷ftv okΓ Y∉Γ) X∉Y∷Γ = rename (ok-∷ftv (ok-∷ftv okΓ (proj₂ (∉-++ X∉Y∷Γ))) Y∉X∷Γ) ρ₁ ρ₂ ⊢L
    where
      Y∉X∷Γ : Y ∉ X ∷ domain-all-ftv Γ
      Y∉X∷Γ = ≢∧∉⇒∉∷ (sym-≢ (∉y∷ys⇒≢y X∉Y∷Γ)) Y∉Γ
      ρ₁ : ∀ {x A}
        → ((Γ , Y) , X) ∋ x ⦂ A
          ---------------------
        → ((Γ , X) , Y) ∋ x ⦂ A
      ρ₁ (T⦂ (T⦂ ∋x)) = T⦂ (T⦂ ∋x)
      ρ₂ : ∀ {W}
        → ((Γ , Y) , X) ∋ W
          -----------------
        → ((Γ , X) , Y) ∋ W
      ρ₂ Z = S Z
      ρ₂ (S Z) = Z
      ρ₂ (S (S ∋W)) = S (S ∋W)


  swap-tm-ty : ∀ {Γ X y L B C}
    → ((Γ , y ⦂ B) , X) ⊢ L ⦂ C
      -------------------------
    → (Γ , X) , y ⦂ B ⊢ L ⦂ C
  swap-tm-ty {Γ} {X} {y} {L} {B} {C} ⊢L with ⊢⇒Ok ⊢L
  ... | ok-∷ftv (ok-∷fv okΓ lc-B B⊆Γ) X∉Γ = rename (ok-∷fv (ok-∷ftv okΓ (proj₂ (∉-++ X∉Γ))) lc-B (⊆⇒⊆∷ B⊆Γ)) ρ₁ ρ₂ ⊢L
    where
      ρ₁ : ∀ {x A}
        → ((Γ , y ⦂ B) , X) ∋ x ⦂ A
          -------------------------
        → ((Γ , X) , y ⦂ B) ∋ x ⦂ A
      ρ₁ (T⦂ (H refl)) = H′
      ρ₁ (T⦂ (T x≢y ∋x)) = T x≢y (T⦂ ∋x)
      ρ₂ : ∀ {Z : List Char}
        → ((Γ , y ⦂ B) , X) ∋ Z
        → ((Γ , X) , y ⦂ B) ∋ Z
      ρ₂ Z = S⦂ Z
      ρ₂ (S (S⦂ ∋Z)) = S⦂ (S ∋Z)

  -- swap-ty-tm : ∀ {Γ X y L A B C}
  --   → ((Γ , X) , y ⦂ A) ⊢ L ⦂ B
  --     -------------------------
  --   → ((Γ , y ⦂ (ty-ty[ X := C ] A)) , X) ⊢ L ⦂ B
  -- swap-ty-tm {Γ} {X} {y} {L} {A} {B} {C} ⊢L with ⊢⇒Ok ⊢L
  -- ... | ok-∷fv (ok-∷ftv okΓ x∉Γ) lc-A A⊆X∷Γ = rename (ok-∷ftv (ok-∷fv okΓ {!!} {!!}) {!!}) {!!} {!!} ⊢L

  -- Dropping shadowed variables.
  drop : ∀ {Γ x L A B C}
    → (Γ , x ⦂ A) , x ⦂ B ⊢ L ⦂ C
      --------------------------
    → Γ , x ⦂ B ⊢ L ⦂ C
  drop {Γ} {x} {L} {A} {B} {C} ⊢L with ⊢⇒Ok ⊢L
  ... | ok-∷fv (ok-∷fv okΓ lc-A A⊆Γ) lc-B B⊆Γ =
    rename (ok-∷fv okΓ lc-B B⊆Γ) ρ-tm ρ-tm-ty ⊢L
    where
      ρ-tm : ∀ {z C}
        → (Γ , x ⦂ A) , x ⦂ B ∋ z ⦂ C
          -------------------------
        → Γ , x ⦂ B ∋ z ⦂ C
      ρ-tm (H refl) = H′
      ρ-tm (T z≢x (H refl)) = contradiction refl z≢x
      ρ-tm (T z≢x (T .z≢x ∋z)) = T z≢x ∋z
      ρ-tm-ty : ∀ {X}
        → ((Γ , x ⦂ A) , x ⦂ B) ∋ X
          -------------------------
        → (Γ , x ⦂ B) ∋ X
      ρ-tm-ty (S⦂ (S⦂ ∋X)) = S⦂ ∋X

  -- Apply term-equality within type judgements.
  ≡-with-⊢-tm : ∀ {Γ L M A}
    → Γ ⊢ L ⦂ A
    → L ≡ M
      ---------
    → Γ ⊢ M ⦂ A
  ≡-with-⊢-tm ⊢L refl = ⊢L

  ≡-with-⊢-ty : ∀ {Γ L A B}
    → Γ ⊢ L ⦂ A
    → A ≡ B
      ---------
    → Γ ⊢ L ⦂ B
  ≡-with-⊢-ty ⊢L refl = ⊢L

  ≡-with-⊢-ctx : ∀ {Γ Δ L A}
    → Γ ⊢ L ⦂ A
    → Γ ≡ Δ
      ---------
    → Δ ⊢ L ⦂ A
  ≡-with-⊢-ctx ⊢L refl = ⊢L
\end{code}

TODO: explain below thing.

\begin{code}
  ftv⊆dom-:= : ∀ {X A C}
    (Γ Δ : Context)
     → Ty-LocallyClosed C
     → ftv-ty C ⊆ domain-ftv Δ
     -- → Ok (((Γ , X) + Δ) , x ⦂ A) -- ((Γ , X) + Δ) ∋ x ⦂ A
     → Ok (Γ + (map (ty-ty[ X := C ]_) Δ))
     → ftv-ty A ⊆ domain-ftv ((Γ , X) + Δ)
     → ftv-ty (ty-ty[ X := C ] A) ⊆ domain-ftv (Γ + map (ty-ty[ X := C ]_) Δ)
  -- ftv⊆dom-:= {x} {X} {A} {C} Γ Δ lc-C (ok-∷fv okΓ,Y+Δ lc-A A⊆Γ,X+Δ) okΓ+map = {!!}
  ftv⊆dom-:= {X} {‵ℕ} {C} Γ Δ lc-C C⊆Δ okΓ+map A⊆dom = All.[]
  ftv⊆dom-:= {X} {t-fr Y} {C} Γ Δ lc-C C⊆Δ okΓ+map (Y∈ All.∷ All.[]) with X ≟lchar Y
  ... | yes refl = All-map (λ x∈ → domain-ftv-++ʳ Γ (map (ty-ty[ X := C ]_) Δ) (∈-≡ x∈ (domain-ftv-map-idempotent {Δ} {ty-ty[ X := C ]_}))) C⊆Δ
  ... | no  X≢Y  = (helper {Δ = Δ} Y∈ (sym-≢ X≢Y)) All.∷ All.[]
    where
      helper : ∀ {Γ Δ X Y}
        → Y ∈ domain-ftv ((Γ , X) + Δ)
        → Y ≢ X
        → Y ∈ domain-ftv (Γ + map (ty-ty[_:=_]_ X C) Δ)
      helper {Γ} {∅} (here refl) Y≢X = contradiction refl Y≢X 
      helper {Γ} {∅} (there Y∈) Y≢X = Y∈
      helper {Γ} {Δ , z ⦂ C} Y∈ Y≢X = helper {Γ} {Δ} Y∈ Y≢X
      helper {Γ} {Δ , W} (here refl) Y≢X = here refl
      helper {Γ} {Δ , W} (there Y∈) Y≢X = there (helper {Δ = Δ} Y∈ Y≢X)
  ftv⊆dom-:= {X} {t-# n} {C} Γ Δ lc-C C⊆Δ okΓ+map A⊆dom = All.[]
  ftv⊆dom-:= {X} {A ⇒ B} {C} Γ Δ lc-C C⊆Δ okΓ+map ⊆dom =
    let ⟨ A⊆ , B⊆ ⟩ = ⊆-++ ⊆dom
    in ++-⊆ (ftv⊆dom-:= {A = A} Γ Δ lc-C C⊆Δ okΓ+map A⊆) (ftv⊆dom-:= {A = B} Γ Δ lc-C C⊆Δ okΓ+map B⊆)
  ftv⊆dom-:= {X} {t-∀ A} {C} Γ Δ lc-C C⊆Δ okΓ+map A⊆dom = ftv⊆dom-:= {X} {A} {C} Γ Δ lc-C C⊆Δ okΓ+map A⊆dom
\end{code}

We will also need a lemma called \texttt{subst-open-context} by \citet{chargueraud_locally_2012}.

\begin{code}
  subst-open-var : ∀ {N : Term} {x y : List Char} {i : ℕ}
    (L : Term)
    → x ≢ y
    → (i ≻tm N)
    → tm-tm[ x := N ] (tm-tm[ i —→ y ] L)
      ≡ tm-tm[ i —→ y ] (tm-tm[ x := N ] L)
  subst-open-var {x = x} (fr z) x≢y i≻u with x ≟lchar z
  ... | yes refl = sym (lemma2·7-2 ⦃ LnsTerm ⦄ ≤-refl i≻u)
  ... | no  _    = refl
  subst-open-var {_} {x} {y} {i} (# k) x≢y i≻u with i ≟ℕ k
  ... | no  _ = refl
  ... | yes refl with x ≟lchar y
  ...    | yes refl = contradiction refl x≢y
  ...    | no  _    = refl
  subst-open-var {i = i} (ƛ L) x≢y i≻u = cong ƛ_
    (subst-open-var L x≢y (lemma2·6 ⦃ LnsTerm ⦄ (n≤1+n i) i≻u))
  subst-open-var (L · M) x≢y i≻u = cong₂ _·_
    (subst-open-var L x≢y i≻u) (subst-open-var M x≢y i≻u)
  subst-open-var (Λ L) x≢y i≻u =
    cong Λ_ (subst-open-var L x≢y i≻u)
  subst-open-var (L [ A ]) x≢y i≻u =
    cong₂ _[_] (subst-open-var L x≢y i≻u) refl
  subst-open-var ‵zero x≢y i≻u = refl
  subst-open-var (‵suc L) x≢y i≻u =
    cong ‵suc_ (subst-open-var L x≢y i≻u)

  subst-open-var-ctx : ∀ {Γ A} {N : Term} {x y : List Char} {i : ℕ}
    (L : Term)
    → x ≢ y
    → (i ≻tm N)
    → Γ ⊢ tm-tm[ x := N ] (tm-tm[ i —→ y ] L) ⦂ A
    → Γ ⊢ tm-tm[ i —→ y ] (tm-tm[ x := N ] L) ⦂ A
  subst-open-var-ctx L x≢y i≻N assump =
    ≡-with-⊢-tm assump (subst-open-var L x≢y i≻N)
\end{code}

\section{Well-typed terms are locally closed}
If the typing rules are correct, then all well-typed terms are locally closed. This is particularly
important because a non-locally closed term has a bound variable which isn't actually bound. The
proof of this in System F is presented below. TODO: Explain more.

We will need some further lemmas regarding local closure.
\begin{code}

  s≻ƛ⇒≻ƛ : ∀ {L i} → (suc i) ≻tm L → i ≻tm (ƛ L)
  s≻ƛ⇒≻ƛ s≻ j = let И⟨ Иe₁ , Иe₂ ⟩ = s≻ (suc j) ⦃ s≤s it ⦄
    in И⟨ Иe₁ , (λ a {a∉} → cong ƛ_ (Иe₂ a {a∉})) ⟩

  ≻ƛ⇒s≻ƛ : ∀ {L i} → i ≻tm (ƛ L) → (suc i) ≻tm L
  ≻ƛ⇒s≻ƛ ≻ƛ (suc j) = let И⟨ Иe₁ , Иe₂ ⟩ = ≻ƛ j ⦃ ≤-pred it ⦄
    in И⟨ Иe₁ , (λ a {a∉} → ƛ-inj (Иe₂ a {a∉})) ⟩

  ·-≻ : ∀ {L M i} → i ≻tm (L · M) → (i ≻tm L) × (i ≻tm M)
  ·-≻ {L} {M} {i} i≻ = ⟨ i≻L , i≻M ⟩
    where
      i≻L : i ≻tm L
      i≻L j = let И⟨ Иe₁ , Иe₂ ⟩ = i≻ j
        in И⟨ Иe₁ , (λ a {a∉} → proj₁ (·-inj (Иe₂ a {a∉}))) ⟩
      i≻M : i ≻tm M
      i≻M j = let И⟨ Иe₁ , Иe₂ ⟩ = i≻ j
        in И⟨ Иe₁ , (λ a {a∉} → proj₂ (·-inj (Иe₂ a {a∉}))) ⟩

  Λ-≻ : ∀ {L i} → i ≻tm (Λ L) → i ≻tm L
  Λ-≻ i≻ j = let И⟨ Иe₁ , Иe₂ ⟩ = i≻ j
    in И⟨ Иe₁ , (λ a {a∉} → Λ-inj (Иe₂ a {a∉})) ⟩

  ≻Λ⇒s≻Λ : ∀ {L i} → i ≻ty-tm (Λ L) → (suc i) ≻ty-tm L
  ≻Λ⇒s≻Λ i≻Λ (suc j) = let И⟨ Иe₁ , Иe₂ ⟩ = i≻Λ j ⦃ ≤-pred it ⦄
    in И⟨ Иe₁ , (λ a {a∉} → Λ-inj (Иe₂ a {a∉})) ⟩

  []-≻ : ∀ {L A i} → i ≻tm (L [ A ]) → i ≻tm L
  []-≻ {L} {A} {i} i≻ j = let И⟨ Иe₁ , Иe₂ ⟩ = i≻ j
    in И⟨ Иe₁ , (λ a {a∉} → proj₁ ([]-inj (Иe₂ a {a∉}))) ⟩

  ‵suc-≻ : ∀ {L i} → i ≻tm (‵suc L) → i ≻tm L
  ‵suc-≻ i≻ j = let И⟨ Иe₁ , Иe₂ ⟩ = i≻ j
    in И⟨ Иe₁ , (λ a {a∉} → ‵suc-inj (Иe₂ a {a∉})) ⟩
\end{code}

An important lemma for proving that substitution preserves types is that all well-typed terms are
locally closed.
\begin{code}
  ⊢⇒lc : ∀ {Γ L A} → Γ ⊢ L ⦂ A → Tm-LocallyClosed L
  ⊢⇒lc (⊢free okΓ lc-A ∋x) j = И⟨ [] , (λ a → refl) ⟩
  ⊢⇒lc (⊢ƛ lc-A И⟨ Иe₁ , Иe₂ ⟩) j = И⟨ Иe₁ , (λ a {a∉} → cong ƛ_
    (open-rec-lc-lemma
      (λ ())
      (lemma2·7-2 ⦃ LnsTerm ⦄ z≤n (⊢⇒lc (Иe₂ a {a∉}))))) ⟩
  ⊢⇒lc {Γ} (⊢· ⊢A⇒B ⊢A) _ = И⟨ [] , (λ _ → cong₂ _·_
    (lemma2·7-2 ⦃ LnsTerm ⦄ z≤n (⊢⇒lc ⊢A⇒B))
    (lemma2·7-2 ⦃ LnsTerm ⦄ z≤n (⊢⇒lc ⊢A))) ⟩
  ⊢⇒lc {Γ} (⊢Λ И⟨ Иe₁ , Иe₂ ⟩) j = И⟨ Иe₁ , (λ a {a∉} → cong Λ_ (lemma2·7-2 ⦃ LnsTerm ⦄ z≤n (helper z≤n (⊢⇒lc (Иe₂ a {a∉}))))) ⟩
    where
      helper : ∀ {N i k q x}
        → k ≥ i
        → i ≻tm (ty-tm[ q —→ x ] N)
        → k ≻tm N
      helper {fr x} {k = k} k≥i i≻ty j = i≻ty k ⦃ k≥i ⦄
      helper {# n} k≥i i≻ty = lemma2·6 ⦃ LnsTerm ⦄ k≥i i≻ty
      helper {ƛ N} {i} {k} {q} {x} k≥i i≻ty j =
        let И⟨ Иe₁ , Иe₂ ⟩ = (helper {k = suc i} (≤-refl) (≻ƛ⇒s≻ƛ i≻ty)) (suc j) ⦃ s≤s (≤-trans k≥i it) ⦄
        in И⟨ Иe₁ , (λ a {a∉} → cong ƛ_ (Иe₂ a {a∉})) ⟩
      helper {N · N₁} k≥i i≻ty j =
        let ⟨ i≻L  , i≻M ⟩ = ·-≻ i≻ty
            И⟨ L-Иe₁ , L-Иe₂ ⟩ = (helper k≥i i≻L) j
            И⟨ M-Иe₁ , M-Иe₂ ⟩ = (helper k≥i i≻M) j
          in И⟨ (L-Иe₁ ++ M-Иe₁) , (λ a {a∉} → cong₂ _·_
            (L-Иe₂ a {proj₁ (∉-++ a∉)})
            (M-Иe₂ a {proj₂ (∉-++ {xs = L-Иe₁} a∉)})) ⟩
      helper {Λ N} k≥i i≻ty j =
        let И⟨ Иe₁ , Иe₂ ⟩ = (helper k≥i (Λ-≻ i≻ty)) j
        in И⟨ Иe₁ , (λ a {a∉} → cong Λ_ (Иe₂ a {a∉})) ⟩
      helper {N [ A ]} k≥i i≻ty j =
        let И⟨ Иe₁ , Иe₂ ⟩ = (helper k≥i ([]-≻ i≻ty)) j
        in И⟨ Иe₁ , (λ a {a∉} → cong _[ A ] (Иe₂ a {a∉})) ⟩
      helper {‵zero} k≥i i≻ty j = И⟨ [] , (λ _ → refl) ⟩
      helper {‵suc N} k≥i i≻ty j =
        let И⟨ Иe₁ , Иe₂ ⟩ = (helper k≥i (‵suc-≻ i≻ty)) j
        in И⟨ Иe₁ , (λ a {a∉} → cong ‵suc_ (Иe₂ a {a∉})) ⟩
  ⊢⇒lc {L = L [ A ]} (⊢[] _ _ ⊢L) j =
    let И⟨ Иe₁ , Иe₂ ⟩ = (⊢⇒lc ⊢L) j
    in И⟨ Иe₁ , (λ a {a∉} → cong _[ A ]  (Иe₂ a {a∉})) ⟩
  ⊢⇒lc (⊢zero _) j = И⟨ [] , (λ _ → refl) ⟩
  ⊢⇒lc (⊢suc ⊢L) j =
    let И⟨ Иe₁ , Иe₂ ⟩ = (⊢⇒lc ⊢L) j
    in И⟨ Иe₁ , (λ a {a∉} → cong ‵suc_ (Иe₂ a {a∉})) ⟩

  ⊢⇒lc-ty : ∀ {Γ L A} → Γ ⊢ L ⦂ A → Ty-LocallyClosed A
  ⊢⇒lc-ty {Γ} {fr x} (⊢free okΓ lc-A ∋x) = lc-A
  ⊢⇒lc-ty {Γ} {ƛ L} (⊢ƛ lc-B И⟨ Иe₁ , Иe₂ ⟩) j =
    let И⟨ B-Иe₁ , B-Иe₂ ⟩ = lc-B j
        И⟨ A-Иe₁ , A-Иe₂ ⟩ = (⊢⇒lc-ty (Иe₂ (fresh Иe₁) {fresh-correct Иe₁})) j
    in И⟨ A-Иe₁ ++ B-Иe₁ , (λ a {a∉} → cong₂ _⇒_
      (B-Иe₂ a {proj₂ (∉-++ {xs = A-Иe₁} a∉)})
      (A-Иe₂ a {proj₁ (∉-++ a∉)})) ⟩
  ⊢⇒lc-ty {Γ} {L · M} (⊢· ⊢L ⊢M) = proj₂ (⇒-≻ (⊢⇒lc-ty ⊢L))
  ⊢⇒lc-ty {Γ} {Λ L} (⊢Λ И⟨ Иe₁ , Иe₂ ⟩) j =
    let induction-hypo = ⊢⇒lc-ty (Иe₂ (fresh Иe₁) {fresh-correct Иe₁})
        И⟨ B-Иe₁ , B-Иe₂ ⟩ = induction-hypo (suc j) ⦃ z≤n ⦄
    in И⟨ B-Иe₁ ++ Иe₁ , (λ a {a∉} → cong t-∀_
      (open-rec-lc-lemma-ty
        (λ ())
        (B-Иe₂ a {proj₁ (∉-++ a∉)}))) ⟩
  ⊢⇒lc-ty {Γ} {L [ B ]} (⊢[] lc-B _ ⊢L) =
    let 1≻A = i≻∀A⇒si≻A (⊢⇒lc-ty ⊢L)
    in helper z≤n 1≻A lc-B
    where
      helper : ∀ {A B i j} → j ≥ i → (suc i) ≻ty A → i ≻ B → j ≻ (ty-ty[ i :→ B ] A)
      helper {‵ℕ} j≥i si≻A lc-B j = И⟨ [] , (λ _ → refl) ⟩
      helper {t-fr x} j≥i si≻A lc-B j = И⟨ [] , (λ _ → refl) ⟩
      helper {t-# n} {_} {i} j≥i si≻A i≻B k with i ≟ℕ n
      ... | yes refl = i≻B k ⦃ ≤-trans j≥i it ⦄
      ... | no  i≢n  with k ≟ℕ n
      ...   | no  _    = И⟨ [] , (λ _ → refl) ⟩
      ...   | yes refl with si≻A n ⦃ ≤∧≢⇒< (≤-trans j≥i it) i≢n ⦄ -- ⦃ ≤∧≢⇒< it 0≢k ⦄
      ...     | И⟨ Иe₁ , Иe₂ ⟩ with n ≟ℕ n
      ...       | yes refl with () ← Иe₂ (fresh Иe₁) {fresh-correct Иe₁}
      ...       | no  n≢n =  contradiction refl n≢n
      helper {A ⇒ C} j≥i si≻A⇒C i≻B k =
        let ⟨ si≻A , si≻C ⟩ = ⇒-≻ si≻A⇒C
            И⟨ A-Иe₁ , A-Иe₂ ⟩ = (helper j≥i si≻A i≻B) k
            И⟨ C-Иe₁ , C-Иe₂ ⟩ = (helper j≥i si≻C i≻B) k
        in И⟨ A-Иe₁ ++ C-Иe₁ , (λ a {a∉} → cong₂ _⇒_
          (A-Иe₂ a {proj₁ (∉-++ a∉)})
          (C-Иe₂ a {proj₂ (∉-++ {xs = A-Иe₁} a∉)})) ⟩
      helper {t-∀ A} {B} {i} j≥i si≻∀A i≻B k =
        let ssi≻A = i≻∀A⇒si≻A si≻∀A
            И⟨ Иe₁ , Иe₂ ⟩ = (helper (s≤s j≥i) ssi≻A (lemma2·6 (n≤1+n i) i≻B)) (suc k) ⦃ s≤s it ⦄
        in И⟨ Иe₁ , (λ a {a∉} → cong t-∀_ (Иe₂ a {a∉})) ⟩
  ⊢⇒lc-ty {Γ} {‵zero} (⊢zero _) = n≻‵ℕ
  ⊢⇒lc-ty {Γ} {‵suc L} (⊢suc ⊢L) = n≻‵ℕ

  ⊢⇒lc-ty-tm : ∀ {Γ L A} → Γ ⊢ L ⦂ A → Ty-Tm-LocallyClosed L
  ⊢⇒lc-ty-tm (⊢free okΓ lc-A ∋x) j = И⟨ [] , (λ _ → refl) ⟩
  ⊢⇒lc-ty-tm (⊢ƛ lc-A И⟨ B-Иe₁ , B-Иe₂ ⟩) j =
    let И⟨ A-Иe₁ , A-Иe₂ ⟩ = lc-A j
        И⟨ Иe₁ , Иe₂ ⟩ = (⊢⇒lc-ty-tm (B-Иe₂ (fresh B-Иe₁) {fresh-correct B-Иe₁})) j
    in И⟨ Иe₁ ++ A-Иe₁ , (λ a {a∉} → cong ƛ_
      (open-rec-lc-lemma-ty-tm-tm-tm (Иe₂ a {proj₁ (∉-++ a∉)}))) ⟩
  ⊢⇒lc-ty-tm {Γ} (⊢· ⊢L ⊢M) j = И⟨ [] , (λ a → cong₂ _·_
    (lemma2·7-2 ⦃ LnsTyTm ⦄ it (⊢⇒lc-ty-tm ⊢L))
    (lemma2·7-2 ⦃ LnsTyTm ⦄ it (⊢⇒lc-ty-tm ⊢M))) ⟩
  ⊢⇒lc-ty-tm {Γ} {Λ L} (⊢Λ И⟨ Иe₁ , Иe₂ ⟩) j =
      let induc-hypo = ⊢⇒lc-ty-tm (Иe₂ (fresh Иe₁) {fresh-correct Иe₁})
          sj≻ty-tm[]L = lemma2·6 ⦃ LnsTyTm ⦄ z≤n induc-hypo
          И⟨ sj≻L-Иe₁ , sj≻L-Иe₂ ⟩ = (helper L z≤n sj≻ty-tm[]L) (suc j) ⦃ s≤s it ⦄
      in И⟨ sj≻L-Иe₁ , (λ a {a∉} → cong Λ_ (sj≻L-Иe₂ a {a∉})) ⟩
    where
      helper : ∀ {i x j} (L : Term) → j ≥ i → (suc i) ≻ty-tm (ty-tm[ i —→ x ] L) → (suc j) ≻ty-tm L
      helper L j≥i si≻[]L k =
        let И⟨ Иe₁ , Иe₂ ⟩ = si≻[]L k ⦃ ≤-trans (s≤s j≥i) it ⦄
            k≢i = sym-≢ (<⇒≢ (≤-trans (s≤s j≥i) it))
        in И⟨ Иe₁ , (λ a {a∉} → open-rec-lc-lemma-ty-tm k≢i (Иe₂ a {a∉})) ⟩
  ⊢⇒lc-ty-tm {Γ} (⊢[] lc _ ⊢L) j = И⟨ [] , (λ a → cong₂ _[_]
    (lemma2·7-2 ⦃ LnsTyTm ⦄ it (⊢⇒lc-ty-tm ⊢L))
    (lemma2·7-2 ⦃ LnsType ⦄ it lc)) ⟩
  ⊢⇒lc-ty-tm (⊢zero _) j = И⟨ [] , (λ _ → refl) ⟩
  ⊢⇒lc-ty-tm {Γ} (⊢suc ⊢L) j = И⟨ [] , (λ a →
    cong ‵suc_ (lemma2·7-2 ⦃ LnsTyTm ⦄ it (⊢⇒lc-ty-tm ⊢L))) ⟩
\end{code}

\begin{code}
  subst-open-ty-tm : ∀ {N : Term} {x y : List Char} {i j : ℕ}
    (L : Term)
    → x ≢ y
    → j ≥ i
    → (i ≻ty-tm N)
    → tm-tm[ x := N ] (ty-tm[ j —→ y ] L)
      ≡ ty-tm[ j —→ y ] (tm-tm[ x := N ] L)
  subst-open-ty-tm {N} {x} {y} {i} {j} (fr z) x≢y j≥i i≻N with x ≟lchar z
  ... | yes refl =
    begin
      N
    ≡⟨ sym (lemma2·7-2 ⦃ LnsTyTm ⦄ ≤-refl (lemma2·6 ⦃ LnsTyTm ⦄ j≥i i≻N)) ⟩
      ty-tm[ j —→ y ] N
    ∎
  ... | no  x≢z  = refl
  subst-open-ty-tm (# k) x≢y j≥i i≻N = refl
  subst-open-ty-tm (ƛ L) x≢y j≥i i≻N
    rewrite subst-open-ty-tm L x≢y j≥i i≻N = refl
  subst-open-ty-tm (L · M) x≢y j≥i i≻N
    rewrite subst-open-ty-tm L x≢y j≥i i≻N
    | subst-open-ty-tm M x≢y j≥i i≻N
    = refl
  subst-open-ty-tm (Λ L) x≢y j≥i i≻N
    rewrite subst-open-ty-tm L x≢y (m≤n⇒m≤1+n j≥i) i≻N = refl
  subst-open-ty-tm (L [ A ]) x≢y j≥i i≻N
    rewrite subst-open-ty-tm L x≢y j≥i i≻N = refl
  subst-open-ty-tm ‵zero x≢y j≥i i≻N = refl
  subst-open-ty-tm (‵suc L) x≢y j≥i i≻N rewrite
    subst-open-ty-tm L x≢y j≥i i≻N = refl

  subst-open-ty-tm-ctx : ∀ {Γ A} {N : Term} {x y : List Char} {i j : ℕ}
    (L : Term)
    → x ≢ y
    → j ≥ i
    → (i ≻ty-tm N)
    → Γ ⊢ tm-tm[ x := N ] (ty-tm[ j —→ y ] L) ⦂ A
    → Γ ⊢ ty-tm[ j —→ y ] (tm-tm[ x := N ] L) ⦂ A
  subst-open-ty-tm-ctx L x≢y j≥i i≻N assump = ≡-with-⊢-tm assump (subst-open-ty-tm L x≢y j≥i i≻N)

  subst-open-ty-tm-tm-tm : ∀ {C : Type} {x y : List Char} {i j : ℕ}
    (L : Term)
    → j ≥ i
    → (i ≻ty C)
    → ty-tm[ x := C ] (tm-tm[ j —→ y ] L)
      ≡ tm-tm[ j —→ y ] (ty-tm[ x := C ] L)
  subst-open-ty-tm-tm-tm (fr y) j≥i i≻C = refl
  subst-open-ty-tm-tm-tm {j = j} (# k) j≥i i≻C with j ≟ℕ k
  ... | yes refl = refl
  ... | no  j≢k  = refl
  subst-open-ty-tm-tm-tm (ƛ L) j≥i i≻C = cong ƛ_
    (subst-open-ty-tm-tm-tm L (m≤n⇒m≤1+n j≥i) i≻C)
  subst-open-ty-tm-tm-tm (L · M) j≥i i≻C = cong₂ _·_
    (subst-open-ty-tm-tm-tm L j≥i i≻C)
    (subst-open-ty-tm-tm-tm M j≥i i≻C)
  subst-open-ty-tm-tm-tm (Λ L) j≥i i≻C = cong Λ_
    (subst-open-ty-tm-tm-tm L j≥i i≻C)
  subst-open-ty-tm-tm-tm (L [ A ]) j≥i i≻C = cong₂ _[_]
    (subst-open-ty-tm-tm-tm L j≥i i≻C)
    refl
  subst-open-ty-tm-tm-tm ‵zero j≥i i≻C = refl
  subst-open-ty-tm-tm-tm (‵suc L) j≥i i≻C = cong ‵suc_
    (subst-open-ty-tm-tm-tm L j≥i i≻C)

  subst-open-ty-tm-tm-tm-ctx : ∀ {Γ A} {C : Type} {x y : List Char} {i j : ℕ}
    (L : Term)
    → j ≥ i
    → (i ≻ty C)
    → Γ ⊢ ty-tm[ x := C ] (tm-tm[ j —→ y ] L) ⦂ A
    → Γ ⊢ tm-tm[ j —→ y ] (ty-tm[ x := C ] L) ⦂ A
  subst-open-ty-tm-tm-tm-ctx L j≥i i≻C assump = ≡-with-⊢-tm assump (subst-open-ty-tm-tm-tm L j≥i i≻C)

  subst-open-ty-ty-ty-ty : ∀ {C : Type} {x y : List Char} {i j : ℕ}
    (A : Type)
    → x ≢ y
    → j ≥ i
    → (i ≻ty C)
    → ty-ty[ x := C ] (ty-ty[ j —→ y ] A)
      ≡ ty-ty[ j —→ y ] (ty-ty[ x := C ] A)
  subst-open-ty-ty-ty-ty ‵ℕ x≢y j≥i i≻C = refl
  subst-open-ty-ty-ty-ty {x = x} {j = j} (t-fr z) x≢y j≥i i≻C with x ≟lchar z
  ... | yes refl = sym (lemma2·7-2 ⦃ LnsType ⦄ j≥i i≻C)
  ... | no  x≢z  = refl
  subst-open-ty-ty-ty-ty {x = x} {y = y} {j = j} (t-# k) x≢y j≥i i≻C with j ≟ℕ k
  ... | no  j≢k  = refl
  ... | yes refl with x ≟lchar y
  ...   | yes refl = contradiction refl x≢y
  ...   | no  _    = refl
  subst-open-ty-ty-ty-ty (A ⇒ B) x≢y j≥i i≻C rewrite
      subst-open-ty-ty-ty-ty A x≢y j≥i i≻C
    | subst-open-ty-ty-ty-ty B x≢y j≥i i≻C
    = refl
  subst-open-ty-ty-ty-ty {j = j} (t-∀ A) x≢y j≥i i≻C = cong t-∀_
    (subst-open-ty-ty-ty-ty {j = suc j} A x≢y (m≤n⇒m≤1+n j≥i) i≻C)

  subst-open-ty-ty-ty-ty-ctx : ∀ {Γ L} {C A : Type} {x y : List Char} {i j : ℕ}
    → x ≢ y
    → j ≥ i
    → (i ≻ty C)
    → Γ ⊢ L ⦂ ty-ty[ x := C ] (ty-ty[ j —→ y ] A)
    → Γ ⊢ L ⦂ ty-ty[ j —→ y ] (ty-ty[ x := C ] A)
  subst-open-ty-ty-ty-ty-ctx {A = A} x≢y j≥i i≻C assump = ≡-with-⊢-ty assump (subst-open-ty-ty-ty-ty A x≢y j≥i i≻C)

  subst-open-ty-tm-ty-tm : ∀ {C : Type} {x y : List Char} {i j : ℕ}
    (L : Term)
    → x ≢ y
    → j ≥ i
    → (i ≻ty C)
    → ty-tm[ x := C ] (ty-tm[ j —→ y ] L)
      ≡ ty-tm[ j —→ y ] (ty-tm[ x := C ] L)
  subst-open-ty-tm-ty-tm (fr z) x≢y j≥i i≻C = refl
  subst-open-ty-tm-ty-tm (# k) x≢y j≥i i≻C = refl
  subst-open-ty-tm-ty-tm (ƛ L) x≢y j≥i i≻C rewrite
    subst-open-ty-tm-ty-tm L x≢y j≥i i≻C = refl
  subst-open-ty-tm-ty-tm (L · M) x≢y j≥i i≻C rewrite
      subst-open-ty-tm-ty-tm L x≢y j≥i i≻C
    | subst-open-ty-tm-ty-tm M x≢y j≥i i≻C
    = refl
  subst-open-ty-tm-ty-tm {j = j} (Λ L) x≢y j≥i i≻C = cong Λ_
    (subst-open-ty-tm-ty-tm {j = suc j} L x≢y (m≤n⇒m≤1+n j≥i) i≻C)
  subst-open-ty-tm-ty-tm (L [ A ]) x≢y j≥i i≻C = cong₂ _[_]
    (subst-open-ty-tm-ty-tm L x≢y j≥i i≻C)
    (subst-open-ty-ty-ty-ty A x≢y j≥i i≻C)
  subst-open-ty-tm-ty-tm ‵zero x≢y j≥i i≻C = refl
  subst-open-ty-tm-ty-tm (‵suc L) x≢y j≥i i≻C rewrite
    (subst-open-ty-tm-ty-tm L x≢y j≥i i≻C) = refl

  subst-open-ty-tm-ty-tm-ctx : ∀ {Γ L A} {C : Type} {x y : List Char} {i j : ℕ}
    → x ≢ y
    → j ≥ i
    → (i ≻ty C)
    → Γ ⊢ ty-tm[ x := C ] (ty-tm[ j —→ y ] L) ⦂ A
    → Γ ⊢ ty-tm[ j —→ y ] (ty-tm[ x := C ] L) ⦂ A
  subst-open-ty-tm-ty-tm-ctx {L = L} x≢y j≥i i≻C assump = ≡-with-⊢-tm assump (subst-open-ty-tm-ty-tm L x≢y j≥i i≻C)

  extract-subst : ∀ {X C A i j}
    (B : Type)
    → j ≥ i
    → i ≻ty C
    → ty-ty[ j :→ ty-ty[ X := C ] A ] (ty-ty[ X := C ] B)
        ≡ ty-ty[ X := C ] (ty-ty[ j :→ A ] B)
  extract-subst ‵ℕ j≥i i≻C = refl
  extract-subst {X} {C} (t-fr Y) j≥i i≻C with X ≟lchar Y
  ... | yes refl = ≻⇒:→-idempotent C j≥i i≻C
  ... | no  X≢Y  = refl
  extract-subst {i = i} {j = j} (t-# n) j≥i i≻C with j ≟ℕ n
  ... | yes refl = refl
  ... | no  j≢n  = refl
  extract-subst (B ⇒ B') j≥i i≻C = cong₂ _⇒_
    (extract-subst B j≥i i≻C) (extract-subst B' j≥i i≻C)
  extract-subst (t-∀ B) j≥i i≻C = cong t-∀_ (extract-subst B (m≤n⇒m≤1+n j≥i) i≻C)

  extract-subst-ctx : ∀ {Γ L X C B A i j}
    → j ≥ i
    → i ≻ty C
    → Γ ⊢ L ⦂ ty-ty[ j :→ ty-ty[ X := C ] A ] (ty-ty[ X := C ] B)
    → Γ ⊢ L ⦂ ty-ty[ X := C ] (ty-ty[ j :→ A ] B)
  extract-subst-ctx {B = B} j≥i i≻C assump = ≡-with-⊢-ty assump (extract-subst B j≥i i≻C)

  map-subst-fresh-id : ∀ {Γ X A} → Ok Γ → X ∉ domain-ftv Γ → Γ ≡ map (ty-ty[ X := A ]_) Γ
  map-subst-fresh-id ok-∅ X∉ = refl
  map-subst-fresh-id (ok-∷fv {A = B} {x = x} okΓ lc-B B⊆Γ) X∉ =
    cong₂ (_, x ⦂_)
    (map-subst-fresh-id okΓ X∉)
    (sym (:=-∉-idempotent λ X∈ftv-B → contradiction (x∈xs∧xs⊆ys⇒x∈ys X∈ftv-B B⊆Γ) X∉))
  map-subst-fresh-id (ok-∷ftv {X = Y} okΓ x) X∉ = cong (_, Y) (map-subst-fresh-id okΓ (∉y∷ys⇒∉ys X∉))

  subst-ty : ∀ {Γ Δ X L B C}
    → Ty-LocallyClosed C
    -- → X ∉ domain-all-ftv Γ
    → ftv-ty C ⊆ domain-ftv Δ
    → Ok (Γ + (map (ty-ty[ X := C ]_) Δ))
    → ((Γ , X) + Δ) ⊢ L ⦂ B
      --------------------
    → (Γ + (map (ty-ty[ X := C ]_) Δ)) ⊢ ty-tm[ X := C ] L ⦂ ty-ty[ X := C ] B
  subst-ty {Γ} {Δ} {X = Y} {C = C} lc-C C⊆Δ okΓ+map (⊢free okΓ+Δ lc-B ∋x)
    with ok-+ {Γ = Γ , Y} okΓ+Δ
  ... | ok-∷ftv okΓ Y∉Γ = ⊢free okΓ+map (:=-≻ z≤n lc-B lc-C) (∋-map-ftv Δ okΓ+Δ ∋x)
  subst-ty {Γ} {Δ} {X = Y} {L = ƛ L} {C = C} lc-C C⊆Δ okΓ+map (⊢ƛ {A = A} lc-A И⟨ Иe₁ , Иe₂ ⟩) =
    ⊢ƛ (:=-≻ z≤n lc-A lc-C) И⟨ Иe₁ , (λ a {a∉} →
      let ⊢tm-tm[0→]L = Иe₂ a {a∉}
          ok = ⊢⇒Ok ⊢tm-tm[0→]L
      in subst-open-ty-tm-tm-tm-ctx L z≤n lc-C
        (subst-ty
          lc-C
          C⊆Δ
          (ok-∷fv
            okΓ+map
            (:=-≻ z≤n lc-A lc-C)
            (ftv⊆dom-:= {A = A} Γ Δ lc-C C⊆Δ okΓ+map (extract-⊆ ok)))
          ⊢tm-tm[0→]L)) ⟩
  subst-ty {X = Y} lc-C C⊆Δ okΓ+map (⊢· ⊢L ⊢M) = ⊢· (subst-ty lc-C C⊆Δ okΓ+map ⊢L) (subst-ty lc-C C⊆Δ okΓ+map ⊢M)
  subst-ty {Γ} {Δ} {X = Y} {B = t-∀ B} {C = C} lc-C C⊆Δ okΓ+map (⊢Λ И⟨ Иe₁ , Иe₂ ⟩) =
    ⊢Λ И⟨ (Y ∷ Иe₁) ++ domain-all-ftv (Γ + map (ty-ty[_:=_]_ Y C) Δ) , (λ a {a∉} →
      let ⟨ a∉Y∷Иe₁ , a∉dom-all-ftv ⟩ = ∉-++ a∉
          ⟨ a∉Y∷[] , a∉Иe₁ ⟩ = ∉-++ a∉Y∷Иe₁
          Y≢a = sym-≢ (∉∷[]⇒≢ a∉Y∷[])
          hypo = subst-ty {Δ = Δ , a} lc-C (⊆⇒⊆∷ C⊆Δ) (ok-∷ftv okΓ+map a∉dom-all-ftv) (Иe₂ a {a∉Иe₁})
      in subst-open-ty-tm-ty-tm-ctx Y≢a z≤n lc-C
        (subst-open-ty-ty-ty-ty-ctx {A = B} {j = 0} Y≢a z≤n lc-C hypo) ) ⟩
  subst-ty {Γ} {Δ} {X = Y} {L = L [ A ]} {B = D} lc-C C⊆Δ okΓ+map (⊢[] {B = B} lc-A A⊆ ⊢L) =
    extract-subst-ctx {X = Y} {B = B} {A = A} {j = 0} z≤n lc-C
      (⊢[] (:=-≻ {x = Y} z≤n lc-A lc-C) (ftv⊆dom-:= {A = A} Γ Δ lc-C C⊆Δ okΓ+map A⊆) (subst-ty lc-C C⊆Δ okΓ+map ⊢L))
  subst-ty {X = Y} lc-C C⊆Δ okΓ+map (⊢zero okΓ+Δ) = ⊢zero okΓ+map
  subst-ty {X = Y} lc-C C⊆Δ okΓ+map (⊢suc ⊢L) = ⊢suc (subst-ty lc-C C⊆Δ okΓ+map ⊢L)

--   {-# TERMINATING #-}
--   subst-ty : ∀ {Γ X L A B}
--     → Ty-LocallyClosed A
--     -- → X ∉ domain-all-ftv Γ
--     → (Γ , X) ⊢ L ⦂ B
--       --------------------
--     → Γ ⊢ ty-tm[ X := A ] L ⦂ ty-ty[ X := A ] B
-- --   subst-ty {X = Y} ⊢L = {!!}
--   subst-ty {X = Y} {B = ‵ℕ} lc-A (⊢free (ok-∷ftv okΓ X∉Γ) lc-B (T⦂ ∋x)) = ⊢free okΓ n≻‵ℕ ∋x
--   subst-ty {X = Y} {B = t-fr X} lc-A (⊢free (ok-∷ftv okΓ X∉Γ) lc-B (T⦂ ∋x)) with Y ≟lchar X
--   ... | yes refl = contradiction (∋⦂⇒∈ ∋x) X∉Γ
--   ... | no  X≢Y  = ⊢free okΓ lc-B ∋x
--   subst-ty {X = Y} {B = t-# x} lc-A (⊢free (ok-∷ftv okΓ X∉Γ) lc-B ∋x) = contradiction lc-B #-never-lc
--   subst-ty {X = Y} {A = A} {B = B ⇒ C} lc-A (⊢free (ok-∷ftv okΓ X∉Γ) lc-B⇒C (T⦂ {x = x} ∋x)) =
--     ⊢free okΓ (:=⇒-≻ B C z≤n lc-B⇒C lc-A) (⇒-:= B C (≡-with-∋-ty ∋x (sym (:=-∉-idempotent (∉-domain-all-∋ X∉Γ ∋x)))))
--   subst-ty {X = Y} {B = t-∀ B} lc-A (⊢free (ok-∷ftv okΓ X∉Γ) lc-B (T⦂ ∋x)) =
--     ⊢free okΓ (:=∀-≻ B z≤n lc-B lc-A) (∀-:= B (≡-with-∋-ty ∋x (sym (:=-∉-idempotent (∉-domain-all-∋ X∉Γ ∋x)))))
-- --   subst-ty {X = Y} {B = B} lc-A (⊢free okΓ,Y lc-B ∋x) = ⊢free {!!} (:=-≻ B z≤n lc-B lc-A) {!!}
--   subst-ty {X = Y} {A = C} lc-C (⊢ƛ lc-A И⟨ Иe₁ , Иe₂ ⟩) = ⊢ƛ (:=-≻ z≤n lc-A lc-C)
--     И⟨ Иe₁ , (λ a {a∉} → {!subst-open-ty-tm-tm-tm-ctx ? ? ? (subst-ty lc-A (Иe₂ a {a∉}))!}) ⟩
--   subst-ty lc-A (⊢· ⊢L ⊢M) = ⊢· (subst-ty lc-A ⊢L) (subst-ty lc-A ⊢M)
--   subst-ty {X = X} {B = t-∀ B} lc-A (⊢Λ И⟨ Иe₁ , Иe₂ ⟩) = ⊢Λ И⟨ X ∷ Иe₁ , (λ a {a∉} →
--     let ⟨ a∉X∷[] , a∉Иe₁ ⟩ = ∉-++ a∉
--         X≢a = sym-≢ (∉∷[]⇒≢ a∉X∷[])
--         hypo = subst-ty lc-A (swap-ty (Иe₂ a {a∉Иe₁}))
--     in subst-open-ty-tm-ty-tm-ctx X≢a z≤n lc-A (subst-open-ty-ty-ty-ty-ctx {A = B} {j = 0} X≢a z≤n lc-A hypo)) ⟩
--   subst-ty {X = X} {L = L [ A ]} {A = C} lc-C (⊢[] {B = B} lc-A ⊢L) =
--     extract-subst-ctx {X = X} {B = B} {A = A} {j = 0} z≤n lc-C (⊢[] (:=-≻ {x = X} z≤n lc-A lc-C) (subst-ty lc-C ⊢L))
--   subst-ty lc-A (⊢zero (ok-∷ftv okΓ _)) = ⊢zero okΓ
--   subst-ty lc-A (⊢suc ⊢L) = ⊢suc (subst-ty lc-A ⊢L)

  {-# TERMINATING #-}
  subst : ∀ {Γ x L N A B}
    → ∅ ⊢ N ⦂ A
    → Γ , x ⦂ A ⊢ L ⦂ B
      --------------------
    → Γ ⊢ tm-tm[ x := N ] L ⦂ B
  subst {x = y} ⊢N (⊢free (ok-∷fv okΓ _ A⊆Γ) lc-A (H refl)) with y ≟lchar y
  ... | yes refl = weaken okΓ ⊢N
  ... | no  y≢y  = contradiction refl y≢y
  subst {x = y} ⊢N (⊢free {x = x} (ok-∷fv okΓ _ A⊆Γ) lc-A (T x≢y ∋x)) with y ≟lchar x
  ... | yes refl = contradiction refl x≢y
  ... | no  _    = ⊢free okΓ lc-A ∋x
  subst {x = y} {L = ƛ L} ⊢N (⊢ƛ lc-A И⟨ Иe₁ , Иe₂ ⟩) =
    ⊢ƛ lc-A И⟨ (y ∷ Иe₁) , (λ a {a∉} →
      let a∉Иe₁ = proj₂ (∉-++ {xs = y ∷ []} a∉)
          y≢a = sym-≢ (∉y∷ys⇒≢y a∉)
          ⊢tm-tm[]L = swap (sym-≢ y≢a) (Иe₂ a {a∉Иe₁})
      in subst-open-var-ctx L y≢a (⊢⇒lc ⊢N) (subst ⊢N ⊢tm-tm[]L)) ⟩
  subst ⊢N (⊢· ⊢L ⊢M) = ⊢· (subst ⊢N ⊢L) (subst ⊢N ⊢M)
  subst {x = y} {L = Λ L} ⊢N (⊢Λ И⟨ Иe₁ , Иe₂ ⟩) =
    ⊢Λ И⟨ y ∷ Иe₁ , (λ A {A∉} →
      let A∉Иe₁ = proj₂ (∉-++ {xs = y ∷ []} A∉)
          y≢A = sym-≢ (∉y∷ys⇒≢y A∉)
          ⊢ty-tm[]L = Иe₂ A {A∉Иe₁}
          induc-hypo = subst ⊢N (swap-tm-ty ⊢ty-tm[]L)
      in subst-open-ty-tm-ctx L y≢A z≤n (⊢⇒lc-ty-tm ⊢N) induc-hypo ) ⟩
  subst ⊢N (⊢[] lc ⊆[] ⊢L) = ⊢[] lc ⊆[] (subst ⊢N ⊢L)
  subst ⊢N (⊢zero (ok-∷fv okΓ _ _)) = ⊢zero okΓ
  subst ⊢N (⊢suc ⊢L) = ⊢suc (subst ⊢N ⊢L)

  subst-intro : ∀ {x : List Char} {i : ℕ} (L N : Term)
    → x ∉ fv-tm L
    → tm-tm[ i :→ N ] L ≡ tm-tm[ x := N ] (tm-tm[ i —→ x ] L)
  subst-intro {x} (fr y) N x∉fv-L with x ≟lchar y
  ... | yes refl = contradiction refl (∉∷[]⇒≢ x∉fv-L)
  ... | no  x≢y  = refl
  subst-intro {x} {i} (# k) N x∉fv-L with i ≟ℕ k
  ... | no  i≢k = refl
  ... | yes refl with x ≟lchar x
  ...   | yes refl = refl
  ...   | no  x≢x  = contradiction refl x≢x
  subst-intro {x} {i} (ƛ L) N x∉fv-L
    rewrite subst-intro {x} {suc i} L N x∉fv-L = refl
  subst-intro (L · M) N x∉ = let ⟨ x∉fv-L , x∉fv-M ⟩ = ∉-++ {xs = fv-tm L} x∉
    in cong₂ _·_ (subst-intro L N x∉fv-L) (subst-intro M N x∉fv-M)
  subst-intro (Λ L) N x∉fv-L = cong Λ_ (subst-intro L N x∉fv-L)
  subst-intro (L [ A ]) N x∉fv-L = cong₂ _[_] (subst-intro L N x∉fv-L) refl
  subst-intro ‵zero N x∉fv-L = refl
  subst-intro (‵suc L) N x∉fv-L = cong ‵suc_ (subst-intro L N x∉fv-L)

  subst-intro-ty-ty : ∀ {x i B} (A : Type)
    → x ∉ ftv-ty A
    → ty-ty[ i :→ B ] A ≡ ty-ty[ x := B ] (ty-ty[ i —→ x ] A)
  subst-intro-ty-ty ‵ℕ x∉ = refl
  subst-intro-ty-ty {x} (t-fr y) x∉ with x ≟lchar y
  ... | yes refl = contradiction refl (∉∷[]⇒≢ x∉)
  ... | no  x≢y  = refl
  subst-intro-ty-ty {x} {i} (t-# k) x∉ with i ≟ℕ k
  ... | no  i≢k  = refl
  ... | yes refl with x ≟lchar x
  ...   | yes refl = refl
  ...   | no  x≢x  = contradiction refl x≢x
  subst-intro-ty-ty (A ⇒ B) x∉ = let ⟨ x∉A , x∉B ⟩ = ∉-++ x∉
    in cong₂ _⇒_ (subst-intro-ty-ty A x∉A) (subst-intro-ty-ty B x∉B)
  subst-intro-ty-ty {i = i} (t-∀ A) x∉ = cong t-∀_ (subst-intro-ty-ty {i = suc i} A x∉)

  subst-intro-ty-tm : ∀ {x i B} (L : Term)
    → x ∉ ftv-tm L
    → ty-tm[ i :→ B ] L ≡ ty-tm[ x := B ] (ty-tm[ i —→ x ] L)
  subst-intro-ty-tm (fr x) x∉ = refl
  subst-intro-ty-tm (# k) x∉ = refl
  subst-intro-ty-tm (ƛ L) x∉ = cong ƛ_ (subst-intro-ty-tm L x∉)
  subst-intro-ty-tm (L · M) x∉ = let ⟨ x∉L , x∉M ⟩ = ∉-++ x∉
    in cong₂ _·_ (subst-intro-ty-tm L x∉L) (subst-intro-ty-tm M x∉M)
  subst-intro-ty-tm (Λ L) x∉ = cong Λ_ (subst-intro-ty-tm L x∉)
  subst-intro-ty-tm (L [ A ]) x∉ = let ⟨ x∉L , x∉A ⟩ = ∉-++ x∉
    in cong₂ _[_] (subst-intro-ty-tm L x∉L) (subst-intro-ty-ty A x∉A )
  subst-intro-ty-tm ‵zero x∉ = refl
  subst-intro-ty-tm (‵suc L) x∉ = cong ‵suc_ (subst-intro-ty-tm L x∉)

  subst-op : ∀ {Γ L N A B}
    → ∅ ⊢ N ⦂ A
    → Γ ⊢ ƛ L ⦂ A ⇒ B
      --------------------
    → Γ ⊢ tm-tm[ 0 :→ N ] L ⦂ B
  subst-op {Γ} {L} {N} ⊢N (⊢ƛ lc-A И⟨ Иe₁ , Иe₂ ⟩) =
    let x                  = fresh (fv-tm L ++ Иe₁)
        ⟨ x∉fv-L , x∉Иe₁ ⟩ = ∉-++ {xs = fv-tm L}
                                (fresh-correct (fv-tm L ++ Иe₁))
    in ≡-with-⊢-tm
      (subst ⊢N (Иe₂ x {x∉Иe₁}))
      (sym (subst-intro L N (x∉fv-L)))

  subst-op-ty : ∀ {L B C}
    → Ty-LocallyClosed C
    → ftv-ty C ⊆ []
    → ∅ ⊢ Λ L ⦂ t-∀ B
      --------------------
    → ∅ ⊢ ty-tm[ 0 :→ C ] L ⦂ ty-ty[ 0 :→ C ] B
  subst-op-ty {L} {B} {C} lc-C C⊆[] (⊢Λ И⟨ Иe₁ , Иe₂ ⟩) =
    let x = fresh (fv-tm L ++ ftv-ty B ++ ftv-tm L ++ Иe₁)
        ⟨ x∉fv-L , x∉ ⟩ = ∉-++ {xs = fv-tm L}
          (fresh-correct (fv-tm L ++ ftv-ty B ++ ftv-tm L ++ Иe₁))
        ⟨ x∉ftv-B , x∉ ⟩ = ∉-++ {xs = ftv-ty B} x∉
        ⟨ x∉ftv-L , x∉Иe₁ ⟩ = ∉-++ {xs = ftv-tm L} x∉
    in ≡-with-⊢-ty
      (≡-with-⊢-tm
        (subst-ty {Δ = ∅} lc-C C⊆[] ok-∅ (Иe₂ x {x∉Иe₁}))
        (sym (subst-intro-ty-tm L x∉ftv-L)))
      (sym (subst-intro-ty-ty B x∉ftv-B))
\end{code}

\section{Evaluation}
\begin{code}
  data Value : Term → Set where
    V-ƛ : ∀ {L} → Value (ƛ L)
    V-Λ : ∀ {L} → Value (Λ L)
    V-zero : Value ‵zero
    V-suc : ∀ {L} → Value L → Value (‵suc L)
\end{code}


\begin{code}
  infix 4 _—→_
  data _—→_ : Term → Term → Set where
    ξ₁ : ∀ {L L' M}
      → L —→ L'
      -- → Tm-LocallyClosed M
        -------------------
      → L · M —→ L' · M

    ξ₂ : ∀ {L M M'}
      → M —→ M'
        ---------
      → L · M —→ L · M'

    ξ-[] : ∀ {L L' A}
      → L —→ L'
        ------------------
      → L [ A ] —→ L' [ A ]

    ξ-suc : ∀ {L L'}
      → L —→ L'
        ------------------
      → ‵suc L —→ ‵suc L'

    β-ƛ : ∀ {L M}
      → 1 ≻tm L
      → Value M
        ---------
      → (ƛ L) · M —→ tm-tm[ 0 :→ M ] L

    β-Λ : ∀ {L A}
      → 1 ≻ty-tm L
      → Ty-LocallyClosed A
      → ftv-ty A ⊆ []
        --------------------------------
      → (Λ L) [ A ] —→ ty-tm[ 0 :→ A ] L
\end{code}

\begin{code}
  infix  2 _—↠_
  infix  1 begin'_
  infixr 2 _—→⟨_⟩_
  infix  3 _∎'
  data _—↠_ : Term → Term → Set where
    _∎' : ∀ M
        ---------
      → M —↠ M

    step—→ : ∀ L {M N}
      → M —↠ N
      → L —→ M
        ---------
      → L —↠ N

  pattern _—→⟨_⟩_ L L—→M M—↠N = step—→ L M—↠N L—→M

  begin'_ : ∀ {M N}
    → M —↠ N
      ------
    → M —↠ N
  begin' M—↠N = M—↠N
\end{code}

\begin{code}
  data Progress (L : Term) : Set where
    step : ∀ {L'}
      → L —→ L'
        ----------
      → Progress L

    done :
        Value L
        ----------
      → Progress L

  progress : ∀ {L A}
    → ∅ ⊢ L ⦂ A
      ----------
    → Progress L
  progress (⊢ƛ lc-A cof) = done V-ƛ
  progress (⊢· ⊢L ⊢M) with progress ⊢L
  ... | step L→L' = step (ξ₁ L→L')
  ... | done V-ƛ with progress ⊢M
  ...   | step M→M' = step (ξ₂ M→M')
  ...   | done val  = step (β-ƛ (≻ƛ⇒s≻ƛ (⊢⇒lc ⊢L)) val)
  progress (⊢Λ x) = done V-Λ
  progress (⊢[] lc-A A⊆[] ⊢L) with progress ⊢L
  ... | step L→L' = step (ξ-[] L→L')
  ... | done V-Λ = step (β-Λ (≻Λ⇒s≻Λ (⊢⇒lc-ty-tm ⊢L)) lc-A A⊆[])
  progress (⊢zero ok∅) = done V-zero
  progress (⊢suc ⊢L) with progress ⊢L
  ... | step L→L'  = step (ξ-suc L→L')
  ... | done val-L = done (V-suc val-L)
\end{code}

\begin{code}

  preserve : ∀ {L L' A}
    → ∅ ⊢ L ⦂ A
    → L —→ L'
      ----------
    → ∅ ⊢ L' ⦂ A
  preserve (⊢· ⊢L ⊢M) (ξ₁ L→L') = ⊢· (preserve ⊢L L→L') ⊢M
  preserve (⊢· ⊢L ⊢M) (ξ₂ M→M') = ⊢· ⊢L (preserve ⊢M M→M')
  preserve (⊢· ⊢L ⊢M) (β-ƛ 1≻L val-M) = subst-op ⊢M ⊢L
  preserve (⊢[] lc-A A⊆[] ⊢L) (ξ-[] L→L') = ⊢[] lc-A A⊆[] (preserve ⊢L L→L')
  preserve (⊢[] lc-A A⊆[] ⊢L) (β-Λ 1≻L _ C⊆[]) = subst-op-ty lc-A C⊆[] ⊢L
  preserve (⊢suc ⊢L) (ξ-suc L→L') = ⊢suc (preserve ⊢L L→L')

  -- preserve : ∀ {t t' A}
  --   → ∅ ⊢ t ⦂ A
  --   → t —→ t'
  --     ----------
  --   → ∅ ⊢ t' ⦂ A
  -- preserve (⊢· ⊢t₁ ⊢t₂) (ξ₁ t→t' _) = ⊢· (preserve ⊢t₁ t→t') ⊢t₂
  -- preserve (⊢· ⊢t₁ ⊢t₂) (ξ₂ t→t') = ⊢· ⊢t₁  (preserve ⊢t₂ t→t')
  -- preserve (⊢· ⊢t₁ ⊢t₂) (β x x₁) = subst-op ⊢t₂ ⊢t₁
  -- preserve (⊢suc ⊢t) (ξ-suc t→t') = ⊢suc (preserve ⊢t t→t')
\end{code}

\chapter{Conclusions}

\section{Final Reminder}

The body of your dissertation, before the references and any appendices,
\emph{must} finish by page~40. The introduction, after preliminary material,
should have started on page~1.

You may not change the dissertation format (e.g., reduce the font size, change
the margins, or reduce the line spacing from the default single spacing). Be
careful if you copy-paste packages into your document preamble from elsewhere.
Some \LaTeX{} packages, such as \texttt{fullpage} or \texttt{savetrees}, change
the margins of your document. Do not include them!

Over-length or incorrectly-formatted dissertations will not be accepted and you
would have to modify your dissertation and resubmit. You cannot assume we will
check your submission before the final deadline and if it requires resubmission
after the deadline to conform to the page and style requirements you will be
subject to the usual late penalties based on your final submission time.

\bibliographystyle{plainnat}
\bibliography{dissertation}


% You may delete everything from \appendix up to \end{document} if you don't need it.
\appendix

\chapter{Miscellaneous Proofs}
\label{appendix:misc_proofs}

Some miscellaneous proofs are required in the main chapters. Since they aren't of interest, they are presented here in the appendix.

These functions are closely adapted from PLFA [TODO: cite].

\begin{code}
module plfa_adaptions where
\end{code}
\begin{comment}
  \begin{code}
  -- Data types (naturals, strings, characters)
  open import Data.Nat using (ℕ; zero; suc; _<_; _≥_; _≤_; _≤?_; _<?_; z≤n; s≤s; _⊔_)
    renaming (_≟_ to _≟ℕ_)
  open import Data.Nat.Properties using (≤-refl; ≤-trans; ≤-<-trans; <-≤-trans; ≤-antisym; ≤-total;
    +-mono-≤; n≤1+n; m≤n⇒m≤1+n; suc-injective; <⇒≢; ≰⇒>; ≮⇒≥)
  open import Data.String using (String; fromList) renaming (_≟_ to _≟str_; _++_ to _++str_;
    length to str-length; toList to ⟪_⟫)
  open import Data.Char using (Char)
  open import Data.Char.Properties using () renaming (_≟_ to _≟char_)
  
  -- Function manipulation.
  open import Function using (_∘_; flip; it; id; case_returning_of_)
  
  -- Relations and predicates/decidability.
  import Relation.Binary.PropositionalEquality as Eq
  open Eq using (_≡_; _≢_; refl; sym; trans; cong; cong-app; cong₂)
  open Eq.≡-Reasoning using (begin_; step-≡-∣; step-≡-⟩; _∎)
  open import Relation.Binary.Definitions using (DecidableEquality)
  open import Relation.Nullary.Decidable using (Dec; yes; no; True; False; toWitnessFalse;
    toWitness; fromWitness; ¬?; ⌊_⌋; From-yes)
  open import Relation.Unary using (Decidable)
  open import Relation.Binary using () renaming (Decidable to BinaryDecidable)
  open import Relation.Nullary.Negation using (¬_; contradiction)
  open import Data.Empty using (⊥-elim)
  
  -- Products and exists quantifier.
  open import Data.Product using (_×_; proj₁; proj₂; ∃-syntax) renaming (_,_ to ⟨_,_⟩)
  
  -- Lists.
  open import Data.List using (List; []; _∷_; _++_; length; filter; map; foldr; head; replicate)
  open import Data.List.Properties using (≡-dec)
  import Data.List.Membership.DecPropositional as DecPropMembership
  open import Data.List.Relation.Unary.All using (All; all?; lookup)
    renaming (fromList to All-fromList; toList to All-toList)
  open import Data.List.Relation.Unary.Any using (Any; here; there)
  open import Data.List.Extrema Data.Nat.Properties.≤-totalOrder using (max; xs≤max)
  
  -- Import list membership using List Char comparisons.
  private
    _≟lchar_ : ∀ (xs ys : List Char) → Dec (xs ≡ ys)
    xs ≟lchar ys = ≡-dec (_≟char_) xs ys
  
  open DecPropMembership _≟lchar_ using (_∈_; _∉_; _∈?_)
  \end{code}
\end{comment}
\begin{code}
  All-++ : ∀ {A : Set} {P : A → Set} (xs ys : List A)
    → All P (xs ++ ys)
      ---------------------
    → (All P xs × All P ys)
  All-++ [] ys Pys = ⟨ All.[] , Pys ⟩
  All-++ (x ∷ xs) ys (Px All.∷ Pxs++ys) with All-++ xs ys Pxs++ys
  ... | ⟨ Pxs , Pys ⟩ = ⟨ Px All.∷ Pxs , Pys ⟩

  All¬⇒¬Any : ∀ {A : Set} {P : A → Set} {xs : List A}
    → All (¬_ ∘ P) xs
      ---------------
    → (¬_ ∘ Any P) xs
  All¬⇒¬Any {xs = x ∷ xs} (¬Px All.∷ All¬P) (here Px) = ¬Px Px
  All¬⇒¬Any {xs = x ∷ xs} (¬Px All.∷ All¬P) (there Pxs) =
    All¬⇒¬Any {xs = xs} All¬P Pxs

  ¬Any⇒All¬ : ∀ {A : Set} {P : A → Set} {xs : List A}
    → (¬_ ∘ Any P) xs
      ---------------
    → All (¬_ ∘ P) xs
  ¬Any⇒All¬ {xs = []} ¬∘AnyP = All.[]
  ¬Any⇒All¬ {xs = (x ∷ xs)} ¬AnyP =
    (λ Px → ¬AnyP (here Px))
      All.∷ ¬Any⇒All¬ {xs = xs} (λ Pxs → ¬AnyP (there Pxs))

  -- We have to use ¬Any⇒All¬ and its inverse because ∉ is an
  -- alias for "¬ Any (_≡ x) xs".  So we have to go through
  -- the process of converting to "All ¬" to apply the
  -- "All-++" lemma.
  ∉-++ : ∀ {s : List Char} {xs ys : List (List Char)}
    → s ∉ xs ++ ys
      --------------------
    → (s ∉ xs) × (s ∉ ys)
  ∉-++ {s} {xs} {ys} s∉xs++ys =
    let ⟨ all¬xs , all¬ys ⟩ = All-++ xs ys (¬Any⇒All¬ s∉xs++ys) in
      ⟨ All¬⇒¬Any {xs = xs} all¬xs
      , All¬⇒¬Any {xs = ys} all¬ys ⟩

  ++-All : ∀ {A : Set} {P : A → Set} (xs ys : List A)
    → All P xs × All P ys
      -------------------
    → All P (xs ++ ys)
  ++-All [] ys ⟨ All.[] , Pys ⟩ = Pys
  ++-All (x ∷ xs) ys ⟨ Px All.∷ Pxs , Pys ⟩ =
    Px All.∷ ++-All xs ys ⟨ Pxs , Pys ⟩

  ++-∉ : ∀ {s : List Char} {xs ys : List (List Char)}
    → s ∉ xs
    → s ∉ ys
      -------------
    → s ∉ xs ++ ys
  ++-∉ {_} {xs} {ys} s∉xs s∉ys = All¬⇒¬Any
    (++-All xs ys ⟨ (¬Any⇒All¬ s∉xs) , (¬Any⇒All¬ s∉ys) ⟩)

  ∉y∷ys⇒≢y : {x y : List Char} {ys : List (List Char)}
    → x ∉ y ∷ ys
    → x ≢ y
  ∉y∷ys⇒≢y x∉ with ¬Any⇒All¬ x∉
  ... | px All.∷ thing = px

  ∉∷[]⇒≢ : {x y : List Char}
    → x ∉ y ∷ []
      -----------
    → x ≢ y
  ∉∷[]⇒≢ = ∉y∷ys⇒≢y

  m+1≤n⇒m≤n : ∀ {m n : ℕ} → (suc m) ≤ n → m ≤ n
  m+1≤n⇒m≤n {zero} {suc n} sm≤n = z≤n
  m+1≤n⇒m≤n {suc m} {suc n} (s≤s sm≤n) = s≤s (m+1≤n⇒m≤n sm≤n)
\end{code}


\chapter{Prior work submitted for TSPL}
\label{appendix:tspl}
\section{Functions on lists and creating strings}
\label{appendix:list_functions}
\begin{code}
module tspl_prior_work where
  open import plfa_adaptions
  -- Import cofinite quantification.
  open import cofinite using (Cof-syntax; Иe₁; Иe₂; И⟨_,_⟩) 
\end{code}
\begin{comment}
  \begin{code}
  -- Data types (naturals, strings, characters)
  open import Data.Nat using (ℕ; zero; suc; _<_; _≥_; _≤_; _≤?_; _<?_; z≤n; s≤s; _⊔_)
    renaming (_≟_ to _≟ℕ_)
  open import Data.Nat.Properties using (≤-refl; ≤-trans; ≤-<-trans; <-≤-trans; ≤-antisym; ≤-total;
    +-mono-≤; n≤1+n; m≤n⇒m≤1+n; suc-injective; <⇒≢; ≰⇒>; ≮⇒≥)
  open import Data.String using (String; fromList) renaming (_≟_ to _≟str_; _++_ to _++str_;
    length to str-length; toList to ⟪_⟫)
  open import Data.Char using (Char)
  open import Data.Char.Properties using () renaming (_≟_ to _≟char_)
  
  -- Function manipulation.
  open import Function using (_∘_; flip; it; id; case_returning_of_)
  
  -- Relations and predicates/decidability.
  import Relation.Binary.PropositionalEquality as Eq
  open Eq using (_≡_; _≢_; refl; sym; trans; cong; cong-app; cong₂)
  open Eq.≡-Reasoning using (begin_; step-≡-∣; step-≡-⟩; _∎)
  open import Relation.Binary.Definitions using (DecidableEquality)
  open import Relation.Nullary.Decidable using (Dec; yes; no; True; False; toWitnessFalse;
    toWitness; fromWitness; ¬?; ⌊_⌋; From-yes)
  open import Relation.Unary using (Decidable)
  open import Relation.Binary using () renaming (Decidable to BinaryDecidable)
  open import Relation.Nullary.Negation using (¬_; contradiction)
  open import Data.Empty using (⊥-elim)
  
  -- Products and exists quantifier.
  open import Data.Product using (_×_; proj₁; proj₂; ∃-syntax) renaming (_,_ to ⟨_,_⟩)
  
  -- Lists.
  open import Data.List using (List; []; _∷_; _++_; length; filter; map; foldr; head; replicate)
  open import Data.List.Properties using (≡-dec)
  import Data.List.Membership.DecPropositional as DecPropMembership
  open import Data.List.Relation.Unary.All using (All; all?; lookup)
    renaming (fromList to All-fromList; toList to All-toList)
  open import Data.List.Relation.Unary.Any using (Any; here; there)
  open import Data.List.Extrema Data.Nat.Properties.≤-totalOrder using (max; xs≤max)
  
  -- Import list membership using List Char comparisons.
  private
    _≟lchar_ : ∀ (xs ys : List Char) → Dec (xs ≡ ys)
    xs ≟lchar ys = ≡-dec (_≟char_) xs ys
  
  open DecPropMembership _≟lchar_ using (_∈_; _∉_; _∈?_)

  \end{code}
  
  Include some infixes.
  
  \begin{code}
  infix  4  _∋_⦂_
  infix  4 _⊢_⦂_
  infixl 5 _,_⦂_
  
  infixr 7 _⇒_
  
  infix  5 ƛ_
  infixl 7 _·_
  infix  9 free_
  infix  9 bound_
  \end{code}
\end{comment}
\begin{code}
  All≤⇒<⇒All< : ∀ {n m : ℕ} (xs : List ℕ)
    → n < m
    → All (_≤ n) xs
      -------------
    → All (_< m) xs
  All≤⇒<⇒All< [] n<m All.[] = All.[]
  All≤⇒<⇒All< (x ∷ xs) n<m (x≤n All.∷ all≤) =
    ≤-<-trans x≤n n<m All.∷ All≤⇒<⇒All< xs n<m all≤

  sym-≢ : ∀ {A : Set} {x y : A}
    → x ≢ y
      -----
    → y ≢ x
  sym-≢ x≢y y≡x = x≢y (sym y≡x)

  ≡⇒<s : ∀ {n m : ℕ} → n ≡ m → n < suc m
  ≡⇒<s {zero} {m} n≡m = s≤s z≤n
  ≡⇒<s {suc n} {suc m} sn≡sm = s≤s (≡⇒<s (suc-injective sn≡sm))

  ∉⇒≢ : ∀ {xs : List (List Char)} {x y : List Char}
    → x ∈ xs
    → y ∉ xs
      -------
    → x ≢ y
  ∉⇒≢ {xs} x∈ y∉ refl = y∉ x∈

  len≠⇒≠ : ∀ {A : Set} (xs ys : List A)
    → length xs ≢ length ys → xs ≢ ys
  len≠⇒≠ xs ys len≢ =
    λ xs≡ys → contradiction (cong length xs≡ys) len≢

  len-replicate : ∀ {A : Set} (n : ℕ) (a : A)
    → length (replicate n a) ≡ n
  len-replicate zero a = refl
  len-replicate (suc n) a = cong suc (len-replicate n a)
\end{code}


We will use \texttt{List Char} rather than the built-in \texttt{String} type, since it's easier to
reason with lists. The \texttt{String} type is opaque in Agda, and reasoning about length and other
properties of \texttt{String} is more difficult than if we just used \texttt{List Char}. To generate
arbitrary strings, we'll provide functions for \texttt{List A} for some set \texttt{A} and then use
the specific where \texttt{A} is \texttt{Char}. The `toList` function for `String` (which I have
renamed to \texttt{⟪\_⟫}) can be used to convert from a \texttt{String}.

\begin{code}
  new-list : ∀ {A : Set} → A → List (List A) → List A
  new-list a xss = a ∷ replicate (max 0 (map length xss)) a

  fresh : List (List Char) → List Char
  fresh xss = new-list 'q' xss

  len-new-list : ∀ {A : Set} (a : A) (xss : List (List A))
    → max 0 (map length xss) < length (new-list a xss)
  len-new-list a xss =
    ≡⇒<s (sym (len-replicate (max 0 (map length xss)) a))
\end{code}

We can create a new string by finding the longest string and creating a string which is one
character longer. Now to prove that this function is correct:

\begin{code}
  new-list-correct :
    ∀ {A : Set} (xss : List (List A)) (a : A)
    → ¬ Any ((new-list a xss) ≡_) xss
  new-list-correct xss a = All¬⇒¬Any (go xss a)
    where
      go : ∀ {A : Set} (xss : List (List A)) (a : A)
        → All ((new-list a xss) ≢_) xss
      go xs a =
        helper
          xs
          (new-list a xs)
          (All≤⇒<⇒All<
            (map length xs)
            (len-new-list a xs)
            (xs≤max 0 (map length xs)))
        where
          helper : ∀ {A : Set} (xss : List (List A)) (ys : List A)
            → All (_< length ys) (map length xss)
            → All (ys ≢_) xss
          helper [] ys All.[] = All.[]
          helper (xs ∷ xss) ys (lenxs<lenys All.∷ all<) =
            sym-≢ (len≠⇒≠ xs ys (<⇒≢ lenxs<lenys))
              All.∷ helper xss ys all<

  fresh-correct : (xss : List (List Char)) → (fresh xss) ∉ xss
  fresh-correct xss = new-list-correct xss 'q'
\end{code}

\section{Syntax of terms}
\label{section:stlc_terms}
Since we will need to have quite a few proofs on local closure, we should define our terms.
\begin{code}
  data Term : Set where
    free_  : List Char → Term
    bound_ : ℕ → Term
    ƛ_     : Term → Term
    _·_    : Term → Term → Term
    ‵zero  : Term
    ‵suc_  : Term → Term

  ƛ-inj : ∀{t t'} → ƛ t ≡ ƛ t' → t ≡ t'
  ƛ-inj refl = refl

  ·-inj : {t₁ t₂ t₁' t₂' : Term}
    → t₁ · t₂ ≡ t₁' · t₂'
      -----------------------
    → (t₁ ≡ t₁') × (t₂ ≡ t₂')
  ·-inj refl = ⟨ refl , refl ⟩

  ‵suc-inj : {t₁ t₂ : Term}
    → ‵suc t₁ ≡ ‵suc t₂
      -------------
    → t₁ ≡ t₂
  ‵suc-inj refl = refl
\end{code}

We have free and bound variables, $\lambda$-abstractions, and applications (written explicitly with
a $\cdot$). Later when doing evaluation, we will only implement weak-head normalisation, so we also
need to include two primitives \texttt{‵zero} and \texttt{‵suc}, as explained in
\ref{section:evaluation_strategy}. To avoid name conflicts with the Agda built-in natural number
keywords, a small \texttt{‵} symbol is used, following what is done in
\citet{wadler_programming_2022}. We also exploit the injectivity of terms to help in proofs later.

\section{Opening and closing}
\label{appendix:opening_and_closing}
Two important operations in the locally nameless representation are
opening and closing terms. This is when an index is replaced with a free variable, and when a free
variable is replaced with an index. We write these as $[k \to x] M$ and $[k \leftarrow x] M$
respectively, for some $k \in \nat$, $x \in \texttt{List Char}$, and $M \in \texttt{Terms}$. These
can be defined recursively.
\begin{code}
  [_—→_]_ : ℕ → List Char → Term → Term
  [ k —→ x ] (free y) = free y
  [ k —→ x ] (bound i) with k ≟ℕ i
  ... | yes _ = free x
  ... | no  _ = bound i
  [ k —→ x ] (ƛ t) = ƛ ([ suc k —→ x ] t)
  [ k —→ x ] (t₁ · t₂) = [ k —→ x ] t₁ · [ k —→ x ] t₂
  [ k —→ x ] ‵zero = ‵zero
  [ k —→ x ] ‵suc t = ‵suc ([ k —→ x ] t)

  [_←—_]_ : ℕ → List Char → Term → Term
  [ k ←— x ] (free y) with x ≟lchar y
  ... | yes _ = bound k
  ... | no  _ = free y
  [ k ←— x ] (bound i) = bound i
  [ k ←— x ] (ƛ t) = ƛ [ suc k ←— x ] t
  [ k ←— x ] (t₁ · t₂) = [ k ←— x ] t₁ · [ k ←— x ] t₂
  [ k ←— x ] ‵zero = ‵zero
  [ k ←— x ] ‵suc t = ‵suc ([ k ←— x ] t)
\end{code}
We take advantage of Agda's ability to use mixfix operators to make the opening and closing
operations look like the notation used in the prose.

\citet{pitts_locally_2023} shows that nine axioms need to be fulfilled for a set to be considered a
locally nameless set. For our terms we only need three: axioms 1, 2, and 5.
\begin{code}
  ax1 : ∀ (i : ℕ) (a b : List Char) (t : Term)
    → [ i —→ a ] ([ i —→ b ] t) ≡ [ i —→ b ] t
  ax1 i a b (free x) = refl
  ax1 i a b (bound k) with i ≟ℕ k
  ... | yes _   = refl
  ... | no  i≢k with i ≟ℕ k
  ...   | yes i≡k = contradiction i≡k i≢k
  ...   | no  _   = refl
  ax1 i a b (ƛ t) rewrite ax1 (suc i) a b t = refl
  ax1 i a b (t₁ · t₂) rewrite ax1 i a b t₁ | ax1 i a b t₂ = refl
  ax1 i a b ‵zero = refl
  ax1 i a b (‵suc t) rewrite ax1 i a b t = refl

  ax1-cor : ∀ (i k : ℕ) (a : List Char) (t : Term)
    → i ≡ k
    → [ k —→ a ] ([ i —→ a ] t) ≡ [ i —→ a ] t
  ax1-cor i .i a t refl = ax1 i a a t

  ax2 : ∀ (i j : ℕ) (x : List Char) (t : Term)
     → [ i ←— x ] ([ j ←— x ] t) ≡ [ j ←— x ] t
  ax2 i j x (free y) with x ≟lchar y
  ... | yes refl = refl
  ... | no  x≢y with x ≟lchar y
  ...   | yes x≡y = contradiction x≡y x≢y
  ...   | no  _   = refl
  ax2 i j x (bound k) = refl
  ax2 i j x (ƛ t) rewrite ax2 (suc i) (suc j) x t = refl
  ax2 i j x (t₁ · t₂) rewrite ax2 i j x t₁ | ax2 i j x t₂ = refl
  ax2 i j x ‵zero = refl
  ax2 i j x (‵suc t) rewrite ax2 i j x t = refl

  suc-preserves-≢ : ∀ {n m : ℕ} → n ≢ m → suc n ≢ suc m
  suc-preserves-≢ {n} {m} n≢m sn≡sm = n≢m (helper n m sn≡sm)
    where
      helper : ∀ (n m : ℕ) → suc n ≡ suc m → n ≡ m
      helper n m refl = refl

  ax5 : ∀ (i j : ℕ) (a b : List Char) (t : Term)
    → (i≢j : i ≢ j)
    → [ i —→ a ] ([ j —→ b ] t) ≡ [ j —→ b ] ([ i —→ a ] t)
  ax5 i j a b (free x) _ = refl
  ax5 i j a b (bound k) i≢j with j ≟ℕ k
  ... | no  j≢k with i ≟ℕ k
  ...   | yes i≡k = refl
  ...   | no  i≢k with j ≟ℕ k
  ...     | yes j≡k = contradiction j≡k j≢k
  ...     | no  j≢k = refl
  ax5 i j a b (bound k) i≢j | yes j≡k with i ≟ℕ k
  ...   | yes i≡k = ⊥-elim (i≢j (trans i≡k (sym j≡k)))
  ...   | no  i≢k with j ≟ℕ k
  ...     | yes j≡k = refl
  ...     | no  j≢k = contradiction j≡k j≢k
  ax5 i j a b (ƛ t) i≢j
    rewrite ax5 (suc i) (suc j) a b t (suc-preserves-≢ i≢j) = refl
  ax5 i j a b (t₁ · t₂) i≢j
    rewrite ax5 i j a b t₁ (i≢j) | ax5 i j a b t₂ (i≢j) = refl
  ax5 i j a b ‵zero i≢j = refl
  ax5 i j a b (‵suc t) i≢j rewrite ax5 i j a b t i≢j = refl
\end{code}

\section{Fresh variables}
Fresh variables are those whose identifier (i.e. \texttt{List Char}) hasn't already been used for
any free variables in a term. We will exploit this when proving properties of substitution. Free
variables can be collected recursively.
\begin{code}
  fv : Term → List (List Char)
  fv (free x) = x ∷ []
  fv (bound i) = []
  fv (ƛ t) = fv t
  fv (t₁ · t₂) = fv t₁ ++ fv t₂
  fv ‵zero = []
  fv (‵suc t) = fv t
\end{code}

A variable is also fresh if a term remains unchanged after it is closed at index $0$ using the
variable. This definition and the previous one imply each other.
\begin{code}
  _#_ : List Char → Term → Set
  x # t = [ 0 ←— x ] t ≡ t

  #⇒∉fv : ∀ (x : List Char) (t : Term) → x # t → x ∉ fv t
  #⇒∉fv x (free y) x#t with x ≟lchar y
  ... | yes x≡y with () ← x#t
  ... | no  x≢y = All¬⇒¬Any (x≢y All.∷ All.[])
  #⇒∉fv x (bound i) x#t = λ ()
  #⇒∉fv x (ƛ t) x#t = #⇒∉fv x t (
    begin
      [ 0 ←— x ] t
    ≡⟨ sym (cong ([ 0 ←— x ]_) (ƛ-inj x#t)) ⟩
      [ 0 ←— x ] ([ 1 ←— x ] t)
    ≡⟨ ax2 0 1 x t ⟩
      [ 1 ←— x ] t
    ≡⟨ ƛ-inj x#t ⟩
      t
    ∎)
  #⇒∉fv x (t₁ · t₂) x#t = let ⟨ x#t₁ , x#t₂ ⟩ = ·-inj x#t in
    ++-∉
      (#⇒∉fv x t₁ x#t₁)
      (#⇒∉fv x t₂ x#t₂)
  #⇒∉fv x (‵suc t) x#t = #⇒∉fv x t (‵suc-inj x#t)

  ∉fv⇒# : ∀ (x : List Char) (t : Term) → x ∉ fv t → x # t
  ∉fv⇒# x (free y) x∉fv with x ≟lchar y
  ... | yes x≡y = ⊥-elim (x∉fv (here x≡y))
  ... | no  x≢y = refl
  ∉fv⇒# x (bound i) x∉fv = refl
  ∉fv⇒# x (ƛ t) x∉fv = cong ƛ_ (
    begin
      [ 1 ←— x ] t
    ≡⟨ sym (cong ([ 1 ←— x ]_) (∉fv⇒# x t x∉fv)) ⟩
      [ 1 ←— x ] ([ 0 ←— x ] t)
    ≡⟨ ax2 1 0 x t ⟩
      [ 0 ←— x ] t
    ≡⟨ ∉fv⇒# x t x∉fv ⟩
      t
    ∎)
  ∉fv⇒# x (t₁ · t₂) x∉fv =
    let ⟨ x∉fv-t₁ , x∉fv-t₂ ⟩ = (∉-++ x∉fv) in
      cong₂ _·_
        (∉fv⇒# x t₁ x∉fv-t₁)
        (∉fv⇒# x t₂ x∉fv-t₂)
  ∉fv⇒# x ‵zero x∉fv = refl
  ∉fv⇒# x (‵suc t) x∉fv rewrite ∉fv⇒# x t x∉fv = refl

  #-ƛ : ∀ {x : List Char} (t : Term)
    → x # (ƛ t)
      -------
    → x # t
  #-ƛ {x} t x#ƛt =
    begin
      [ 0 ←— x ] t
    ≡⟨ sym (cong ([ 0 ←— x ]_) (ƛ-inj x#ƛt)) ⟩
      [ 0 ←— x ] ([ 1 ←— x ] t)
    ≡⟨ ax2 0 1 x t ⟩
      [ 1 ←— x ] t
    ≡⟨ ƛ-inj x#ƛt ⟩
      t
    ∎

  #-· : ∀ {x : List Char} (t₁ t₂ : Term)
    → x # (t₁ · t₂)
      ---------------
    → x # t₁ × x # t₂
  #-· {x} t₁ t₂ x#t₁t₂ with ∉-++ (#⇒∉fv x (t₁ · t₂) x#t₁t₂)
  ... | ⟨ x∉fv-t₁ , x∉fv-t₂ ⟩
    = ⟨ (∉fv⇒# x t₁ x∉fv-t₁) , ∉fv⇒# x t₂ x∉fv-t₂ ⟩

  #-‵suc : ∀ {x : List Char} (t : Term)
    → x # (‵suc t)
      -------
    → x # t
  #-‵suc {x} t x#‵suc-t = ∉fv⇒# x t (#⇒∉fv x (‵suc t) x#‵suc-t)
\end{code}

\section{Local closure}
\label{appendix:local_closure_proofs}
\begin{code}
  _≻_ : ℕ → Term → Set
  i ≻ t = (j : ℕ) ⦃ _ : j ≥ i ⦄ → И a , ([ j —→ a ] t ≡ t)

  LocallyClosed_ : Term → Set
  LocallyClosed t = 0 ≻ t
\end{code}

As mentioned in \citet{pitts_locally_2023}, the predicate that \citet{chargueraud_locally_2012}
calls `body' is equivalent to $1 \succ M$ for some term $M$.

We can show some interesting lemmas which will also be needed later. These are named after the lemma
number used in \citet{pitts_locally_2023}.
\begin{code}
  lemma2·6 : ∀ {i j : ℕ} {t : Term}
    → j ≥ i
    → i ≻ t
      -----
    → j ≻ t
  lemma2·6 {i} {j} {t} j≥i i≻t k = i≻t k ⦃ ≤-trans j≥i it ⦄

  ≻⇒s≻ : ∀ {i : ℕ} {t : Term}
    → i ≻ t
      -----
    → (suc i) ≻ t
  ≻⇒s≻ {i} i≻t = lemma2·6 (n≤1+n i) i≻t

  lemma2·7-1 : ∀ {i : ℕ} {x y : List Char} {t : Term}
    → [ i —→ x ] t ≡ t
      ----------------
    → [ i —→ y ] t ≡ t
  lemma2·7-1 {i} {x} {y} {t} [i>x]t≡t =
    begin
      ([ i —→ y ] t)
    -- use the fact that t ≡ [ i —→ x ] t
    ≡⟨ sym (cong ([ i —→ y ]_) [i>x]t≡t) ⟩
      [ i —→ y ] ([ i —→ x ] t)
    ≡⟨ ax1 i y x t ⟩
      [ i —→ x ] t
    ≡⟨ [i>x]t≡t ⟩
      t
    ∎

  lemma2·7-2 : ∀ {i j : ℕ} {x : List Char} {t : Term}
    → j ≥ i
    → i ≻ t
      ----------------
    → [ j —→ x ] t ≡ t
  lemma2·7-2 {j = j} j≥i i≻t with (i≻t j ⦃ j≥i ⦄)
  ... | И⟨ Иe₁ , Иe₂ ⟩ =
    lemma2·7-1 (Иe₂ (fresh Иe₁) {fresh-correct Иe₁})

  open-rec-lc : ∀ {t : Term} {i : ℕ} {x : List Char}
    → LocallyClosed t
      ----------------
    → [ i —→ x ] t ≡ t
  open-rec-lc lc-t = lemma2·7-2 z≤n lc-t
\end{code}

Note, for this next property, \texttt{open-rec-lc-lemma}, in the paper
\citep{chargueraud_locally_2012}, the assumption is written like so (with the notation adapted):

\texttt{[ i —→ u ] ([ j —→ v ] t) ≡ [ i —→ u ] t}

However, in the Coq source code \citep{chargueraud_lambda_jar_paperv_2023}, the assumption is as below (notice that the right side of the
equality is \texttt{[ j —→ v ] t}). I decided to use the assumption as in the Coq source code.
\begin{code}
  open-rec-lc-lemma : ∀ {t : Term} {i j : ℕ} {u v : List Char}
    → i ≢ j
    → [ i —→ u ] ([ j —→ v ] t) ≡ [ j —→ v ] t
    → [ i —→ u ] t ≡ t
  open-rec-lc-lemma {free x} i≢j assump = refl
  open-rec-lc-lemma {bound k} {i} {j} i≢j assump
    with i ≟ℕ j | i ≟ℕ k
  ... | yes i≡j | yes i≡k = contradiction i≡j i≢j
  ... | yes i≡j | no  i≢k = contradiction i≡j i≢j
  ... | no  _   | no  _   = refl
  ... | no i≢j' | yes i≡k with j ≟ℕ k
  ...   | yes j≡k = ⊥-elim (i≢j (trans i≡k (sym j≡k)))
  ...   | no  j≢k with i ≟ℕ k
  ...     | yes i≡k' with () ← assump
  ...     | no  i≢k  = contradiction i≡k i≢k
  open-rec-lc-lemma {ƛ t} {i} {j} i≢j assump
    rewrite open-rec-lc-lemma {t} {suc i} {suc j}
        (suc-preserves-≢ i≢j)
        (ƛ-inj assump)
      = refl
  open-rec-lc-lemma {t₁ · t₂} i≢j assump
    rewrite
      open-rec-lc-lemma {t₁} i≢j (proj₁ (·-inj assump))
    | open-rec-lc-lemma {t₂} i≢j (proj₂ (·-inj assump))
    = refl
  open-rec-lc-lemma {‵zero} _ _ = refl
  open-rec-lc-lemma {‵suc t} i≢j assump
    rewrite open-rec-lc-lemma {t} i≢j (‵suc-inj assump) = refl

  lemma2·13 : ∀ {t : Term} {a : List Char} {i : ℕ} (j : ℕ)
    → j ≥ i
    → i ≻ t
    → i ≻ ([ j —→ a ] t)
  lemma2·13 {t} {a} {i} j j≥i i≻t k
    with j ≟ℕ k | Иe₁ (i≻t j ⦃ j≥i ⦄)
  ... | yes refl | l = И⟨ l , (λ b → ax1 j b a t) ⟩
  ... | no  j≢k  | l = И⟨ l , (λ b →
    begin
      [ k —→ b ] ([ j —→ a ] t)
    ≡⟨ ax5 k j b a t (sym-≢ j≢k) ⟩
      [ j —→ a ] ([ k —→ b ] t)
    ≡⟨ cong([ j —→ a ]_) (lemma2·7-2 it i≻t) ⟩
      [ j —→ a ] t
    ∎) ⟩
\end{code}

\citet{pitts_locally_2023} uses an existential quantification for this property instead of the
cofinite quantification I used. To show that these two are equivalent, we'll use a recursively
defined local closure predicate (as is done in \citet{chargueraud_locally_2012}). It's sufficient
for us to show an iff relation between the cofinite and recursive definitions, since
\citet{pitts_locally_2023} shows an iff relation between the existential and recursive definitions
(proposition 4.3); thus, we can prove that the two different quantification definitions imply each
other.

\begin{code}
  data Lc-at : ℕ → Term → Set where
    lc-at-bound : ∀ {i j : ℕ} ⦃ _ : j < i ⦄ → Lc-at i (bound j)
    lc-at-free : ∀ {i : ℕ} {a : List Char} → Lc-at i (free a)
    lc-at-lam : ∀ {i : ℕ} {t : Term}
      → Lc-at (suc i) t
        ------------------
      → Lc-at i (ƛ t)
    lc-at-app : ∀ {i : ℕ} {t₁ t₂ : Term}
      → Lc-at i t₁
      → Lc-at i t₂
        -------------------
      → Lc-at i (t₁ · t₂)
    lc-at-‵zero : ∀ {i : ℕ} → Lc-at i ‵zero
    lc-at-‵suc : ∀ {i : ℕ} {t : Term}
      → Lc-at i t
        ------------------
      → Lc-at i (‵suc t)

  ≻⇒lc-at : ∀ (i : ℕ) (t : Term) → i ≻ t → Lc-at i t
  ≻⇒lc-at i (free x) i≻t = lc-at-free
  ≻⇒lc-at i (bound j) i≻t with j <? i
  ... | yes j<i = lc-at-bound ⦃ j<i ⦄
  ... | no  j≮i with
    (Иe₂ (i≻t j ⦃ ≮⇒≥ j≮i ⦄))
      (fresh (Иe₁ (i≻t j ⦃ ≮⇒≥ j≮i ⦄)))
      {fresh-correct (Иe₁ (i≻t j ⦃ ≮⇒≥ j≮i ⦄))}
  ...   | q with j ≟ℕ j
  ...     | yes refl with () ← q
  ...     | no  j≢j  = contradiction refl j≢j
  ≻⇒lc-at i (ƛ t) i≻t = lc-at-lam (≻⇒lc-at (suc i) t helper)
    where
      helper : suc i ≻ t
      helper (suc j) ⦃ s≤s j≥i ⦄ =
        И⟨ Иe₁ (i≻t j ⦃ j≥i ⦄)
        , (λ a {a∉} → ƛ-inj ((Иe₂ (i≻t j ⦃ j≥i ⦄)) a {a∉})) ⟩
  ≻⇒lc-at i (t₁ · t₂) i≻t =
    lc-at-app (≻⇒lc-at i t₁ i≻t₁) (≻⇒lc-at i t₂ i≻t₂)
    where
      i≻t₁ : i ≻ t₁
      i≻t₁ j ⦃ j≥i ⦄ =
        И⟨ Иe₁ (i≻t j ⦃ j≥i ⦄)
        , (λ a {a∉} → proj₁ (·-inj ((Иe₂ (i≻t j ⦃ j≥i ⦄)) a {a∉})))
        ⟩
      i≻t₂ : i ≻ t₂
      i≻t₂ j ⦃ j≥i ⦄ =
        И⟨ Иe₁ (i≻t j ⦃ j≥i ⦄)
        , (λ a {a∉} → proj₂ (·-inj ((Иe₂ (i≻t j ⦃ j≥i ⦄)) a {a∉})))
        ⟩
  ≻⇒lc-at _ ‵zero _ = lc-at-‵zero
  ≻⇒lc-at i (‵suc t) i≻t = lc-at-‵suc (≻⇒lc-at i t (λ j →
    И⟨ (Иe₁ (i≻t j ⦃ it ⦄))
    , (λ a {a∉} → ‵suc-inj ((Иe₂ (i≻t j ⦃ it ⦄)) a {a∉})) ⟩))

  lc-at⇒≻ : ∀ (i : ℕ) (t : Term) → Lc-at i t → i ≻ t
  -- Here we use "[]" because we don't really care what string
  -- we use to test the locally closedness.
  lc-at⇒≻ i (bound k) lc-at-bound j ⦃ i≤j ⦄ with j ≟ℕ k
  ... | yes j≡k = contradiction (sym j≡k) (<⇒≢ (≤-trans it i≤j))
  ... | no  j≢k = И⟨ [] , (λ a → refl) ⟩
  lc-at⇒≻ i (free x) lc-at-free j = И⟨ [] , (λ a → refl) ⟩
  lc-at⇒≻ i (ƛ t) (lc-at-lam lc-at) j
    with lc-at⇒≻ (suc i) t lc-at
  ... | si≻t = И⟨ [] , (λ a → cong ƛ_ (lemma2·7-2 (s≤s it) si≻t)) ⟩
  lc-at⇒≻ i (t₁ · t₂) (lc-at-app lc-at₁ lc-at₂) j =
    И⟨ []
    , (λ a → cong₂ _·_
        (lemma2·7-2 it (lc-at⇒≻ i t₁ lc-at₁))
        (lemma2·7-2 it (lc-at⇒≻ i t₂ lc-at₂))) ⟩
  lc-at⇒≻ _ (‵zero) (lc-at-‵zero) j = И⟨ [] , (λ _ → refl) ⟩
  lc-at⇒≻ i (‵suc t) (lc-at-‵suc lc-at) j =
    И⟨ []
    , (λ _ → cong ‵suc_ (lemma2·7-2 it (lc-at⇒≻ i t lc-at))) ⟩
\end{code}

We can also use this recrusive definition to show some other properties of the locally closed syntax
which will be very convenient in the future.
\begin{code}
  bound-never-lc : ∀ (n : ℕ) → ¬ LocallyClosed (bound n)
  bound-never-lc n x with ≻⇒lc-at 0 (bound n) x
  ... | lc-at-bound ⦃ () ⦄ -- This implies that n < 0
                          -- which is never true.

  free-lc : ∀ {x : List Char} → LocallyClosed (free x)
  free-lc _ = И⟨ [] , (λ _ → refl) ⟩

  i≻ƛt⇒si≻t : ∀ {i : ℕ} {t : Term} → i ≻ (ƛ t) → suc i ≻ t
  i≻ƛt⇒si≻t {i} {t} lc-t with ≻⇒lc-at i (ƛ t) lc-t
  ... | lc-at-lam lc-at-si-t = lc-at⇒≻ (suc i) t lc-at-si-t

  ·-≻ : ∀ {t₁ t₂ : Term} {i : ℕ}
    → i ≻ (t₁ · t₂) → (i ≻ t₁) × (i ≻ t₂)
  ·-≻ {t₁} {t₂} {i} i≻· with ≻⇒lc-at i (t₁ · t₂) i≻·
  ... | lc-at-app t₁-at t₂-at =
    ⟨ lc-at⇒≻ i t₁ t₁-at
    , lc-at⇒≻ i t₂ t₂-at ⟩

  ‵zero-≻ : ∀ {i : ℕ} → i ≻ ‵zero
  ‵zero-≻ j = И⟨ [] , (λ _ → refl) ⟩

  ‵suc-≻ : ∀ {t : Term} {i : ℕ} → i ≻ (‵suc t) → i ≻ t
  ‵suc-≻ {t} {i} i≻‵suc-t with ≻⇒lc-at i (‵suc t) i≻‵suc-t
  ... | lc-at-‵suc lc-at-i = lc-at⇒≻ i t lc-at-i
\end{code}

\section{Substitution of terms}
\label{appendix:substitution_proofs}
We can substitute terms for free variables with this recursive definition.
\begin{code}
  [_:=_]_ : List Char → Term → Term → Term
  [ x := u ] (free y) with x ≟lchar y
  ... | yes _ = u
  ... | no  _ = free y
  [ x := u ] (bound i) = bound i
  [ x := u ] (ƛ t) = ƛ [ x := u ] t
  [ x := u ] (t₁ · t₂) = [ x := u ] t₁ · [ x := u ] t₂
  [ x := u ] (‵zero) = ‵zero
  [ x := u ] (‵suc t) = ‵suc ([ x := u ] t)
\end{code}

While there are many additional properties of free-variable substitution, as proven by
\citet{chargueraud_locally_2012}, we only need this one which shows how substitution interacts with
term opening. It is named following what it's called in \citet{chargueraud_locally_2012}.
\begin{code}
  subst-open-var : ∀ {u : Term} {x y : List Char} {i : ℕ} (t : Term)
    → x ≢ y
    → i ≻ u
    → [ x := u ] ([ i —→ y ] t) ≡ [ i —→ y ] ([ x := u ] t)
  subst-open-var {x = x} (free z) x≢y i≻u with x ≟lchar z
  ... | yes x≡z = sym (lemma2·7-2 ≤-refl i≻u)
  ... | no  x≢z = refl
  subst-open-var {_} {x} {y} {i} (bound k) x≢y lc-u with i ≟ℕ k
  ... | no  i≢k = refl
  ... | yes i≡k with x ≟lchar y
  ...   | yes x≡y = contradiction x≡y x≢y
  ...   | no  x≢y = refl
  subst-open-var (ƛ t) x≢y lc-u =
    cong ƛ_ (subst-open-var t x≢y (≻⇒s≻ lc-u))
  subst-open-var (t₁ · t₂) x≢y lc-u =
    cong₂ _·_
      (subst-open-var t₁ x≢y lc-u)
      (subst-open-var t₂ x≢y lc-u)
  subst-open-var (‵zero) x≢y lc-u = refl
  subst-open-var (‵suc t) x≢y lc-u =
    cong ‵suc_ (subst-open-var t x≢y lc-u)

  subst-open-lc : ∀ {t u : Term} {x y : List Char}
    → x ≢ y
    → LocallyClosed u
    → [ x := u ] ([ 0 —→ y ] t) ≡ [ 0 —→ y ] ([ x := u ] t)
  subst-open-lc {t} x≢y lc-u = subst-open-var t x≢y lc-u
\end{code}

During the development, I ended up proving more locally nameless substitution theorems than I
needed, so some haven't been referenced above. These aren't strictly needed.

\begin{code}
  subst-fresh : ∀ {t u : Term} {x : List Char}
    → x # t
    → [ x := u ] t ≡ t
  subst-fresh {free y} {u} {x} x#t with x ≟lchar y
  ... | yes _ with () ← x#t
  ... | no  _ = refl
  subst-fresh {bound i} {u} {x} x#t = refl
  subst-fresh {ƛ t} {u} x#t =
    cong ƛ_ (subst-fresh (#-ƛ t x#t))
  subst-fresh {t₁ · t₂} {u} {x} x#· =
    let ⟨ x#t₁ , x#t₂ ⟩ = #-· t₁ t₂ x#· in
      cong₂ _·_ (subst-fresh x#t₁) (subst-fresh x#t₂)
  subst-fresh {‵zero} {u} x#t = refl
  subst-fresh {‵suc t} {u} x#t =
    cong ‵suc_ (subst-fresh (#-‵suc t x#t))

  subst-≻ : ∀ {t u : Term} {i : ℕ} (x : List Char)
    → i ≻ t
    → i ≻ u
      ----------------
    → i ≻ ([ x := u ] t)
  subst-≻ {free y} x i≻t i≻u j with x ≟lchar y
  ... | yes _ = i≻u j
  ... | no  _ = И⟨ [] , (λ _ → refl) ⟩
  subst-≻ {bound k} x i≻t i≻u j with j ≟ℕ k
  ... | no  _   = И⟨ [] , (λ _ → refl) ⟩
  ... | yes j≡k with i≻t j
  ...   | И⟨ Иe₁ , Иe₂ ⟩ with j ≟ℕ k
  ...     | no  j≢k = contradiction j≡k j≢k
  ...     | yes _   with () ← Иe₂ (fresh Иe₁) {fresh-correct Иe₁}
  subst-≻ {ƛ t} {u} x i≻t i≻u j with i≻ƛt⇒si≻t i≻t
  ... | si≻t =
    И⟨ (x ∷ (Иe₁ (si≻t (suc j) ⦃ s≤s it ⦄)))
    , (λ a {a∉} → cong ƛ_ ((
      begin
        [ suc j —→ a ] ([ x := u ] t)
      ≡⟨ sym (subst-open-var
            t
            (sym-≢ (∉∷[]⇒≢ (proj₁ (∉-++ {xs = x ∷ []} a∉))))
            (lemma2·6 (m≤n⇒m≤1+n it) i≻u)) ⟩
        [ x := u ] ([ suc j —→ a ] t)
      ≡⟨ cong ([ x := u ]_)
          ((Иe₂ (si≻t (suc j) ⦃ s≤s it ⦄))
            a
            {proj₂ (∉-++ {xs = x ∷ []} a∉)}) ⟩
        [ x := u ] t
      ∎))) ⟩
  subst-≻ {t₁ · t₂} {i = i} x i≻t i≻u j =
    let ⟨ i≻t₁ , i≻t₂ ⟩ = ·-≻ i≻t in
      И⟨ (Иe₁ ((subst-≻ x i≻t₁ i≻u) j)
        ++ Иe₁ ((subst-≻ x i≻t₂ i≻u) j))
      , (λ a {a∉} → cong₂ _·_
          (lemma2·7-2 it (subst-≻ x i≻t₁ i≻u))
          (lemma2·7-2 it (subst-≻ x i≻t₂ i≻u))) ⟩
  subst-≻ {‵zero} {u} x i≻t i≻u j = И⟨ [] , (λ _ → refl) ⟩
  subst-≻ {‵suc t} {u} x i≻t i≻u j =
    let И⟨ Иe₁ , Иe₂ ⟩ = (subst-≻ {t} x (‵suc-≻ i≻t) i≻u) j ⦃ it ⦄
      in И⟨ Иe₁ , (λ a {a∉} → cong ‵suc_ (Иe₂ a {a∉})) ⟩

  subst-lc : ∀ {t u : Term} (x : List Char)
    → LocallyClosed t
    → LocallyClosed u
      --------------------------
    → LocallyClosed [ x := u ] t
  subst-lc = subst-≻
\end{code}

\section{Types and contexts}
\label{appendix:typing_stlc}
We will use \texttt{‵ℕ} as the base type to work together with the
\texttt{‵zero} and \texttt{‵suc} primitives. Other than the base type, we also have function (arrow)
types. We will closely follow what is done in \citet{wadler_programming_2022}.
\begin{code}
  data Type : Set where
    ‵ℕ : Type
    _⇒_ : Type → Type → Type

  data Context : Set where
    ∅ : Context
    _,_⦂_ : Context → List Char → Type → Context
\end{code}

\citet{aydemir_engineering_2008} and \citet{chargueraud_locally_2012} include a predicate to tell
when a context is `ok', that is, that there are no duplicate names. Instead of worrying about
keeping track of the strings in the context, we can use `shadowing' (always using the latest
matching variable in the context).

To access the context, we can use these accessors. We use H and T to mirror the \texttt{List.Any}
construct's `here' and `there'.
\begin{code}
  data _∋_⦂_ : Context → List Char → Type → Set where
    H : ∀ {Γ x y A}
      → x ≡ y
        ------------------
      → Γ , x ⦂ A ∋ y ⦂ A

    T : ∀ {Γ x y A B}
      → x ≢ y
      → Γ ∋ x ⦂ A
        -----------------
      → Γ , y ⦂ B ∋ x ⦂ A
\end{code}

Like in \citet{wadler_programming_2022}, we can use some helper functions to try and use Agda's type
inference to find the required evidence itself.
\begin{code}
  H′ : ∀ {Γ x A}
    → Γ , x ⦂ A ∋ x ⦂ A
  H′ = H refl

  T′ : ∀ {Γ x y A B}
    → {x≢y : False (x ≟lchar y)}
    → Γ ∋ x ⦂ A
      ------------------
    → Γ , y ⦂ B ∋ x ⦂ A
  T′ {x≢y = x≢y} x = T (toWitnessFalse x≢y) x
\end{code}

We need to have a function to get all the variables in the context.
\begin{code}
  domain : Context → List (List Char)
  domain ∅ = []
  domain (Γ , x ⦂ A) = x ∷ domain Γ
\end{code}

\section{Type judgements}
\label{appendix:type_judgements}
The type judgements are the same as in
\citet[chapter~Lambda]{wadler_programming_2022}, so we won't go into much detail here. The one
difference is how we define $\lambda$-abstraction, where we use a cofinite quantifier.

In \citet[chapter~Lambda]{wadler_programming_2022}, $\lambda$-abstractions are handled as follows:
the bound variable is added to the context and the expression is typechecked with the bound variable
now treated as if it were free. Here, we open the term to replace it with a free variable (and then
we add this free variable to the context). Thus, we need to find a \texttt{List Char} which isn't in
the context yet, otherwise we would shadow a previous variable with the same identifier. Since bound
variables become free, we don't have a typing judgement for bound variables.
\begin{code}
  data _⊢_⦂_ : Context → Term → Type → Set where
    ⊢free : ∀ {Γ x A}
      → Γ ∋ x ⦂ A
        ---------
      → Γ ⊢ free x ⦂ A

    ⊢ƛ : ∀ {Γ t A B}
      → И x , ((Γ , x ⦂ A) ⊢ [ 0 —→ x ] t ⦂ B)
        ---------------------------
      → Γ ⊢ ƛ t ⦂ (A ⇒ B)

    ⊢· : ∀ {Γ t₁ t₂ A B}
      → Γ ⊢ t₁ ⦂ (A ⇒ B)
      → Γ ⊢ t₂ ⦂ A
        ---------
      → Γ ⊢ t₁ · t₂ ⦂ B

    ⊢zero : ∀ {Γ}
        -------
      → Γ ⊢ ‵zero ⦂ ‵ℕ

    ⊢suc : ∀ {Γ t}
      → Γ ⊢ t ⦂ ‵ℕ
        ----------------
      → Γ ⊢ ‵suc t ⦂ ‵ℕ

  -- Apply term-equality within type judgements.
  ≡-with-⊢ : ∀ {Γ t u A}
    → Γ ⊢ t ⦂ A
    → t ≡ u
      ----------
    → Γ ⊢ u ⦂ A
  ≡-with-⊢ ⊢t refl = ⊢t
\end{code}

As a consequence of these type judgements, only locally closed terms are well-typed.
\begin{code}
  ⊢⇒lc : ∀ {Γ t A} → Γ ⊢ t ⦂ A → LocallyClosed t
  ⊢⇒lc {Γ} {t} {A} (⊢free Γ∋A) = free-lc
  ⊢⇒lc {Γ} {ƛ t} {A} (⊢ƛ И⟨ Иe₁ ,  Иe₂ ⟩) j =
    И⟨ Иe₁ , (λ a {a∉} → cong ƛ_
      (open-rec-lc-lemma
        (λ ())
        (open-rec-lc (⊢⇒lc (Иe₂ a {a∉}))))) ⟩
  ⊢⇒lc {Γ} {t₁ · t₂} (⊢· ⊢A⇒B ⊢A) _ =
    И⟨ domain Γ , (λ _ → cong₂ _·_
      (open-rec-lc (⊢⇒lc ⊢A⇒B)) (open-rec-lc (⊢⇒lc ⊢A))) ⟩
  ⊢⇒lc {Γ} {‵zero} ⊢zero = ‵zero-≻
  ⊢⇒lc {Γ} {‵suc t} (⊢suc ⊢t) j =
    И⟨ domain Γ , (λ a {a∉} →
      cong ‵suc_ (open-rec-lc (⊢⇒lc ⊢t))) ⟩
\end{code}

Using the type judgements, we can show that these two terms are well-typed. Using more familiar
notation, we would write this as $x \colon \nat \vdash (\lambda \colon \nat \to \nat. 0x) \colon \nat$
and $\vdash (\lambda \colon \nat. \lambda \colon \nat \to \nat. 01) \colon \nat \to (\nat
\to \nat) \to \nat$ respectively, or using only the named representation, $x \colon \nat \vdash
(\lambda a \colon \nat \to \nat. ax) \colon \nat$ and $\vdash (\lambda x \colon \nat. \lambda
f \colon \nat \to \nat. fx) \colon \nat \to (\nat \to \nat) \to \nat$.
\begin{code}
  _ : (∅ , ⟪ "x" ⟫ ⦂ ‵ℕ) ⊢ (ƛ (bound 0)) · (free ⟪ "x" ⟫) ⦂ ‵ℕ
  _ = ⊢· (⊢ƛ И⟨ ⟪ "x" ⟫ ∷ [] , (λ a → ⊢free H′) ⟩) (⊢free H′)

  ex-for-all-contexts : ∀ {Γ} → Γ ⊢ ƛ ƛ (bound 0) · (bound 1) ⦂ (‵ℕ ⇒ (‵ℕ ⇒ ‵ℕ) ⇒ ‵ℕ)
  ex-for-all-contexts {Γ} = ⊢ƛ (
    И⟨ [] , (λ a → ⊢ƛ (
      И⟨ a ∷ []
      , (λ b {b∉} → ⊢· (⊢free (H′)) (⊢free (T (∉⇒≢ (here refl) b∉) H′))) ⟩ )) ⟩)
\end{code}

\section{Typing properties}
Since the type judgements are very similar to those in named simply typed lambda calculus,
we have many of the same properties as in \citet[chapter~Lambda]{wadler_programming_2022}.
\begin{code}
  -- Extending contexts.
  ext : ∀ {Γ Δ}
    → (∀ {x A}     →         Γ ∋ x ⦂ A →         Δ ∋ x ⦂ A)
      ----------------------------------------------------
    → (∀ {x y A B} → Γ , y ⦂ B ∋ x ⦂ A → Δ , y ⦂ B ∋ x ⦂ A)
  ext ρ (H refl) = H refl
  ext ρ (T x≢y ∋x) = T x≢y (ρ ∋x)

  -- Renaming (aka. "rebasing") of contexts.
  rename : ∀ {Γ Δ}
    → (∀ {x A} → Γ ∋ x ⦂ A → Δ ∋ x ⦂ A)
      --------------------------------
    → (∀ {M A} → Γ ⊢ M ⦂ A → Δ ⊢ M ⦂ A)
  rename ρ (⊢free ∋A) = ⊢free (ρ ∋A)
  rename {Δ = Δ} ρ (⊢ƛ И⟨ Иe₁ , Иe₂ ⟩ ) =
    ⊢ƛ И⟨ (domain Δ ++ Иe₁) , (λ a {a∉} →
      rename (ext ρ) (Иe₂ a {proj₂ (∉-++ a∉)})) ⟩
  rename ρ (⊢· ⊢A⇒B ⊢A) = ⊢· (rename ρ ⊢A⇒B) (rename ρ ⊢A)
  rename {Δ = Δ} ρ ⊢zero = ⊢zero
  rename {Δ = Δ} ρ (⊢suc ⊢t) = ⊢suc (rename ρ ⊢t)

  -- Weakening of contexts.
  weaken : ∀ {Γ t A}
    → ∅ ⊢ t ⦂ A
      ---------
    → Γ ⊢ t ⦂ A
  weaken {Γ} ⊢A = rename (λ ()) ⊢A

  -- Swapping variables in a context.
  swap : ∀ {Γ x y t A B C}
    → x ≢ y
    → Γ , y ⦂ B , x ⦂ A ⊢ t ⦂ C
      ------------------------
    → Γ , x ⦂ A , y ⦂ B ⊢ t ⦂ C
  swap {Γ} {x} {y} {M} {A} {B} {C} x≢y ⊢t = rename ρ ⊢t
    where
      ρ : ∀ {z C}
        → Γ , y ⦂ B , x ⦂ A ∋ z ⦂ C
          --------------------------
        → Γ , x ⦂ A , y ⦂ B ∋ z ⦂ C
      ρ (H refl) = T x≢y (H′)
      ρ (T z≢x (H refl)) = H′
      ρ (T z≢x (T z≢y ∋z)) = T z≢y (T z≢x ∋z)

  -- Dropping shadowed variables.
  drop : ∀ {Γ x M A B C}
    → Γ , x ⦂ A , x ⦂ B ⊢ M ⦂ C
      ------------------------
    → Γ , x ⦂ B ⊢ M ⦂ C
  drop {Γ} {x} {M} {A} {B} {C} ⊢M = rename ρ ⊢M
    where
      ρ : ∀ {z C}
        → Γ , x ⦂ A , x ⦂ B ∋ z ⦂ C
          -------------------------
        → Γ , x ⦂ B ∋ z ⦂ C
      ρ (H refl) = H refl
      ρ (T z≢x (H x≡z)) = contradiction (sym x≡z) z≢x
      ρ (T z≢x (T _ ∋z)) = T z≢x ∋z

  -- subst-open-var adapted to work with judgements.
  subst-open-context : ∀ {Γ A} {t u : Term} {x y : List Char}
    → x ≢ y
    → LocallyClosed u
    → Γ ⊢ [ x := u ] ([ 0 —→ y ] t) ⦂ A
      ---------------------------------
    → Γ ⊢ [ 0 —→ y ] ([ x := u ] t) ⦂ A
  subst-open-context {t = t} x≢y lc-u sub-open =
    ≡-with-⊢ sub-open (subst-open-var t x≢y lc-u)
\end{code}



\chapter{Compilation instructions}
\label{appendix:compilation_instructions}

This document is a literate Agda file. It has been tested to work with
\begin{itemize}
  \item Agda 2.7.0,
  \item the Agda Standard Library 2.1 \citep{the_agda_community_agda_2024},
  \item XeLaTeX 3.141592653-2.6-0.999996 (TeX Live 2024/Arch Linux 2024.2-4).
\end{itemize}

The full source code is available at [TODO: TO BE MADE PUBLIC LATER].

Since this document uses the Minted package [TODO: cite], XeLaTeX needs to be run with the
\texttt{--shell-escape} option. While Agda does provide its own typesetting of Agda code, it uses a
sans-serif typeface. I decided to use the Minted package to provide a monospace typeface for the
code blocks.

To typecheck the document, run Agda with the \texttt{--latex} option. If successful, it will give no
output and return an exit code \texttt{0}.

\end{document}

% UG project example file, February 2024
%
%   Added the "online" option for equal margins, February 2024 [Hiroshi Shimodaira, Iain Murray]
%   A minor change in citation, September 2023 [Hiroshi Shimodaira]
%
% Do not change the first two lines of code, except you may delete "logo," if causing problems.
% Understand any problems and seek approval before assuming it's ok to remove ugcheck.
\documentclass[logo,bsc,singlespacing,parskip,online]{infthesis}
\usepackage{ugcheck}


% Include any packages you need below, but don't include any that change the page
% layout or style of the dissertation. By including the ugcheck package above,
% you should catch most accidental changes of page layout though.

\usepackage{microtype} % recommended, but you can remove if it causes problems
\usepackage[round]{natbib} % recommended for citations

% === Custom packages === %

% Syntax highlighting
\usepackage{minted}
% BNF
\usepackage{simplebnf}
% Inference rules
\usepackage{mathpartir}
% Agda
\usepackage{agda}
\AgdaNoSpaceAroundCode{}
% Hide in output
\usepackage{comment}
% Colour
\usepackage{xcolor}

% Unicode

\usepackage{fontspec}
\usepackage{newunicodechar}
\newfontface{\notosansmono}{NotoSansMono-Regular.ttf}[Path = fonts/]
\newfontface{\ibmplexmath}{IBMPlexMath-Regular.otf}[Path = fonts/]
\newfontface{\hack}{HackNerdFontMono-Regular.ttf}[Path = fonts/]
\newunicodechar{ℕ}{{\notosansmono{ℕ}}}
\newunicodechar{∀}{{\notosansmono{∀}}}
\newunicodechar{≡}{{\notosansmono{≡}}}
\newunicodechar{≥}{{\notosansmono{≥}}}
\newunicodechar{≤}{{\notosansmono{≤}}}
\newunicodechar{≰}{{\notosansmono{≰}}}
\newunicodechar{⊔}{{\notosansmono{⊔}}}
\newunicodechar{≟}{{\notosansmono{≟}}}
\newunicodechar{⇒}{{\hack{⇒}}}
\newunicodechar{≢}{{\notosansmono{≢}}}
\newunicodechar{≮}{{\notosansmono{≮}}}
\newunicodechar{⟪}{{\notosansmono{⟪}}}
\newunicodechar{⟫}{{\notosansmono{⟫}}}
\newunicodechar{⊤}{{\notosansmono{⊤}}}
\newunicodechar{⊥}{{\notosansmono{⊥}}}
\newunicodechar{∣}{{\notosansmono{∣}}}
\newunicodechar{⟨}{{\notosansmono{⟨}}}
\newunicodechar{⟩}{{\notosansmono{⟩}}}
\newunicodechar{∎}{{\notosansmono{∎}}}
\newunicodechar{⌊}{{\notosansmono{⌊}}}
\newunicodechar{⌋}{{\notosansmono{⌋}}}
\newunicodechar{₁}{{\notosansmono{₁}}}
\newunicodechar{₂}{{\notosansmono{₂}}}
\newunicodechar{∃}{{\notosansmono{∃}}}
\newunicodechar{∷}{{\notosansmono{∷}}}
\newunicodechar{∈}{{\notosansmono{∈}}}
\newunicodechar{∉}{{\notosansmono{∉}}}
\newunicodechar{λ}{{\notosansmono{λ}}}
\newunicodechar{И}{{\notosansmono{И}}}
\newunicodechar{∘}{{\notosansmono{∘}}}
\newunicodechar{≠}{{\notosansmono{≠}}}
\newunicodechar{‵}{{\notosansmono{‵}}}
\newunicodechar{ƛ}{{\notosansmono{ƛ}}}
\newunicodechar{≻}{{\ibmplexmath{≻}}}
\newunicodechar{⦃}{{\ibmplexmath{⦃}}}
\newunicodechar{⦄}{{\ibmplexmath{⦄}}}
\newunicodechar{′}{{\notosansmono{′}}}
\newunicodechar{∋}{{\notosansmono{∋}}}
\newunicodechar{⦂}{{\ibmplexmath{⦂}}}
\newunicodechar{∅}{{\hack{∅}}}
\newunicodechar{⊢}{{\hack{⊢}}}
\newunicodechar{ξ}{{\notosansmono{ξ}}}
\newunicodechar{β}{{\notosansmono{β}}}
\newunicodechar{↠}{{\hack{↠}}}
\newunicodechar{ρ}{{\notosansmono{ρ}}}
\newunicodechar{σ}{{\notosansmono{σ}}}
\newunicodechar{⊆}{{\notosansmono{⊆}}}
\newunicodechar{⊎}{{\notosansmono{⊎}}}
\newunicodechar{∧}{{\notosansmono{∧}}}
\newunicodechar{ʳ}{{\notosansmono{ʳ}}}
\newunicodechar{ˡ}{{\notosansmono{ˡ}}}

% Maths:

\usepackage{mathtools}
\usepackage{amssymb}
\usepackage{enumitem}
\usepackage{hyperref}
\usepackage{xurl}
\DeclareMathOperator{\lcm}{lcm}
\DeclareMathOperator{\Real}{Re}
\DeclareMathOperator{\Imag}{Im}
\DeclareMathOperator{\complex}{\mathbb{C}}
\DeclareMathOperator{\reals}{\mathbb{R}}
\DeclareMathOperator{\nat}{\mathbb{N}}
\DeclareMathOperator{\integer}{\mathbb{Z}}
\DeclareMathOperator{\rational}{\mathbb{Q}}
\DeclareMathOperator{\Log}{Log}
\DeclareMathOperator{\Arg}{Arg}
\DeclareMathOperator{\cof}{\text{И}}

% Use minted for Agda

% \let\oldcode\code
% \NewCommandCopy{\mintedcopy}{\minted}
% \NewCommandCopy{\endmintedcopy}{\endminted}
% % Adapted from https://tex.stackexchange.com/a/488451/202867
% \renewenvironment{code}{\mintedcopy[breaklines,breaksymbolleft=\;]{agda}}{\endmintedcopy}

\begin{document}
\begin{preliminary}

\title{System F in Agda via Pitts}

\author{Charlotte Ausel}

% CHOOSE YOUR DEGREE a):
% please leave just one of the following un-commented
% \course{Artificial Intelligence}
%\course{Artificial Intelligence and Computer Science}
%\course{Artificial Intelligence and Mathematics}
%\course{Artificial Intelligence and Software Engineering}
%\course{Cognitive Science}
%\course{Computer Science}
%\course{Computer Science and Management Science}
\course{Computer Science and Mathematics}
%\course{Computer Science and Physics}
%\course{Software Engineering}
%\course{Master of Informatics} % MInf students

% CHOOSE YOUR DEGREE b):
% please leave just one of the following un-commented
%\project{MInf Project (Part 1) Report}  % 4th year MInf students
%\project{MInf Project (Part 2) Report}  % 5th year MInf students
\project{4th Year Project Report}        % all other UG4 students


\date{\today}

\abstract{
This skeleton demonstrates how to use the \texttt{infthesis} style for
undergraduate dissertations in the School of Informatics. It also emphasises the
page limit, and that you must not deviate from the required style.
The file \texttt{skeleton.tex} generates this document and should be used as a
starting point for your thesis. Replace this abstract text with a concise
summary of your report.
}

\maketitle

\newenvironment{ethics}
   {\begin{frontenv}{Research Ethics Approval}{\LARGE}}
   {\end{frontenv}\newpage}

\begin{ethics}
This project was planned in accordance with the Informatics Research
Ethics policy. It did not involve any aspects that required approval
from the Informatics Research Ethics committee.

\standarddeclaration
\end{ethics}


\begin{acknowledgements}
Any acknowledgements go here.
\end{acknowledgements}


\tableofcontents
\end{preliminary}


\chapter{Introduction}

\textit{System F} is a theoretical model of computation first described independently by
Jean \citet{girard_interpretation_1972} and John \citet{goos_towards_1974}. A previous formalisation of this
in the proof-assistant Agda has been presented in \citep{hutton_system_2019}, but the authors used a
syntax involving only De Bruijn indices.

The \textit{locally nameless representation} is an alternative representation for the
$\lambda$-calculus which uses De Bruijn indices for bound variables and identifiers (as in, regular
names) for free variables. As described by Arthur \citet{chargueraud_locally_2012}, this can also be
used for System F.

Most of the prior literature in this representation has been made in Coq. A generalisation of this
syntax, called \textit{locally nameless sets}, has been explored in Agda previously by Andrew
\citet{pitts_locally_2023}. While this generalisation didn't specifically focus on the
$\lambda$-calculus, it proves that the syntax is able to be used effectively in Agda.

We present the first formalisation of System F using the locally nameless representation in Agda.

This document is a literate Agda file and uses {\color{violet}colour}. Please see appendix
\ref{appendix:compilation_instructions} for details.

\begin{code}
module dissertation where
\end{code}

\chapter{Background}

\section{Agda}
Agda is a dependently-typed functional programming language based on Martin--Löf type theory, which
makes it suitable as a proof-assistant using intuitionistic logic \citep{norell_towards_2007}.

In Agda, most definitions are done inductively, that is, using recursion. For example, following the
Peano axioms for the natural numbers $\nat$ \citep{boolos_freges_1995}, we may define them like so.

\begin{code}
module Example where
  data ℕ : Set where
    zero : ℕ
    suc  : ℕ → ℕ
\end{code}


\paragraph*{Propositions as types.} First noted by William A. Howard in 1969
\citep{howard_formulae-as-types_1980}, there is a direct correspondence between proofs and programs.
\citet{wadler_propositions_2015} calls this correspondence \textit{propositions as types}, known as
the Curry-Howard correspondence by some authors. Thanks to this correspondence, a correct Agda
function with the appropriate type signature matching the claim (proposition) suffices for a proof.
For example, a proof that addition is associative would use recursion, which corresponds to
induction, as shown below.

\begin{code}
  -- We can import from the standard library, here we're using
  -- the reflexive and congruence properties of equality.
  open import Relation.Binary.PropositionalEquality
    using (_≡_; refl; cong)

  _+_ : ℕ → ℕ → ℕ
  zero  + m = m
  suc n + m = suc (n + m)

  +-assoc : ∀ (m n p : ℕ) → (m + n) + p ≡ m + (n + p)
  +-assoc zero    n p = refl
  +-assoc (suc m) n p = cong suc (+-assoc m n p)
\end{code}

Imports from the standard library \citep{the_agda_community_agda_2024} are omitted, but are
available in the full source file (see appendix \ref{appendix:compilation_instructions}).
\begin{comment}
\begin{code}
-- Data types (naturals, strings, characters)
open import Data.Nat using (ℕ; zero; suc; _<_; _≥_; _≤_; _≤?_; _<?_; z≤n; s≤s; _⊔_)
  renaming (_≟_ to _≟ℕ_)
open import Data.Nat.Properties using (≤-refl; ≤-trans; ≤-<-trans; <-≤-trans;
  ≤-pred; +-mono-≤; <-trans; n≤1+n; m≤n⇒m≤1+n; n≮n; <⇒≢; ≰⇒>; ≮⇒≥; ≤∧≢⇒<; 1+n≢n)
open import Data.String using (String; fromList; toList) renaming (_≟_ to _≟str_;
  _++_ to _++str_; length to str-length)
open import Data.Char using (Char)
open import Data.Char.Properties using () renaming (_≟_ to _≟char_)

-- Function manipulation.
open import Function using (_∘_; flip; it)

-- Relations and predicates/decidability.
import Relation.Binary.PropositionalEquality as Eq
open Eq using (_≡_; _≢_; refl; sym; trans; cong; cong-app; cong₂)
open Eq.≡-Reasoning using (begin_; step-≡-∣; step-≡-⟩; _∎)
open import Relation.Binary.Definitions using (DecidableEquality)
open import Relation.Nullary.Decidable using (Dec; yes; no; True; False; toWitnessFalse;
  toWitness; fromWitness; ¬?; ⌊_⌋)
open import Relation.Nullary.Negation using (¬_; contradiction)
open import Data.Empty using (⊥-elim)

-- Products and exists quantifier.
open import Data.Product using (_×_; proj₁; proj₂; ∃-syntax) renaming (_,_ to ⟨_,_⟩)
open import Data.Sum using (_⊎_; inj₁; inj₂)

-- Lists.
open import Data.List using (List; []; _∷_; _++_; length)
open import Data.List.Properties using (≡-dec)
import Data.List.Membership.DecPropositional as DecPropMembership
open import Data.List.Relation.Unary.All using (All; all?; lookup)
  renaming (fromList to All-fromList; toList to All-toList; map to All-map)
open import Data.List.Relation.Unary.Any using (Any; here; there)
open import Data.List.Extrema Data.Nat.Properties.≤-totalOrder using (max; xs≤max)

-- Import list membership using List Char comparisons.
_≟lchar_ : ∀ (xs ys : List Char) → Dec (xs ≡ ys)
xs ≟lchar ys = ≡-dec (_≟char_) xs ys

open DecPropMembership _≟lchar_ using (_∈_; _∉_; _∈?_)
\end{code}
\end{comment}

\section{Cofinite Quantification}

\input{cofinite.tex}

\section{The Lambda Calculus and System F}
\subsection{The lambda calculus.}
The $\lambda$-calculus (pronounced \textit{lambda calculus}), is a theoretical model of computation
developed by Alonzo Church in the 1930s \citep{church_set_1932} and is the basis for System F. It
looks and works similarly to familiar functional programming languages, yet its definition is as
minimal as possible while still being Turing-complete. Although Turing machines would be invented
after the $\lambda$-calculus \citep{turing_computable_1937}, `turing-complete' has become a
shorthand for `universal method of computation'. Such a universal method was not Church's initial
goal, but is why we're still interested in the $\lambda$-calculus.

The $\lambda$-calculus includes three reductions. $\alpha$-conversion is the renaming of variables
such that the semantics of the program are not changed; terms which are semantically identical but
use different variable names are called $\alpha$-equivalent, and as the name would imply, this is an
equivalence relation \citep{pierce_types_2002}.

$\beta$-reduction describes function application. The evaluation rules for the $\lambda$-calculus
are as follows \citep{wadler_programming_2022} (where $N[M/x]$ means replacing the term $M$ for any occurrence of $x$ in the term $N$).

\begin{equation}
\label{equation:untyped_beta_rules}
\begin{split}
\inferrule{ }{(\lambda \, x. N)M \mapsto_{\beta} N[M/x]} \; (\beta) \quad
\inferrule{L \mapsto_{\beta} L'}{LM \mapsto_{\beta} L'M} \; (\xi_1) \quad\\
\inferrule{M \mapsto_{\beta} M'}{LM \mapsto_{\beta} LM'} \; (\xi_2) \quad
\inferrule{N \mapsto_{\beta} N'}{(\lambda \, x. N) \mapsto_{\beta} (\lambda \, x. N')} \; (\zeta)
\end{split}
\end{equation}

The last reduction is $\eta$-reduction; for all $\lambda$-abstractions $\lambda \, x. L x$, we have
that $\lambda \, x. f x \mapsto_{\eta} f$.

If further $\beta$-reducations do not simplify a term, we say that it is in its \textit{normal
form}. We shall represent an arbitrary number of successive $\beta$-reductions using the Kleene star
$\mapsto_{\beta^{\star}}$. The Church-Rosser theorem states that for any term $L$, if it
$\beta$-reduces to two terms $M$ and $N$, then there exists a common term $L'$ which both $M$ and
$N$ eventually $\beta$-reduce to \citep{church_properties_1936}. This will be an important
consideration when choosing an evaluation strategy (see section \ref{background:evaluation_strategy}.)

Notably, some terms may not have a normal form. A well-known example is \textit{little omega}
$\omega \triangleq \lambda \, x. (x x)$ and \textit{omega} (sometimes called the \textit{omega
combinator}) $\Omega \, \triangleq \, \omega \omega$. When $\Omega$ is applied to itself, it doesn't
reduce any further.

\begin{equation*}
  (\lambda \, x. (x x)) (\lambda \, x. (x x)) \quad
  \mapsto_{\beta} \quad (\lambda \, x. (x x)) (\lambda \, x. (x x))
\end{equation*}

The \textit{y-combinator} $\mathcal{Y} \triangleq (\lambda \, f. (\lambda \, x. f (x x )) (\lambda
\, x. f (xx)))$ allows for recursion. It will $\beta$-reduce to the argument applied to itself,
$\mathcal{Y} L \mapsto_{\beta^{\star}} L (\mathcal{Y} L)$.

We say that these terms \textit{do not have a normal form}.

\subsection{The simply-typed lambda-calculus}
The simply-typed $\lambda$-calculus (STLC, sometimes given the symbol $\lambda^{\rightarrow}$) is an
extension of the untyped $\lambda$-calculus developed by Alonzo \citet{church_formulation_1940}. It
requires each term to have a \textit{type}. In the STLC, terms which do not have a normal form
cannot be given a type, thus, all expressions will eventually reduce to an irreducible
expression---their normal form. This is also called \textit{strong normalisation}
\citep{pierce_types_2002}. However, this means we can longer represent all of the terms of the
untyped $\lambda$-calculus (such as $\Omega$), thus, Turing-completeness is lost.

In the STLC, some base types need to be chosen. These are indeterminates which are not given
definitions. While Church originally used the symbols $\iota$ and $\sigma$ for base types, we will
use $\nat$, for reasons given in section \ref{background:evaluation_strategy}. Without base types,
our computational model becomes \textit{degenerate} (that is, there are no terms)
\citep{pierce_types_2002}. As well as the base type, we also have a function (or arrow) type. For
some types $\tau$ and $\sigma$, writing $\tau \to \sigma$ means that this term can be applied using
a term of type $\tau$ and result in a type $\sigma$.

We will also need a \textit{type context} (also called \textit{type environment}), which will
usually be given the symbol $\Gamma$ or $\Delta$ and is separated using $\vdash$. This is a partial
map from variables to types $\Gamma \colon V \to T$. Types are written next to terms using a colon.
For example, if the context contained the mapping $x \colon \nat$, then we could write a function
which applies $x$ to its argument;
\begin{equation*}
  x \colon \nat \in \Gamma \vdash (\lambda \, f \colon \nat \to \nat . f x) \colon \nat.
\end{equation*}

Sometimes we may choose to omit the type of the bound variable for clarity, so we could write
$(\lambda \, f. \lambda \, x. fx) \colon (\nat \to \nat) \to \nat$.

In the STLC, we can't give a type to little omega, since we can't give a type to both the argument
and the argument applied to itself ($(\lambda \, x  \colon ? . x x) \colon ??$). The typing rules for
STLC are given in section \ref{appendix:type_judgements}.

\subsection{System F.}
System F has been the formal background to what many modern programming
languages call \textit{generics}. For an anachronistic example, in Rust, we could write a function
which applies a function twice to an argument.

\begin{minted}[samepage]{rust}
fn twice<T, F>(f: F, x: T) -> T
where
    F: Fn(T) -> T,
{
    f(f(x))
}
\end{minted}

Since we used a \textit{type parameter} in the function's type signature (here \texttt{T}), we can
use any appropriate function which has the type signature $\texttt{T} \to \texttt{T}$. One such
function is \texttt{u64::isqrt}, the (flooring) square root function. If this function was invoked
with \texttt{twice(u64::isqrt, 81)}, its output would be \texttt{3}. In this case, the compiler can
infer that the type for \texttt{T} should be \texttt{u64}, so we don't need to specify it
explicitly.

System F is the STLC equipped with \textit{polymorphic types}, another term for type parameters. It
was independently discovered by Jean-Yves \citet{girard_interpretation_1972} and John
\citet{goos_towards_1974}. We can write this \textit{twice} function like so in System F:

\begin{equation*}
  (\Lambda \, T. \lambda \, f \colon T \to T . \lambda \, x \colon T . f (f x))
  \colon \forall T . (T \to T) \to T \to T.
\end{equation*}

Like in the Rust example, if we wanted to use this function, we would usually need to explicitly
perform a type application to specify what type we're using, though if it can be understood by the
reader from context, we will omit it. We write type applications using [square brackets]. For
instance,
\begin{align*}
  s \colon \nat \to \nat, z \colon \nat \in \Gamma \vdash
  &(\Lambda \, T. \lambda \, f \colon T \to T . \lambda \, x \colon T . f (f x))[\nat] s z \colon \nat\\
  &\mapsto_{\beta} (\lambda \, f \colon \nat \to \nat . \lambda \, x \colon \nat . f (f x)) s z \colon \nat\\
  &\mapsto_{\beta^{\star}} s s z \colon \nat.
\end{align*}


\section{Locally Nameless Representation}

\subsection{De Bruijn indices}
\label{section:background_debruijn}
Consider this expression,
\begin{equation*}
  x \colon \nat \in \Gamma \vdash (\lambda \, y \colon \nat \to \nat. y x) \colon \nat.
\end{equation*}

Suppose we move this expression into a context where $y$ is already defined,
\begin{equation*}
  x \colon \nat, y \colon \nat \to \nat \in \Gamma \vdash (\lambda \, y \colon \nat \to \nat. y x) \colon \nat.
\end{equation*}

It's unclear whether we are referring to the bound $y$ or the free $y$. One way to resolve this is
to always choose the most recently declared variable. This is called \textit{shadowing}, since the
older definitions are `in the shadow of' the most recent occurence \citep{wadler_programming_2022}.
Another approach is to use an $\alpha$-conversion. Here we can apply $y \mapsto_{\alpha} f$ to the
$\lambda$-abstraction,
\begin{equation*}
  x \colon \nat, y \colon \nat \to \nat \in \Gamma \vdash (\lambda \, f. f x) \colon \nat,
\end{equation*}
which solves the problem. We can add further assumptions to our context without affecting the
semantics of the expression \citep{pitts_locally_2023} (for example, adding $k \colon \nat$ to
$\Gamma$ doesn't cause any name conflicts). I shall call this property weakening-invariance (taken
from proof theory, where extra assumptions make a theorem weaker \citep{buss_handbook_1998}).

Since we have these $\alpha$-equivalent expressions, we can say that we have \textit{quotient}
inductive definitions \citep{aydemir_engineering_2008}; we have both an inductive definition of the
$\lambda$-calculus, but also (infinitely) many $\alpha$-equivalence classes which induce a quotient
set. In some proof assistants, this makes proofs difficult, as they don't handle quotient
definitions well \citep{pitts_locally_2023}.

We can solve this by using \textit{De Bruijn indices}, which ensure variables in the context don't
cause name conflicts and accidental shadowing with bound variables. In fact, $\alpha$-conversions
cannot be performed anymore. In our example,
\begin{equation*}
  x \colon \nat, y \colon \nat \to \nat \in \Gamma \vdash (\lambda \, 0 2) \colon \nat.
\end{equation*}

However, if the context changes, the indices would need to change too. For example, if we remove the
$y$, then we need to reindex the $2$ to a $1$:
\begin{equation*}
  x \colon \nat \in \Gamma \vdash (\lambda \, 0 1) \colon \nat,
\end{equation*}
We have lost weakening-invariance \citep{aydemir_engineering_2008}. This leaves a demand for a
syntax which solves both the issue of quotient definitions and weakening-invariance.

\subsection{Locally Nameless Representation}
Using \textit{locally nameless representation}, we can get both purely inductive definitions
\textit{and} weakening-invariance. Free variables use variable names while bound variables use
indices. Our example becomes
\begin{equation*}
  x \colon \nat, \in \Gamma \vdash (\lambda \, 0 x) \colon \nat,
\end{equation*}
or with another variable in the context,
\begin{equation*}
  x \colon \nat, y \colon \nat \to \nat \in \Gamma \vdash (\lambda \, 0 x) \colon \nat.
\end{equation*}

As part of using locally nameless terms, common operations and properties (called `infrastructure'
by \citet{aydemir_engineering_2008}) need to be defined for the target language. A recent article by
Andrew \citet{pitts_locally_2023} uses Agda to explore an abstraction of this representation called
locally nameless sets, and shows that this infrastructure can be defined in a syntax-agnostic way.

The drawback of using this representation is that we need to define infrastructure and prove some
properties of the syntax before we can prove the properties of the metalangauge. But usually, this
drawback is too small to outweight the benefits.

\paragraph*{Opening and closing.} Two fundamental operations on locally nameless terms are
\textit{term opening} and \textit{term closing} \citep{pitts_locally_2023}. Opening will replace all
occurrences of a bound variable with a free variable, written $[i \to x] L$ for some De Bruijn
index $i$, name $x$, and term $L$. For example,
\begin{equation*}
  [0 \to y] (0 q (\lambda \, 1 t 0)) \mapsto y q (\lambda \, y t 0).
\end{equation*}
Every occurence of the bound variable $0$ is replaced with the free variable $y$. Note how after we
go under a new $\lambda$-abstraction, we have to increment this index to $1$ to keep referring to
the same bound variable.

Closing works similarly, and is the inverse of opening; it replaces all occurrences of a free
variable to a given index. For example,
\begin{equation*}
  [0 \leftarrow y] (y q (\lambda \, y t 0)) \mapsto 0 q (\lambda \, 1 t 0).
\end{equation*}

\paragraph*{Local closure.} A term is said to be \textit{locally closed} up to level $i$ if it
remains unchanged after opening it at index $i$ with an arbitrary string. For some $i \in \nat$ and
term $L$, we write that $L$ is locally closed at level $i$ as $i \succ L$. Formally,
\begin{equation}
  \label{equation:local_closure}
  i \succ L \triangleq \forall j \geq i, \; \cof a , \; [j \to a] L = L.
\end{equation}
If it is locally closed at level $0$, we simply call it \textit{locally closed}. Local closure is
defined for System F in section \ref{chapter4:local_closure}.

\paragraph*{Induction Principle.} Often with locally nameless terms, the induction principle is
similar to how it is for named terms in all cases but the $\lambda$-abstraction case. This is
because it is often defined using cofinite quantification and opening. As an example, take type
judgements for the STLC, defined in appendix \ref{appendix:type_judgements}. In regular notation,
these are:
\begin{equation}
\begin{gathered}
  \inferrule
    { }
    {\Gamma \vdash \texttt{‵zero} \colon \nat}
    \; (\vdash\texttt{zero})
  \quad
  \inferrule{x \colon A \in \Gamma}
    {\Gamma \vdash x \colon A}
    \; (\vdash\text{free})
  \quad
  \inferrule
    {\cof x , \; (\Gamma , \, x \colon A \vdash [0 \to x] L \colon B)}
    {\Gamma \vdash \lambda \, L \colon A \to B}
    \; (\vdash \lambda)
  \\
  \inferrule
    {\Gamma \vdash L \colon (A \to B) \\ \Gamma \vdash M \colon A}
    {\Gamma \vdash LM \colon B}
    \; (\vdash\text{app})
  \quad
  \quad
  \inferrule
    {\Gamma \vdash L \colon \nat}
    {\Gamma \vdash \texttt{‵suc} \, L \colon \nat}
    \; (\vdash\texttt{suc})
\end{gathered}
\end{equation}

The only difference to the named representation is the rule for $\lambda$-abstractions
($\vdash\lambda$). Like with the named representation, we intend to prove that the term $L$ is
well-typed with a free variable $x$ added to the context in place of the bound variable
\citep[chapter~Lambda]{wadler_programming_2022}. However, to replace the bound varaible $0$ with
$x$, we first need to open the term $L$ with the name $x$ at index $0$. Furthermore, we are
replacing the bound $0$ with an arbitrary name $x$, so to make sure that the choice of $x$ is
arbitrary, we need cofinite quantification (we cannot use $\forall$ because otherwise we would run
into unintentional shadowing of free variables in the context).

So, when proving the $\lambda$-abstraction case, the induction principle would allow us to assume
that the term $L$ is well-typed when, for cofinite $x$, the term $L$ is opened $[0 \to x] L$. Or in
other words, given a property $P$ over type judgements, we need to prove that $\cof x , \; P(\Gamma
, \, x \colon A \vdash [0 \to x] L \colon B) \implies P(\Gamma \vdash \lambda \, L \colon A \to B)$.
An example of this is given in chapter \ref{chapter3:stlc_sub_and_eval}.


\section{Evaluation strategy}
\label{background:evaluation_strategy}
The idea of reducible expressions, called a \textit{redex}, was introduced by Alonzo Church
\cite[p.~56]{pierce_types_2002}. We refer to any expression of the form $(\lambda \, x. M) N$ as a
redex, since we can perform a $\beta$-reduction (see equation \ref{equation:untyped_beta_rules}).

There are several ways of evaluating the $\lambda$-calculus. One important decision is whether to
treat $\lambda$-abstractions as values, or to reduce these if they are a redex.

When evaluating, we need to consider some terms to be \textit{values}. These cannot be reduced any
further. One method is to consider $\lambda$-abstractions to be values and to only evaluate
\textit{closed} terms, that is, terms without any free variables. This is called weak-head
reduction, and the values are called weak-head normal forms \citep{wadler_programming_2022}. This
method is also used by popular programming languages such as Haskell \citep{hutchison_sharing_2005}.
The definition of $\lambda$-calculus as Church first described it uses what is nowadays referred to
as \textit{full normalisation} \citep{wadler_programming_2022}. The reduction rule which is absent
in weak-head reduction is the $\zeta$ rule (see equation \ref{equation:untyped_beta_rules}), also
called \textit{reducing under a $\lambda$-abstraction}.

Using weak-head reduction, since evaluation stops at $\lambda$-abstractions, the Church-Rosser
property no longer applies (and, in fact, the system is no longer Turing complete). One of the early
explorations of weak-head reduction was made by \citet{cagman_combinatory_1998}, comparing it to
combinatory logic. They presented the necessity of another reduction rule which dictates when one
can reduce under a $\lambda$-abstraction, which restores the Church-Rosser property
\citep{hutchison_sharing_2005}.

However, another, simpler, method is to introduce primitives \citep{wadler_programming_2022}. These
primitives are a small subset of Dana Scott and Gordon Plotkin's Programming Computable Functions
(PCF) \citep{plotkin_lcf_1977}; we introduce a value called \texttt{‵zero} of type $\nat$ and a
function \texttt{‵suc} of type $\nat \to \nat$ (since these are modelled after the naturals, we use
the base type $\nat$). Further description is in section \ref{appendix:stlc_terms}.

Weak-head reduction was used by \citet{chargueraud_locally_2012} and
\citet[chapter~Lambda]{wadler_programming_2022}, so we choose to follow the same approach here.


\section{Prior research}
System F was previously formalised in Agda by \citet{hutton_system_2019}. However, the authors
formalised a variant of System F with language extensions known as \textit{System F$_{\omega \mu}$}.
They also used a different approach, opting to make use of De Bruijn indices only. This paper will
use the locally nameless representation, which hasn't been used for System F in Agda before.

As mentioned before, \citet{pitts_locally_2023} explored a generalisation called Locally Nameless
Sets in Agda. Since the scope of his work was only to prove properties of the syntax and not of any
lanugage in particular, the helpfulness of his development is limited. By relying on only the
opening and closing operations, one cannot manipulate language constructs enough to prove properties
such as \texttt{subst-open}, as explained by \citet{kiselyov_generic_2025}. Although they present a
library which does prove properties such as \texttt{subst-open}, their work is written in Coq and so
cannot be used here.

I have previously submitted some work on STLC using locally nameless representation for a course
called \textit{Types and Semantics of Programming Languages} (TSPL) \citep{ausel_simply_2025}. While
this was very helpful in preparing my formalisation of System F, detailing it is not a prerequisite.
The only functions which will be reused here are those for generating fresh variables (see appendix
\ref{appendix:tspl}).

\chapter{System F}
Combining System F with the locally nameless representation was previously described by
\citet{chargueraud_locally_2012}.
\begin{code}
module chapter4 where
  open import cofinite
  open import plfa_adaptions using (All-++; ++-All; ∉-++;
    ++-∉; ∈-++ˡ; ∈-++ʳ; ∈-swap; ∉y∷ys⇒≢y; ∉y∷ys⇒∉ys;
    m+1≤n⇒m≤n; ∉∷[]⇒≢; ≢∧∉⇒∉∷; ∈∷[]⇒≡; ∉y∷ys⇒∉y∷[];
    ∈y∷ys∧≢y⇒∈ys; ∈-≡; ++-idʳ; _⊆_; ≡-⊆; ++-⊆; ⊆-++; ⊆⇒⊆∷;
    ⊆⇒⊆++; x∈xs∧xs⊆ys⇒x∈ys)
  open import tspl_prior_work
    using (suc-preserves-≢; sym-≢; fresh; fresh-correct)
\end{code}

\section{Syntax of types and terms}
To represent identifiers, we'll use \texttt{List Char}, and thus define a type alias. I have
previously worked on functions to generate new identifiers which can be found in appendix
\ref{appendix:list_functions}. \texttt{fresh} generates a new identifier and \texttt{fresh-correct}
gives a proof that this identifier does not exist in the given list.
\begin{code}
  open import tspl_prior_work
    using (suc-preserves-≢; sym-≢; fresh; fresh-correct)
  Id : Set
  Id = List Char
\end{code}

In System F, types become part of the syntax, as they now become a part of how terms are built.
Note, some authors use $\Pi$ for the $\forall$ type instead, such as \citet{hutton_system_2019}, and
some use the lowercase $\lambda$ for both term and type abstraction, such as
\citet{pierce_types_2002}.
\begin{code}
  data Type : Set where
    ‵ℕ      : Type                 -- Base type.
    t-fr     : Id → Type           -- Free type variables.
    t-#      : ℕ → Type            -- Bound type variables.
    _⇒_      : Type → Type → Type  -- Arrow types.
    t-∀_     : Type → Type         -- "For all" type.

  data Term : Set where
    fr     : Id → Term
    #      : ℕ → Term
    ƛ_     : Term → Term
    _·_    : Term → Term → Term
    Λ_     : Term → Term
    _[_]   : Term → Type → Term
    ‵zero  : Term
    ‵suc_  : Term → Term
\end{code}

We also define some helper functions for reasoning about equality.
\begin{code}
  ⇒-inj : ∀ {A B A' B'}→ A ⇒ B ≡ A' ⇒ B' → (A ≡ A') × (B ≡ B')
  ⇒-inj refl = ⟨ refl , refl ⟩
  ∀-inj : ∀ {A B} → t-∀ A ≡ t-∀ B → A ≡ B
  ∀-inj refl = refl
  ƛ-inj : ∀ {L M} → ƛ L ≡ ƛ M → (L ≡ M)
  ƛ-inj refl = refl
  ·-inj : ∀ {L M L' M'} → L · M ≡ L' · M' → (L ≡ L') × (M ≡ M')
  ·-inj refl = ⟨ refl , refl ⟩
  Λ-inj : ∀ {L M} → Λ L ≡ Λ M → L ≡ M
  Λ-inj refl = refl
  []-inj : ∀ {L M A B} → L [ A ] ≡ M [ B ] → (L ≡ M) × (A ≡ B)
  []-inj refl = ⟨ refl , refl ⟩
  ‵suc-inj : ∀ {L M} → ‵suc L ≡ ‵suc M → L ≡ M
  ‵suc-inj refl = refl
\end{code}

Since there are two different types of free and bound variables now, outside of Agda code, I shall
refer to free variables in lowercase and free type variables in UPPERCASE, and bound variables using
regular arabic numerals ($0$) and bound type variables using bold arabic numerals ($\mathbf{0}$). So
for example,

\begin{equation*}
  \text{id} \triangleq (\Lambda \lambda 0) \colon \forall \mathbf{0} \to \mathbf{0}
\end{equation*}
\begin{code}
  id : Term
  id = Λ ƛ # 0
\end{code}

\section{Opening}
In System F, we only require the opening operation \citep{chargueraud_locally_2012}. There are three
kinds of opening:
\begin{itemize}
  \item opening a type at an index by replacing the bound type variable at that index with a free
        type variable,
  \item opening a term at an index by replacing the bound type variable at that index in the term
        with a free type variable, and
  \item opening a term at an index by replacing the bound variable at that index with a free
        variable.
\end{itemize}

\begin{code}
  ty-ty[_—→_]_ : ℕ → Id → Type → Type
  ty-ty[ i —→ x ] ‵ℕ = ‵ℕ
  ty-ty[ i —→ x ] (t-fr y) = t-fr y
  ty-ty[ i —→ x ] (t-# n) with i ≟ℕ n
  ... | yes _ = t-fr x
  ... | no  _ = t-# n
  ty-ty[ i —→ x ] (A ⇒ B) = (ty-ty[ i —→ x ] A) ⇒ (ty-ty[ i —→ x ] B)
  ty-ty[ i —→ x ] t-∀ A = t-∀ (ty-ty[ suc i —→ x ] A)

  ty-tm[_—→_]_ : ℕ → Id → Term → Term
  ty-tm[ i —→ x ] (fr y) = fr y
  ty-tm[ i —→ x ] (# n) = # n
  ty-tm[ i —→ x ] (ƛ L) = ƛ (ty-tm[ i —→ x ] L)
  ty-tm[ i —→ x ] (L · M) = (ty-tm[ i —→ x ] L) · (ty-tm[ i —→ x ] M)
  ty-tm[ i —→ x ] (Λ L) = Λ ty-tm[ suc i —→ x ] L
  ty-tm[ i —→ x ] (L [ A ]) = (ty-tm[ i —→ x ] L) [ (ty-ty[ i —→ x ] A) ]
  ty-tm[ i —→ x ] ‵zero = ‵zero
  ty-tm[ i —→ x ] (‵suc L) = ‵suc ty-tm[ i —→ x ] L

  tm-tm[_—→_]_ : ℕ → Id → Term → Term
  tm-tm[ i —→ x ] (fr y) = fr y
  tm-tm[ i —→ x ] (# n) with i ≟ℕ n
  ... | yes _ = fr x
  ... | no  _ = # n
  tm-tm[ i —→ x ] (ƛ L) = ƛ tm-tm[ (suc i) —→ x ] L
  tm-tm[ i —→ x ] (L · M) =
    (tm-tm[ i —→ x ] L) · (tm-tm[ i —→ x ] M)
  tm-tm[ i —→ x ] (Λ L) = Λ (tm-tm[ i —→ x ] L)
  tm-tm[ i —→ x ] (L [ A ]) = (tm-tm[ i —→ x ] L) [ A ]
  tm-tm[ i —→ x ] ‵zero = ‵zero
  tm-tm[ i —→ x ] (‵suc L) = ‵suc tm-tm[ i —→ x ] L
\end{code}

When \citet{pitts_locally_2023} defines locally nameless sets, he requires that they follow 9
axioms. In our development we will only need one of these, axiom 1. Proving the others is possible,
but not necessary. Axiom 1 states that applying two openings of the same index to a type is the same
as applying the first one only. The proof is done by induction on the structure of the type with
each case being trivial (proven using reflexivity).
\begin{code}
  ax1-ty : ∀ (i : ℕ) (a b : Id) (A : Type)
    → ty-ty[ i —→ a ] (ty-ty[ i —→ b ] A) ≡ ty-ty[ i —→ b ] A
  ax1-ty i a b ‵ℕ = refl
  ax1-ty i a b (t-fr x) = refl
  ax1-ty i a b (t-# k) with i ≟ℕ k
  ... | yes _   = refl
  ... | no  i≢k with i ≟ℕ k
  ...   | yes refl = contradiction refl i≢k
  ...   | no  _    = refl
  ax1-ty i a b (A ⇒ B)
    rewrite ax1-ty i a b A | ax1-ty i a b B = refl
  ax1-ty i a b (t-∀ A) rewrite ax1-ty (suc i) a b A = refl
\end{code}

The same can be said of the other kinds of opening. Repeating proofs for the three kinds of opening
will be a recurring theme. The proofs for the other two opening operations are very similar and left
in appendix \ref{appendix:f_local_closure}.
\begin{code}
  ax1-tm : ∀ (i : ℕ) (a b : Id) (L : Term)
    → tm-tm[ i —→ a ] (tm-tm[ i —→ b ] L) ≡ tm-tm[ i —→ b ] L
  ax1-ty-tm : ∀ (i : ℕ) (a b : Id) (L : Term)
    → ty-tm[ i —→ a ] (ty-tm[ i —→ b ] L) ≡ ty-tm[ i —→ b ] L
\end{code}

\section{Local closure}
\label{chapter3:local_closure}
Recall that a term is locally closed at level $i$ if it remains unchanged after opening it at $i$,
see equation \ref{equation:local_closure}. We will have three kinds of local closure, one for each
kind of opening. In Agda, we can make use of \textit{instance arguments} to automatically search for
a evidence that is in scope. In the above definitions, we try and automatically find evidence that
$j \geq i$. If Agda cannot find such evidence, it has to be provided using ⦃ these unicode curly
brackets ⦄ or \{\{ double curly brackets \}\}.

\begin{code}
  _≻ty_ : ℕ → Type → Set
  i ≻ty A = (j : ℕ) ⦃ _ : j ≥ i ⦄ → И T , (ty-ty[ j —→ T ] A ≡ A)

  _≻tm_ : ℕ → Term → Set
  i ≻tm L = (j : ℕ) ⦃ _ : j ≥ i ⦄ → И a , (tm-tm[ j —→ a ] L ≡ L)

  _≻ty-tm_ : ℕ → Term → Set
  i ≻ty-tm L = (j : ℕ) ⦃ _ : j ≥ i ⦄ → И T , (ty-tm[ j —→ T ] L ≡ L)

  Ty-LocallyClosed : Type → Set
  Ty-LocallyClosed A = 0 ≻ty A

  Tm-LocallyClosed : Term → Set
  Tm-LocallyClosed L = 0 ≻tm L

  Ty-Tm-LocallyClosed : Term → Set
  Ty-Tm-LocallyClosed L = 0 ≻ty-tm L
\end{code}

\citet{pitts_locally_2023} proves three lemmas about local closure that we will make use of later.
As I simply adapted these proofs with the syntax of System F defined earlier, the proofs are
detailed in the appendix \ref{appendix:f_local_closure}.

\textbf{Lemma 2.6} states that if a type is locally closed at level $i$, then it is also locally
closed at any level greater than $i$. The proof follows from the definition of local closure.
\begin{code}
  lemma2·6-ty : ∀ {i j L}
    → j ≥ i
    → i ≻ty L
      -------
    → j ≻ty L
\end{code}
The same can be said of the other kinds of local closure.
\begin{code}
  lemma2·6-tm : ∀ {i j L}
    → j ≥ i → i ≻tm L → j ≻tm L
  lemma2·6-ty-tm : ∀ {i j L}
    → j ≥ i → i ≻ty-tm L → j ≻ty-tm L
\end{code}

\textbf{Lemma 2.7-1} states that if a type with a string leaves it unchanged, then opening it with
any other string will also leave it unchanged. Once again, the same can be said of the other local
closure properties.
\begin{code}
  lemma2·7-1-ty : ∀ {i x y L}
    → ty-ty[ i —→ x ] L ≡ L
    → ty-ty[ i —→ y ] L ≡ L

  lemma2·7-1-tm : ∀ {i x y L}
    → tm-tm[ i —→ x ] L ≡ L → tm-tm[ i —→ y ] L ≡ L
  lemma2·7-1-ty-tm : ∀ {i x y L}
    → ty-tm[ i —→ x ] L ≡ L → ty-tm[ i —→ y ] L ≡ L
\end{code}

And \textbf{Lemma 2.7-2} states that if a type is locally closed at level $i$, then opening it at
$i$ or any greater index will leave the type unchanged. And the same applies to the other local
closure properties.
\begin{code}
  lemma2·7-2-ty : ∀ {i j x A}
    → j ≥ i       → i ≻ty A
      ---------------------
    → ty-ty[ j —→ x ] A ≡ A

  lemma2·7-2-tm : ∀ {i j x L} → j ≥ i
    → i ≻tm L → tm-tm[ j —→ x ] L ≡ L
  lemma2·7-2-ty-tm : ∀ {i j x L} → j ≥ i
    → i ≻ty-tm L → ty-tm[ j —→ x ] L ≡ L
\end{code}

\textit{As as aside, we could use instances so these three lemmas need only be proven once each. But
since this is the only repetition it would save, it is not worth the effort for this implementation.
See [TODO] for more detail.}

We also have some local closure properties which will become useful in later proofs. I would like to
draw particular attention to the first property, that if a type $\forall A$ is locally closed up to
$i$, then $A$ is locally closed up to $i + 1$. Intuitively, this is because we increment the local
closure index when we go under a $\Lambda$-abstraction. The property does not follow from Lemma 2.6
as that would prove $i \succ \forall A$ implies $i + 1 \succ \forall A$ without removing the
$\forall$.
\begin{code}
  i≻∀A⇒si≻A : ∀ {A i} → i ≻ty (t-∀ A) → (suc i) ≻ty A
  i≻∀A⇒si≻A {A} i≻∀ (suc j) =
    -- "it" retrieves instance evidence that is not in scope.
    let И⟨ Иe₁ , Иe₂ ⟩ = i≻∀ j ⦃ ≤-pred it ⦄
    in И⟨ Иe₁ , (λ a {a∉} → ∀-inj (Иe₂ a {a∉})) ⟩
\end{code}

Recall the definition of local closure. To prove that something is locally closed up to $i$, we take
a $j \in \nat$ that is greater than $i$ as an argument, and then prove that the type remains
unchanged when opened at $j$ with a cofinite identifier. Since we need to prove that $A$ is locally
closed up to $i + 1$, we can deconstruct and write \texttt{suc j} instead. Using $j$ with the
hypothesis gives a proof that $\forall ([suc j \to a] A) = \forall A$. To obtain this proof, we need
to choose some identifier which is not in the set \texttt{Иe₁}. By constructing a cofinite
quantification ourselves, we can use the set \texttt{Иe₁} as our finite set of exceptions and choose
an identifier that isn't in \texttt{Иe₁}, which we give the variable name $a$. Then we can use the
injectivity of \texttt{t-∀} to get a proof of $([suc j \to a] A) = A$, which is exactly what is
required.

We can prove a similar properties for \texttt{ƛ} and \texttt{Λ} in the same way.
\begin{code}
  ≻ƛ⇒s≻ƛ : ∀ {L i} → i ≻tm (ƛ L) → (suc i) ≻tm L
  ≻ƛ⇒s≻ƛ ≻ƛ (suc j) = let И⟨ Иe₁ , Иe₂ ⟩ = ≻ƛ j ⦃ ≤-pred it ⦄
    in И⟨ Иe₁ , (λ a {a∉} → ƛ-inj (Иe₂ a {a∉})) ⟩

  ≻Λ⇒s≻Λ : ∀ {L i} → i ≻ty-tm (Λ L) → (suc i) ≻ty-tm L
  ≻Λ⇒s≻Λ i≻Λ (suc j) = let И⟨ Иe₁ , Иe₂ ⟩ = i≻Λ j ⦃ ≤-pred it ⦄
    in И⟨ Иe₁ , (λ a {a∉} → Λ-inj (Иe₂ a {a∉})) ⟩
\end{code}

For some terms, opening will not increment the index, and so the proof is even simpler.
\begin{code}
  Λ-≻ : ∀ {L i} → i ≻tm (Λ L) → i ≻tm L
  Λ-≻ i≻ j = let И⟨ Иe₁ , Иe₂ ⟩ = i≻ j
    in И⟨ Иe₁ , (λ a {a∉} → Λ-inj (Иe₂ a {a∉})) ⟩

  ‵suc-≻ : ∀ {L i} → i ≻tm (‵suc L) → i ≻tm L
  ‵suc-≻ i≻ j = let И⟨ Иe₁ , Иe₂ ⟩ = i≻ j
    in И⟨ Иe₁ , (λ a {a∉} → ‵suc-inj (Иe₂ a {a∉})) ⟩

  []-≻ : ∀ {L A i} → i ≻tm (L [ A ]) → i ≻tm L
  []-≻ {L} {A} {i} i≻ j = let И⟨ Иe₁ , Иe₂ ⟩ = i≻ j
    in И⟨ Иe₁ , (λ a {a∉} → proj₁ ([]-inj (Иe₂ a {a∉}))) ⟩
\end{code}

The two below properties are trivial, since the opening operation will do nothing as there are no
indices.
\begin{code}
  n≻‵ℕ : ∀ {n : ℕ} → n ≻ty ‵ℕ
  n≻‵ℕ j = И⟨ [] , (λ _ → refl) ⟩

  ty-fr-lc : ∀ {A} → Ty-LocallyClosed (t-fr A)
  ty-fr-lc j = И⟨ [] , (λ _ → refl) ⟩
\end{code}

And finally, we have a proof that for syntax which consists of two subexpressions, local closure can
be moved between the main expression and the subexpressions. This follows the same method of using
the hypothesis and injectivity.
\begin{code}
  ⇒-≻ : ∀ {A B i} → i ≻ty (A ⇒ B) → (i ≻ty A) × (i ≻ty B)
  ⇒-≻ {A} {B} {i} i≻A⇒B = ⟨ i≻A , i≻B ⟩
    where
      i≻A : i ≻ty A
      i≻A j = let И⟨ Иe₁ , Иe₂ ⟩ = i≻A⇒B j
        in И⟨ Иe₁ , (λ a {a∉} → proj₁ (⇒-inj (Иe₂ a {a∉}))) ⟩
      i≻B : i ≻ty B
      i≻B j = let И⟨ Иe₁ , Иe₂ ⟩ = i≻A⇒B j
        in И⟨ Иe₁ , (λ a {a∉} → proj₂ (⇒-inj (Иe₂ a {a∉}))) ⟩

  ≻-⇒ : ∀ {A B i} → i ≻ty A → i ≻ty B → i ≻ty (A ⇒ B)
  ≻-⇒ i≻A i≻B j =
    let И⟨ A-Иe₁ , A-Иe₂ ⟩ = i≻A j
        И⟨ B-Иe₁ , B-Иe₂ ⟩ = i≻B j
    in И⟨ A-Иe₁ ++ B-Иe₁ , (λ a {a∉} →
      let ⟨ a∉A , a∉B ⟩ = ∉-++ a∉
      in cong₂ _⇒_ (A-Иe₂ a {a∉A}) (B-Иe₂ a {a∉B})) ⟩

  ·-≻ : ∀ {L M i} → i ≻tm (L · M) → (i ≻tm L) × (i ≻tm M)
  ·-≻ {L} {M} {i} i≻ = ⟨ i≻L , i≻M ⟩
    where
      i≻L : i ≻tm L
      i≻L j = let И⟨ Иe₁ , Иe₂ ⟩ = i≻ j
        in И⟨ Иe₁ , (λ a {a∉} → proj₁ (·-inj (Иe₂ a {a∉}))) ⟩
      i≻M : i ≻tm M
      i≻M j = let И⟨ Иe₁ , Иe₂ ⟩ = i≻ j
        in И⟨ Иe₁ , (λ a {a∉} → proj₂ (·-inj (Иe₂ a {a∉}))) ⟩
\end{code}

One important lemma which will be used later is called \texttt{open-rec-lc-lemma} by
\citet{chargueraud_locally_2012} (his \texttt{open-rec-lc} is in fact Pitts lemma 2.6 proven above).
It states that if applying an opening to an opened type leaves it unchanged, then the opening will
have no effect on the type. This is, in fact, true for combinations of term-opening too, so we need
several versions of the proof.

We cannot use a generic proof because it is proven by induction on the structure of the term or
type. For the base cases, the proof is immediate by reflexivity, and for the inductive cases, the
inductive hypothesis immediately gives the result.
\begin{code}
  open-rec-lc-lemma : ∀ {L : Term} {i j u v} → i ≢ j
    → tm-tm[ i —→ u ] (tm-tm[ j —→ v ] L) ≡ tm-tm[ j —→ v ] L
    → tm-tm[ i —→ u ] L ≡ L
  open-rec-lc-lemma {fr x} i≢j assump = refl
  open-rec-lc-lemma {# k} {i} {j} i≢j assump
    with i ≟ℕ j | i ≟ℕ k
  ... | yes refl | _ = contradiction refl i≢j
  ... | no _     | no _ = refl
  ... | no _     | yes refl with j ≟ℕ k
  ...   | yes refl = contradiction refl i≢j
  ...   | no j≢k with k ≟ℕ k
  ...     | yes refl with () ← assump
  ...     | no  k≢k  = contradiction refl k≢k
  open-rec-lc-lemma {ƛ L} {i} {j} i≢j assump
    rewrite open-rec-lc-lemma {L} {suc i} {suc j}
      (suc-preserves-≢ i≢j)
      (ƛ-inj assump) = refl
  open-rec-lc-lemma {L · M} i≢j assump rewrite
      open-rec-lc-lemma {L} i≢j (proj₁ (·-inj assump))
    | open-rec-lc-lemma {M} i≢j (proj₂ (·-inj assump)) = refl
  open-rec-lc-lemma {Λ L} i≢j assump
    rewrite open-rec-lc-lemma {L} i≢j (Λ-inj assump) = refl
  open-rec-lc-lemma {L [ x ]} i≢j assump
    rewrite open-rec-lc-lemma {L} i≢j (proj₁ ([]-inj assump)) = refl
  open-rec-lc-lemma {‵zero} i≢j assump = refl
  open-rec-lc-lemma {‵suc L} i≢j assump
    rewrite open-rec-lc-lemma {L} i≢j (‵suc-inj assump) = refl
\end{code}

The proofs of the related versions of this lemma are left in appendix \ref{appendix:f_local_closure} as they are very similar.
\begin{code}
  open-rec-lc-lemma-ty : ∀ {A : Type} {i j u v} → i ≢ j
    → ty-ty[ i —→ u ] (ty-ty[ j —→ v ] A) ≡ ty-ty[ j —→ v ] A
    → ty-ty[ i —→ u ] A ≡ A
  open-rec-lc-lemma-ty-tm : ∀ {L : Term} {i j u v} → i ≢ j
    → ty-tm[ i —→ u ] (ty-tm[ j —→ v ] L) ≡ ty-tm[ j —→ v ] L
    → ty-tm[ i —→ u ] L ≡ L
  open-rec-lc-lemma-ty-tm-tm-tm : ∀ {L : Term} {i j u v}
    → ty-tm[ i —→ u ] (tm-tm[ j —→ v ] L) ≡ tm-tm[ j —→ v ] L
    → ty-tm[ i —→ u ] L ≡ L
\end{code}

\section{Free variables}
We want to collect free variables in terms and free type variables in terms and types. As before,
there are three such functons we need to define. The full definitions of the other two functions is
left in appendix \ref{appendix:f_free_variables}.
\begin{code}
  fv-tm : Term → List Id
  fv-tm (fr x) = x ∷ []
  fv-tm (# i) = []
  fv-tm (ƛ L) = fv-tm L
  fv-tm (L · M) = fv-tm L ++ fv-tm M
  fv-tm (Λ L) = fv-tm L
  fv-tm (L [ A ]) = fv-tm L
  fv-tm ‵zero = []
  fv-tm (‵suc L) = fv-tm L

  ftv-ty : Type → List Id
  ftv-tm : Term → List Id
\end{code}
\begin{comment}
\begin{code}
  -- We're actually gonna something really sneaky here.
  -- We're including the full body here because Agda needs it to
  -- be defined so that it can use the definition in later proofs.
  -- So I'm just gonna copy-paste the body as non-Agda code later.

  ftv-ty ‵ℕ = []
  ftv-ty (t-fr x) = x ∷ []
  ftv-ty (t-# i) = []
  ftv-ty (A ⇒ B) = ftv-ty A ++ ftv-ty B
  ftv-ty (t-∀ A) = ftv-ty A

  ftv-tm (fr x) = []
  ftv-tm (# i) = []
  ftv-tm (ƛ L) = ftv-tm L
  ftv-tm (L · M) = ftv-tm L ++ ftv-tm M
  ftv-tm (Λ L) = ftv-tm L
  ftv-tm (L [ A ]) = ftv-tm L ++ ftv-ty A
  ftv-tm ‵zero = []
  ftv-tm (‵suc L) = ftv-tm L
\end{code}
\end{comment}

\section{Substitution of types and terms}
We can substitute a term for a free variable, and a type for a free type variable in a term and a
type. Thus, we have three functions. We define them recursively.
\begin{code}
  tm-tm[_:=_]_ : Id → Term → Term → Term
  tm-tm[ x := N ] (fr y) with x ≟lchar y
  ... | yes refl = N
  ... | no  _    = fr y
  tm-tm[ x := N ] (# k) = # k
  tm-tm[ x := N ] (ƛ L) = ƛ tm-tm[ x := N ] L
  tm-tm[ x := N ] (L · M) = (tm-tm[ x := N ] L) · (tm-tm[ x := N ] M)
  tm-tm[ x := T ] (Λ L) = Λ (tm-tm[ x := T ] L)
  tm-tm[ x := T ] (L [ A ]) = (tm-tm[ x := T ] L) [ A ]
  tm-tm[ x := N ] ‵zero = ‵zero
  tm-tm[ x := N ] (‵suc L) = ‵suc tm-tm[ x := N ] L
\end{code}
The definition of the other two substitution functions is in appendix
\ref{appendix:f_substitution_of_terms}.
\begin{code}
  ty-ty[_:=_]_ : Id → Type → Type → Type
  ty-tm[_:=_]_ : Id → Type → Term → Term
\end{code}
\begin{comment}
\begin{code}
  ty-ty[ X := T ] ‵ℕ = ‵ℕ
  ty-ty[ X := T ] (t-fr Y) with X ≟lchar Y
  ... | yes refl = T
  ... | no  _    = t-fr Y
  ty-ty[ X := T ] (t-# k) = t-# k
  ty-ty[ X := T ] (A ⇒ B) = (ty-ty[ X := T ] A) ⇒ (ty-ty[ X := T ] B)
  ty-ty[ X := T ] (t-∀ A) = t-∀ (ty-ty[ X := T ] A)

  ty-tm[ X := T ] (fr x) = fr x
  ty-tm[ X := T ] (# k) = # k
  ty-tm[ X := T ] (ƛ L) = ƛ (ty-tm[ X := T ] L)
  ty-tm[ X := T ] (L · M) = (ty-tm[ X := T ] L) · (ty-tm[ X := T ] M)
  ty-tm[ X := T ] (Λ L) = Λ (ty-tm[ X := T ] L)
  ty-tm[ X := T ] (L [ A ]) = (ty-tm[ X := T ] L) [ ty-ty[ X := T ] A ]
  ty-tm[ X := T ] ‵zero = ‵zero
  ty-tm[ X := T ] (‵suc L) = ‵suc ty-tm[ X := T ] L
\end{code}
\end{comment}

This substitution is central to proving evaluation properties, so we'll require two lemmas about how
it interacts with local closure. The first proves that if types $A$ and $C$ are locally closed up to
$i$, then the substitution $[X := C] A$ is locally closed up to $i$ too. The proof is a consequence
of lemma 2.6 and is done by induction on the structure of types. Since it's a very straight-forward
proof, it has been detailed in appendix \ref{appendix:f_substitution_of_terms}.
\begin{code}
  :=-≻ : ∀ {A X C i j} → j ≥ i → i ≻ty A → i ≻ty C
    → j ≻ty (ty-ty[ X := C ] A)
\end{code}

The second is how substitution is idempotent if the free type variable that is being substituted for
isn't present in the type. While this is also true for the other forms of substitution, we only need
the result for type-in-type substitution. The proof is a trivial induction.
\begin{code}
  :=-∉-idempotent : ∀ {A X B} → X ∉ ftv-ty A
    → (ty-ty[ X := B ] A) ≡ A
  :=-∉-idempotent {‵ℕ} X∉A = refl
  :=-∉-idempotent {t-fr Y} {X} X∉A with X ≟lchar Y
  ... | yes refl = contradiction refl (∉∷[]⇒≢ X∉A)
  ... | no  X≢Y  = refl
  :=-∉-idempotent {t-# k} X∉A = refl
  :=-∉-idempotent {A ⇒ B} {X} {B = C} X∉ = cong₂ _⇒_
    (:=-∉-idempotent {A} (proj₁ (∉-++ X∉)))
    (:=-∉-idempotent {B} (proj₂ (∉-++ X∉)))
  :=-∉-idempotent {t-∀ A} X∉A =
    cong t-∀_ (:=-∉-idempotent {A} X∉A)
\end{code}

And we also have an alternative substitution; substituting a term for a bound variable.
\citet{chargueraud_locally_2012} also calls this operation opening, since it's the same as opening
but instead of replacing the index with a free (type) varaible, it is replaced with a term. I
believe this operation is closer to substitution than opening, and thus will call it substitution.
Notice that we still increase the index when going under a $\lambda$-abstraction.
\begin{code}
  tm-tm[_:→_]_ : ℕ → Term → Term → Term
  tm-tm[ k :→ N ] (fr x) = fr x
  tm-tm[ k :→ N ] (# i) with k ≟ℕ i
  ... | yes refl = N
  ... | no  _    = # i
  tm-tm[ k :→ N ] (ƛ L) = ƛ tm-tm[ suc k :→ N ] L
  tm-tm[ k :→ N ] (L · M) = (tm-tm[ k :→ N ] L) · (tm-tm[ k :→ N ] M)
  tm-tm[ k :→ N ] (Λ L) = Λ tm-tm[ k :→ N ] L
  tm-tm[ k :→ N ] (L [ A ]) = (tm-tm[ k :→ N ] L) [ A ]
  tm-tm[ k :→ N ] ‵zero = ‵zero
  tm-tm[ k :→ N ] (‵suc L) = ‵suc tm-tm[ k :→ N ] L
\end{code}
The definition of the other two substitution functions is in the appendix \ref{appendix:f_substitution_of_terms}.
\begin{code}
  ty-ty[_:→_]_ : ℕ → Type → Type → Type
  ty-tm[_:→_]_ : ℕ → Type → Term → Term
\end{code}
\begin{comment}
\begin{code}
  ty-ty[ k :→ T ] ‵ℕ = ‵ℕ
  ty-ty[ k :→ T ] (t-fr x) = t-fr x
  ty-ty[ k :→ T ] (t-# i) with k ≟ℕ i
  ... | yes refl = T
  ... | no  _    = t-# i
  ty-ty[ k :→ T ] (A ⇒ B) = (ty-ty[ k :→ T ] A) ⇒ (ty-ty[ k :→ T ] B)
  ty-ty[ k :→ T ] (t-∀ A) = t-∀ (ty-ty[ (suc k) :→ T ] A)

  ty-tm[ k :→ T ] (fr x) = fr x
  ty-tm[ k :→ T ] (# i) = # i
  ty-tm[ k :→ T ] (ƛ L) = ƛ ty-tm[ k :→ T ] L
  ty-tm[ k :→ T ] (L · M) = (ty-tm[ k :→ T ] L) · (ty-tm[ k :→ T ] M)
  ty-tm[ k :→ T ] (Λ L) = Λ ty-tm[ suc k :→ T ] L
  ty-tm[ k :→ T ] (L [ A ]) = (ty-tm[ k :→ T ] L) [ ty-ty[ k :→ T ] A ]
  ty-tm[ k :→ T ] ‵zero = ‵zero
  ty-tm[ k :→ T ] (‵suc L) = ‵suc ty-tm[ k :→ T ] L
\end{code}
\end{comment}

Similar to before, we want to prove that if an index doesn't occur in a type, then substituting for
that index will leave the type unchanged. The proof once again is a trivial induction and so left in appendix \ref{appendix:f_substitution_of_terms}.
\begin{code}
  ≻⇒:→-idempotent : ∀ {C i j} (A : Type)
    → j ≥ i       → i ≻ty A
      ---------------------
    → ty-ty[ j :→ C ] A ≡ A
\end{code}

\section{Type contexts}
Type contexts work as they did for the STLC. We can either add a free variable of some type, or a
free type variable to the context.
\begin{code}
  data Context : Set where
    ∅ : Context
    _,_⦂_ : Context → Id → Type → Context
    _,_ : Context → Id → Context
\end{code}

Contexts are isomorphic to lists \citep{wadler_programming_2022}, so we can concatenate them and
define map over the types of free variables. We will need this later in proofs.
\begin{code}
  _+_ : Context → Context → Context
  Γ + ∅ = Γ
  Γ + (Δ , x ⦂ A) = (Γ + Δ) , x ⦂ A
  Γ + (Δ , X) = (Γ + Δ) , X

  map : ∀ (f : Type → Type) → Context → Context
  map f ∅ = ∅
  map f (Γ , x ⦂ A) = (map f Γ) , x ⦂ f A
  map f (Γ , X) = (map f Γ) , X
\end{code}

\paragraph*{Ok environments.} We need a predicate to tell when an environment is well-formed.
Namely, we want to ensure we don't have duplicate free type variables and that any free variables
whose type contains a free type variable only occurs \textit{after} said free type variable has been
added to the context (or in other words, that the free type variable is \textit{in scope}). Since
the STLC has no free type variables, there wasn't a need for this predicate.

Before we can define the predicate, however, we need to be able to reason about the domain of a
context. We're only interested in the free type variables that are contained in a context
(\texttt{domain-ftv}) and those contained in the types of the free variables in the context
(\texttt{domain-all-ftv}). We also have a few properties of concatenation and mapping.
\begin{code}
  domain-ftv : Context → List Id
  domain-ftv ∅ = []
  domain-ftv (ctx , x ⦂ A) = domain-ftv ctx
  domain-ftv (ctx , X) = X ∷ domain-ftv ctx

  domain-all-ftv : Context → List Id
  domain-all-ftv ∅ = []
  domain-all-ftv (ctx , x ⦂ A) = (ftv-ty A) ++ domain-all-ftv ctx
  domain-all-ftv (ctx , X) = X ∷ domain-all-ftv ctx

  domain-ftv-map-idempotent : ∀ {Γ f}
    → domain-ftv Γ ≡ (domain-ftv (map f Γ))
  domain-ftv-map-idempotent {∅} = refl
  domain-ftv-map-idempotent {Γ , x ⦂ A} =
    domain-ftv-map-idempotent {Γ}
  domain-ftv-map-idempotent {Γ , X} =
    cong (X ∷_) (domain-ftv-map-idempotent {Γ})

  domain-++ : ∀ (Γ Δ : Context)
    → (domain-ftv Δ) ++ (domain-ftv Γ) ≡ domain-ftv (Γ + Δ)
  domain-++ Γ ∅ = refl
  domain-++ Γ (Δ , x ⦂ A) = domain-++ Γ Δ
  domain-++ Γ (Δ , Y) = cong (Y ∷_) (domain-++ Γ Δ)

  domain-ftv-++ʳ : ∀ {X} (Γ Δ : Context)
    → X ∈ (domain-ftv Δ) → X ∈ domain-ftv (Γ + Δ)
  domain-ftv-++ʳ {X} Γ Δ X∈Δ = ∈-≡ (∈-++ˡ X∈Δ) (domain-++ Γ Δ)
\end{code}

We closely follow how \citet{chargueraud_locally_2012} defines the `Ok' predicate.
\begin{code}
  data Ok : Context → Set where
    ok-∅ : Ok ∅
    ok-∷fv : ∀ {Γ A x} → Ok Γ → Ty-LocallyClosed A
      → ftv-ty A ⊆ domain-ftv Γ → Ok (Γ , x ⦂ A)
    ok-∷ftv : ∀ {Γ X} → Ok Γ → X ∉ domain-all-ftv Γ → Ok (Γ , X)
\end{code}

For example, we can show that a context is `Ok' like so.
\begin{code}
  _ : Ok ((∅ , ('T' ∷ [])) + (∅ , ('x' ∷ []) ⦂ (t-fr ('T' ∷ []))))
  _ = ok-∷fv
    (ok-∷ftv ok-∅ (λ ())) ty-fr-lc ((here refl) All.∷ All.[])
\end{code}

We also have two convenience functions which are simple consequences of the definition of `Ok'.
\begin{code}
  ok-+ : ∀ {Γ Δ} → Ok (Γ + Δ) → Ok Γ
  ok-+ {Γ} {∅} okΓ = okΓ
  ok-+ {Γ} {Δ , x ⦂ A} (ok-∷fv ok+ _ _) = ok-+ ok+
  ok-+ {Γ} {Δ , X} (ok-∷ftv ok+ _) = ok-+ ok+

  extract-⊆ : ∀ {Γ x A} → Ok (Γ , x ⦂ A) → ftv-ty A ⊆ domain-ftv Γ
  extract-⊆ (ok-∷fv okΓ lc-A A⊆Γ) = A⊆Γ
\end{code}

Accessing the context is done in two different ways:
\begin{itemize}
  \item If a free variable is required, then the accessor is the constructor \texttt{\_∋\_⦂\_}.
  \item If a free type variable is required, then the accessor is the constructor \texttt{\_∋\_}.
\end{itemize}

Each accessor has three constructors; one for getting the variable at the top of the context, one
for going deeper, and one for changing between free and free type variables. As for the STLC, we
provide some convenience functions with the \texttt{′} character to find the relevant evidence
automatically. These are adapted from \citet[chapter~Lambda]{wadler_programming_2022}.
\begin{code}
  data _∋_ : Context → Id → Set where
    Z : ∀ {Γ X} → (Γ , X) ∋ X
    S : ∀ {Γ X Y} → Γ ∋ X → (Γ , Y) ∋ X
    S⦂ : ∀ {Γ X y B} → Γ ∋ X → (Γ , y ⦂ B) ∋ X

  infix 4 _∋_⦂_
  data _∋_⦂_ : Context → Id → Type → Set where
    H : ∀ {Γ x y A} → x ≡ y → (Γ , x ⦂ A) ∋ y ⦂ A
    T : ∀ {Γ x y A B}
      → x ≢ y → (Γ ∋ x ⦂ A) → (Γ , y ⦂ B) ∋ x ⦂ A
    T⦂ : ∀ {Γ x Y A} → (Γ ∋ x ⦂ A) → (Γ , Y) ∋ x ⦂ A

  H′ : ∀ {Γ x A} → (Γ , x ⦂ A) ∋ x ⦂ A
  H′ = H refl

  T′ : ∀ {Γ x y A B} {x≢y : False (x ≟lchar y)}
    → Γ ∋ x ⦂ A → (Γ , y ⦂ B) ∋ x ⦂ A
  T′ { x≢y = x≢y } x = T (toWitnessFalse x≢y) x
\end{code}

Here's an example of accessing the context.
\begin{code}
  _ : ((∅ , ('f' ∷ []) ⦂ (‵ℕ ⇒ ‵ℕ)) , ('T' ∷ []))
        , ('x' ∷ []) ⦂ t-fr ('T' ∷ [])
    ∋ ('f' ∷ []) ⦂ (‵ℕ ⇒ ‵ℕ)
  _ = T′ (T⦂ H′)
\end{code}

We also need some functions for switching between evidence that a variable is in the context and
evidence that a variable is in the domain of the context.
\begin{code}
  ∋⇒∈ : ∀ {Γ X} → Γ ∋ X → X ∈ domain-ftv Γ
  ∋⇒∈ Z = here refl
  ∋⇒∈ (S ∋X) = there (∋⇒∈ ∋X)
  ∋⇒∈ (S⦂ ∋X) = ∋⇒∈ ∋X

  ∈⇒∋ : ∀ {Γ X} → X ∈ domain-ftv Γ → Γ ∋ X
  ∈⇒∋ {Γ , x ⦂ A} X∈Γ = S⦂ (∈⇒∋ X∈Γ)
  ∈⇒∋ {Γ , Y} {X = X} (here X≡Y) with X ≟lchar Y
  ... | yes refl = Z
  ... | no  X≢Y  = contradiction X≡Y X≢Y
  ∈⇒∋ {Γ , Y} (there X∈Γ) = S (∈⇒∋ X∈Γ)

  ∉-domain-all-∋ : ∀ {Γ X x A} → X ∉ domain-all-ftv Γ → Γ ∋ x ⦂ A
    → X ∉ ftv-ty A
  ∉-domain-all-∋ X∉Γ (H x) = proj₁ (∉-++ X∉Γ)
  ∉-domain-all-∋ X∉Γ (T _ ∋x) =
    ∉-domain-all-∋ (proj₂ (∉-++ X∉Γ)) ∋x
  ∉-domain-all-∋ X∉Γ (T⦂ ∋x) =
    ∉-domain-all-∋ (proj₂ (∉-++ X∉Γ)) ∋x

  ⊆-change-ctx : ∀ {Γ A Δ} → ftv-ty A ⊆ domain-ftv Γ
    → (∀ {X} → Γ ∋ X → Δ ∋ X)
    → ftv-ty A ⊆ domain-ftv Δ
  ⊆-change-ctx {Γ} {A} A⊆Γ ρ =
    All-map (λ px → ∋⇒∈ (ρ (∈⇒∋ px))) A⊆Γ

  ≡-with-∋-ty : ∀ {Γ x A B} → Γ ∋ x ⦂ A → A ≡ B → Γ ∋ x ⦂ B
  ≡-with-∋-ty ∋x refl = ∋x
\end{code}

We also want to prove that (for an `Ok' context) removing a free type variable from the context can
be done by applying a substitution, and importantly, that this changes the type of a free variable.
We express the context as $\Gamma, X, \Delta$ and induct on the structure of $\Delta$.
\begin{code}
  ∋-map-ftv : ∀ {Γ X x A C} (Δ : Context)
    → Ok ((Γ , X) + Δ)
    → ((Γ , X) + Δ) ∋ x ⦂ A
    → Γ + (map (ty-ty[ X := C ]_) Δ) ∋ x ⦂ (ty-ty[ X := C ] A)
  ∋-map-ftv ∅ (ok-∷ftv okΓ X∉Γ) (T⦂ ∋x) =
    ≡-with-∋-ty ∋x
      (sym (:=-∉-idempotent (∉-domain-all-∋ X∉Γ ∋x)))
\end{code}
If $\Delta$ is empty, then $X$ cannot occur in the type of $x$, as $X$ hasn't been bound when $x$
was added to the context.
\begin{code}
  ∋-map-ftv (Δ , y ⦂ B) _ (H refl) = H′
\end{code}
If the free variable is at the top of $\Delta$, then the map is applied and the variable has the
type we want, so we can simply use \texttt{H′}.
\begin{code}
  ∋-map-ftv (Δ , y ⦂ B) (ok-∷fv okΓ+Δ _ _) (T x≢y ∋x) =
    T x≢y (∋-map-ftv Δ okΓ+Δ ∋x)
  ∋-map-ftv (Δ , Y) (ok-∷ftv okΓ,X+Δ Y∉) (T⦂ ∋x) =
    T⦂ (∋-map-ftv Δ okΓ,X+Δ ∋x)
\end{code}
In the other cases, we simply induct.

The other proof we need is that if the free type variables of $A$ are in $\Gamma, X, \Delta$, then
after applying a substitution, they will still be in new context. We prove this by inducting on the
structure of the type $A$.
\begin{code}
  ftv⊆dom-:= : ∀ {X A C} (Γ Δ : Context)
     → Ty-LocallyClosed C
     → ftv-ty C ⊆ domain-ftv Δ
     → Ok (Γ + (map (ty-ty[ X := C ]_) Δ))
     → ftv-ty A ⊆ domain-ftv ((Γ , X) + Δ)
       -------------------------------------------
     → ftv-ty (ty-ty[ X := C ] A)
       ⊆ domain-ftv (Γ + map (ty-ty[ X := C ]_) Δ)
  ftv⊆dom-:= {X} {‵ℕ} Γ Δ lc-C C⊆Δ okΓ+map A⊆dom = All.[]
\end{code}
If the type if $\nat$, then it's vacuously true.
\begin{code}
  ftv⊆dom-:= {X} {t-fr Y} {C} Γ Δ _ C⊆Δ _ (Y∈ All.∷ All.[])
    with X ≟lchar Y
  ... | yes refl = All-map
    (λ x∈ → domain-ftv-++ʳ Γ (map (ty-ty[ X := C ]_) Δ)
      (∈-≡ x∈ (domain-ftv-map-idempotent {Δ} {ty-ty[ X := C ]_})))
    C⊆Δ
  ... | no  X≢Y  = (helper Δ Y∈ (sym-≢ X≢Y)) All.∷ All.[]
    where
      helper : ∀ {Γ X Y} (Δ : Context)
        → Y ∈ domain-ftv ((Γ , X) + Δ)
        → Y ≢ X
        → Y ∈ domain-ftv (Γ + map (ty-ty[_:=_]_ X C) Δ)
      helper ∅ (here refl) Y≢X = contradiction refl Y≢X
      helper ∅ (there Y∈) Y≢X = Y∈
      helper (Δ , z ⦂ C) Y∈ Y≢X = helper Δ Y∈ Y≢X
      helper (Δ , W) (here refl) Y≢X = here refl
      helper (Δ , W) (there Y∈) Y≢X = there (helper Δ Y∈ Y≢X)
\end{code}
If it's a free type variable, then we need to check whether it's equal to $X$. If it is, we use the
assumption \texttt{ftv-ty C ⊆ domain-ftv Δ} and apply the mapping to the context.
\begin{code}
  ftv⊆dom-:= {X} {t-# n} {C} Γ Δ lc-C C⊆Δ okΓ+map A⊆dom = All.[]
\end{code}
If it's a bound type variable, it is again vacuously true.
\begin{code}
  ftv⊆dom-:= {X} {A ⇒ B} {C} Γ Δ lc-C C⊆Δ okΓ+map ⊆dom =
    let ⟨ A⊆ , B⊆ ⟩ = ⊆-++ ⊆dom
    in ++-⊆
      (ftv⊆dom-:= {A = A} Γ Δ lc-C C⊆Δ okΓ+map A⊆)
      (ftv⊆dom-:= {A = B} Γ Δ lc-C C⊆Δ okΓ+map B⊆)
  ftv⊆dom-:= {X} {t-∀ A} {C} Γ Δ lc-C C⊆Δ okΓ+map A⊆dom =
    ftv⊆dom-:= {X} {A} {C} Γ Δ lc-C C⊆Δ okΓ+map A⊆dom
\end{code}
The last cases are the inductive steps and are trivial.

\section{Type judgements}
\label{chapter3:type_judgements}
Modifying the ones presented in \citet{chargueraud_locally_2012}, since he uses subtyping, System F
has these type judgements (a version with more familiar mathematical notation is given in appendix \ref{appendix:f_type_judgements}):
\begin{code}
  infix  4 _⊢_⦂_
  data _⊢_⦂_ : Context → Term → Type → Set where
    ⊢free : ∀ {Γ x A}
      → Ok Γ
      → Ty-LocallyClosed A
      → Γ ∋ x ⦂ A
        ------------------
      → Γ ⊢ fr x ⦂ A

    ⊢ƛ : ∀ {Γ L A B}
      → Ty-LocallyClosed A
      → И x , (Γ , x ⦂ A ⊢ tm-tm[ 0 —→ x ] L ⦂ B)
        ----------------------------------------
      → Γ ⊢ ƛ L ⦂ (A ⇒ B)

    ⊢· : ∀ {Γ L M A B}
      → Γ ⊢ L ⦂ (A ⇒ B)
      → Γ ⊢ M ⦂ A
        ---------------
      → Γ ⊢ L · M ⦂ B

    ⊢Λ : ∀ {Γ L B}
      → И T , ((Γ , T) ⊢ ty-tm[ 0 —→ T ] L ⦂ (ty-ty[ 0 —→ T ] B))
        --------------------------------------------------------
      → Γ ⊢ Λ L ⦂ t-∀ B

    ⊢[] : ∀ {Γ L A B}
      → Ty-LocallyClosed A
      → ftv-ty A ⊆ domain-ftv Γ
      → Γ ⊢ L ⦂ t-∀ B
        -------------------------------
      → Γ ⊢ L [ A ] ⦂ ty-ty[ 0 :→ A ] B

    ⊢zero : ∀ {Γ}
      → Ok Γ
        --------------
      → Γ ⊢ ‵zero ⦂ ‵ℕ

    ⊢suc : ∀ {Γ L}
      → Γ ⊢ L ⦂ ‵ℕ
        ---------------
      → Γ ⊢ ‵suc L ⦂ ‵ℕ
\end{code}

The first five rules are the same as they were for the STLC (with the exception of the locally
closed requirement ($0 \succ A$) and that the ($\vdash \lambda$) case was adapted to the new syntax,
it includes a `$\colon A$' after the $\lambda$ in the consequent). Notice how the two new rules for
type abstraction are similar to the two rules for $\lambda$-abstraction. For the ($\vdash\Lambda$)
judgement, both the term and the type need to be opened with the cofinite type variable $T$ because
bound type variables can be referenced in both the term and the type. For the ($\vdash[]$)
judgement, if one were to rewrite the antecedant type in fully named syntax, it would be $\Gamma
\vdash L \colon \forall A. B$, which more closely mirrors the ($\vdash\text{app}$) judgement. There
is only one part to the antecedant because we don't keep track of free type variables in the
context.

The only real difference between these and the rules presented by \citet{chargueraud_locally_2012}
(besides the substyping) is that we only allow a type-application for types where the free variables
are entirely contained in the context.

We can give a type to \textit{twice} and $\Omega$ now.
\begin{align}
\label{equation:twice_big_omega}
  \text{twice} &\triangleq (\Lambda \lambda \colon \mathbf{0} \to \mathbf{0}. \lambda \colon \mathbf{0}. 1(10))
    \colon \forall (\mathbf{0} \to \mathbf{0}) \to \mathbf{0} \to \mathbf{0}\\
  \Omega &\triangleq (\lambda \colon (\forall \mathbf{0} \to \mathbf{0}). (0 [\forall \mathbf{0} \to \mathbf{0}]) 0)
    \colon (\forall \mathbf{0} \to \mathbf{0}) \to (\forall \mathbf{0} \to \mathbf{0}) \notag
\end{align}
\begin{code}
  module example where
    twice : ∅ ⊢ (Λ ƛ (ƛ ((# 1) · ((# 1) · (# 0)))))
        ⦂ t-∀ (((t-# 0) ⇒ (t-# 0))
          ⇒ ((t-# 0) ⇒ (t-# 0)))
    twice = ⊢Λ И⟨ [] , (λ X {X∉} → ⊢ƛ fr⇒fr-lc И⟨ [] , (λ f {f∉} →
      ⊢ƛ ty-fr-lc И⟨ (f ∷ []) , (λ x {x∉} → ⊢·
        (⊢free ok-ctx fr⇒fr-lc (T (f≢x x∉) H′))
        (⊢·
          (⊢free ok-ctx fr⇒fr-lc (T (f≢x x∉) H′))
          (⊢free ok-ctx ty-fr-lc H′))) ⟩) ⟩) ⟩
      where
        fr⇒fr-lc : ∀ {A} → Ty-LocallyClosed (t-fr A ⇒ t-fr A)
        fr⇒fr-lc j = И⟨ [] , (λ _ → refl) ⟩
        f≢x : ∀ {f x} → x ∉ f ∷ [] → f ≢ x
        f≢x x∉ = sym-≢ (∉∷[]⇒≢ x∉)
        ok-ctx : ∀ {f X x}
          → Ok (((∅ , X) , f ⦂ (t-fr X ⇒ t-fr X)) , x ⦂ t-fr X)
        ok-ctx = ok-∷fv
                   (ok-∷fv (ok-∷ftv ok-∅ (λ ()))
                     fr⇒fr-lc
                     (here refl All.∷ here refl All.∷ All.[]))
                   ty-fr-lc
                   ((here refl) All.∷ All.[])

    Ω : ∅ ⊢ (ƛ (((# 0) [ t-∀ ((t-# 0) ⇒ (t-# 0)) ]) · (# 0)))
        ⦂ (t-∀ ((t-# 0) ⇒ (t-# 0))) ⇒ (t-∀ ((t-# 0) ⇒ (t-# 0)))
    Ω = ⊢ƛ lc-forall И⟨ [] , (λ x {x∉} →
      ⊢·
        (⊢[] lc-forall All.[] (⊢free ok-ctx lc-forall H′))
        (⊢free ok-ctx lc-forall H′)) ⟩
      where
        lc-forall : Ty-LocallyClosed (t-∀ (t-# 0 ⇒ t-# 0))
        lc-forall j = И⟨ [] , (λ _ → refl) ⟩
        ok-ctx : ∀ {x} →  Ok (∅ , x ⦂ (t-∀ (t-# 0 ⇒ t-# 0)))
        ok-ctx = ok-∷fv ok-∅ lc-forall All.[]
\end{code}

\section{Typing properties}
We first would like to show that all well-typed terms have an `Ok' environment, a simple induction
on the type judgement.
\begin{code}
  ⊢⇒Ok : ∀ {Γ L A} → Γ ⊢ L ⦂ A → Ok Γ
  ⊢⇒Ok (⊢free okΓ lc-A ∋x) = okΓ
  ⊢⇒Ok (⊢ƛ lc-A И⟨ Иe₁ , Иe₂ ⟩)
    with ⊢⇒Ok (Иe₂ (fresh Иe₁) {fresh-correct Иe₁})
  ... | ok-∷fv OkΓ _ _ = OkΓ
  ⊢⇒Ok (⊢· ⊢L ⊢LM) = ⊢⇒Ok ⊢L
  ⊢⇒Ok (⊢Λ И⟨ Иe₁ , Иe₂ ⟩)
    with ⊢⇒Ok (Иe₂ (fresh Иe₁) {fresh-correct Иe₁})
  ... | ok-∷ftv OkΓ _ = OkΓ
  ⊢⇒Ok (⊢[] lc-A _ ⊢L) = ⊢⇒Ok ⊢L
  ⊢⇒Ok (⊢zero OkΓ) = OkΓ
  ⊢⇒Ok (⊢suc ⊢L) = ⊢⇒Ok ⊢L
\end{code}

The proofs of these next properties is similar to those in \citet{wadler_programming_2022}, except
that we require more \texttt{ext} lemmas for the renaming property.
\begin{code}
  -- Extending contexts.
  ext-tm : ∀ {Γ Δ}
    → (∀ {x A}     →        Γ ∋ x ⦂ A →         Δ ∋ x ⦂ A)
    → (∀ {x y A B} → Γ , y ⦂ B ∋ x ⦂ A → Δ , y ⦂ B ∋ x ⦂ A)
  ext-tm ρ (H refl) = H′
  ext-tm ρ (T x≢y ∋x) = T x≢y (ρ ∋x)

  ext-ty : ∀ {Γ Δ}
    → (∀ {X}   →       Γ ∋ X →       Δ ∋ X)
    → (∀ {X Y} → (Γ , Y) ∋ X → (Δ , Y) ∋ X)
  ext-ty ρ Z = Z
  ext-ty ρ (S ∋X) = S (ρ ∋X)

  ext-tm-ty : ∀ {Γ Δ}
    → (∀ {x A}   →       Γ ∋ x ⦂ A →       Δ ∋ x ⦂ A)
    → (∀ {x Y A} → (Γ , Y) ∋ x ⦂ A → (Δ , Y) ∋ x ⦂ A)
  ext-tm-ty ρ (T⦂ ∋x) = T⦂ (ρ ∋x)

  ext-ty-tm : ∀ {Γ Δ}
    → (∀ {X}     →          Γ ∋ X →           Δ ∋ X)
    → (∀ {X y A} → (Γ , y ⦂ A) ∋ X → (Δ , y ⦂ A) ∋ X)
  ext-ty-tm ρ (S⦂ ∋X) = S⦂ (ρ ∋X)
\end{code}

The renaming property is a good example to explain the inductive hypothesis when dealing with
cofinite quantification.
\begin{code}
  -- Renaming (a.k.a. rebasing) contexts.
  rename : ∀ {Γ Δ}
    → Ok Δ
    → (∀ {x A} → Γ ∋ x ⦂ A → Δ ∋ x ⦂ A)
    → (∀ {X} → Γ ∋ X → Δ ∋ X)
      ---------------------------------
    → (∀ {L A} → Γ ⊢ L ⦂ A → Δ ⊢ L ⦂ A)
  rename okΔ ρ-tm ρ-ty (⊢free okΓ lc-A x) = ⊢free okΔ lc-A (ρ-tm x)
  rename {Γ} {Δ} okΔ ρ-tm ρ-ty {_} {A ⇒ B} (⊢ƛ lc-A И⟨ Иe₁ , Иe₂ ⟩) =
    ⊢ƛ lc-A И⟨ Иe₁ , (λ a {a∉} →
      rename
        (OkΔ,x a a∉)
        (ext-tm ρ-tm)
        (ext-ty-tm ρ-ty)
        (Иe₂ a {proj₂ (∉-++ a∉)})) ⟩
    where
      OkΔ,x : (x : Id) → x ∉ Иe₁ → Ok (Δ , x ⦂ A)
      OkΔ,x x x∉ with ⊢⇒Ok (Иe₂ x {x∉})
      ... | ok-∷fv OkΓ lc-A ftvA⊆Γ =
        ok-∷fv okΔ lc-A (⊆-change-ctx {Γ} {A} ftvA⊆Γ ρ-ty)
  rename okΔ ρ-tm ρ-ty (⊢· ⊢A⇒B ⊢A) =
    ⊢· (rename okΔ ρ-tm ρ-ty ⊢A⇒B) (rename okΔ ρ-tm ρ-ty ⊢A)
  rename {Γ} {Δ} okΔ ρ-tm ρ-ty (⊢Λ И⟨ Иe₁ , Иe₂ ⟩) =
    ⊢Λ И⟨ Иe₁ ++ domain-all-ftv Δ , (λ T {T∉} →
      let ⟨ T∉Иe₁ , T∉Δ ⟩ = ∉-++ T∉
      in rename
        (ok-∷ftv okΔ T∉Δ)
        (ext-tm-ty ρ-tm)
        (ext-ty ρ-ty)
        (Иe₂ T {T∉Иe₁}) ) ⟩
  rename okΔ ρ-tm ρ-ty (⊢[] lc A⊆Γ ⊢L) = ⊢[]
    lc
    (All-map (λ px → ∋⇒∈ (ρ-ty (∈⇒∋ px))) A⊆Γ)
    (rename okΔ ρ-tm ρ-ty ⊢L)
  rename okΔ ρ-tm ρ-ty (⊢zero okΓ) = ⊢zero okΔ
  rename okΔ ρ-tm ρ-ty (⊢suc ⊢L) = ⊢suc (rename okΔ ρ-tm ρ-ty ⊢L)
\end{code}
Specifically, take the case of $\vdash\Lambda$. Here, we have a proof that, if we use an identifier
which is not in the list \texttt{Иe₁} (suppose $T$), then \texttt{Иe₂} will supply us the type
judgement $\Gamma , T \vdash [0 \to T] L \colon [0 \to T] B$. We can then call the inductive
hypothesis on this type judgement; notice how we call \texttt{rename} with the argument \texttt{Иe₂
T {T∉Иe₁}}.

We have a few remaining properties. Again, the proof here is similar to how they are proven in
\citet{wadler_programming_2022}, just with the addition of the `Ok' predicate. They are all a
consequence of the \texttt{rename} lemma. The full proof for \texttt{swap} and \texttt{drop} is left
in appendix \ref{appendix:f_type_properties}.
\begin{code}
  -- Weakening contexts.
  weaken : ∀ {Γ L A} → Ok Γ → ∅ ⊢ L ⦂ A → Γ ⊢ L ⦂ A
  weaken okΓ ⊢L = rename okΓ (λ ()) (λ ()) ⊢L

  -- Swapping variables in a context.
  swap : ∀ {Γ x y L A B C}
    → x ≢ y
    → (Γ , y ⦂ B) , x ⦂ A ⊢ L ⦂ C
    → (Γ , x ⦂ A) , y ⦂ B ⊢ L ⦂ C

  swap-tm-ty : ∀ {Γ X y L B C}
    → ((Γ , y ⦂ B) , X) ⊢ L ⦂ C
    → (Γ , X) , y ⦂ B ⊢ L ⦂ C

  drop : ∀ {Γ x L A B C}
    → (Γ , x ⦂ A) , x ⦂ B ⊢ L ⦂ C
    → Γ , x ⦂ B ⊢ L ⦂ C

  -- Apply equality within type judgements.
  ≡-with-⊢-tm : ∀ {Γ L M A} → Γ ⊢ L ⦂ A → L ≡ M → Γ ⊢ M ⦂ A
  ≡-with-⊢-tm ⊢L refl = ⊢L
  ≡-with-⊢-ty : ∀ {Γ L A B} → Γ ⊢ L ⦂ A → A ≡ B → Γ ⊢ L ⦂ B
  ≡-with-⊢-ty ⊢L refl = ⊢L
  ≡-with-⊢-ctx : ∀ {Γ Δ L A} → Γ ⊢ L ⦂ A → Γ ≡ Δ → Δ ⊢ L ⦂ A
  ≡-with-⊢-ctx ⊢L refl = ⊢L
\end{code}

We will also need a lemma called \texttt{subst-open} by \citet{chargueraud_locally_2012}. This will
be used in the main substitution properties later (\texttt{subst} and \texttt{subst-ty}). The proof
uses a simple induction on the structure of the term.
\begin{code}
  subst-open : ∀ {N x y i} (L : Term)
    → x ≢ y → (i ≻tm N)
    → tm-tm[ x := N ] (tm-tm[ i —→ y ] L)
      ≡ tm-tm[ i —→ y ] (tm-tm[ x := N ] L)
  subst-open {x = x} (fr z) x≢y i≻u with x ≟lchar z
  ... | yes refl = sym (lemma2·7-2-tm ≤-refl i≻u)
  ... | no  _    = refl
  subst-open {_} {x} {y} {i} (# k) x≢y i≻u with i ≟ℕ k
  ... | no  _ = refl
  ... | yes refl with x ≟lchar y
  ...    | yes refl = contradiction refl x≢y
  ...    | no  _    = refl
  subst-open {i = i} (ƛ L) x≢y i≻u = cong ƛ_
    (subst-open L x≢y (lemma2·6-tm (n≤1+n i) i≻u))
  subst-open (L · M) x≢y i≻u = cong₂ _·_
    (subst-open L x≢y i≻u) (subst-open M x≢y i≻u)
  subst-open (Λ L) x≢y i≻u = cong Λ_ (subst-open L x≢y i≻u)
  subst-open (L [ A ]) x≢y i≻u =
    cong₂ _[_] (subst-open L x≢y i≻u) refl
  subst-open ‵zero x≢y i≻u = refl
  subst-open (‵suc L) x≢y i≻u = cong ‵suc_ (subst-open L x≢y i≻u)

  subst-open-ctx : ∀ {Γ A N x y i} (L : Term)
    → x ≢ y → (i ≻tm N)
    → Γ ⊢ tm-tm[ x := N ] (tm-tm[ i —→ y ] L) ⦂ A
    → Γ ⊢ tm-tm[ i —→ y ] (tm-tm[ x := N ] L) ⦂ A
  subst-open-ctx L x≢y i≻N assump =
    ≡-with-⊢-tm assump (subst-open L x≢y i≻N)
\end{code}

We also have these variations whose proof is placed in the appendix
\ref{appendix:f_type_properties}.
\begin{code}
  subst-open-ty-tm : ∀ {N x y i j} (L : Term)
    → x ≢ y → j ≥ i → (i ≻ty-tm N)
    → tm-tm[ x := N ] (ty-tm[ j —→ y ] L)
      ≡ ty-tm[ j —→ y ] (tm-tm[ x := N ] L)
  subst-open-ty-tm-ctx : ∀ {Γ A N x y i j} (L : Term)
    → x ≢ y → j ≥ i → (i ≻ty-tm N)
    → Γ ⊢ tm-tm[ x := N ] (ty-tm[ j —→ y ] L) ⦂ A
    → Γ ⊢ ty-tm[ j —→ y ] (tm-tm[ x := N ] L) ⦂ A

  subst-open-ty-tm-tm-tm : ∀ {C x y i j} (L : Term)
    → j ≥ i → (i ≻ty C)
    → ty-tm[ x := C ] (tm-tm[ j —→ y ] L)
      ≡ tm-tm[ j —→ y ] (ty-tm[ x := C ] L)
  subst-open-ty-tm-tm-tm-ctx : ∀ {Γ A C x y i j} (L : Term)
    → j ≥ i → (i ≻ty C)
    → Γ ⊢ ty-tm[ x := C ] (tm-tm[ j —→ y ] L) ⦂ A
    → Γ ⊢ tm-tm[ j —→ y ] (ty-tm[ x := C ] L) ⦂ A

  subst-open-ty-ty-ty-ty : ∀ {C x y i j} (A : Type)
    → x ≢ y → j ≥ i → (i ≻ty C)
    → ty-ty[ x := C ] (ty-ty[ j —→ y ] A)
      ≡ ty-ty[ j —→ y ] (ty-ty[ x := C ] A)
  subst-open-ty-ty-ty-ty-ctx : ∀ {Γ L C A x y i j}
    → x ≢ y → j ≥ i → (i ≻ty C)
    → Γ ⊢ L ⦂ ty-ty[ x := C ] (ty-ty[ j —→ y ] A)
    → Γ ⊢ L ⦂ ty-ty[ j —→ y ] (ty-ty[ x := C ] A)

  subst-open-ty-tm-ty-tm : ∀ {C x y i j} (L : Term)
    → x ≢ y → j ≥ i → (i ≻ty C)
    → ty-tm[ x := C ] (ty-tm[ j —→ y ] L)
      ≡ ty-tm[ j —→ y ] (ty-tm[ x := C ] L)
  subst-open-ty-tm-ty-tm-ctx : ∀ {Γ L A C x y i j}
    → x ≢ y → j ≥ i → (i ≻ty C)
    → Γ ⊢ ty-tm[ x := C ] (ty-tm[ j —→ y ] L) ⦂ A
    → Γ ⊢ ty-tm[ j —→ y ] (ty-tm[ x := C ] L) ⦂ A
\end{code}

To actually swap between the two substitutions, we have a lemma called \texttt{subst-intro}. It says
that, so long as we choose a fresh identifier $x$, we can swap between \texttt{:→} and \texttt{:=}
with opening. The proof is a trivial induction, and so detailed in appendix \ref{appendix:f_type_properties}.
\begin{code}
  subst-intro : ∀ {x i} (L N : Term) → x ∉ fv-tm L
    → tm-tm[ i :→ N ] L ≡ tm-tm[ x := N ] (tm-tm[ i —→ x ] L)
  subst-intro-ty-ty : ∀ {x i B} (A : Type) → x ∉ ftv-ty A
    → ty-ty[ i :→ B ] A ≡ ty-ty[ x := B ] (ty-ty[ i —→ x ] A)
  subst-intro-ty-tm : ∀ {x i B} (L : Term) → x ∉ ftv-tm L
    → ty-tm[ i :→ B ] L ≡ ty-tm[ x := B ] (ty-tm[ i —→ x ] L)
\end{code}

\section{Well-typed terms are locally closed}
If the typing rules are correct, then all well-typed terms are locally closed. Intuitively, since we
only allow closed terms in our language, then all well-typed terms are locally closed. We prove this by inducting on the type judgement.
\begin{code}
  ⊢⇒lc : ∀ {Γ L A} → Γ ⊢ L ⦂ A → Tm-LocallyClosed L
  ⊢⇒lc (⊢free okΓ lc-A ∋x) j = И⟨ [] , (λ a → refl) ⟩
  ⊢⇒lc (⊢ƛ lc-A И⟨ Иe₁ , Иe₂ ⟩) j = И⟨ Иe₁ , (λ a {a∉} → cong ƛ_
    (open-rec-lc-lemma (λ ())
      (lemma2·7-2-tm z≤n (⊢⇒lc (Иe₂ a {a∉}))))) ⟩
\end{code}
For the $\vdash\lambda$ case, since the type judgement is in the form of a cofinite quantification,
we need to create a cofinite quantification ourselves and use the same finite set for our
exceptions. Then we're given an identifier $a$ which is not in \texttt{Иe₁}. We can use this $a$ to
use the inductive hypothesis, which is called using \texttt{⊢⇒lc Иe₂ a {a∉}}. Then we simply apply
\texttt{open-rec-lc-lemma} to complete the proof for this case.
\begin{code}
  ⊢⇒lc {Γ} (⊢· ⊢A⇒B ⊢A) _ = И⟨ [] , (λ _ → cong₂ _·_
    (lemma2·7-2-tm z≤n (⊢⇒lc ⊢A⇒B))
    (lemma2·7-2-tm z≤n (⊢⇒lc ⊢A))) ⟩
  ⊢⇒lc {Γ} (⊢Λ И⟨ Иe₁ , Иe₂ ⟩) j =
    И⟨ Иe₁ , (λ a {a∉} →
      cong Λ_ (lemma2·7-2-tm z≤n
        (helper z≤n (⊢⇒lc (Иe₂ a {a∉}))))) ⟩
    where
      helper : ∀ {N i k q x} → k ≥ i → i ≻tm (ty-tm[ q —→ x ] N)
        → k ≻tm N
      helper {fr x} {k = k} k≥i i≻ty j = i≻ty k ⦃ k≥i ⦄
      helper {# n} k≥i i≻ty = lemma2·6-tm k≥i i≻ty
      helper {ƛ N} {i} {k} {q} {x} k≥i i≻ty j =
        let induc-hypo = helper {k = suc i} (≤-refl) (≻ƛ⇒s≻ƛ i≻ty)
            И⟨ Иe₁ , Иe₂ ⟩ = induc-hypo (suc j) ⦃ s≤s (≤-trans k≥i it) ⦄
        in И⟨ Иe₁ , (λ a {a∉} → cong ƛ_ (Иe₂ a {a∉})) ⟩
      helper {N · N₁} k≥i i≻ty j =
        let ⟨ i≻L  , i≻M ⟩ = ·-≻ i≻ty
            И⟨ L-Иe₁ , L-Иe₂ ⟩ = (helper k≥i i≻L) j
            И⟨ M-Иe₁ , M-Иe₂ ⟩ = (helper k≥i i≻M) j
          in И⟨ (L-Иe₁ ++ M-Иe₁) , (λ a {a∉} → cong₂ _·_
            (L-Иe₂ a {proj₁ (∉-++ a∉)})
            (M-Иe₂ a {proj₂ (∉-++ {xs = L-Иe₁} a∉)})) ⟩
      helper {Λ N} k≥i i≻ty j =
        let И⟨ Иe₁ , Иe₂ ⟩ = (helper k≥i (Λ-≻ i≻ty)) j
        in И⟨ Иe₁ , (λ a {a∉} → cong Λ_ (Иe₂ a {a∉})) ⟩
      helper {N [ A ]} k≥i i≻ty j =
        let И⟨ Иe₁ , Иe₂ ⟩ = (helper k≥i ([]-≻ i≻ty)) j
        in И⟨ Иe₁ , (λ a {a∉} → cong _[ A ] (Иe₂ a {a∉})) ⟩
      helper {‵zero} k≥i i≻ty j = И⟨ [] , (λ _ → refl) ⟩
      helper {‵suc N} k≥i i≻ty j =
        let И⟨ Иe₁ , Иe₂ ⟩ = (helper k≥i (‵suc-≻ i≻ty)) j
        in И⟨ Иe₁ , (λ a {a∉} → cong ‵suc_ (Иe₂ a {a∉})) ⟩
\end{code}
For the $\vdash\Lambda$ case, we need a helper function which is very similar to lemma 2.6. It says
that, if a term is locally closed at $i$ after opening, it will also be locally closed at $i$ (and
any greater number) for the unopened term. This is itself proven by induction on the term.
\begin{code}
  ⊢⇒lc {L = L [ A ]} (⊢[] _ _ ⊢L) j =
    let И⟨ Иe₁ , Иe₂ ⟩ = (⊢⇒lc ⊢L) j
    in И⟨ Иe₁ , (λ a {a∉} → cong _[ A ]  (Иe₂ a {a∉})) ⟩
  ⊢⇒lc (⊢zero _) j = И⟨ [] , (λ _ → refl) ⟩
  ⊢⇒lc (⊢suc ⊢L) j = let И⟨ Иe₁ , Иe₂ ⟩ = (⊢⇒lc ⊢L) j
    in И⟨ Иe₁ , (λ a {a∉} → cong ‵suc_ (Иe₂ a {a∉})) ⟩
\end{code}
The remaining cases are a simple induction, once again, using cofinite quantification by reusing the
same set of exceptions.

The proofs for the other two local closure predicates are presented in the appendix \ref{appendix:f_well_typed_lc}.
\begin{code}
  ⊢⇒lc-ty : ∀ {Γ L A} → Γ ⊢ L ⦂ A → Ty-LocallyClosed A
  ⊢⇒lc-ty-tm : ∀ {Γ L A} → Γ ⊢ L ⦂ A → Ty-Tm-LocallyClosed L
\end{code}

\section{Substitution preserves types}
Before proving the main substitution theorem, we'll need a lemma for extracting one kind of
substitution out of another. We only need the type substitution version of this property. The proof
is a simple induction and a consequence of the previously proven \texttt{≻⇒:→-idempotent}.
\begin{code}
  extract-subst : ∀ {X C A i j}
    (B : Type)
    → j ≥ i
    → i ≻ty C
    → ty-ty[ j :→ ty-ty[ X := C ] A ] (ty-ty[ X := C ] B)
        ≡ ty-ty[ X := C ] (ty-ty[ j :→ A ] B)
  extract-subst ‵ℕ j≥i i≻C = refl
  extract-subst {X} {C} (t-fr Y) j≥i i≻C with X ≟lchar Y
  ... | yes refl = ≻⇒:→-idempotent C j≥i i≻C
  ... | no  X≢Y  = refl
  extract-subst {i = i} {j = j} (t-# n) j≥i i≻C with j ≟ℕ n
  ... | yes refl = refl
  ... | no  j≢n  = refl
  extract-subst (B ⇒ B') j≥i i≻C = cong₂ _⇒_
    (extract-subst B j≥i i≻C) (extract-subst B' j≥i i≻C)
  extract-subst (t-∀ B) j≥i i≻C =
    cong t-∀_ (extract-subst B (m≤n⇒m≤1+n j≥i) i≻C)

  extract-subst-ctx : ∀ {Γ L X C B A i j}
    → j ≥ i
    → i ≻ty C
    → Γ ⊢ L ⦂ ty-ty[ j :→ ty-ty[ X := C ] A ] (ty-ty[ X := C ] B)
    → Γ ⊢ L ⦂ ty-ty[ X := C ] (ty-ty[ j :→ A ] B)
  extract-subst-ctx {B = B} j≥i i≻C assump =
    ≡-with-⊢-ty assump (extract-subst B j≥i i≻C)
\end{code}

\texttt{subst-ty} shows that we can substitute a free type variable and have the type of the term
effected accordingly, and \texttt{subst} shows that we can apply a substitution to eliminate a
function type of a term.

This first proof is done with an induction on the type judgement.
\begin{code}
  subst-ty : ∀ {Γ Δ X L B C}
    → Ty-LocallyClosed C
    → ftv-ty C ⊆ domain-ftv Δ
    → Ok (Γ + (map (ty-ty[ X := C ]_) Δ))
    → ((Γ , X) + Δ) ⊢ L ⦂ B
      --------------------
    → (Γ + (map (ty-ty[ X := C ]_) Δ))
        ⊢ ty-tm[ X := C ] L ⦂ ty-ty[ X := C ] B
  subst-ty {Γ} {Δ} {X} lc-C C⊆Δ okΓ+map (⊢free okΓ+Δ lc-B ∋x)
    with ok-+ {Γ = Γ , X} okΓ+Δ
  ... | ok-∷ftv okΓ X∉Γ =
    ⊢free okΓ+map (:=-≻ z≤n lc-B lc-C) (∋-map-ftv Δ okΓ+Δ ∋x)
\end{code}
In the $\vdash\text{free}$ case, we use the previously proven \texttt{∋-map-ftv} property to apply
the substitution to the type of the free variable.
\begin{code}
  subst-ty {Γ} {Δ} {L = ƛ L} lc-C C⊆Δ okΓ+map (⊢ƛ {A = A} lc-A И⟨ Иe₁ , Иe₂ ⟩) =
    ⊢ƛ (:=-≻ z≤n lc-A lc-C) И⟨ Иe₁ , (λ a {a∉} →
      let ⊢tm-tm[0→]L = Иe₂ a {a∉}
          ok = ⊢⇒Ok ⊢tm-tm[0→]L
      in subst-open-ty-tm-tm-tm-ctx L z≤n lc-C
        (subst-ty lc-C C⊆Δ (ok-∷fv
            okΓ+map
            (:=-≻ z≤n lc-A lc-C)
            (ftv⊆dom-:= {A = A} Γ Δ lc-C C⊆Δ okΓ+map (extract-⊆ ok)))
          ⊢tm-tm[0→]L)) ⟩
\end{code}
In the $\vdash\lambda$ case, we simply need to use the inductive hypothesis. The only tricky part
comes in proving that the context is still okay when adding the new free variable to it. Luckily, we
have already done the hard work when we proved \texttt{ftv⊆dom-:=} earlier.
\begin{code}
  subst-ty lc-C C⊆Δ okΓ+map (⊢· ⊢L ⊢M) =
    ⊢· (subst-ty lc-C C⊆Δ okΓ+map ⊢L) (subst-ty lc-C C⊆Δ okΓ+map ⊢M)
  subst-ty {Γ} {Δ} {X} {B = t-∀ B} {C = C} lc-C C⊆Δ okΓ+map (⊢Λ И⟨ Иe₁ , Иe₂ ⟩) =
    ⊢Λ И⟨ (X ∷ Иe₁) ++ domain-all-ftv (Γ + map (ty-ty[_:=_]_ X C) Δ) , (λ a {a∉} →
      let ⟨ a∉X∷Иe₁ , a∉dom-all-ftv ⟩ = ∉-++ a∉
          ⟨ a∉X∷[] , a∉Иe₁ ⟩ = ∉-++ a∉X∷Иe₁
          X≢a = sym-≢ (∉∷[]⇒≢ a∉X∷[])
          hypo = subst-ty {Δ = Δ , a} lc-C (⊆⇒⊆∷ C⊆Δ)
              (ok-∷ftv okΓ+map a∉dom-all-ftv) (Иe₂ a {a∉Иe₁})
      in subst-open-ty-tm-ty-tm-ctx X≢a z≤n lc-C
        (subst-open-ty-ty-ty-ty-ctx {A = B} {j = 0} X≢a z≤n lc-C hypo) ) ⟩
\end{code}
The $\vdash\text{app}$ case is simple. For $\vdash\Lambda$, after application of the inductive
hypothesis, we end up with a term which is of the form $\vdash [Y := C] ([i \to a] L) \colon [...]$.
But to type a $\Lambda$-abstraction, we need the opening operation to be on the outside. Luckily we
can use \texttt{subst-open} to swap these two around.
\begin{code}
  subst-ty {Γ} {Δ} {X = Y} {L = L [ A ]} {B = D}
      lc-C C⊆Δ okΓ+map (⊢[] {B = B} lc-A A⊆ ⊢L)
    = extract-subst-ctx {X = Y} {B = B} {A = A} {j = 0} z≤n lc-C
      (⊢[]
        (:=-≻ {X = Y} z≤n lc-A lc-C)
        (ftv⊆dom-:= {A = A} Γ Δ lc-C C⊆Δ okΓ+map A⊆)
        (subst-ty lc-C C⊆Δ okΓ+map ⊢L))
\end{code}
For $\vdash[]$, after applying the inductive hypothesis and the $\vdash[]$ type judgement, we get an
expression of the type \texttt{(ty-ty[ 0 :→ ty-ty[ Y := C ] A ] (ty-ty[ Y := C ] B))}. Extracting
the \texttt{:=} substitution out of the \texttt{:→} one is exactly the reason why we need to call
\texttt{extract-subst-ctx}.
\begin{code}
  subst-ty lc-C C⊆Δ okΓ+map (⊢zero okΓ+Δ) = ⊢zero okΓ+map
  subst-ty lc-C C⊆Δ okΓ+map (⊢suc ⊢L) =
    ⊢suc (subst-ty lc-C C⊆Δ okΓ+map ⊢L)
\end{code}
The last two cases are a simple application of the inductive hypothesis.

For \texttt{subst}, our proof is very similar to how it is in \citet{wadler_programming_2022}, with
cofinite induction and \texttt{subst-open} being the only notable additions. As mentioned in the
STLC section, Agda believes that the $\vdash\lambda$ case doesn't terminate. However, as we're
recursing on the deconstructed type judgement, we are ensuring that we're recursing on a strict
subexpression of the argument, which is exactly the requirement for termination that Agda uses
\citep{the_agda_community_termination_2024}.
\begin{code}
  {-# TERMINATING #-}
  subst : ∀ {Γ x L N A B}
    → ∅ ⊢ N ⦂ A
    → Γ , x ⦂ A ⊢ L ⦂ B
      --------------------
    → Γ ⊢ tm-tm[ x := N ] L ⦂ B
  subst {x = y} ⊢N (⊢free (ok-∷fv okΓ _ A⊆Γ) lc-A (H refl)) with y ≟lchar y
  ... | yes refl = weaken okΓ ⊢N
  ... | no  y≢y  = contradiction refl y≢y
  subst {x = y} ⊢N (⊢free {x = x} (ok-∷fv okΓ _ A⊆Γ) lc-A (T x≢y ∋x))
    with y ≟lchar x
  ... | yes refl = contradiction refl x≢y
  ... | no  _    = ⊢free okΓ lc-A ∋x
  subst {x = y} {L = ƛ L} ⊢N (⊢ƛ lc-A И⟨ Иe₁ , Иe₂ ⟩) =
    ⊢ƛ lc-A И⟨ (y ∷ Иe₁) , (λ a {a∉} →
      let a∉Иe₁ = proj₂ (∉-++ {xs = y ∷ []} a∉)
          y≢a = sym-≢ (∉y∷ys⇒≢y a∉)
          ⊢tm-tm[]L = swap (sym-≢ y≢a) (Иe₂ a {a∉Иe₁})
      in subst-open-ctx L y≢a (⊢⇒lc ⊢N) (subst ⊢N ⊢tm-tm[]L)) ⟩
  subst ⊢N (⊢· ⊢L ⊢M) = ⊢· (subst ⊢N ⊢L) (subst ⊢N ⊢M)
  subst {x = y} {L = Λ L} ⊢N (⊢Λ И⟨ Иe₁ , Иe₂ ⟩) =
    ⊢Λ И⟨ y ∷ Иe₁ , (λ A {A∉} →
      let A∉Иe₁ = proj₂ (∉-++ {xs = y ∷ []} A∉)
          y≢A = sym-≢ (∉y∷ys⇒≢y A∉)
          ⊢ty-tm[]L = Иe₂ A {A∉Иe₁}
          induc-hypo = subst ⊢N (swap-tm-ty ⊢ty-tm[]L)
      in subst-open-ty-tm-ctx L y≢A z≤n (⊢⇒lc-ty-tm ⊢N) induc-hypo ) ⟩
  subst ⊢N (⊢[] lc ⊆[] ⊢L) = ⊢[] lc ⊆[] (subst ⊢N ⊢L)
  subst ⊢N (⊢zero (ok-∷fv okΓ _ _)) = ⊢zero okΓ
  subst ⊢N (⊢suc ⊢L) = ⊢suc (subst ⊢N ⊢L)
\end{code}

And to put these two properties to use, we need to replace the \texttt{:→} substitution with the
\texttt{:=} substitution. To do this, we combine the above proofs and the previously proven
\texttt{subst-intro} lemma.
\begin{code}
  subst-op : ∀ {Γ L N A B}
    → ∅ ⊢ N ⦂ A
    → Γ ⊢ ƛ L ⦂ A ⇒ B
      --------------------
    → Γ ⊢ tm-tm[ 0 :→ N ] L ⦂ B
  subst-op {Γ} {L} {N} ⊢N (⊢ƛ lc-A И⟨ Иe₁ , Иe₂ ⟩) =
    let x                  = fresh (fv-tm L ++ Иe₁)
        ⟨ x∉fv-L , x∉Иe₁ ⟩ = ∉-++ {xs = fv-tm L}
                                (fresh-correct (fv-tm L ++ Иe₁))
    in ≡-with-⊢-tm (subst ⊢N (Иe₂ x {x∉Иe₁}))
      (sym (subst-intro L N (x∉fv-L)))

  subst-op-ty : ∀ {L B C}
    → Ty-LocallyClosed C
    → ftv-ty C ⊆ []
    → ∅ ⊢ Λ L ⦂ t-∀ B
      --------------------
    → ∅ ⊢ ty-tm[ 0 :→ C ] L ⦂ ty-ty[ 0 :→ C ] B
  subst-op-ty {L} {B} {C} lc-C C⊆[] (⊢Λ И⟨ Иe₁ , Иe₂ ⟩) =
    let x = fresh (fv-tm L ++ ftv-ty B ++ ftv-tm L ++ Иe₁)
        ⟨ x∉fv-L , x∉ ⟩ = ∉-++ {xs = fv-tm L}
          (fresh-correct (fv-tm L ++ ftv-ty B ++ ftv-tm L ++ Иe₁))
        ⟨ x∉ftv-B , x∉ ⟩ = ∉-++ {xs = ftv-ty B} x∉
        ⟨ x∉ftv-L , x∉Иe₁ ⟩ = ∉-++ {xs = ftv-tm L} x∉
    in ≡-with-⊢-ty (≡-with-⊢-tm
          (subst-ty {Δ = ∅} lc-C C⊆[] ok-∅ (Иe₂ x {x∉Иe₁}))
          (sym (subst-intro-ty-tm L x∉ftv-L)))
      (sym (subst-intro-ty-ty B x∉ftv-B))
\end{code}

\section{Evaluation}
\label{chapter3:evaluation}
As mentioned earlier, we will stop evaluating at abstractions.
\begin{code}
  data Value : Term → Set where
    V-ƛ : ∀ {L} → Value (ƛ L)
    V-Λ : ∀ {L} → Value (Λ L)
    V-zero : Value ‵zero
    V-suc : ∀ {L} → Value L → Value (‵suc L)
\end{code}

We have the following evaluation rules.
\begin{code}
  infix 4 _—→_
  data _—→_ : Term → Term → Set where
    ξ₁ : ∀ {L L' M}
      → L —→ L'
        -------------------
      → L · M —→ L' · M

    ξ₂ : ∀ {L M M'}
      → M —→ M'
        ---------
      → L · M —→ L · M'

    ξ-[] : ∀ {L L' A}
      → L —→ L'
        ------------------
      → L [ A ] —→ L' [ A ]

    ξ-suc : ∀ {L L'}
      → L —→ L'
        ------------------
      → ‵suc L —→ ‵suc L'

    β-ƛ : ∀ {L M}
      → 1 ≻tm L              → Value M
        ------------------------------
      → (ƛ L) · M —→ tm-tm[ 0 :→ M ] L

    β-Λ : ∀ {L A}
      → 1 ≻ty-tm L  → Ty-LocallyClosed A
      → ftv-ty A ⊆ []
        --------------------------------
      → (Λ L) [ A ] —→ ty-tm[ 0 :→ A ] L
\end{code}
Notice how we only allow type applications which do not have any free type variables. This is
because in our language, the only way to receive a free type variable is in the type judgement of a
$\Lambda$-abstraction, evaluation will only work on empty contexts.

We have reflexivity and transitivity of evaluation.
\begin{code}
  data _—↠_ : Term → Term → Set where
    _∎' : ∀ M → M —↠ M
    step—→ : ∀ L {M N} → M —↠ N → L —→ M → L —↠ N
  pattern _—→⟨_⟩_ L L—→M M—↠N = step—→ L M—↠N L—→M

  begin'_ : ∀ {M N} → M —↠ N → M —↠ N
  begin' M—↠N = M—↠N
\end{code}

The proof of progress is quite straight-forward. Compared to the proof as presented in
\citet{wadler_programming_2022}, we only need to add some properties about local closure, which is
easily obtained by \texttt{⊢⇒lc}.
\begin{code}
  data Progress (L : Term) : Set where
    done : Value L → Progress L
    step : ∀ {L'} → L —→ L' → Progress L

  progress : ∀ {L A} → ∅ ⊢ L ⦂ A → Progress L
  progress (⊢ƛ lc-A cof) = done V-ƛ
  progress (⊢· ⊢L ⊢M) with progress ⊢L
  ... | step L→L' = step (ξ₁ L→L')
  ... | done V-ƛ with progress ⊢M
  ...   | step M→M' = step (ξ₂ M→M')
  ...   | done val  = step (β-ƛ (≻ƛ⇒s≻ƛ (⊢⇒lc ⊢L)) val)
  progress (⊢Λ x) = done V-Λ
  progress (⊢[] lc-A A⊆[] ⊢L) with progress ⊢L
  ... | step L→L' = step (ξ-[] L→L')
  ... | done V-Λ = step (β-Λ (≻Λ⇒s≻Λ (⊢⇒lc-ty-tm ⊢L)) lc-A A⊆[])
  progress (⊢zero ok∅) = done V-zero
  progress (⊢suc ⊢L) with progress ⊢L
  ... | step L→L'  = step (ξ-suc L→L')
  ... | done val-L = done (V-suc val-L)
\end{code}

The proof of preservation, however, proves difficult, because it requires the proofs of substitution
which were presented earlier.
\begin{code}
  preserve : ∀ {L L' A} → ∅ ⊢ L ⦂ A → L —→ L' → ∅ ⊢ L' ⦂ A
  preserve (⊢· ⊢L ⊢M) (ξ₁ L→L') = ⊢· (preserve ⊢L L→L') ⊢M
  preserve (⊢· ⊢L ⊢M) (ξ₂ M→M') = ⊢· ⊢L (preserve ⊢M M→M')
  preserve (⊢· ⊢L ⊢M) (β-ƛ 1≻L val-M) = subst-op ⊢M ⊢L
  preserve (⊢[] lc-A A⊆[] ⊢L) (ξ-[] L→L') =
    ⊢[] lc-A A⊆[] (preserve ⊢L L→L')
  preserve (⊢[] lc-A A⊆[] ⊢L) (β-Λ 1≻L _ C⊆[]) =
    subst-op-ty lc-A C⊆[] ⊢L
  preserve (⊢suc ⊢L) (ξ-suc L→L') = ⊢suc (preserve ⊢L L→L')
\end{code}

Now we can make an evaluation function.
\begin{code}
  record Gas : Set where
    pattern
    constructor gas
    field
      amount : ℕ

  data Finished (N : Term) : Set where
    out-of-gas : Finished N
    done : Value N → Finished N

  data Steps (L : Term) : Set where
    steps : ∀ {N} → L —↠ N → Finished N → Steps L

  eval : ∀ {L A} → Gas → ∅ ⊢ L ⦂ A → Steps L
  eval {L} (gas zero) ⊢L = steps (L ∎') out-of-gas
  eval {L} (gas (suc n)) ⊢L with progress ⊢L
  ... | done val = steps (L ∎') (done val)
  ... | step {M} L→M with eval (gas n) (preserve ⊢L L→M)
  ...   | steps M→N fin = steps (L —→⟨ L→M ⟩ M→N) fin
\end{code}

As an example, we can apply the earlier defined \texttt{twice} function to apply $\lambda \;
\texttt{‵suc} \; 0$ twice to \texttt{‵zero} to get \texttt{‵suc (‵suc ‵zero)}.
\begin{code}
  ⊢twice-suc-zero :
    ∅ ⊢ (((Λ ƛ (ƛ ((# 1) · ((# 1) · (# 0))))) [ ‵ℕ ])
          · (ƛ ‵suc (# 0))) · ‵zero
        ⦂ ‵ℕ
  ⊢twice-suc-zero = ⊢·
    (⊢· (⊢[] n≻‵ℕ All.[] example.twice)
      (⊢ƛ n≻‵ℕ И⟨ [] , (λ a →
        ⊢suc (⊢free (ok-∷fv ok-∅ n≻‵ℕ All.[]) n≻‵ℕ H′)) ⟩))
    (⊢zero ok-∅)
\end{code}

The Agda-generated code is ommitted here, but one could figure this out interactively by using
Agda's normalise feature. For example, in Emacs, one can press C-c C-n and type \texttt{eval (gas 10) ⊢twice-suc-zero} to see the derivation being automatically generated.

\chapter{Conclusions}
We have successfully developed an evaluator for System F using the locally nameless representation.

\paragraph*{Future work} Most, if not all, of the prior literature on locally nameless
representation, including the work presented here, uses an extrinsically typed approach. This means
that we defined the syntax of the langauge first and then restricted ourselves to only work with
well-typed terms. Using intrinsic typing, we instead define the typing rules and immediately
restrict ourselves to only well-typed terms. For the STLC,
\citet[chapter~De~Bruijn]{wadler_programming_2022} found that this results in some theorems becoming
trivial, such as preservation, and the overall code being more concise. As preservation proved to be
the most difficult theorem in this work, this would certainly be welcome. The main problem with this
is that the locally nameless representation requires a lot of properties of syntax to be proven. We
would need two contexts, one for free variables and one for bound variables, and this may diminish
the conciseness that an intrinsic typed approach is supposed to give.

The language we presented here is plain System F, which isn't usually a subject of interest. The
POPLMark challenge asks for an implementation of System F with subtyping
\citep{chargueraud_locally_2012}; implementing this shouldn't be too hard, as it would only require
a few extra type judgements for subtyping, extra evaluation rules, and some further restrictions on
the `Ok' predicate. System F$_\omega$ is an important variant which has previously been studied by
\citet{hutton_system_2019} in Agda, but a locally nameless approach is yet to be used for such a
more complex langauge.

\bibliographystyle{plainnat}
\bibliography{dissertation}

\appendix

\chapter{Miscellaneous Proofs}
\label{appendix:misc_proofs}

Some miscellaneous proofs are required in the main chapters. Since they aren't of interest, they are presented here in the appendix.

\input{plfa_adaptions.tex}

\chapter{Miscellaneous System F proofs}
\section{Local Closure}
\label{appendix:f_local_closure}
\begin{code}
  -- lemma2·6-ty : ∀ {i j A}
  --   → j ≥ i
  --   → i ≻ty A
  --     -------
  --   → j ≻ty A
  lemma2·6-ty j≥i i≻A k =
    i≻A k ⦃ ≤-trans j≥i it ⦄

  -- lemma2·6-tm : ∀ {i j L}
  --   → j ≥ i → i ≻tm L → j ≻tm L
  lemma2·6-tm j≥i i≻L k =
    i≻L k ⦃ ≤-trans j≥i it ⦄

  -- lemma2·6-ty-tm : ∀ {i j L}
  --   → j ≥ i → i ≻ty-tm L → j ≻ty-tm L
  lemma2·6-ty-tm j≥i i≻L k =
    i≻L k ⦃ ≤-trans j≥i it ⦄

  -- lemma2·7-1-ty : ∀ {i x y A}
  --   → ty-ty[ i —→ x ] A ≡ A
  --   → ty-ty[ i —→ y ] A ≡ A
  lemma2·7-1-ty {i} {x} {y} {A} assump =
    begin
      ty-ty[ i —→ y ] A
    ≡⟨ sym (cong (ty-ty[ i —→ y ]_) assump) ⟩
      ty-ty[ i —→ y ] (ty-ty[ i —→ x ] A)
    ≡⟨ ax1-ty i y x A ⟩
      ty-ty[ i —→ x ] A
    ≡⟨ assump ⟩
      A
    ∎

  -- lemma2·7-1-tm : ∀ {i x y L}
  --   → tm-tm[ i —→ x ] L ≡ L → tm-tm[ i —→ y ] L ≡ L
  lemma2·7-1-tm {i} {x} {y} {L} assump =
    begin
      tm-tm[ i —→ y ] L
    ≡⟨ sym (cong (tm-tm[ i —→ y ]_) assump) ⟩
      tm-tm[ i —→ y ] (tm-tm[ i —→ x ] L)
    ≡⟨ ax1-tm i y x L ⟩
      tm-tm[ i —→ x ] L
    ≡⟨ assump ⟩
      L
    ∎

  -- lemma2·7-1-ty-tm : ∀ {i x y L}
  --   → ty-tm[ i —→ x ] L ≡ L → ty-tm[ i —→ y ] L ≡ L
  lemma2·7-1-ty-tm {i} {x} {y} {L} assump =
    begin
      ty-tm[ i —→ y ] L
    ≡⟨ sym (cong (ty-tm[ i —→ y ]_) assump) ⟩
      ty-tm[ i —→ y ] (ty-tm[ i —→ x ] L)
    ≡⟨ ax1-ty-tm i y x L ⟩
      ty-tm[ i —→ x ] L
    ≡⟨ assump ⟩
      L
    ∎

  -- lemma2·7-2-ty : ∀ {i j x A}
  --   → j ≥ i       → i ≻ty A
  --     ---------------------
  --   → ty-ty[ j —→ x ] L ≡ A
  lemma2·7-2-ty {j = j} j≥i i≻A =
    let И⟨ Иe₁ , Иe₂ ⟩ = i≻A j ⦃ j≥i ⦄ in
      lemma2·7-1-ty (Иe₂ (fresh Иe₁) {fresh-correct Иe₁})

  -- lemma2·7-2-tm : ∀ {i j x L} → j ≥ i
  --   → i ≻tm L → tm-tm[ j —→ x ] L ≡ L
  lemma2·7-2-tm {j = j} j≥i i≻L =
    let И⟨ Иe₁ , Иe₂ ⟩ = i≻L j ⦃ j≥i ⦄ in
      lemma2·7-1-tm (Иe₂ (fresh Иe₁) {fresh-correct Иe₁})

  -- lemma2·7-2-ty-tm : ∀ {i j x L} → j ≥ i
  --   → i ≻ty-tm L → ty-tm[ j —→ x ] L ≡ L
  lemma2·7-2-ty-tm {j = j} j≥i i≻L =
    let И⟨ Иe₁ , Иe₂ ⟩ = i≻L j ⦃ j≥i ⦄ in
      lemma2·7-1-ty-tm (Иe₂ (fresh Иe₁) {fresh-correct Иe₁})
\end{code}
\begin{code}
  -- ax1-tm : ∀ (i : ℕ) (a b : Id) (L : Term)
  --   → tm-tm[ i —→ a ] (tm-tm[ i —→ b ] L) ≡ tm-tm[ i —→ b ] L
  ax1-tm i a b (fr x) = refl
  ax1-tm i a b (# k) with i ≟ℕ k
  ... | yes refl = refl
  ... | no  i≢k  with i ≟ℕ k
  ... |   yes refl = contradiction refl i≢k
  ... |   no  _    = refl
  ax1-tm i a b (ƛ L) rewrite ax1-tm (suc i) a b L = refl
  ax1-tm i a b (L · M)
    rewrite ax1-tm i a b L | ax1-tm i a b M = refl
  ax1-tm i a b (Λ L) rewrite ax1-tm i a b L = refl
  ax1-tm i a b (L [ A ]) rewrite ax1-tm i a b L = refl
  ax1-tm i a b ‵zero = refl
  ax1-tm i a b (‵suc L) rewrite ax1-tm i a b L = refl

  -- ax1-ty-tm : ∀ (i : ℕ) (a b : Id) (L : Term)
  --   → ty-tm[ i —→ a ] (ty-tm[ i —→ b ] L) ≡ ty-tm[ i —→ b ] L
  ax1-ty-tm i a b (fr x) = refl
  ax1-ty-tm i a b (# k) = refl
  ax1-ty-tm i a b (ƛ L) rewrite
    ax1-ty-tm i a b L = refl
  ax1-ty-tm i a b (L · M) rewrite
    ax1-ty-tm i a b L | ax1-ty-tm i a b M = refl
  ax1-ty-tm i a b (Λ L)
    rewrite ax1-ty-tm (suc i) a b L = refl
  ax1-ty-tm i a b (L [ A ])
    rewrite ax1-ty i a b A | ax1-ty-tm i a b L = refl
  ax1-ty-tm i a b ‵zero = refl
  ax1-ty-tm i a b (‵suc L)
    rewrite ax1-ty-tm i a b L = refl
\end{code}

\begin{code}
  -- open-rec-lc-lemma-ty : ∀ {A i j u v}
  --   → i ≢ j
  --   → ty-ty[ i —→ u ] (ty-ty[ j —→ v ] A) ≡ ty-ty[ j —→ v ] A
  --   → ty-ty[ i —→ u ] A ≡ A
  open-rec-lc-lemma-ty {‵ℕ} i≢j assump = refl
  open-rec-lc-lemma-ty {t-fr x} i≢j assump = refl
  open-rec-lc-lemma-ty {t-# k} {i} {j} i≢j assump with i ≟ℕ k
  ... | no  i≢k  = refl
  ... | yes refl with j ≟ℕ k
  ...   | yes refl = contradiction refl i≢j
  ...   | no  j≢k with k ≟ℕ k
  ...     | yes refl with () ← assump
  ...     | no  k≢k = contradiction refl k≢k
  open-rec-lc-lemma-ty {A ⇒ B} i≢j assump rewrite
      open-rec-lc-lemma-ty {A} i≢j (proj₁ (⇒-inj assump))
    | open-rec-lc-lemma-ty {B} i≢j (proj₂ (⇒-inj assump))
    = refl
  open-rec-lc-lemma-ty {t-∀ A} {i} {j} i≢j assump
    rewrite open-rec-lc-lemma-ty {A} {suc i} {suc j}
      (suc-preserves-≢ i≢j)
      (∀-inj assump)
        = refl

  -- open-rec-lc-lemma-ty-tm : ∀ {L i j u v}
  --   → i ≢ j
  --   → ty-tm[ i —→ u ] (ty-tm[ j —→ v ] L) ≡ ty-tm[ j —→ v ] L
  --   → ty-tm[ i —→ u ] L ≡ L
  open-rec-lc-lemma-ty-tm {fr x} i≢j assump = refl
  open-rec-lc-lemma-ty-tm {# k} i≢j assump = refl
  open-rec-lc-lemma-ty-tm {ƛ L} i≢j assump rewrite
    open-rec-lc-lemma-ty-tm {L} i≢j (ƛ-inj assump) = refl
  open-rec-lc-lemma-ty-tm {L · M} i≢j assump rewrite
      open-rec-lc-lemma-ty-tm {L} i≢j (proj₁ (·-inj assump))
    | open-rec-lc-lemma-ty-tm {M} i≢j (proj₂ (·-inj assump)) = refl
  open-rec-lc-lemma-ty-tm {Λ L} i≢j assump rewrite
    open-rec-lc-lemma-ty-tm {L} (suc-preserves-≢ i≢j) (Λ-inj assump) = refl
  open-rec-lc-lemma-ty-tm {L [ A ]} i≢j assump rewrite
      open-rec-lc-lemma-ty-tm {L} i≢j (proj₁ ([]-inj assump))
    | open-rec-lc-lemma-ty {A} i≢j (proj₂ ([]-inj assump)) = refl
  open-rec-lc-lemma-ty-tm {‵zero} i≢j assump = refl
  open-rec-lc-lemma-ty-tm {‵suc L} i≢j assump rewrite
    open-rec-lc-lemma-ty-tm {L} i≢j (‵suc-inj assump) = refl

  -- open-rec-lc-lemma-ty-tm-tm-tm : ∀ {L i j u v}
  --   → ty-tm[ i —→ u ] (tm-tm[ j —→ v ] L) ≡ tm-tm[ j —→ v ] L
  --   → ty-tm[ i —→ u ] L ≡ L
  open-rec-lc-lemma-ty-tm-tm-tm {fr x} assump = refl
  open-rec-lc-lemma-ty-tm-tm-tm {# k} assump = refl
  open-rec-lc-lemma-ty-tm-tm-tm {ƛ L} assump rewrite
    open-rec-lc-lemma-ty-tm-tm-tm {L} (ƛ-inj assump) = refl
  open-rec-lc-lemma-ty-tm-tm-tm {L · M} assump rewrite
      open-rec-lc-lemma-ty-tm-tm-tm {L} (proj₁ (·-inj assump))
    | open-rec-lc-lemma-ty-tm-tm-tm {M} (proj₂ (·-inj assump))
    = refl
  open-rec-lc-lemma-ty-tm-tm-tm {Λ L} assump rewrite
    open-rec-lc-lemma-ty-tm-tm-tm {L} (Λ-inj assump) = refl
  open-rec-lc-lemma-ty-tm-tm-tm {L [ A ]} assump rewrite
      proj₂ ([]-inj assump)
    | open-rec-lc-lemma-ty-tm-tm-tm {L} (proj₁ ([]-inj assump))
    = refl
  open-rec-lc-lemma-ty-tm-tm-tm {‵zero} assump = refl
  open-rec-lc-lemma-ty-tm-tm-tm {‵suc L} assump rewrite
    open-rec-lc-lemma-ty-tm-tm-tm {L} (‵suc-inj assump) = refl
\end{code}

\section{Free variables}
\label{appendix:f_free_variables}
\begin{minted}{agda}
  -- ftv-ty : Term → List Id
  ftv-ty ‵ℕ = []
  ftv-ty (t-fr x) = x ∷ []
  ftv-ty (t-# i) = []
  ftv-ty (A ⇒ B) = ftv-ty A ++ ftv-ty B
  ftv-ty (t-∀ A) = ftv-ty A

  -- ftv-tm : Term → List Id
  ftv-tm (fr x) = []
  ftv-tm (# i) = []
  ftv-tm (ƛ L) = ftv-tm L
  ftv-tm (L · M) = ftv-tm L ++ ftv-tm M
  ftv-tm (Λ L) = ftv-tm L
  ftv-tm (L [ A ]) = ftv-tm L ++ ftv-ty A
  ftv-tm ‵zero = []
  ftv-tm (‵suc L) = ftv-tm L
\end{minted}

\section{Substitution of types and terms}
\label{appendix:f_substitution_of_terms}
\begin{minted}{agda}
  -- ty-ty[_:=_]_ : Id → Type → Type → Type
  ty-ty[ X := T ] ‵ℕ = ‵ℕ
  ty-ty[ X := T ] (t-fr Y) with X ≟lchar Y
  ... | yes refl = T
  ... | no  _    = t-fr Y
  ty-ty[ X := T ] (t-# k) = t-# k
  ty-ty[ X := T ] (A ⇒ B) = (ty-ty[ X := T ] A) ⇒ (ty-ty[ X := T ] B)
  ty-ty[ X := T ] (t-∀ A) = t-∀ (ty-ty[ X := T ] A)

  -- ty-tm[_:=_]_ : Id → Type → Term → Term
  ty-tm[ X := T ] (fr x) = fr x
  ty-tm[ X := T ] (# k) = # k
  ty-tm[ X := T ] (ƛ L) = ƛ (ty-tm[ X := T ] L)
  ty-tm[ X := T ] (L · M) = (ty-tm[ X := T ] L) · (ty-tm[ X := T ] M)
  ty-tm[ X := T ] (Λ L) = Λ (ty-tm[ X := T ] L)
  ty-tm[ X := T ] (L [ A ]) = (ty-tm[ X := T ] L) [ ty-ty[ X := T ] A ]
  ty-tm[ X := T ] ‵zero = ‵zero
  ty-tm[ X := T ] (‵suc L) = ‵suc ty-tm[ X := T ] L
\end{minted}

\begin{code}
  -- ≻⇒:→-idempotent : ∀ {C i j} (A : Type)
  --   → j ≥ i       → i ≻ty A
  --     ---------------------
  --   → ty-ty[ j :→ C ] A ≡ A
  ≻⇒:→-idempotent ‵ℕ j≥i i≻A = refl
  ≻⇒:→-idempotent (t-fr x) j≥i i≻A = refl
  ≻⇒:→-idempotent {C} {i} {j} (t-# n) j≥i i≻A with j ≟ℕ n
  ... | no  j≢n  = refl
  ... | yes refl with i≻A j ⦃ j≥i ⦄
  ...   | И⟨ Иe₁ , Иe₂ ⟩ with n ≟ℕ n
  ...     | yes refl with () ← Иe₂ (fresh Иe₁) {fresh-correct Иe₁}
  ...     | no  n≢n  = contradiction refl n≢n
  ≻⇒:→-idempotent (A ⇒ B) j≥i i≻ = let ⟨ i≻A , i≻B ⟩ = ⇒-≻ i≻
    in cong₂ _⇒_ (≻⇒:→-idempotent A j≥i i≻A) (≻⇒:→-idempotent B j≥i i≻B)
  ≻⇒:→-idempotent {i = i} {j = j} (t-∀ A) j≥i i≻ = cong t-∀_
    (≻⇒:→-idempotent A (s≤s j≥i) (i≻∀A⇒si≻A i≻))
\end{code}

\begin{code}
  -- :=-≻ : ∀ {A X C i j} → j ≥ i → i ≻ty A → i ≻ty C
  --   → j ≻ty (ty-ty[ X := C ] A)
  :=-≻ {‵ℕ} j≥i i≻A i≻C = n≻‵ℕ
  :=-≻ {t-fr Y} {X} j≥i i≻A i≻C with X ≟lchar Y
  ... | yes refl = lemma2·6-ty j≥i i≻C
  ... | no  X≢Y  = lemma2·6-ty j≥i i≻A
  :=-≻ {t-# k} j≥i i≻A i≻C = lemma2·6-ty j≥i i≻A
  :=-≻ {A ⇒ B} j≥i i≻ i≻C k =
    let ⟨ i≻A , i≻B ⟩ = ⇒-≻ i≻
        И⟨ A-Иe₁ , A-Иe₂ ⟩ = (:=-≻ j≥i i≻A i≻C) k
        И⟨ B-Иe₁ , B-Иe₂ ⟩ = (:=-≻ j≥i i≻B i≻C) k
    in И⟨ A-Иe₁ ++ B-Иe₁ , (λ a {a∉} →
      let ⟨ a∉A , a∉B ⟩ = ∉-++ a∉
      in cong₂ _⇒_ (A-Иe₂ a {a∉A}) (B-Иe₂ a {a∉B})) ⟩
  :=-≻ {t-∀ A} {i = i}  j≥i i≻A i≻C k =
    let И⟨ Иe₁ , Иe₂ ⟩ = (:=-≻ (s≤s j≥i) (i≻∀A⇒si≻A i≻A) (lemma2·6-ty (n≤1+n i) i≻C)) (suc k) ⦃ s≤s it ⦄
    in И⟨ Иe₁ , (λ a {a∉} → cong t-∀_ (Иe₂ a {a∉})) ⟩
\end{code}

\begin{minted}{agda}
  -- ty-ty[_:→_]_ : ℕ → Type → Type → Type
  ty-ty[ k :→ T ] ‵ℕ = ‵ℕ
  ty-ty[ k :→ T ] (t-fr x) = t-fr x
  ty-ty[ k :→ T ] (t-# i) with k ≟ℕ i
  ... | yes refl = T
  ... | no  _    = t-# i
  ty-ty[ k :→ T ] (A ⇒ B) = (ty-ty[ k :→ T ] A) ⇒ (ty-ty[ k :→ T ] B)
  ty-ty[ k :→ T ] (t-∀ A) = t-∀ (ty-ty[ (suc k) :→ T ] A)

  -- ty-tm[_:→_]_ : ℕ → Type → Term → Term
  ty-tm[ k :→ T ] (fr x) = fr x
  ty-tm[ k :→ T ] (# i) = # i
  ty-tm[ k :→ T ] (ƛ L) = ƛ ty-tm[ k :→ T ] L
  ty-tm[ k :→ T ] (L · M) = (ty-tm[ k :→ T ] L) · (ty-tm[ k :→ T ] M)
  ty-tm[ k :→ T ] (Λ L) = Λ ty-tm[ suc k :→ T ] L
  ty-tm[ k :→ T ] (L [ A ]) = (ty-tm[ k :→ T ] L) [ ty-ty[ k :→ T ] A ]
  ty-tm[ k :→ T ] ‵zero = ‵zero
  ty-tm[ k :→ T ] (‵suc L) = ‵suc ty-tm[ k :→ T ] L
\end{minted}

\section{Type judgements}
\label{appendix:f_type_judgements}
\begin{equation}
\begin{gathered}
  \inferrule
    {\text{Ok} \, \Gamma}
    {\Gamma \vdash \texttt{‵zero} \colon \nat}
    \; (\vdash\texttt{zero})\quad
  \inferrule
    {\Gamma \vdash L \colon \nat}
    {\Gamma \vdash \texttt{‵suc} \, L \colon \nat}
    \; (\vdash\texttt{suc})\quad
  \\
  \inferrule
    {\text{Ok} \, \Gamma\\0 \succ A\\x \colon A \in \Gamma}
    {\Gamma \vdash x \colon A}
    \; (\vdash\text{free})
  \\
  \inferrule
    {0 \succ A\\\cof x , \; (\Gamma , \, x \colon A \vdash [0 \to x] L \colon B)}
    {\Gamma \vdash (\lambda \colon A. L) \colon A \to B}
    \; (\vdash\lambda)\quad
  \inferrule
    {\Gamma \vdash L \colon A \to B\\\Gamma \vdash M \colon A}
    {\Gamma \vdash L M \colon B}
    \; (\vdash\text{app})
  \\
  \inferrule
    {\cof T, \; (\Gamma \vdash ([0 \to T] L) \colon [0 \to T] B)}
    {\Gamma \vdash (\Lambda \, L) \colon \forall B}
    \; (\vdash\Lambda)
  \\
  \inferrule
    {0 \succ A\\\text{ftv}(A) \subseteq \text{dom-ftv}(\Gamma)\\\Gamma \vdash L \colon \forall B}
    {\Gamma \vdash L [ A ] \colon [0 \to A] B}
    \; (\vdash[])
\end{gathered}
\end{equation}

\section{Typing properties}
\label{appendix:f_type_properties}
\begin{code}
  -- swap : ∀ {Γ x y L A B C}
  --   → x ≢ y
  --   → (Γ , y ⦂ B) , x ⦂ A ⊢ L ⦂ C
  --     -------------------------
  --   → (Γ , x ⦂ A) , y ⦂ B ⊢ L ⦂ C
  swap {Γ} {x} {y} {L} {A} {B} {C} x≢y ⊢L with ⊢⇒Ok ⊢L
  ... | ok-∷fv (ok-∷fv okΓ lc-B B⊆Γ) lc-A A⊆Γ =
    rename (ok-∷fv (ok-∷fv okΓ lc-A A⊆Γ) lc-B B⊆Γ) ρ₁ ρ₂ ⊢L
    where
      ρ₁ : ∀ {z C}
        → (Γ , y ⦂ B) , x ⦂ A ∋ z ⦂ C
          -------------------------
        → (Γ , x ⦂ A) , y ⦂ B ∋ z ⦂ C
      ρ₁ (H refl) = T x≢y H′
      ρ₁ (T z≢x (H refl)) = H′
      ρ₁ (T z≢x (T z≢y ∋z)) = T z≢y (T z≢x ∋z)
      ρ₂ : ∀ {X}
        → ((Γ , y ⦂ B) , x ⦂ A) ∋ X
          -------------------------
        → ((Γ , x ⦂ A) , y ⦂ B) ∋ X
      ρ₂ (S⦂ (S⦂ ∋X)) = S⦂ (S⦂ ∋X)

  -- swap-tm-ty : ∀ {Γ X y L B C}
  --   → ((Γ , y ⦂ B) , X) ⊢ L ⦂ C
  --     -------------------------
  --   → (Γ , X) , y ⦂ B ⊢ L ⦂ C
  swap-tm-ty {Γ} {X} {y} {L} {B} {C} ⊢L with ⊢⇒Ok ⊢L
  ... | ok-∷ftv (ok-∷fv okΓ lc-B B⊆Γ) X∉Γ = rename (ok-∷fv (ok-∷ftv okΓ (proj₂ (∉-++ X∉Γ))) lc-B (⊆⇒⊆∷ B⊆Γ)) ρ₁ ρ₂ ⊢L
    where
      ρ₁ : ∀ {x A}
        → ((Γ , y ⦂ B) , X) ∋ x ⦂ A
          -------------------------
        → ((Γ , X) , y ⦂ B) ∋ x ⦂ A
      ρ₁ (T⦂ (H refl)) = H′
      ρ₁ (T⦂ (T x≢y ∋x)) = T x≢y (T⦂ ∋x)
      ρ₂ : ∀ {Z : Id}
        → ((Γ , y ⦂ B) , X) ∋ Z
        → ((Γ , X) , y ⦂ B) ∋ Z
      ρ₂ Z = S⦂ Z
      ρ₂ (S (S⦂ ∋Z)) = S⦂ (S ∋Z)

  -- drop : ∀ {Γ x L A B C}
  --   → (Γ , x ⦂ A) , x ⦂ B ⊢ L ⦂ C
  --     --------------------------
  --   → Γ , x ⦂ B ⊢ L ⦂ C
  drop {Γ} {x} {L} {A} {B} {C} ⊢L with ⊢⇒Ok ⊢L
  ... | ok-∷fv (ok-∷fv okΓ lc-A A⊆Γ) lc-B B⊆Γ =
    rename (ok-∷fv okΓ lc-B B⊆Γ) ρ₁ ρ₂ ⊢L
    where
      ρ₁ : ∀ {z C}
        → (Γ , x ⦂ A) , x ⦂ B ∋ z ⦂ C
          -------------------------
        → Γ , x ⦂ B ∋ z ⦂ C
      ρ₁ (H refl) = H′
      ρ₁ (T z≢x (H refl)) = contradiction refl z≢x
      ρ₁ (T z≢x (T .z≢x ∋z)) = T z≢x ∋z
      ρ₂ : ∀ {X}
        → ((Γ , x ⦂ A) , x ⦂ B) ∋ X
          -------------------------
        → (Γ , x ⦂ B) ∋ X
      ρ₂ (S⦂ (S⦂ ∋X)) = S⦂ ∋X
\end{code}

\begin{code}
  -- subst-open-ty-tm : ∀ {N : Term} {x y : Id} {i j : ℕ}
  --   (L : Term)
  --   → x ≢ y
  --   → j ≥ i
  --   → (i ≻ty-tm N)
  --   → tm-tm[ x := N ] (ty-tm[ j —→ y ] L)
  --     ≡ ty-tm[ j —→ y ] (tm-tm[ x := N ] L)
  subst-open-ty-tm {N} {x} {y} {i} {j} (fr z) x≢y j≥i i≻N with x ≟lchar z
  ... | yes refl =
    begin
      N
    ≡⟨ sym (lemma2·7-2-ty-tm ≤-refl (lemma2·6-ty-tm j≥i i≻N)) ⟩
      ty-tm[ j —→ y ] N
    ∎
  ... | no  x≢z  = refl
  subst-open-ty-tm (# k) x≢y j≥i i≻N = refl
  subst-open-ty-tm (ƛ L) x≢y j≥i i≻N
    rewrite subst-open-ty-tm L x≢y j≥i i≻N = refl
  subst-open-ty-tm (L · M) x≢y j≥i i≻N
    rewrite subst-open-ty-tm L x≢y j≥i i≻N
    | subst-open-ty-tm M x≢y j≥i i≻N
    = refl
  subst-open-ty-tm (Λ L) x≢y j≥i i≻N
    rewrite subst-open-ty-tm L x≢y (m≤n⇒m≤1+n j≥i) i≻N = refl
  subst-open-ty-tm (L [ A ]) x≢y j≥i i≻N
    rewrite subst-open-ty-tm L x≢y j≥i i≻N = refl
  subst-open-ty-tm ‵zero x≢y j≥i i≻N = refl
  subst-open-ty-tm (‵suc L) x≢y j≥i i≻N rewrite
    subst-open-ty-tm L x≢y j≥i i≻N = refl

  -- subst-open-ty-tm-ctx : ∀ {Γ A} {N : Term} {x y : Id} {i j : ℕ}
  --   (L : Term)
  --   → x ≢ y
  --   → j ≥ i
  --   → (i ≻ty-tm N)
  --   → Γ ⊢ tm-tm[ x := N ] (ty-tm[ j —→ y ] L) ⦂ A
  --   → Γ ⊢ ty-tm[ j —→ y ] (tm-tm[ x := N ] L) ⦂ A
  subst-open-ty-tm-ctx L x≢y j≥i i≻N assump = ≡-with-⊢-tm assump (subst-open-ty-tm L x≢y j≥i i≻N)

  -- subst-open-ty-tm-tm-tm : ∀ {C : Type} {x y : Id} {i j : ℕ}
  --   (L : Term)
  --   → j ≥ i
  --   → (i ≻ty C)
  --   → ty-tm[ x := C ] (tm-tm[ j —→ y ] L)
  --     ≡ tm-tm[ j —→ y ] (ty-tm[ x := C ] L)
  subst-open-ty-tm-tm-tm (fr y) j≥i i≻C = refl
  subst-open-ty-tm-tm-tm {j = j} (# k) j≥i i≻C with j ≟ℕ k
  ... | yes refl = refl
  ... | no  j≢k  = refl
  subst-open-ty-tm-tm-tm (ƛ L) j≥i i≻C = cong ƛ_
    (subst-open-ty-tm-tm-tm L (m≤n⇒m≤1+n j≥i) i≻C)
  subst-open-ty-tm-tm-tm (L · M) j≥i i≻C = cong₂ _·_
    (subst-open-ty-tm-tm-tm L j≥i i≻C)
    (subst-open-ty-tm-tm-tm M j≥i i≻C)
  subst-open-ty-tm-tm-tm (Λ L) j≥i i≻C = cong Λ_
    (subst-open-ty-tm-tm-tm L j≥i i≻C)
  subst-open-ty-tm-tm-tm (L [ A ]) j≥i i≻C = cong₂ _[_]
    (subst-open-ty-tm-tm-tm L j≥i i≻C)
    refl
  subst-open-ty-tm-tm-tm ‵zero j≥i i≻C = refl
  subst-open-ty-tm-tm-tm (‵suc L) j≥i i≻C = cong ‵suc_
    (subst-open-ty-tm-tm-tm L j≥i i≻C)

  -- subst-open-ty-tm-tm-tm-ctx : ∀ {Γ A} {C : Type} {x y : Id} {i j : ℕ}
  --   (L : Term)
  --   → j ≥ i
  --   → (i ≻ty C)
  --   → Γ ⊢ ty-tm[ x := C ] (tm-tm[ j —→ y ] L) ⦂ A
  --   → Γ ⊢ tm-tm[ j —→ y ] (ty-tm[ x := C ] L) ⦂ A
  subst-open-ty-tm-tm-tm-ctx L j≥i i≻C assump = ≡-with-⊢-tm assump (subst-open-ty-tm-tm-tm L j≥i i≻C)

  -- subst-open-ty-ty-ty-ty : ∀ {C : Type} {x y : Id} {i j : ℕ}
  --   (A : Type)
  --   → x ≢ y
  --   → j ≥ i
  --   → (i ≻ty C)
  --   → ty-ty[ x := C ] (ty-ty[ j —→ y ] A)
  --     ≡ ty-ty[ j —→ y ] (ty-ty[ x := C ] A)
  subst-open-ty-ty-ty-ty ‵ℕ x≢y j≥i i≻C = refl
  subst-open-ty-ty-ty-ty {x = x} {j = j} (t-fr z) x≢y j≥i i≻C with x ≟lchar z
  ... | yes refl = sym (lemma2·7-2-ty j≥i i≻C)
  ... | no  x≢z  = refl
  subst-open-ty-ty-ty-ty {x = x} {y = y} {j = j} (t-# k) x≢y j≥i i≻C with j ≟ℕ k
  ... | no  j≢k  = refl
  ... | yes refl with x ≟lchar y
  ...   | yes refl = contradiction refl x≢y
  ...   | no  _    = refl
  subst-open-ty-ty-ty-ty (A ⇒ B) x≢y j≥i i≻C rewrite
      subst-open-ty-ty-ty-ty A x≢y j≥i i≻C
    | subst-open-ty-ty-ty-ty B x≢y j≥i i≻C
    = refl
  subst-open-ty-ty-ty-ty {j = j} (t-∀ A) x≢y j≥i i≻C = cong t-∀_
    (subst-open-ty-ty-ty-ty {j = suc j} A x≢y (m≤n⇒m≤1+n j≥i) i≻C)

  -- subst-open-ty-ty-ty-ty-ctx : ∀ {Γ L} {C A : Type} {x y : Id} {i j : ℕ}
  --   → x ≢ y
  --   → j ≥ i
  --   → (i ≻ty C)
  --   → Γ ⊢ L ⦂ ty-ty[ x := C ] (ty-ty[ j —→ y ] A)
  --   → Γ ⊢ L ⦂ ty-ty[ j —→ y ] (ty-ty[ x := C ] A)
  subst-open-ty-ty-ty-ty-ctx {A = A} x≢y j≥i i≻C assump = ≡-with-⊢-ty assump (subst-open-ty-ty-ty-ty A x≢y j≥i i≻C)

  -- subst-open-ty-tm-ty-tm : ∀ {C : Type} {x y : Id} {i j : ℕ}
  --   (L : Term)
  --   → x ≢ y
  --   → j ≥ i
  --   → (i ≻ty C)
  --   → ty-tm[ x := C ] (ty-tm[ j —→ y ] L)
  --     ≡ ty-tm[ j —→ y ] (ty-tm[ x := C ] L)
  subst-open-ty-tm-ty-tm (fr z) x≢y j≥i i≻C = refl
  subst-open-ty-tm-ty-tm (# k) x≢y j≥i i≻C = refl
  subst-open-ty-tm-ty-tm (ƛ L) x≢y j≥i i≻C rewrite
    subst-open-ty-tm-ty-tm L x≢y j≥i i≻C = refl
  subst-open-ty-tm-ty-tm (L · M) x≢y j≥i i≻C rewrite
      subst-open-ty-tm-ty-tm L x≢y j≥i i≻C
    | subst-open-ty-tm-ty-tm M x≢y j≥i i≻C
    = refl
  subst-open-ty-tm-ty-tm {j = j} (Λ L) x≢y j≥i i≻C = cong Λ_
    (subst-open-ty-tm-ty-tm {j = suc j} L x≢y (m≤n⇒m≤1+n j≥i) i≻C)
  subst-open-ty-tm-ty-tm (L [ A ]) x≢y j≥i i≻C = cong₂ _[_]
    (subst-open-ty-tm-ty-tm L x≢y j≥i i≻C)
    (subst-open-ty-ty-ty-ty A x≢y j≥i i≻C)
  subst-open-ty-tm-ty-tm ‵zero x≢y j≥i i≻C = refl
  subst-open-ty-tm-ty-tm (‵suc L) x≢y j≥i i≻C rewrite
    (subst-open-ty-tm-ty-tm L x≢y j≥i i≻C) = refl

  -- subst-open-ty-tm-ty-tm-ctx : ∀ {Γ L A} {C : Type} {x y : Id} {i j : ℕ}
  --   → x ≢ y
  --   → j ≥ i
  --   → (i ≻ty C)
  --   → Γ ⊢ ty-tm[ x := C ] (ty-tm[ j —→ y ] L) ⦂ A
  --   → Γ ⊢ ty-tm[ j —→ y ] (ty-tm[ x := C ] L) ⦂ A
  subst-open-ty-tm-ty-tm-ctx {L = L} x≢y j≥i i≻C assump = ≡-with-⊢-tm assump (subst-open-ty-tm-ty-tm L x≢y j≥i i≻C)
\end{code}

\begin{code}
  -- subst-intro : ∀ {x i} (L N : Term)
  --   → x ∉ fv-tm L
  --   → tm-tm[ i :→ N ] L ≡ tm-tm[ x := N ] (tm-tm[ i —→ x ] L)
  subst-intro {x} (fr y) N x∉fv-L with x ≟lchar y
  ... | yes refl = contradiction refl (∉∷[]⇒≢ x∉fv-L)
  ... | no  x≢y  = refl
  subst-intro {x} {i} (# k) N x∉fv-L with i ≟ℕ k
  ... | no  i≢k = refl
  ... | yes refl with x ≟lchar x
  ...   | yes refl = refl
  ...   | no  x≢x  = contradiction refl x≢x
  subst-intro {x} {i} (ƛ L) N x∉fv-L
    rewrite subst-intro {x} {suc i} L N x∉fv-L = refl
  subst-intro (L · M) N x∉ =
    let ⟨ x∉fv-L , x∉fv-M ⟩ = ∉-++ {xs = fv-tm L} x∉
    in cong₂ _·_ (subst-intro L N x∉fv-L) (subst-intro M N x∉fv-M)
  subst-intro (Λ L) N x∉fv-L = cong Λ_ (subst-intro L N x∉fv-L)
  subst-intro (L [ A ]) N x∉fv-L =
    cong₂ _[_] (subst-intro L N x∉fv-L) refl
  subst-intro ‵zero N x∉fv-L = refl
  subst-intro (‵suc L) N x∉fv-L =
    cong ‵suc_ (subst-intro L N x∉fv-L)

  -- subst-intro-ty-ty : ∀ {x i B} (A : Type)
  --   → x ∉ ftv-ty A
  --   → ty-ty[ i :→ B ] A ≡ ty-ty[ x := B ] (ty-ty[ i —→ x ] A)
  subst-intro-ty-ty ‵ℕ x∉ = refl
  subst-intro-ty-ty {x} (t-fr y) x∉ with x ≟lchar y
  ... | yes refl = contradiction refl (∉∷[]⇒≢ x∉)
  ... | no  x≢y  = refl
  subst-intro-ty-ty {x} {i} (t-# k) x∉ with i ≟ℕ k
  ... | no  i≢k  = refl
  ... | yes refl with x ≟lchar x
  ...   | yes refl = refl
  ...   | no  x≢x  = contradiction refl x≢x
  subst-intro-ty-ty (A ⇒ B) x∉ = let ⟨ x∉A , x∉B ⟩ = ∉-++ x∉
    in cong₂ _⇒_ (subst-intro-ty-ty A x∉A) (subst-intro-ty-ty B x∉B)
  subst-intro-ty-ty {i = i} (t-∀ A) x∉ = cong t-∀_ (subst-intro-ty-ty {i = suc i} A x∉)

  -- subst-intro-ty-tm : ∀ {x i B} (L : Term)
  --   → x ∉ ftv-tm L
  --   → ty-tm[ i :→ B ] L ≡ ty-tm[ x := B ] (ty-tm[ i —→ x ] L)
  subst-intro-ty-tm (fr x) x∉ = refl
  subst-intro-ty-tm (# k) x∉ = refl
  subst-intro-ty-tm (ƛ L) x∉ = cong ƛ_ (subst-intro-ty-tm L x∉)
  subst-intro-ty-tm (L · M) x∉ = let ⟨ x∉L , x∉M ⟩ = ∉-++ x∉
    in cong₂ _·_ (subst-intro-ty-tm L x∉L) (subst-intro-ty-tm M x∉M)
  subst-intro-ty-tm (Λ L) x∉ = cong Λ_ (subst-intro-ty-tm L x∉)
  subst-intro-ty-tm (L [ A ]) x∉ = let ⟨ x∉L , x∉A ⟩ = ∉-++ x∉
    in cong₂ _[_] (subst-intro-ty-tm L x∉L) (subst-intro-ty-ty A x∉A )
  subst-intro-ty-tm ‵zero x∉ = refl
  subst-intro-ty-tm (‵suc L) x∉ = cong ‵suc_ (subst-intro-ty-tm L x∉)
\end{code}

\section{Well-typed terms are locally closed}
\label{appendix:f_well_typed_lc}
\begin{code}
  -- ⊢⇒lc-ty : ∀ {Γ L A} → Γ ⊢ L ⦂ A → Ty-LocallyClosed A
  ⊢⇒lc-ty {Γ} {fr x} (⊢free okΓ lc-A ∋x) = lc-A
  ⊢⇒lc-ty {Γ} {ƛ L} (⊢ƛ lc-B И⟨ Иe₁ , Иe₂ ⟩) j =
    let И⟨ B-Иe₁ , B-Иe₂ ⟩ = lc-B j
        И⟨ A-Иe₁ , A-Иe₂ ⟩ =
          (⊢⇒lc-ty (Иe₂ (fresh Иe₁) {fresh-correct Иe₁})) j
    in И⟨ A-Иe₁ ++ B-Иe₁ , (λ a {a∉} → cong₂ _⇒_
      (B-Иe₂ a {proj₂ (∉-++ {xs = A-Иe₁} a∉)})
      (A-Иe₂ a {proj₁ (∉-++ a∉)})) ⟩
  ⊢⇒lc-ty {Γ} {L · M} (⊢· ⊢L ⊢M) = proj₂ (⇒-≻ (⊢⇒lc-ty ⊢L))
  ⊢⇒lc-ty {Γ} {Λ L} (⊢Λ И⟨ Иe₁ , Иe₂ ⟩) j =
    let induction-hypo = ⊢⇒lc-ty (Иe₂ (fresh Иe₁) {fresh-correct Иe₁})
        И⟨ B-Иe₁ , B-Иe₂ ⟩ = induction-hypo (suc j) ⦃ z≤n ⦄
    in И⟨ B-Иe₁ ++ Иe₁ , (λ a {a∉} → cong t-∀_
      (open-rec-lc-lemma-ty
        (λ ())
        (B-Иe₂ a {proj₁ (∉-++ a∉)}))) ⟩
  ⊢⇒lc-ty {Γ} {L [ B ]} (⊢[] lc-B _ ⊢L) =
    let 1≻A = i≻∀A⇒si≻A (⊢⇒lc-ty ⊢L)
    in helper z≤n 1≻A lc-B
    where
      helper : ∀ {A B i j}
        → j ≥ i
        → (suc i) ≻ty A
        → i ≻ty B → j ≻ty (ty-ty[ i :→ B ] A)
      helper {‵ℕ} j≥i si≻A lc-B j = И⟨ [] , (λ _ → refl) ⟩
      helper {t-fr x} j≥i si≻A lc-B j = И⟨ [] , (λ _ → refl) ⟩
      helper {t-# n} {_} {i} j≥i si≻A i≻B k with i ≟ℕ n
      ... | yes refl = i≻B k ⦃ ≤-trans j≥i it ⦄
      ... | no  i≢n  with k ≟ℕ n
      ...   | no  _    = И⟨ [] , (λ _ → refl) ⟩
      ...   | yes refl with si≻A n ⦃ ≤∧≢⇒< (≤-trans j≥i it) i≢n ⦄ -- ⦃ ≤∧≢⇒< it 0≢k ⦄
      ...     | И⟨ Иe₁ , Иe₂ ⟩ with n ≟ℕ n
      ...       | yes refl with () ← Иe₂ (fresh Иe₁) {fresh-correct Иe₁}
      ...       | no  n≢n =  contradiction refl n≢n
      helper {A ⇒ C} j≥i si≻A⇒C i≻B k =
        let ⟨ si≻A , si≻C ⟩ = ⇒-≻ si≻A⇒C
            И⟨ A-Иe₁ , A-Иe₂ ⟩ = (helper j≥i si≻A i≻B) k
            И⟨ C-Иe₁ , C-Иe₂ ⟩ = (helper j≥i si≻C i≻B) k
        in И⟨ A-Иe₁ ++ C-Иe₁ , (λ a {a∉} → cong₂ _⇒_
          (A-Иe₂ a {proj₁ (∉-++ a∉)})
          (C-Иe₂ a {proj₂ (∉-++ {xs = A-Иe₁} a∉)})) ⟩
      helper {t-∀ A} {B} {i} j≥i si≻∀A i≻B k =
        let ssi≻A = i≻∀A⇒si≻A si≻∀A
            И⟨ Иe₁ , Иe₂ ⟩ = (helper (s≤s j≥i) ssi≻A (lemma2·6-ty (n≤1+n i) i≻B)) (suc k) ⦃ s≤s it ⦄
        in И⟨ Иe₁ , (λ a {a∉} → cong t-∀_ (Иe₂ a {a∉})) ⟩
  ⊢⇒lc-ty {Γ} {‵zero} (⊢zero _) = n≻‵ℕ
  ⊢⇒lc-ty {Γ} {‵suc L} (⊢suc ⊢L) = n≻‵ℕ

  -- ⊢⇒lc-ty-tm : ∀ {Γ L A} → Γ ⊢ L ⦂ A → Ty-Tm-LocallyClosed L
  ⊢⇒lc-ty-tm (⊢free okΓ lc-A ∋x) j = И⟨ [] , (λ _ → refl) ⟩
  ⊢⇒lc-ty-tm (⊢ƛ lc-A И⟨ B-Иe₁ , B-Иe₂ ⟩) j =
    let И⟨ A-Иe₁ , A-Иe₂ ⟩ = lc-A j
        И⟨ Иe₁ , Иe₂ ⟩ = (⊢⇒lc-ty-tm (B-Иe₂ (fresh B-Иe₁) {fresh-correct B-Иe₁})) j
    in И⟨ Иe₁ ++ A-Иe₁ , (λ a {a∉} → cong ƛ_
      (open-rec-lc-lemma-ty-tm-tm-tm (Иe₂ a {proj₁ (∉-++ a∉)}))) ⟩
  ⊢⇒lc-ty-tm {Γ} (⊢· ⊢L ⊢M) j = И⟨ [] , (λ a → cong₂ _·_
    (lemma2·7-2-ty-tm it (⊢⇒lc-ty-tm ⊢L))
    (lemma2·7-2-ty-tm it (⊢⇒lc-ty-tm ⊢M))) ⟩
  ⊢⇒lc-ty-tm {Γ} {Λ L} (⊢Λ И⟨ Иe₁ , Иe₂ ⟩) j =
      let induc-hypo = ⊢⇒lc-ty-tm (Иe₂ (fresh Иe₁) {fresh-correct Иe₁})
          sj≻ty-tm[]L = lemma2·6-ty-tm z≤n induc-hypo
          И⟨ sj≻L-Иe₁ , sj≻L-Иe₂ ⟩ = (helper L z≤n sj≻ty-tm[]L) (suc j) ⦃ s≤s it ⦄
      in И⟨ sj≻L-Иe₁ , (λ a {a∉} → cong Λ_ (sj≻L-Иe₂ a {a∉})) ⟩
    where
      helper : ∀ {i x j} (L : Term) → j ≥ i → (suc i) ≻ty-tm (ty-tm[ i —→ x ] L) → (suc j) ≻ty-tm L
      helper L j≥i si≻[]L k =
        let И⟨ Иe₁ , Иe₂ ⟩ = si≻[]L k ⦃ ≤-trans (s≤s j≥i) it ⦄
            k≢i = sym-≢ (<⇒≢ (≤-trans (s≤s j≥i) it))
        in И⟨ Иe₁ , (λ a {a∉} → open-rec-lc-lemma-ty-tm k≢i (Иe₂ a {a∉})) ⟩
  ⊢⇒lc-ty-tm {Γ} (⊢[] lc _ ⊢L) j = И⟨ [] , (λ a → cong₂ _[_]
    (lemma2·7-2-ty-tm it (⊢⇒lc-ty-tm ⊢L))
    (lemma2·7-2-ty it lc)) ⟩
  ⊢⇒lc-ty-tm (⊢zero _) j = И⟨ [] , (λ _ → refl) ⟩
  ⊢⇒lc-ty-tm {Γ} (⊢suc ⊢L) j = И⟨ [] , (λ a →
    cong ‵suc_ (lemma2·7-2-ty-tm it (⊢⇒lc-ty-tm ⊢L))) ⟩
\end{code}

\chapter{Prior work submitted for TSPL}
\label{appendix:tspl}
\input{tspl_prior_work.tex}

\chapter{Compilation instructions}
\label{appendix:compilation_instructions}

This document is a literate Agda file. It has been tested to work with
\begin{itemize}
  \item Agda 2.7.0,

  \item the Agda Standard Library 2.1 \citep{the_agda_community_agda_2024},

  \item XeLaTeX 3.141592653-2.6-0.999997 (TeX Live 2026/dev/Arch Linux) (using Arch Linux package
  \texttt{texlive-xetex} 2025.2-1).
\end{itemize}

The full source code is available at \url{https://github.com/ettolrach/system-f-in-agda-via-pitts}.

Please see \texttt{./README.md} in the above linked repository for full compilation instructions.

\end{document}

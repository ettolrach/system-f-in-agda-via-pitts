The idea of reducible expressions, called a \textit{redex}, was introduced by Alonzo Church
\cite[p.~56]{pierce_types_2002}. We refer to any expression of the form $(\lambda \, x. M) N$ (or
$(\Lambda M) [A]$ in System F) as a redex, since we can perform a $\beta$-reduction (see equation
\ref{equation:untyped_beta_rules}).

There are several ways of evaluating the $\lambda$-calculus. One important decision is whether to
treat $\lambda$-abstractions as values, or to reduce these if they are a redex.

When evaluating, we need to consider some terms to be \textit{values}. These cannot be reduced any
further. One method is to consider $\lambda$-abstractions to be values and to only evaluate
\textit{closed} terms, that is, terms without any free variables. This is called weak-head
reduction, and the values are called weak-head normal forms \citep{wadler_programming_2022}. This
method is also used by popular programming languages such as Haskell \citep{hutchison_sharing_2005}.
The definition of the $\lambda$-calculus as Church first described it uses what is nowadays referred
to as \textit{full normalisation} \citep{wadler_programming_2022}. The reduction rule which is
absent in weak-head reduction is the $\zeta$ rule (see equation \ref{equation:untyped_beta_rules}),
also called \textit{reducing under a $\lambda$-abstraction}.

Using weak-head reduction, since evaluation stops at $\lambda$-abstractions, the Church-Rosser
property no longer applies (and, in fact, the system is no longer Turing complete). One of the early
explorations of weak-head reduction was made by \citet{cagman_combinatory_1998}, comparing it to
combinatory logic. They presented the necessity of another reduction rule which dictates when one
can reduce under a $\lambda$-abstraction, which restores the Church-Rosser property
\citep{hutchison_sharing_2005}.

However, another, simpler method is to introduce primitives \citep{wadler_programming_2022}. These
are a small subset of Dana Scott and Gordon Plotkin's Programming Computable Functions (PCF)
\citep{plotkin_lcf_1977}; we introduce a value \texttt{‵zero} of type $\nat$ and a function
\texttt{‵suc} of type $\nat \to \nat$ (since these are modelled after the naturals, we use the base
type $\nat$). As this approach was used by \citet{chargueraud_locally_2012} and
\citet{wadler_programming_2022}, we choose to follow the same approach here.

We refer to any expression of the form $(\lambda \, x. M) N$ (or $(\Lambda M) [A]$ in System F) as a
\textit{redex} (short for reducible expressions), named so by Alonzo Church
\citep[p.~56]{pierce_types_2002}. These expressions are reducible because we can perform a
$\beta$-reduction (see equation \ref{equation:untyped_beta_rules}).

There are several ways of evaluating the $\lambda$-calculus. One important decision is whether to
treat $\lambda$-abstractions as values, or to reduce these if they are a redex.

When evaluating, we need to consider some terms to be \textit{values}. These cannot be reduced any
further. One method is to consider $\lambda$-abstractions to be values and to only evaluate
\textit{closed} terms, that is, terms without any free variables or free type variables. This is
called weak-head reduction, and the values are called weak-head normal forms
\citep{wadler_programming_2022}. This method is also used by popular programming languages such as
Haskell \citep{hutchison_sharing_2005}. The definition of the $\lambda$-calculus as Church first
described it uses what is nowadays referred to as \textit{full normalisation}
\citep{wadler_programming_2022}. The reduction rule which is absent in weak-head reduction is the
$\zeta$ rule (see equation \ref{equation:untyped_beta_rules}), also called \textit{reducing under a
$\lambda$-abstraction}.

When using weak-head reduction, we run into an issue. Consider the following expression;
\begin{align*}
  \lambda \, q. \lambda \, r. q ((\lambda \, f. \lambda \, x . x) q r).
\end{align*}
We ought to continue evaluating this expression to $\lambda \, q. \lambda \, r. q r$. However, since
the entire expression is in a $\lambda$-abstraction, and we consider these to be values, evaluation
would stop.

One of the early explorations of weak-head reduction was made by \citet{cagman_combinatory_1998},
wherein they presented the necessity of another reduction rule which dictates when one can reduce
under a $\lambda$-abstraction, which would help evaluate the above expression
\citep{hutchison_sharing_2005}. However, adding more reduction rules would cause proofs about
evaluation to become longer.

Instead, we will follow \citet{wadler_programming_2022} and introduce two primitive values:
\texttt{‵zero} and \texttt{‵suc}. These are inspired by the same constructs in Dana Scott and Gordon
Plotkin's Programming Computable Functions (PCF) \citep{plotkin_lcf_1977}. As the choice of symbol
for the base type is arbitrary, we shall change it to be $\nat$ instead of $\iota$ to reflect the
connection that the two new primitives have to the natural numbers. \texttt{‵zero} will be of type
$\nat$ and \texttt{‵suc} will be of type $\nat \to \nat$.

Now we can evaluate the above expression by modifying it slightly;
\begin{align*}
  &(\lambda \, q. \lambda \, r. q ((\lambda \, f. \lambda \, x . x) q r))
    \texttt{‵suc} \;\texttt{‵zero}\\
  &\mapsto_{\beta} (
      \lambda \, r. \texttt{‵suc} \; ((\lambda \, f. \lambda \, x . x) \texttt{‵suc} \; r)
    )
    \texttt{‵zero}\\
  &\mapsto_{\beta} \texttt{‵suc} ((\lambda \, f. \lambda \, x . x) \texttt{‵suc} \; \texttt{‵zero})\\
  &\mapsto_{\beta} \texttt{‵suc} \; \texttt{‵zero}.
\end{align*}

Although this results in a different expression, it allows us to reduce the redex, which we couldn't
before. This approach is more concise than adding the previously mentioned extra evaluation rules.
In my develpoment of System F, I will consider $\lambda$-abstractions, $\Lambda$-abstractions,
\texttt{‵zero}, and \texttt{‵suc} to be values. The evaluation rules used in my development are
detailed in section \ref{chapter3:evaluation}.

% UG project example file, February 2024
%
%   Added the "online" option for equal margins, February 2024 [Hiroshi Shimodaira, Iain Murray]
%   A minor change in citation, September 2023 [Hiroshi Shimodaira]
%
% Do not change the first two lines of code, except you may delete "logo," if causing problems.
% Understand any problems and seek approval before assuming it's ok to remove ugcheck.
\documentclass[logo,bsc,singlespacing,parskip,online]{infthesis}
\usepackage{ugcheck}


% Include any packages you need below, but don't include any that change the page
% layout or style of the dissertation. By including the ugcheck package above,
% you should catch most accidental changes of page layout though.

\usepackage{microtype} % recommended, but you can remove if it causes problems
\usepackage[round]{natbib} % recommended for citations

% === Custom packages === %

% Maths:

\usepackage{mathtools}
\usepackage{amssymb}
\usepackage{enumitem}
\usepackage{hyperref}
\DeclareMathOperator{\lcm}{lcm}
\DeclareMathOperator{\Real}{Re}
\DeclareMathOperator{\Imag}{Im}
\DeclareMathOperator{\complex}{\mathbb{C}}
\DeclareMathOperator{\reals}{\mathbb{R}}
\DeclareMathOperator{\nat}{\mathbb{N}}
\DeclareMathOperator{\integer}{\mathbb{Z}}
\DeclareMathOperator{\Log}{Log}
\DeclareMathOperator{\Arg}{Arg}

% Others:

% Syntax highlighting
\usepackage{minted}
% BNF
\usepackage{simplebnf}
% Unicode
\usepackage[utf8]{inputenc}

% Unicode characters:
\DeclareUnicodeCharacter{2115}{$\nat$}

\begin{document}
\begin{preliminary}

\title{System F in Agda via Pitt}

\author{Charlotte Ausel}

% CHOOSE YOUR DEGREE a):
% please leave just one of the following un-commented
% \course{Artificial Intelligence}
%\course{Artificial Intelligence and Computer Science}
%\course{Artificial Intelligence and Mathematics}
%\course{Artificial Intelligence and Software Engineering}
%\course{Cognitive Science}
%\course{Computer Science}
%\course{Computer Science and Management Science}
\course{Computer Science and Mathematics}
%\course{Computer Science and Physics}
%\course{Software Engineering}
%\course{Master of Informatics} % MInf students

% CHOOSE YOUR DEGREE b):
% please leave just one of the following un-commented
%\project{MInf Project (Part 1) Report}  % 4th year MInf students
%\project{MInf Project (Part 2) Report}  % 5th year MInf students
\project{4th Year Project Report}        % all other UG4 students


\date{\today}

\abstract{
This skeleton demonstrates how to use the \texttt{infthesis} style for
undergraduate dissertations in the School of Informatics. It also emphasises the
page limit, and that you must not deviate from the required style.
The file \texttt{skeleton.tex} generates this document and should be used as a
starting point for your thesis. Replace this abstract text with a concise
summary of your report.
}

\maketitle

\newenvironment{ethics}
   {\begin{frontenv}{Research Ethics Approval}{\LARGE}}
   {\end{frontenv}\newpage}

\begin{ethics}
This project was planned in accordance with the Informatics Research
Ethics policy. It did not involve any aspects that required approval
from the Informatics Research Ethics committee.

\standarddeclaration
\end{ethics}


\begin{acknowledgements}
Any acknowledgements go here.
\end{acknowledgements}


\tableofcontents
\end{preliminary}


\chapter{Introduction}

The preliminary material of your report should contain:
\begin{itemize}
\item
The title page.
\item
An abstract page.
\item
Declaration of ethics and own work.
\item
Optionally an acknowledgements page.
\item
The table of contents.
\end{itemize}

As in this example \texttt{skeleton.tex}, the above material should be
included between:
\begin{verbatim}
\begin{preliminary}
    ...
\end{preliminary}
\end{verbatim}
This style file uses roman numeral page numbers for the preliminary material.

The main content of the dissertation, starting with the first chapter,
starts with page~1. \emph{\textbf{The main content must not go beyond page~40.}}

The report then contains a bibliography and any appendices, which may go beyond
page~40. The appendices are only for any supporting material that's important to
go on record. However, you cannot assume markers of dissertations will read them.

You may not change the dissertation format (e.g., reduce the font size, change
the margins, or reduce the line spacing from the default single spacing). Be
careful if you copy-paste packages into your document preamble from elsewhere.
Some \LaTeX{} packages, such as \texttt{fullpage} or \texttt{savetrees}, change
the margins of your document. Do not include them!

Over-length or incorrectly-formatted dissertations will not be accepted and you
would have to modify your dissertation and resubmit. You cannot assume we will
check your submission before the final deadline and if it requires resubmission
after the deadline to conform to the page and style requirements you will be
subject to the usual late penalties based on your final submission time.

\section{Using Sections}

Divide your chapters into sub-parts as appropriate.

\section{Citations}

When citing work using author names, your sentences should still read
as correct English if any parts in parenthesis are removed.
Use {\tt {\textbackslash}citet} when citing the name as text that's part of your sentence: 
\begin{quote}
  We follow the method of \citet{P1}.
\end{quote}
and use {\tt {\textbackslash}citep} for a parenthetical citation:
\begin{quote}
  It's possible to learn first-order Horn sequences from entailment \citep{P2}.
\end{quote}

You may use any consistent reference style that you prefer, including ``[1]'' numerical citations. 

\chapter{Background}

\section{Agda}
Agda is a dependently-typed functional programming language, which makes it suitable as a proof-assistent for intuitionistic logic. Its syntax is very similar to Haskell, and in fact, the two are quite closely related. Agda compiles to Haskell and can use Haskell's standard library.

We won't present a detailed description of Agda and instead encourage the reader to learn more about it themselves. Initial code examples should be simple enough to follow, and if prior experience in other theorem provers like Coq or Lean is present, then it shouldn't be too difficult to understand the code in this paper.

A lot of definitions of datatypes in Agda use inductive definitions. For example, following the Peano axioms for the natural numbers $\nat$, we may define them like so.

\begin{minted}{agda}
data Nat : Set where
  zero : Nat
  suc  : Nat -> Nat
\end{minted}

Note how we here we use \texttt{Nat} and \texttt{->}, words which are composed of ASCII characters. Agda also supports non-ASCII (unicode) characters, and in fact, encourages its use. So it would be more idiomatic to write the above using $\nat$ and $\rightarrow$.

\begin{minted}{agda}
data ℕ : Set where
  zero : ℕ
  suc  : ℕ → ℕ
\end{minted}

\section{The $\lambda$-Calculus and System F}

The \textit{$\lambda$-calculus} (pronounced \textit{lambda calculus}) is a theoretical model of computation which looks and works similar to familiar functional programming languages while still being as simple as possible to be fully Turing-complete. We recommend that resource for readers who aren't as familiar with the $\lambda$-calculus to use a resource such as Pierce TODO: CITE PIERCE. We will summarise the results we're interested in below.

We have the following familiar BNF grammar for any $\lambda$-calculus term $t$.

\begin{center}
\begin{bnf}
  $t$ ::=
  | $x$ : variables
  | $\lambda \, x. t$ : ($\lambda$)-abstractions
  | $t_1 t_2$ : function application
\end{bnf}
\end{center}

We will take for granted the following facts.

\begin{itemize}
  \item The familiar syntax of the $\lambda$-calculus, e.g. function application is left-associative (so for all abstractions $f$, $f a b = (f a) b$), and $\lambda$-abstractions extend as far as possible (e.g. $\lambda f. f a b = \lambda f. (f a b)$).

  \item We have the familiar $\alpha$-conversion, $\beta$-reduction, and $\eta$-reduction.

  \item $\beta$-reduction is a congruence, and so we can choose any redex (reducible expression) to perform a $\beta$-reduction.
  
  \item The Church-Rosser theorem.
  
  \item $\alpha$-equivalence is an equivalence relation, and so is $\alpha\beta$-equivalence and $\alpha\beta\eta$-equivalence. We write this as $=_{\alpha}$, $=_{\alpha\beta}$, and $=_{\alpha\beta\eta}$ repectively.
\end{itemize}

\chapter{Conclusions}

\section{Final Reminder}

The body of your dissertation, before the references and any appendices,
\emph{must} finish by page~40. The introduction, after preliminary material,
should have started on page~1.

You may not change the dissertation format (e.g., reduce the font size, change
the margins, or reduce the line spacing from the default single spacing). Be
careful if you copy-paste packages into your document preamble from elsewhere.
Some \LaTeX{} packages, such as \texttt{fullpage} or \texttt{savetrees}, change
the margins of your document. Do not include them!

Over-length or incorrectly-formatted dissertations will not be accepted and you
would have to modify your dissertation and resubmit. You cannot assume we will
check your submission before the final deadline and if it requires resubmission
after the deadline to conform to the page and style requirements you will be
subject to the usual late penalties based on your final submission time.

% \bibliographystyle{plain}
\bibliographystyle{plainnat}
\bibliography{mybibfile}


% You may delete everything from \appendix up to \end{document} if you don't need it.
\appendix

\chapter{First appendix}

\section{First section}

Any appendices, including any required ethics information, should be included
after the references.

Markers do not have to consider appendices. Make sure that your contributions
are made clear in the main body of the dissertation (within the page limit).

\chapter{Participants' information sheet}

If you had human participants, include key information that they were given in
an appendix, and point to it from the ethics declaration.

\chapter{Participants' consent form}

If you had human participants, include information about how consent was
gathered in an appendix, and point to it from the ethics declaration.
This information is often a copy of a consent form.


\end{document}

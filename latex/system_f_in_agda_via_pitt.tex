% UG project example file, February 2024
%
%   Added the "online" option for equal margins, February 2024 [Hiroshi Shimodaira, Iain Murray]
%   A minor change in citation, September 2023 [Hiroshi Shimodaira]
%
% Do not change the first two lines of code, except you may delete "logo," if causing problems.
% Understand any problems and seek approval before assuming it's ok to remove ugcheck.
\documentclass[logo,bsc,singlespacing,parskip,online]{infthesis}
\usepackage{ugcheck}


% Include any packages you need below, but don't include any that change the page
% layout or style of the dissertation. By including the ugcheck package above,
% you should catch most accidental changes of page layout though.

\usepackage{microtype} % recommended, but you can remove if it causes problems
\usepackage[round]{natbib} % recommended for citations

% === Custom packages === %

% Maths:

\usepackage{mathtools}
\usepackage{amssymb}
\usepackage{enumitem}
\usepackage{hyperref}
\DeclareMathOperator{\lcm}{lcm}
\DeclareMathOperator{\Real}{Re}
\DeclareMathOperator{\Imag}{Im}
\DeclareMathOperator{\complex}{\mathbb{C}}
\DeclareMathOperator{\reals}{\mathbb{R}}
\DeclareMathOperator{\nat}{\mathbb{N}}
\DeclareMathOperator{\integer}{\mathbb{Z}}
\DeclareMathOperator{\Log}{Log}
\DeclareMathOperator{\Arg}{Arg}

% Others:

% Syntax highlighting
\usepackage{minted}
% BNF
\usepackage{simplebnf}
% Unicode
\usepackage[utf8]{inputenc}
% Inference rules
\usepackage{mathpartir}

% Unicode characters:
\DeclareUnicodeCharacter{2115}{$\nat$}
\DeclareUnicodeCharacter{2200}{$\forall$}
\DeclareUnicodeCharacter{2261}{$\equiv$}

\begin{document}
\begin{preliminary}

\title{System F in Agda via Pitt}

\author{Charlotte Ausel}

% CHOOSE YOUR DEGREE a):
% please leave just one of the following un-commented
% \course{Artificial Intelligence}
%\course{Artificial Intelligence and Computer Science}
%\course{Artificial Intelligence and Mathematics}
%\course{Artificial Intelligence and Software Engineering}
%\course{Cognitive Science}
%\course{Computer Science}
%\course{Computer Science and Management Science}
\course{Computer Science and Mathematics}
%\course{Computer Science and Physics}
%\course{Software Engineering}
%\course{Master of Informatics} % MInf students

% CHOOSE YOUR DEGREE b):
% please leave just one of the following un-commented
%\project{MInf Project (Part 1) Report}  % 4th year MInf students
%\project{MInf Project (Part 2) Report}  % 5th year MInf students
\project{4th Year Project Report}        % all other UG4 students


\date{\today}

\abstract{
This skeleton demonstrates how to use the \texttt{infthesis} style for
undergraduate dissertations in the School of Informatics. It also emphasises the
page limit, and that you must not deviate from the required style.
The file \texttt{skeleton.tex} generates this document and should be used as a
starting point for your thesis. Replace this abstract text with a concise
summary of your report.
}

\maketitle

\newenvironment{ethics}
   {\begin{frontenv}{Research Ethics Approval}{\LARGE}}
   {\end{frontenv}\newpage}

\begin{ethics}
This project was planned in accordance with the Informatics Research
Ethics policy. It did not involve any aspects that required approval
from the Informatics Research Ethics committee.

\standarddeclaration
\end{ethics}


\begin{acknowledgements}
Any acknowledgements go here.
\end{acknowledgements}


\tableofcontents
\end{preliminary}


\chapter{Introduction}

TODO

\chapter{Background}

\section{Agda}
Agda is a dependently-typed functional programming language, which makes it
suitable as a proof-assistant for intuitionistic logic
\citep{norell_towards_2007}, similar to other such proof-assistants like Coq or
Lean. Its syntax is very similar to Haskell, and in fact, the two are closely
related; the Agda compiler is a transpiler to Haskell, and the Haskell standard
library can be used in Agda \citep{kusee_compiling_2017}. Yet, Agda has stricter
limitations on recursive functions and some other such language features which,
if included, would make it harder to reason about proofs
\citep{berghofer_brief_2009}.

Agda most commonly uses inductive definitions. For example, following the Peano
axioms for the natural numbers $\nat$ \citep{boolos_freges_1995}, we may define
them like so.

\begin{minted}{agda}
data ℕ : Set where
  zero : ℕ
  suc  : ℕ → ℕ
\end{minted}

Taking advantage of the Curry-Howard correspondence, a proof in Agda is simply a
function with an appropriate type signature and function body
\citep{wadler_propositions_2015}. So, a proof that addition is associative would
use recursion, which corresponds to induction, as shown below.

\begin{minted}{agda}
open import Relation.Binary.PropositionalEquality
  using (_≡_; refl; cong)
open import Data.Nat using (ℕ; zero; suc; _+_)

+-assoc : ∀ (m n p : ℕ) → (m + n) + p ≡ m + (n + p)
+-assoc zero n p = refl
+-assoc (suc m) n p = cong suc (+-assoc m n p)
\end{minted}

\section{The $\lambda$-Calculus and System F}

\paragraph*{The $\lambda$-calculus.} The $\lambda$-calculus (pronounced
\textit{lambda calculus}), is a theoretical model of computation developed by
Alonzo Church in the 1930s (first described in \citet{church_set_1932}) and is
what System F is based upon. It looks and works similar to familiar functional
programming languages, yet its defintion is as minimal as possible while still
being Turing-complete. As described in \cite{pierce_types_2002}, the most
relevant results are summarised below.

We have the following familiar BNF grammar for any $\lambda$-calculus term $t$.

\begin{center}
\begin{bnf}
  $t$ ::=
  | $x$ : variables
  | $\lambda \, x. t$ : ($\lambda$)-abstractions
  | $t_1 t_2$ : function application
\end{bnf}
\end{center}

Function application is left-associative (so for all abstractions $f$, $f a b =
(f a) b$), and $\lambda$-abstractions extend as far as possible (e.g. $\lambda
f. f a b = \lambda f. (f a b)$).

The $\lambda$-calculus includes three reductions. $\alpha$-conversion is the
renaming of variables such that the semantics of the program are not changed;
terms which are semantically identical but use different variable bindings are
called $\alpha$-equivalent, and $\alpha$-equivalence is an equivalence-relation.
$\beta$-reduction is how functions are applied, we write a reduction as $t
\mapsto_{\beta} t'$. For any function application $(\lambda \, x. t) u$, we
replace all occurrences of $x$ in $t$ with $u$. $\beta$-reduction is a
congruence, so if $t \mapsto_{\beta} t'$, then $st \mapsto_{\beta} st'$, and $ts
\mapsto_{\beta} t's$.

Finally, we have $\eta$-reduction; for all $\lambda$-abstractions $\lambda \, x.
f x$, we have that $\lambda \, x. f x \mapsto_{\eta} f$ (this is the idea of
function extensionality).
  
If further $\beta$-reducations do not simplify a term, we say that it is in its
\textit{normal form}. We shall represent an arbitrary number of successive
$\beta$-reductions as $\mapsto_{\beta^{\star}}$.

Lastly, the Church-Rosser theorem states that for any term $t$, if it
$\beta$-reduces to two terms $a$ and $b$, then there exists a common term $t'$
which both $a$ and $b$ eventually $\beta$-reduce to
\citep{church_properties_1936}.

Notably, some terms may not have a normal form. A famous example is
\textit{omega} (sometimes called the \textit{omega combinator}) defined as
$(\lambda \, x. (x x)) (\lambda \, x. (x x))$, which when applied to itself
doesn't reduce any further.

\begin{equation*}
  (\lambda \, x. (x x)) (\lambda \, x. (x x)) \quad \mapsto_{\beta} \quad (\lambda \, x. (x x)) (\lambda \, x. (x x))
\end{equation*}

Just the first $\lambda \, x. (x x)$ part on its own is called \textit{little
omega}. Another famous example is the fixpoint called the
\textit{y-combinator}\footnote{Defined as $\mathcal{Y} \triangleq (\lambda \, f.
(\lambda \, x. f (x x )) (\lambda \, x. f (xx)))$}. Given any argument, it will
$\beta$-reduce to the argument applied to itself.

We say that these terms \textit{do not have a normal form}.

\paragraph*{The simply-typed $\lambda$-calculus.} While not being
Turing-complete, the simply-typed $\lambda$-calculus (STLC for short and
sometimes given the symbol $\lambda^{\rightarrow}$) is an extension to the
untyped $\lambda$-calculus described above that was developed in
\citet{church_formulation_1940} which requires each term to have a
\textit{type}. Terms which do not have a normal form cannot be given a type. As
such, we only remain with `nice' expressions, that is, all expressions will
after successive $\beta$-reductions result in an irreducible expression---their
normal form. This is also called \textit{strong normalisation}.
\citep{pierce_types_2002}

Since we can no longer represent all of the terms of the untyped
$\lambda$-calculus, we've lost Turing-completeness. Nevertheless, STLC is still
useful, since we can guarantee that any term which can be given a type will have
a normal form. Having a normal form is the $\lambda$-calculus equivalent of a
Turing machine encoding halting. The popular saying goes, `well-typed programs
can't go wrong!' \citep{milner_theory_1978}

The base types we will commonly be using are $\mathbb{B}$ and $\nat$ (we could
use just one, but using two will make the examples easier to
understand)\footnote{If we didn't include any base types, then our computational
model becomes \textit{degenerate} (that is, there are no terms). This is because
everything has to have a type, and if there are no types, then nothing can
exist.}. We shall call the set of base types $T$, so in our case, $T = \{
\mathbb{B} , \nat \}$. The set of variables will be $V$. Our \textit{type
context} (also called \textit{type environment}) will be given the symbol
$\Gamma$, which is a map from variables to types $\Gamma \colon V \to T$
(alternatively, we can see it as a sequence indexed by variables $(T_v)_{v \in
V}$). We closely follow the syntax that is used by \citet{pierce_types_2002}.

In the STLC, we couldn't give a type to little omega, since we can't give a type
to both the argument and the argument applied to itself ($\lambda \, x  \colon ?
. x x \colon ??$). If our context contained the mapping $x \colon \nat$, then we
could write an function which applies its argument to $x$ like so,

\begin{equation*}
  x \colon \nat \in \Gamma \vdash (\lambda \, f \colon \nat \to \nat . f x) \colon \nat.
\end{equation*}

Crucially, if $x$ had type $\mathbb{B}$, we would fail to determine the type of
the function, and so we can guarantee that all of our typed expressions `can't
go wrong'. \citep{milner_theory_1978}

This paper will build upon and borrow notation from the textbook
\textit{Programming Language Foundations in Agda}
\citep{wadler_programming_2022}, which includes an implementation of the STLC
and also serves as further background.

\paragraph*{System F.} System F has been the formal background to what many
modern programming languages call \textit{generics}. For example, in Rust, we
could write a function which checks if a list's length is two.

\begin{minted}{rust}
fn is_length_two<A>(slice: &[A]) -> bool {
    slice.len() == 2
}
\end{minted}

This function will accept any kind of list. This is because we used a
\textit{type parameter} in the function's type signature, in this case called
\texttt{A}.

System F is the STLC equipped with \textit{polymorphic types}, another word for
type parameters. It was independently discovered by Jean-Yves
\citet{girard_interpretation_1972} and John \citet{goos_towards_1974} (who gave
it the more straight-forward name, \textit{the polymorphic $\lambda$-calulus}).
We could write a function which applies a function twice to an argument,

\begin{equation*}
  \Lambda \, X. \lambda \, f \colon X \to X . \lambda \, x \colon X . f (f x) \colon \forall X . (X \to X) \to X \to X.
\end{equation*}

If we wanted to use this function, we would need to instantiate it with a
specific type, notated using [square brackets]. For instance,

\begin{align*}
  s \colon \nat \to \nat, z \colon \nat \in \Gamma \vdash &(\Lambda \, X. \lambda \, f \colon X \to X . \lambda \, x \colon X . f (f x))[\nat] s z \colon \nat\\
  &\mapsto_{\beta} (\lambda \, f \colon \nat \to \nat . \lambda \, x \colon \nat . f (f x)) s z \colon \nat\\
  &\mapsto_{\beta^{\star}} s s z \colon \nat.
\end{align*}

When trying to formalise System F, we will face a choice of using an
intrinsically- or extrinsically-typed approach; although these were first
described by Alonzo Church and Haskell Curry respectively, we will use the
former two terms. The different approaches are described in detail in
\citet{gries_what_2003}, but to summarise, an intrinsic approach will define the
types of terms before the terms themselves, whereas an extrinsic approach will
define terms first (without types), and only later justify that the type system
is consistent.

Using the intrinsic approach will require less code and effort, so we shall
proceed that way.

\section{De Bruijn Indicies and Locally Nameless Sets}

Suppose we had the following expression,

\begin{equation*}
  x \colon \nat \in \Gamma \vdash \lambda \, y \colon \nat \to \nat. y x \colon \nat.
\end{equation*}

This takes in a function $y$ and applies it to the $x$ that is in the context.
Now suppose we move this expression into a context where we already have a bound
$y$.

\begin{equation*}
  x \colon \nat, y \colon \nat \to \nat \in \Gamma \vdash \lambda \, y \colon \nat \to \nat. y x \colon \nat.
\end{equation*}

It's unclear whether we are referring to the local $y$ or the previously bound
$y$ that is in our context (if we were to use actual functions on the naturals,
then we could have a real problem if the outer $y$ is the squaring function and
the inner $y$ the successor function, for instance). We can use an
$\alpha$-conversion and resolve this issue. We shall apply the conversion $y
\mapsto_{\alpha} q$ to our inner expression,

\begin{equation*}
  x \colon \nat, y \colon \nat \to \nat \in \Gamma \vdash \lambda \, q \colon \nat \to \nat. q x \colon \nat,
\end{equation*}

which solves our problem. We can add further assumptions to our context without
affecting the semantics of the expression (for example, adding $k \colon \nat$
to $\Gamma$ won't change the semantics of the expression), this is called
weakening-invariance (taken from proof theory where extra assumptions make a
theorem weaker).

Since we have these $\alpha$-equivalent expressions, we can say that we have
\textit{quotient} inductive definitions \citep{aydemir_engineering_2008}, since
we have both an inductive defintion of the $\lambda$-calculus, but also
(infinitely) many $\alpha$-equivalence classes which we need to deal
with\footnote{The name \textit{quotient} is taken from other areas of
mathematics where equivalence classes produce quotient objects. For example, in
ring theory, for a ring $R$ and ideal $I \subseteq R$, the quotient ring $R/I$
is the set of equivalence classes where for all $a, b \in R$ we have the
relation $a \sim b \iff a - b \in I$. If we let $R = \integer$ and $I =
2\integer$, the relation is $a \sim b \iff a - b \in 2\integer \iff \text{they have
the same parity}$. The quotient ring $R/I$ is just the set $\{0, 1\}$. In our
case, the relation is $a \sim b \iff a =_{\alpha} b$, and our quotient becomes the
set of possible terms which are semantically distinct}. This becomes difficult
in Agda, since Agda primarily relies on inductive definitions
\citep{pitts_locally_2023}.

We can solve this by using \textit{De Bruijn indicies}, where, no matter what
variables we have in our context, we don't have any local variable names which
could cause issues, since we only use indicies that directly reference the
scope. In our example,

\begin{equation*}
  x \colon \nat, y \colon \nat \to \nat \in \Gamma \vdash \lambda \nat \to \nat. 0 2 \colon \nat.
\end{equation*}

We can't use any $\alpha$-conversions since each bound variable is indexed by a
(unique) natural number. However, if we were to change the context, we would
need to change the index too. So we have a purely inductive deifnition. For
example, if we remove the $y$,

\begin{equation*}
  x \colon \nat, \in \Gamma \vdash \lambda \nat \to \nat. 0 1 \colon \nat,
\end{equation*}

then we need to reindex the $2$ to a $1$. We have lost weakening invariance
\citep{aydemir_engineering_2008}.

Using \textit{locally nameless sets}, we can get both purely inductive
defintions \textit{and} weakening-invariance. Free variables will use variable
names while bound variables will use indicies. Our example becomes

\begin{align*}
  x \colon \nat, \in \Gamma &\vdash \lambda \nat \to \nat. 0 x \colon \nat,\quad \text{or with another variable in the context,}\\
  x \colon \nat, y \colon \nat \to \nat \in \Gamma &\vdash \lambda \nat \to \nat. 0 x \colon \nat.
\end{align*}

This approach has been of research interest as of late. As part of using locally
nameless terms, common operations (called `infrastructure' by
\citet{aydemir_engineering_2008}) need to be defined for the language used as
part of the metatheory. A recent article by \citet{pitts_locally_2023} explores
locally nameless sets in Agda, and proves that this infrastructure can be
defined in a syntax-agnostic way. This will form part of the basis of our
approach to implementing System F in Agda.

\section{Prior research}
System F was previously formalised in Agda by \citet{hutton_system_2019}.
However, the authors of that paper formalised a variant of System F with
language extensions known as \textit{System F$_{\omega \mu}$}. They also used a
different approach, opting to make use of De Bruijn indicies. This paper will
feature a novel approach using locally nameless sets.

\chapter{Conclusions}

\section{Final Reminder}

The body of your dissertation, before the references and any appendices,
\emph{must} finish by page~40. The introduction, after preliminary material,
should have started on page~1.

You may not change the dissertation format (e.g., reduce the font size, change
the margins, or reduce the line spacing from the default single spacing). Be
careful if you copy-paste packages into your document preamble from elsewhere.
Some \LaTeX{} packages, such as \texttt{fullpage} or \texttt{savetrees}, change
the margins of your document. Do not include them!

Over-length or incorrectly-formatted dissertations will not be accepted and you
would have to modify your dissertation and resubmit. You cannot assume we will
check your submission before the final deadline and if it requires resubmission
after the deadline to conform to the page and style requirements you will be
subject to the usual late penalties based on your final submission time.

% \bibliographystyle{plain}
\bibliographystyle{plainnat}
% \bibliography{mybibfile}
\bibliography{system_f_in_agda_via_pitt}


% You may delete everything from \appendix up to \end{document} if you don't need it.
\appendix

\chapter{First appendix}

\section{First section}

Any appendices, including any required ethics information, should be included
after the references.

Markers do not have to consider appendices. Make sure that your contributions
are made clear in the main body of the dissertation (within the page limit).

\chapter{Participants' information sheet}

If you had human participants, include key information that they were given in
an appendix, and point to it from the ethics declaration.

\chapter{Participants' consent form}

If you had human participants, include information about how consent was
gathered in an appendix, and point to it from the ethics declaration.
This information is often a copy of a consent form.


\end{document}
